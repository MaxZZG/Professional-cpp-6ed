通过阅读这一章,学会了如何设计可重用的代码。了解了可重用代码的思想,“一次编写,多次使用”,并学到了可重用代码应该既有通用性又易于使用,设计可重用代码需要使用抽象、适当组织代码,并设计良好的接口。

这一章提供了具体的方法来组织代码:避免组合不相关或逻辑上分离的概念,使用模板为泛型数据结构和算法,提供适当的检查和防护措施,以及为可扩展性设计。

这一章还介绍了六种设计接口的策略:遵循熟悉的工作方式,不要省略所需的功能,提供整洁的接口,提供文档,提供执行相同功能的多种方式,以及提供可定制性。它还讨论了如何调和通常相互冲突的通用性和易用性的要求。

这一章以 SOLID 这个容易记住的首字母缩略词作为结尾,描述了本书和其他章节中讨论的最重要的设计原则。

这是第二部分的最后一章,该部分专注于在较高层次上讨论设计主题。下一部分将深入探讨软件工程过程的实现阶段,包括 C++ 编程的详细信息。