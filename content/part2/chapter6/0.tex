\noindent
\textbf{WHAT’S IN THIS CHAPTER?}

\begin{itemize}
\item
The reuse philosophy: Why you should design code for reuse

\item
How to design reusable code

\item
How to use abstraction

\item
Strategies for structuring your code for reuse

\item
Six strategies for designing usable interfaces

\item
How to reconcile generality with ease of use

\item
The SOLID principles
\end{itemize}

Reusing libraries and other code in your programs as discussed in Chapter 4, “Designing Professional C++ Programs,” is an important design strategy. However, it is only half of the reuse strategy. The other half is designing and writing your own code that you can reuse in your programs. As you’ve probably discovered, there is a significant difference between well-designed and poorly designed libraries. Well-designed libraries are a pleasure to use, while poorly designed libraries can prod you to give up in disgust and write the code yourself. Whether you’re writing a library explicitly designed for use by other programmers or merely deciding on a class hierarchy, you should design your code with reuse in mind. You never know when you’ll need a similar piece of functionality in a subsequent project.

Chapter 4 introduces the design theme of reuse and explains how to apply this theme by incorporating libraries and other code into your designs, but it doesn’t explain how to design reusable code. That is the topic of this chapter. It builds on the object-oriented design principles described in Chapter 5, “Designing with Classes.”
