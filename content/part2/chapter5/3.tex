
从程序化思维过程过渡到面向对象,常常需要将属性和行为组合成类。一些开发者会回顾自己程序的设计,并将某些部分重写为类。其他人可能会受到诱惑,将所有代码扔掉,并重新开始编写一个完全面向对象的程序。

使用类开发软件有两种主要方法,类仅仅是数据和功能的优雅封装。开发者的程序中使用类,以使代码更具可读性和易于维护。采取这种方法的程序员会像外科医生植入起搏器一样,将孤立的部分代码替换为类。这种方法并没有本质上的错误。这些人认为,类是一个有益的工具。程序的某些部分“感觉就像一个类”,比如股票报价。这是可以隔离,并用现实的描述。

还有一些开发者完全采用面向对象编程范式,并将所有内容都转换为类。在他们看来,一些类对应于现实世界中的事物,比如橙子或股票报价,而其他类则封装了更抽象的概念,比如排序器或还原类。

两种极端并非理想的方式。我们第一个面向对象的程序,可能真的只是一个带有几个类的传统程序化程序。可能会全力以赴,将一切都变成一个类,从代表 int 的类到代表主应用程序的类。随着时间的推移,会找到其中的平衡点。

\mySubsubsection{5.3.1.}{过度分类}

设计面向对象系统时,往往有一条界限,就是是否将每一件小事都变成一个类,从而让团队中的其他人感到烦恼。正如弗洛伊德曾经说过的那样,有时候一个变量就是一个变量。好吧,这是对他的话的转述。

也许正在设计下一个畅销的井字游戏,对这个游戏采用了全面的面向对象设计,所以坐下来,拿着一杯咖啡和一张便签纸,开始规划类和对象。像这样的游戏中,通常有一个类负责监督游戏玩法,并能够检测获胜者。为了表示游戏板,可能会涉及一个 Grid 类,它会跟踪标记及其位置。

网格的组成部分可能是 Piece 类,代表一个 X 或一个 O。 稍等,请退一步!这个设计提出要有一个类来代表一个 X 或一个 O。这可能是一种过度设计。毕竟,一个 char 不是同样可以代表一个 X 或一个 O 吗?更好的是,为什么 Grid 不能简单地使用一个二维枚举类型的数组?一个 Piece 类真的会使代码复杂化吗?看看下面 Piece 类的表格:

% Please add the following required packages to your document preamble:
% \usepackage{longtable}
% Note: It may be necessary to compile the document several times to get a multi-page table to line up properly
\begin{longtable}{|l|l|l|l|}
\hline
\textbf{类} & \textbf{相关组件} & \textbf{属性} & \textbf{行为} \\ \hline
\endfirsthead
%
\endhead
%
Piece          & 无                           & X 或 O              & 无               \\ \hline
\end{longtable}

这个表格有点空,暗示这里粒度太细,其无法成为一个完整的类。

另一方面,一个有远见的开发者可能会争辩说,尽管 Piece 类目前看起来相当单薄,但将其做成一个类,可以降低未来的扩展成本。

也许在将来,可以添加其他的属性,比如 Piece 的颜色或 Piece 是否是最新的移动。

另一种解决方案可能是考虑网格方格的状态,而不是使用棋子。方格的状态可以是空、X或O。为了使设计具有前瞻性,可以设计一个抽象基类 State,以及具体的派生类 StateEmpty、StateX 和 StateO。采用这种设计,将来可以在基类或派生类中添加其他属性。

这显然没有正确答案。重要的是,这些是在设计应用程序时应该考虑的问题。记住,类存在是为了帮助开发者管理代码。如果仅因“更面向对象”而使用类,那么就本末倒置了。

\mySubsubsection{5.3.2.}{过于泛化的类}

另一个更令人烦恼的是过于泛化的类。所有面向对象编程的学生都以“橙子”为例——这是毋庸置疑的类。在现实生活中的编码中,类可以相当抽象。许多面向对象的程序都有一个“应用程序类”。因为应用程序本身具有一些属性和行为,所以将应用程序表示为一个类可能有用。

过于泛化的类是一个根本不表示特定事物的类。开发者可能试图创建一个灵活或可重用的类,但最终得到的却是一个令人困惑的类。想象一个程序,它可以组织和显示媒体文件,可以对照片进行分类,组织数字音乐和电影收藏,并作为个人日记。过于泛化的方法是将所有这些事物都视为“media”对象,并构建一个可以容纳所有支持的格式的单类。这个单一类可能有一个名为“data”的属性,包含图像、歌曲、电影或日记条目的原始位,具体取决于媒体的类型。该类可能有一个名为“perform”的行为,根据不同类型的媒体适当绘制图像、播放歌曲、播放电影或打开日记条目以进行编辑。

这个单类过于泛化的线索在于属性和行为的名称。单词 data 本身没有意义——因为这个类可扩展为三种不同的用途,需要在这里使用一个通用的术语,perform 对于不同的类型将执行不同的操作。显然,这个类试图做的事情太多了。






















