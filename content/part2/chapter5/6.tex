By solving the following exercises, you can practice the material discussed in this chapter. Solutions to all exercises are available with the code download on the book’s website at www.wiley.com/go/ proc++6e. However, if you are stuck on an exercise, first reread parts of this chapter to try to find an answer yourself before looking at the solution from the website.

For the exercises in this chapter, there is no single correct solution. As you have learned in the course of this chapter, a specific problem often has several design solutions with different trade-offs. The solutions accompanying these exercises explain one possible design, but that doesn’t mean the solutions you came up with must match those.

\begin{itemize}
\item
Exercise 5-1: Suppose you want to write a car racing game. You will need some kind of model for the car itself. Assume for this exercise there is only one type of car. Each instance of that car needs to keep track of several properties, such as the current power output of its engine, the current fuel usage, the tire pressure, whether or not its driving lights are switched on, whether the windshield wipers are active, and so on. The game should allow players to configure their car with different engines, different tires, custom driving lights and windshield wipers, and so on. How would you model such a car and why?

\item
Exercise 5-2: Continuing the racing game from Exercise 5-1, you of course want to include support for human-driven cars, but also cars driven by an artificial intelligence (AI). How would you model this in your game?

\item
Exercise 5-3: Suppose part of a human resources (HR) application has the following three classes:

\begin{itemize}
\item
Employee: Keeping track of employee ID, salary, date when employee started working, and so on

\item
Person: Keeping track of a name and address

\item
Manager: Keeping track of which employees are in their team
\end{itemize}

What do you think of the high-level class diagram in Figure 5.12? Are there any changes you would make to it? The diagram doesn’t show any properties or behaviors of the different classes, as that’s the topic of Exercise 5-4.

\myGraphic{0.4}{content/part2/chapter5/images/12.png}{图 5.12}

\item
Exercise 5-4: Start from the final class diagram for Exercise 5-3. Add a couple of behaviors and properties to the class diagram. Finally, model the fact that a manager manages a team of employees.
\end{itemize}







