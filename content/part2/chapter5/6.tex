通过解决下面的练习,可以练习本章讨论的内容。所有练习的解决方案都可以在本书的网站\url{www.wiley.com/go/proc++6e}下载到源码。若在练习中卡住了,可以考虑先重读本章的部分内容,试着自己找到答案,再在从网站上寻找解决方案

对于本章的练习,没有唯一的正确解决方案。正如在本章中学到的,特定问题通常有几种设计解决方案,各有不同。这些练习中附带的解决方案解释了一种可能的设计。

\begin{itemize}
\item
\textbf{练习 5-1}: 假设想要编写一个赛车游戏,需要某种模型来表示汽车本身。假设在这个练习中只有一种类型的汽车。这种汽车每个实例都需要跟踪几个属性,例如其引擎的当前功率输出、当前的燃料消耗、轮胎压力、是否打开驾驶灯、雨刷是否激活等。游戏应该允许玩家配置他们的汽车,包括不同的引擎、不同的轮胎、定制的驾驶灯和雨刷等。应该如何建模这样的汽车,为什么?

\item
\textbf{练习 5-2}: 继续练习 5-1 的赛车游戏,当然希望包括由人类驾驶的汽车,以及由人工智能(AI)驾驶的汽车。游戏中,各位读者会如何建模这一点?

\item
\textbf{练习 5-3}: 假设人力资源(HR)应用的一部分包含以下三个类:

\begin{itemize}
\item
Employee: 跟踪员工ID、薪水、员工开始工作的日期等信息

\item
Person: 跟踪姓名和地址

\item
Manager: 跟踪他们团队的员工
\end{itemize}

你认为图 5.12 中的高级类图怎么样?会对它进行修改吗?图 5.12 没有显示不同类的属性或行为,因为那是练习 5-4 的主题。

\myGraphic{0.4}{content/part2/chapter5/images/12.png}{图 5.12}

\item
\textbf{练习 5-4}: 从练习 5-3 的最终类图开始。向类图中添加几个属性和行为。最后,建模一个经理管理一个员工团队的情况。
\end{itemize}







