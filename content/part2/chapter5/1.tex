C这样的过程式语言中,代码划分为小块,每个小块(理想情况下)都完成一个单一的任务。如果没有在C中使用函数,所有代码都会合并在一起,位于main()函数中,这样的代码难以阅读。

计算机不在乎代码是否全部在main()中,或者是否分割成具有描述性名称和注释的小块。函数是帮助程序员思考和维护代码的一种抽象。这个概念围绕着这样一个基本问题构建:“这个程序做什么?”用英语回答这个问题就是以过程式思维。在设计一个股票选择程序时,可能会这样开始思考:首先,程序从互联网获取股票报价;然后,根据特定指标对数据进行排序;接着,对排序后的数据进行分析;最后,输出买入和卖出的建议。开始编码时,可能会直接将这个心理模型转换为C函数:retrieveQuotes()、sortQuotes()、analyzeQuotes()和outputRecommendations()。

\begin{myNotic}{NOTE}
尽管C将函数称为“函数”,但C并不是一个函数式语言。在像Lisp这样的语言中,函数式指的是完全不同的抽象。
\end{myNotic}

过程式方法在程序遵循特定步骤列表时,往往工作得很好。然而,在大型现代应用程序中,通常不存在一个线性的事件序列。用户通常可以在任何时间执行任何命令,过程式思维也不涉及数据表示。在之前的例子中,没有讨论股票报价实际上是什么。

如果过程式思维模式符合您对程序的思维方式,不要担心。当意识到面向对象编程(OOP)只是另一种更灵活的思考软件的方式,就会很自然地使用了。







