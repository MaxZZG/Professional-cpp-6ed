\noindent
\textbf{WHAT’S IN THIS CHAPTER?}

\begin{itemize}
\item
What object-oriented programming design is

\item
What classes, objects, properties, and behaviors are

\item
How you can define relationships between different classes
\end{itemize}

Now that you have developed an appreciation for good software design from Chapter 4, “Designing Professional C++ Programs,” it’s time to pair the notion of classes with the concept of good design. The difference between programmers who use classes in their code and those who truly grasp object-oriented programming comes down to the way their classes relate to each other and to the overall design of the program.

This chapter begins with a brief description of procedural programming (C-style), followed by a detailed discussion of object-oriented programming (OOP). Even if you’ve been using classes for years, you will want to read this chapter for some new ideas regarding how to think about classes. I will discuss the different kinds of relationships between classes, including pitfalls programmers often succumb to when building an object-oriented program.

When thinking about procedural programming or object-oriented programming, the most important point to remember is that they just represent different ways of reasoning about what’s going on in your program. Too often, programmers get bogged down in the syntax and jargon of OOP before they adequately understand what a class is and what objects are. This chapter is light on code and heavy on concepts and ideas. Chapters 8, “Gaining Proficiency with Classes and Objects,” 9, “Mastering Classes and Objects,” and 10, “Discovering Inheritance Techniques,” go deeper in on C++ class syntax.















