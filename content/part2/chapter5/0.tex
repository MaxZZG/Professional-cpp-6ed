\noindent
\textbf{内筒概要}

\begin{itemize}
\item
什么是面向对象的程序设计

\item
什么是类、对象、属性和行为

\item
如何定义不同类之间的关系
\end{itemize}

现在已经通过第4章对良好软件设计有了理解,现在是时候将类的概念与良好设计相结合。使用类编写代码的开发者和真正掌握面向对象编程的开发者之间的区别在于,他们各自的类如何相互关联,以及与整个程序设计的关系。

本章首先简要描述了过程式编程(C风格),然后详细讨论了面向对象编程(OOP)。即使已经使用类多年,也会想阅读本章以获取关于如何考虑类的全新想法。我将讨论类之间的不同类型的关系,包括开发者在构建面向对象程序时常常陷入的陷阱。

考虑过程式编程或面向对象编程时,最重要的是要记住,其只是代表了对程序中发生的事情进行推理的不同方式。开发者常常在完全理解类和对象是什么之前,就陷入了OOP的语法和术语的泥潭,本章轻代码而重概念和想法。第8章、第9章和第10章会更深入地探讨C++类的语法。















