
作为开发者,无疑会遇到不同类之间具有共同特征,或者在某种程度上相互关联的情况。面向对象的语言为处理类之间的这种关系提供了多种机制,难点在于理解这种关系实际上是什么。类之间有两种主要的关系类型——has-a关系和is-a关系。

\mySubsubsection{5.4.1.}{Has-a关系}

具有 has-a 关系的类遵循 “A 中有一个 B” 或 “A 包含一个 B” 的模式。这种关系中,可以将一个类视为另一个类的部分,描述由其他类组成的类,所以组件通常代表 has-a 关系。

实际的例子可能是动物园和猴子之间的关系。可以说是动物园有一个猴子,或动物园包含一个猴子。代码中的动物园模拟将有一个动物园类,该类有一个猴子组件。

经常考虑用户界面场景有助于理解类之间的关系。尽管不是所有的用户界面都是用面向对象编程实现的(尽管现在大多数都是),屏幕上的视觉元素很容易转换为类。has-a 关系的 UI 类是一个包含按钮的窗口,按钮和窗口显然是两个不同的类,但它们在某种程度上相关(按钮在窗口内),所以你说窗口有一个按钮。

图 5.3 展示了现实世界和用户界面的 has-a 关系。

\myGraphic{0.7}{content/part2/chapter5/images/3.png}{图 5.3}

有两种类型的has-a关系:

\begin{itemize}
\item
聚合:聚合的对象(组件)在聚合者销毁时仍然可以继续存在。假设一个动物园对象包含了一群动物对象。当动物园对象因为破产而销毁时,动物对象(理想情况下)并不会销毁,会转移到另一个动物园。

\item
组合:如果销毁一个由其他对象组成的对象,那些其他对象也会销毁。如果销毁一个包含按钮的窗口对象,那些按钮对象也会销毁。
\end{itemize}

\mySubsubsection{5.4.2.}{Is-a关系(继承)}

is-a 关系是面向对象编程的一个基本概念,因此它有多个名称,包括派生、子类化、扩展和继承。类模拟了现实世界中具有属性和行为的对象。继承模拟了这些对象倾向于组织成层次结构,这些层次结构表示了 is-a 关系。

从根本上讲,继承遵循 “A 是 B” 或 “A 确实有点像 B” 的模式——这可能有些难懂。简单起见,再次回到动物园,但假设除了猴子之外还有其他动物。这个陈述本身已经构建了关系——一只猴子是一种动物。同样,长颈鹿是一种动物,袋鼠是一种动物,企鹅是一种动物,这又意味着什么呢?当猴子、长颈鹿、袋鼠和企鹅有一些共同点时,继承的魔力就显现出来了。这些共同点是动物共有的特性。

对于程序员来说,可以定义一个 Animal 类,封装了所有适用于每种动物的属性(大小、位置、饮食等)和行为(移动、进食、睡觉)。特定的动物,如猴子,成为 Animal 的派生类,因为一只猴子包含了一切动物的特性。猴子是一种动物,加上一些使它独特的特性。图 5.4 显示了动物的继承图,箭头表示 is-a 关系的方向。

\myGraphic{0.5}{content/part2/chapter5/images/4.png}{图 5.4}

正如猴子和大象是不同类型的动物一样,用户界面往往有不同的按钮类型。例如,复选框是一种按钮。假设按钮只是一个可以点击以执行操作的 UI 元素,复选框通过添加状态(复选框是选中还是未选中)来扩展 Button 类。

当相关联的类在 is-a 关系中时,一个目标是将共同的功能性因素提取到基类中,即其他类继承的类。如果发现所有派生类都有类似的或完全相同的代码,考虑如何将一些或所有代码移动到基类中。这样,做出的更改只发生在同一个地方,未来的派生类可以“免费”获得共享功能。

\begin{myNotic}{NOTE}
有时,has-a 或 is-a的关系明确。在其他时候,则不然。如果可以选择,has-a 关系优于 is-a 关系。
\end{myNotic}

\mySamllsection{继承}

前面的例子涵盖了继承中使用的几种技术。使用派生类时,开发者可以通过几种方式区分一个类与其父类,也称为基类或超类。派生类可以使用这些技术中的一种或多种,并且通过完成句子“A是B,则……”来鉴别。

\mySamllsection{增加功能}

派生类可以通过添加额外的功能来增强父类。例如,猴子是一种动物,能够从树上荡秋千。除了具有 Animal 类的所有成员函数之外,Monkey 类还具有 swingFromTrees() 成员函数,这是 Monkey 类特有的行为。

\mySamllsection{替换功能}

派生类可以完全替换或重写其父类的成员函数。大多数动物通过行走来移动,可以给 Animal 类一个模拟行走的 move() 成员函数。袋鼠是一种动物,它是通过跳跃,而非行走来移动。 Animal 基类的所有其他属性和成员函数仍然适用,但 Kangaroo 派生类只是改变了 move() 成员函数的工作方式。如果发现自己替换了基类所有的功能,这可能表明继承并不是正确的做法,除非基类是一个抽象基类。抽象基类迫使每个派生类实现所有抽象基类中没有实现的成员函数,并且不能创建抽象基类的实例。

\mySamllsection{添加属性}

派生类还可以向从基类继承的属性中添加新的属性。例如,企鹅具有动物的所有属性,但也有喙的长度属性。

\mySamllsection{替换属性}

C++ 提供了一种类似于覆盖成员函数的方式来覆盖属性。但这样做通常不太合适,因为它隐藏了基类的属性,基类可以为具有特定名称的属性指定一个特定的值,而派生类可以为另一个具有相同名称的属性指定另一个值。重要的是区分替换属性与派生类具有不同值属性是两个概念。例如,所有动物都有一个饮食属性,表示它们吃什么。猴子吃香蕉,企鹅吃鱼,但都没有替换饮食属性——只是在属性上分配了不同的值。

\mySamllsection{多态性}

多态性是指遵循一组标准属性和成员函数的对象可以互换使用。类定义就像对象和与之交互的代码之间的合同。根据定义,Monkey 对象都必须支持 Monkey 类的属性和成员函数。

这个概念也适用于基类。因为所有的猴子都是动物,Monkey 对象也支持 Animal 类的属性和成员函数。

多态性是面向对象编程的一部分,利用了继承所提供的东西。动物园模拟中,可以编程地遍历动物园中的所有动物,并让每个动物移动一次。因为所有的动物都是 Animal 类的成员,都知道如何移动。一些动物已经重写了移动成员函数,但这正是最好的部分——因为每个动物都以已知的方式移动,所以代码只是告诉每个动物移动,而不关心具体是哪种动物。

\mySubsubsection{5.4.3.}{Has-a和Is-a的区别}

现实中,很容易区分对象之间的 has-a 和 is-a 关系。没有人会说一个橙子有一个水果——橙子是一种水果。代码中,事情有时并不那么清楚。

考虑一个假设的类,代表一个关联数组,这是一种数据结构,可有效地将键映射到值。例如,一个保险公司可以使用 AssociativeArray 类将会员 ID 映射到名称,以便给定一个 ID,可以轻松地找到相应的会员名称。会员 ID 是键,会员名称是值。

标准的关联数组实现中,键与一个值相关联。如果 ID 14534 映射到会员名称“Kleper, Scott”,就不能同时映射到会员名称“Kleper, Marni”。在大多数实现中,如果尝试为已经有一个值的键添加第二个值,第一个值就会消失。如果 ID 14534 映射到“Kleper, Scott”,然后将 ID 14534 分配给“Kleper, Marni”,那么 Scott 实际上就没有保险了。这在以下序列中得到了演示,该序列展示了两个 insert() 成员函数的调用,以及关联数组的结果内容:

\begin{cpp}
myArray.insert(14534, "Kleper, Scott");
\end{cpp}

% Please add the following required packages to your document preamble:
% \usepackage{longtable}
% Note: It may be necessary to compile the document several times to get a multi-page table to line up properly
\begin{longtable}{|l|l|}
\hline
\textbf{键} & \textbf{值}              \\ \hline
\endfirsthead
%
\endhead
%
14534         & “Kleper, Scott” {[}string{]} \\ \hline
\end{longtable}

\begin{cpp}
myArray.insert(14534, "Kleper, Marni");
\end{cpp}

% Please add the following required packages to your document preamble:
% \usepackage{longtable}
% Note: It may be necessary to compile the document several times to get a multi-page table to line up properly
\begin{longtable}{|l|l|}
\hline
\textbf{键} & \textbf{值}              \\ \hline
\endfirsthead
%
\endhead
%
14534         & “Kleper, Marni” {[}string{]} \\ \hline
\end{longtable}

不难想象一种数据结构,类似于关联数组,但允许为给定键设置多个值。保险示例中,一个家庭可能有几个与同一个 ID 对应的名称。因为这样的数据结构类似于关联数组,所以能够以某种方式利用该功能是很不错的。关联数组只能有一个值作为键,但这个值可以是任何东西。值可以是包含多个键值对的集合(例如vector),每次为现有 ID 添加新成员时,可将名称添加到集合中:

\begin{cpp}
Collection collection; // Make a new collection.
collection.insert("Kleper, Scott"); // Add a new element to the collection.
myArray.insert(14534, collection); // Insert the collection into the array.
\end{cpp}

% Please add the following required packages to your document preamble:
% \usepackage{longtable}
% Note: It may be necessary to compile the document several times to get a multi-page table to line up properly
\begin{longtable}{|l|l|}
\hline
\textbf{键} & \textbf{值}                      \\ \hline
\endfirsthead
%
\endhead
%
14534         & \{“Kleper, Scott”\} {[}Collection{]} \\ \hline
\end{longtable}

\begin{cpp}
Collection collection { myArray.get(14534) }; // Retrieve the existing collection.
collection.insert("Kleper, Marni"); // Add a new element to the collection.
myArray.insert(14534, collection); // Replace the collection with the updated one.
\end{cpp}

% Please add the following required packages to your document preamble:
% \usepackage{longtable}
% Note: It may be necessary to compile the document several times to get a multi-page table to line up properly
\begin{longtable}{|l|l|}
\hline
\textbf{键} & \textbf{值}                                       \\ \hline
\endfirsthead
%
\endhead
%
14534         & \{“Kleper, Scott”, “Kleper, Marni”\} {[}Collection{]} \\ \hline
\end{longtable}

与字符串打交道比与集合打交道更麻烦,而且需要编写大量重复的代码。更可取的是将这种多值功能封装在单独的类中,可能称之为 MultiAssociativeArray。MultiAssociativeArray 类与 AssociativeArray 类的工作方式完全相同,只是存储每个值的方式不同,而是将每个值存储为一个字符串集合。显然,MultiAssociativeArray 与 Associative Array 之间有某种关系,因为它仍在使用关联数组来存储数据,但不太清楚这构成了一个 is-a 关系还是一个 has-a 关系。

首先,考虑 is-a 关系。假设 MultiAssociativeArray 是 AssociativeArray 的派生类。这将证明是一个糟糕的想法,MultiAssociativeArray 必须重写添加数组条目的成员函数,以便它要么创建一个集合并添加新元素,要么检索现有的集合并将其添加到其中,还须重写检索值的成员函数。但有一个问题:重写的 get() 成员函数应该返回一个单一的值,而不是一个集合。MultiAssociativeArray 应该返回哪个值?一个选项是返回与给定键相关联的第一个值。可以添加一个 getAll() 成员函数来检索与键相关联的所有值,这可能看起来是一个合理的设计。尽管写了基类的所有成员函数,但仍然在派生类中使用了基类的成员函数。图 5.5 展示了这种方法的 UML 图。

\myGraphic{0.3}{content/part2/chapter5/images/5.png}{图 5.5}

现在考虑 has-a 关系。MultiAssociativeArray 是一个独立的类,但它包含一个 AssociativeArray 对象。可能有一个与 AssociativeArray 类似的接口,但不必相同。当用户向 MultiAssociativeArray 添加某些内容时,实际上会包裹在集合中,并放入一个 AssociativeArray 对象中。这看起来完全合理,如图 5.6 所示。

\myGraphic{0.7}{content/part2/chapter5/images/6.png}{图 5.6}

那么,哪种解决方案是正确的?似乎没有明确的答案,但几十年的经验告诉我们,has-a 通常是这两种选择中更好的一个。主要原因是允许修改公开的接口,而不用担心维护关联数组功能。例如,在图 5.6 中,get() 成员函数更改为 getAll(),清楚地表明这个函数获取 MultiAssociativeArray 中特定键的所有值。此外,具有 has-a 关系的类,不必担心关联数组功能渗透进来。例如,具有 is-a 关系的类,如果关联数组类支持一个可以获取值总数的成员函数,它将报告集合的数量。

然而,有人可能会尝试论证,MultiAssociativeArray 实际上是一个具有某些新功能的 AssociativeArray,应该是一个 is-a 关系。这两种关系之间有时有一条界限,需要考虑这个类将如何使用,以及所构建的是否只是利用了另一个类的某些功能,还是真的就是那个类,但具有修改或新功能。

以下表格代表了对 MultiAssociativeArray 类采用两种方法的正反论据:

% Please add the following required packages to your document preamble:
% \usepackage{longtable}
% Note: It may be necessary to compile the document several times to get a multi-page table to line up properly
\begin{longtable}{|l|l|l|}
\hline
\textbf{} &
\textbf{IS-A} &
\textbf{HAS-A} \\ \hline
\endfirsthead
%
\endhead
%
同意的原因 &
\begin{tabular}[c]{@{}l@{}}从根本上说,它是具有不同特征的\\相同抽象。提供(几乎)与\\AssociativeArray相同的成员函数。\end{tabular} &
\begin{tabular}[c]{@{}l@{}}MultiAssociativeArray可以有成员函数,\\而不必担心AssociativeArray有什么成\\员函数。\\ 实现可以更改为AssociativeArray以外\\的其他内容,而无需更改公开的成员\\函数。\end{tabular} \\ \hline
\begin{tabular}[c]{@{}l@{}}反对的原因\end{tabular} &
\begin{tabular}[c]{@{}l@{}}根据定义,关联数组每个键有\\一个值。说MultiAssociativeArray\\是一个关联数组是亵渎神明!\\MultiAssociativeArray覆盖\\AssociativeArray的两个成员函数,\\这是一个强烈的信号,表明设计\\有问题。\\ AssociativeArray的未知或不适当\\的属性或成员函数可能“溢出”到\\MultiAssociativeArray。\end{tabular} &
\begin{tabular}[c]{@{}l@{}}MultiAssociativeArray通过提出新的成员\\函数,是在重新发明轮子。\\ AssociativeArray的一些附加属性和成员\\函数则可能是有用的。\end{tabular} \\ \hline
\end{longtable}

这种情况下,反对使用 is-a 关系的理由非常强烈。此外,Liskov 替换原则(LSP)可以帮助决定是使用 is-a 关系,还是 has-a 关系。这个原则指出,应该能够使用派生类代替基类而不改变行为。应用于这个例子,表明这必须是一个 has-a 关系,不能简单地使用 MultiAssociativeArray 代替之前使用的 AssociativeArray。如果这样做,行为将会改变。AssociativeArray 的 insert() 成员函数会删除数组中已有的具有相同键的早期值,而 MultiAssociativeArray 不会删除这样的值。

本节中详细解释的两种解决方案,实际上并不仅有两种可能的解决方案。

其他选项可以是让 AssociativeArray 包含一个 MultiAssociativeArray,或者让 AssociativeArray 和 MultiAssociativeArray 都继承自一个共同的基类等。对于一个特定的设计,通常可以想出多种解决方案。

\begin{myWarning}{WARNING}
如果确实有选择两种关系类型的机会,根据多年的经验,我建议选择 has-a ,而非 is-a。
\end{myWarning}

请注意,AssociativeArray 和 MultiAssociativeArray 是在这里用来演示 is-a 和 has-a 之间的区别。代码中,建议使用标准关联数组类。C++ 标准库提供了std::map用于代替 AssociativeArray,并使用std::multimap代替 MultiAssociativeArray。

\mySubsubsection{5.4.4.}{Not-a关系}

考虑类之间具有哪种类型的关系时,应该考虑它们是否具有关系。不要让面向对象设计的热情变成许多不必要的类/派生类关系。

陷阱发生在明显相关的事物在代码中,实际上没有关系。面向对象的层次结构需要模拟功能关系,而不是人为的关系。图 5.7 显示了作为本体或层次结构的有意义的联系,但在代码中的联系可能无意义。

\myGraphic{0.7}{content/part2/chapter5/images/7.png}{图 5.7}

避免不必要的继承的最佳方法是,首先绘制出你的设计。对于每个类和派生类,写下打算放入类的属性和成员函数。如果发现一个类没有特别的属性和成员函数,或者其所有属性和成员函数都可使用派生类完全覆盖,除了前面提到的抽象基类之外,应该考虑重新设计。

\mySubsubsection{5.4.5.}{层次结构}

就像类 A 可以是类 B 的基类一样,B 也可以是类 C 的基类。面向对象的层次结构可以模拟这种多级关系。一个动物园模拟程序可能会为每个动物设计公共 Animal 类的派生类,如图 5.8 所示。

\myGraphic{0.5}{content/part2/chapter5/images/8.png}{图 5.8}

当为这些派生类编写代码时,可能会发现它们中有有很多相似。这种情况下,应该考虑将相同的部分放入一个共同的父类中。狮子和豹子都以相同的方式移动,并具有相同的饮食可能会表明需要一个可能的 BigCat 类。可以进一步将 Animal 类细分为 WaterAnimal 和 Marsupial。图 5.9 展示了一个更层次化的设计,利用了这种共同性。

\myGraphic{0.7}{content/part2/chapter5/images/9.png}{图 5.9}

一位生物学家看着这个层次结构可能会感到失望——企鹅实际上并不和海豚在同一个家族,这强调了一个重要观点——代码中,需要平衡与现实世界的关系。即使两件事在现实世界中可能密切相关,但在代码中可能是一个 not-a 的关系。可以将动物分为哺乳动物和鱼类,但这不会为基类带来共性。

另一个重要观点是,可能有其他组织层次结构的方法。前面的设计主要按照动物的运动方式来组织。如果按照动物的饮食或身高来组织,层次结构可能会非常不同。最终,重要的是类将如何使用,需求将决定类层次结构的设计。

一个好的面向对象层次结构实现了以下目标:

\begin{itemize}
\item
将类组织成有意义的功能关系

\item
将常见功能移到基类中,支持代码重用

\item
避免派生类覆盖大部分父类的功能,除非父类是抽象基类
\end{itemize}

\mySubsubsection{5.4.6.}{多重继承}

到目前为止的所有示例都只有一个继承链,给定的类最多只有一个直接父类。可以通过多重继承,一个类可以有多个基类。

图 5.10 显示了一个多重继承设计。仍然有一个名为 Animal 的基类,根据大小进一步细分为不同的类。另一个层次结构根据饮食进行分类,第三个则处理运动。每种动物都是这三个类的派生类,如图中不同线条所示

\myGraphic{0.9}{content/part2/chapter5/images/10.png}{图 5.10}

用户界面的上下文中,想象一个用户可以点击的图像。这个类看起来既是按钮又是图像,所以实现可能涉及从 Image 类和 Button 类同时继承,如图 5.11 所示。

\myGraphic{0.3}{content/part2/chapter5/images/11.png}{图 5.11}

多重继承在某些情况下是有用的,但应该始终记住它也有许多缺点。许多开发者不喜欢多重继承。C++ 支持多重继承,Java 语言完全去除了多重继承,除非从多个接口(抽象基类)继承。多重继承的批评者指出了几个原因。

首先,多重继承的视觉化很复杂。如图 5.10 所示,即使是一个简单的类图,当有多个层次结构和交叉线时,会变得复杂。

类层次结构本应使开发者更容易理解代码之间的关系,但多重继承意味着一个类可能有几个毫无关系的父类。由于这么多类都为对象贡献代码,能确定发生了什么吗?

其次,多重继承可能会破坏原本干净的层次结构。在动物示例中,切换到多重继承方法意味着 Animal 基类变得不那么有意义,因为描述动物的代码现在分成了三个单独的层次结构。尽管图 5.10 所示的设计展示了三个干净的层次结构,就可能会变得混乱。如果意识到所有跳跃者不仅以相同的方式移动,而且也吃相同的东西,因为存在单独的层次结构,所以无法将运动和饮食的概念结合起来,除非添加另一个派生类。

第三,多重继承的实现很复杂。如果两个基类以不同的方式实现同一个成员函数,有两个基类,它们本身就是从一个共同基类派生的吗?这些可能性使得实现变得复杂,因为编写这样的复杂关系代码对作者和读者来说都很难。

其他语言可以完全去掉多重继承,通过重新思考层次结构,可以在项目设计时避免引入多重继承。

\mySubsubsection{5.4.7.}{混合类}

混合类代表了类之间另一种类型的关系。在 C++ 中,实现混合类的方法在语法上与多重继承相似,但语义上有很大的不同。混合类回答了“这个类还能做什么?”的问题,答案往往以“-able”结尾。混合类是一种可以向类添加功能的类,而不必以 is-a 关系的方式,可以将其视为一种共享关系。

回到动物园的例子,想引入一种动物可以“抚摸”的概念。有一些动物游客可以在动物园抚摸,前提是他们不会被咬伤或攻击。可能希望所有可抚摸的动物都支持“被抚摸”的行为。由于可抚摸的动物没有什么其他共同点,而且不想破坏设计好的层次结构,那么Pettable 就是一个很好的混合类。

混合类在用户界面中经常使用。与其说 PictureButton 类既是 Image 又是 Button,不如说它是一个可以点击的 Image,桌面上的文件夹图标可能是一个可以拖动和点击的 Image。软件开发人员倾向于编造很多有趣的形容词。

混合类与基类之间的区别更多地在于如何思考类,而不是代码差异。通常,混合类比多重继承更容易理解,因为其范围非常有限。Pettable 混合类只是向现有类添加一个行为,Clickable 混合类可能只添加“鼠标按下”和“鼠标释放”行为。此外,混合类很少有大的层次结构,所以功能上不存在交叉污染。





























