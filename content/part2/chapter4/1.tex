开始一个新的程序或现有程序的新特性时,第一步是分析需求,这包括与相关人员进行讨论。这个分析阶段的关键成果是一个功能需求文档,描述了代码需要做什么,但不解释如何做到这一点。需求分析也可能导致一个非功能需求文档,描述最终系统应该是什么样的,它应该做什么相比。非功能需求的例子包括系统需要安全、可扩展、满足某些性能标准等。

收集了所有需求,项目的设计阶段就可以开始了。程序设计或软件设计,是实现所有功能和非功能需求的程序架构的规范,设计是计划如何编写程序的方式,应该以设计文档的形式写下相应的设计。尽管每个公司或项目都有自己的设计文档格式,但大多数设计文档都有相同的通用布局,包括两个主要部分:

\begin{itemize}
\item
将程序划分为子系统,包括子系统之间的接口和依赖关系、子系统之间的数据流、每个子系统的输入和输出,以及通用的线程模型

\item
每个子系统的详细信息,包括划分为类、类层次结构、数据结构、算法、特定的线程模型,以及错误处理的细节
\end{itemize}

设计文档通常包括展示子系统交互和类层次结构的图表和表格。统一建模语言(UML)是此类图表的行业标准,并在本章和后续章节中用于图表。请参阅附录D,以了解UML语法的简要介绍。设计文档的确切格式并不重要,重要的是思考开发者设计的过程。

\begin{myNotic}{NOTE}
设计的目的在于在编写代码之前进行思考。
\end{myNotic}

通常应该在开始编码之前尽量设计好,设计应该为开发者提供程序的路线。开始编码并遇到之前没有想到的问题,设计不可避免地需要修改。软件工程流程已经设计好了,可以很灵活地应对这些改变。敏捷软件开发方法论之一,Scrum就是一个这样的迭代过程的例子,应用程序在称为冲刺的周期内开发。每次冲刺,设计都可以修改,新需求也可以考虑进去。第28章更详细地描述了各种软件工程的流程模型。






























