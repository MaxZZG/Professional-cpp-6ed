
本节介绍了一种使用C++系统性方法设计的国际象棋。为了提供一个完整的示例,其中一些步骤涉及后面章节中介绍的概念。可以先了解一下这个示例,以获得设计过程的大致了解,之后再去了解那些还未涉及的概念。

\mySubsubsection{4.6.1.}{前置信息}

开始设计之前,清晰了解程序的功能和效率的需求非常重要。理想情况下,这些要求将以需求规范的形式记录下来。国际象棋需要包含以下类型的规范:

\begin{itemize}
\item
支持国际象棋的标准规则。

\item
支持两名人类玩家,不提供计算机玩家。

\item
提供一个基于文本的界面:
\begin{itemize}
\item
以纯文本形式渲染棋盘和棋子。

\item
通过输入代表棋盘位置的数字来表示棋子的移动。
\end{itemize}
\end{itemize}

这些需求确保设计的程序能够满足用户期望。

\mySubsubsection{4.6.2.}{设计步骤}

设计程序时,应该采取一种系统化的方法,从一般到具体逐步进行。以下步骤并不适用于所有程序,只是提供了一种一般性指导原则。设计应包括适当的图表和表格,UML是行业标准。可以参考附录D了解简要介绍,UML定义了多种标准的图,可以使用它们来记录软件设计,如类图、序列图等。我建议在适用的情况下使用UML或类似UML的图,但我不主张严格遵循UML语法,因为一个清晰、易于理解的图,比拥有一个语法正确的图更为重要。

\mySamllsection{将程序划分为子系统}

第一步是将程序划分为功能性子系统,并指定子系统之间的接口和交互。不要担心数据结构和算法的具体细节,甚至是类。只是在尝试了解程序的各个部分,以及它们之间如何交互。可以在一个表格中列出子系统,该表格表达子系统的高层行为或功能,子系统向其他子系统导出接口,以及使用其他子系统的接口。

对于这个国际象棋游戏,推荐的设计是使用模型-视图-控制器(MVC)范式,以清晰地区分存储数据和显示数据。这个范式模拟了这样一个情况:许多应用程序通常处理一组数据,该数据的一个或多个视图,以及该数据的操作。 MVC 中,一组数据称为模型,视图是模型的特定可视化,控制器是响应某些事件而改变模型的代码片段。MVC 的三个组件相互作用形成一个反馈循环:动作由控制器处理,控制器调整模型,导致视图或视图的变化。控制器也可以直接修改视图,例如 UI 元素。图 4.4 可视化这种交互。使用这个范式,清晰地分离不同的组件,允许修改一个组件而不影响其他组件。可以轻松地在基于文本的界面和图形用户界面之间切换,或者在桌面 PC 上运行的界面和手机上运行的界面之间切换,而不必触及底层的数据模型或逻辑。

\myGraphic{0.7}{content/part2/chapter4/images/4.png}{图4.4}

下表显示了象棋游戏的子系统:

% Please add the following required packages to your document preamble:
% \usepackage{longtable}
% Note: It may be necessary to compile the document several times to get a multi-page table to line up properly
\begin{longtable}{|l|l|l|l|l|}
\hline
\textbf{子系统名称} &
\textbf{实例} &
\textbf{功能} &
\textbf{\begin{tabular}[c]{@{}l@{}}导出的接口\end{tabular}} &
\textbf{\begin{tabular}[c]{@{}l@{}}使用的接口\end{tabular}} \\ \hline
\endfirsthead
%
\endhead
%
GamePlay &
1 &
\begin{tabular}[c]{@{}l@{}}开始游戏\\ 控制游戏流程\\ 控制绘图\\ 宣布获胜者\\ 结束游戏\end{tabular} &
Game Over &
\begin{tabular}[c]{@{}l@{}}Take Turn (on Player)\\ Draw (on ChessBoardView)\end{tabular} \\ \hline
ChessBoard &
1 &
\begin{tabular}[c]{@{}l@{}}储存棋子\\ 检查棋局状态\end{tabular} &
\begin{tabular}[c]{@{}l@{}}Get Piece At\\ Set Piece At\end{tabular} &
\begin{tabular}[c]{@{}l@{}}Game Over (on GamePlay)\end{tabular} \\ \hline
ChessBoardView &
1 &
\begin{tabular}[c]{@{}l@{}}绘制相关的\\ChessBoard\end{tabular} &
Draw &
\begin{tabular}[c]{@{}l@{}}Draw (on ChessPieceView)\end{tabular} \\ \hline
ChessPiece &
32 &
\begin{tabular}[c]{@{}l@{}}移动\\ \\检查移动是否\\符合规则\end{tabular} &
\begin{tabular}[c]{@{}l@{}}Move\\ Check Move\end{tabular} &
\begin{tabular}[c]{@{}l@{}}Get Piece At (on ChessBoard)\\ Set Piece At (on ChessBoard)\end{tabular} \\ \hline
ChessPieceView &
32 &
\begin{tabular}[c]{@{}l@{}}绘制相关的\\ChessPiece\end{tabular} &
Draw &
无 \\ \hline
Player &
2 &
\begin{tabular}[c]{@{}l@{}}通过提示用户\\移动,并获取\\用户的移动信\\息进行交互\\\\ 移动棋子\end{tabular} &
Take Turn &
\begin{tabular}[c]{@{}l@{}}Get Piece At (on ChessBoard)\\ Move (on ChessPiece)\\ Check Move (on ChessPiece)\end{tabular} \\ \hline
ErrorLogger &
1 &
\begin{tabular}[c]{@{}l@{}}将错误消息写\\入日志文件\end{tabular} &
Log Error &
无 \\ \hline
\end{longtable}

如表所示,这个国际象棋游戏的功能子系统,包括一个 GamePlay 子系统,一个 ChessBoard 和 ChessBoardView,32 个 ChessPieces 和 ChessPieceViews,两个 Players 和一个 ErrorLogger。在软件设计,就像编程本身一样,通常有许多不同的方法可以实现相同的目标。并非所有解决方案都是平等的;有些显然比其他的更好,通常有多个有效的解决方案。

好的子系统划分将程序划分为其基本的功能部分,玩家是与 ChessBoard、ChessPieces 或 GamePlay 不同的子系统。将玩家归入 GamePlay 子系统没有意义,它们在逻辑上是独立的子系统,其他可能没有那么明显。

这个 MVC 设计中,ChessBoard 和 ChessPiece 子系统是模型的一部分。ChessBoardView 和 ChessPieceView 是视图的一部分,而 Player 是控制器的一部分。

由于很难从表格中可视化子系统关系,因此在图中显示程序的子系统通常很有帮助,其中线条代表从一个子系统到另一个子系统的调用。图 4.5 显示了国际象棋游戏子系统,可视化为基于 UML 的图。

\myGraphic{0.9}{content/part2/chapter4/images/5.png}{图 4.5}

\mySamllsection{选择线程模型}

设计的早期阶段,考虑如何编写算法中实现特定循环的多线程还为时过早。在这一步中,应该选择程序中的高层线程数量,并指定它们的交互。高层线程的例子包括 UI 线程、音频播放线程、网络通信线程等。

多线程设计中,应尽可能避免共享数据,这会使设计更简单、更安全。如果无法避免共享数据,应使用锁。

如果不熟悉多线程程序,或者平台不支持多线程,则应该使程序单线程运行。如果程序有多个不同的任务,每个任务都可以并行工作,那么将有可能是多线程化。图形用户界面应用程序通常有一个线程执行主要的应用程序工作,另一个线程等待用户按下按钮或选择菜单项。多线程编程在第 27 章中进行介绍。

国际象棋程序只需要一个线程来控制游戏流程。

\mySamllsection{指定类层次结构}

这一步中,需要确定在程序中编写的类层次结构。国际象棋程序可以使用一个类层次结构来表示棋子。层次结构可以如图 4.6 所示那样,通用的 ChessPiece 类作为抽象基类。ChessPieceView 类也需要一个类似的层次结构。

\myGraphic{0.8}{content/part2/chapter4/images/6.png}{图 4.6}

另一个类层次结构可以用于 ChessBoardView 类,以使其能够具有基于文本的界面或游戏的图形用户界面。图 4.7 展示了一个示例层次结构,棋盘以文本形式显示在控制台上,或者使用 2D 或 3D 图形用户界面显示。ChessPieceView 层次结构的各个类也需要一个类似的层次结构。

\myGraphic{0.8}{content/part2/chapter4/images/7.png}{图 4.7}

第 5 章解释了设计和类层次结构的细节。

\mySamllsection{指定类、数据结构、算法和模式}

这一步中,需要考虑更详细的层次,并指定每个子系统的具体细节,包括将为每个子系统编写的具体类。很可能会发现自己将每个子系统本身建模为一个类。这些信息可以总结在一个表格中。

% Please add the following required packages to your document preamble:
% \usepackage{longtable}
% Note: It may be necessary to compile the document several times to get a multi-page table to line up properly
\begin{longtable}{|l|l|l|l|l|}
\hline
\textbf{子系统名称} &
\textbf{类} &
\textbf{\begin{tabular}[c]{@{}l@{}}数据结构\end{tabular}} &
\textbf{算法} &
\textbf{模式} \\ \hline
\endfirsthead
%
\endhead
%
GamePlay &
GamePlay类 &
\begin{tabular}[c]{@{}l@{}}GamePlay对\\象包括一个\\ChessBoard、\\对象和两个\\Player 对象\end{tabular} &
\begin{tabular}[c]{@{}l@{}}每个玩家一个回合\end{tabular} &
无 \\ \hline
ChessBoard &
ChessBoard类 &
\begin{tabular}[c]{@{}l@{}}ChessBoard对\\象存储一个二\\维的8x8网格,\\包含32个\\ChessPieces\end{tabular} &
\begin{tabular}[c]{@{}l@{}}走棋后判断胜负\end{tabular} &
无 \\ \hline
ChessBoardView &
\begin{tabular}[c]{@{}l@{}}抽象基类:\\ChessBoardView\\ 具体的派生类:\\ChessBoardViewConsole,\\ChessBoardViewGUI2D等\end{tabular} &
\begin{tabular}[c]{@{}l@{}}存储有关\\如何绘制\\棋盘的信息\end{tabular} &
\begin{tabular}[c]{@{}l@{}}画个棋盘\end{tabular} &
观察者 \\ \hline
ChessPiece &
\begin{tabular}[c]{@{}l@{}}抽象基类:\\ChessPiece\\ 具体的派生类:\\Rook, Bishop, Knight,\\ King, Pawn和Queen\end{tabular} &
\begin{tabular}[c]{@{}l@{}}每个棋子\\都将其位\\置存储在\\棋盘上\end{tabular} &
\begin{tabular}[c]{@{}l@{}}通过检查盘上的\\棋子在不同的\\位置确定移动是\\否符合规则。\end{tabular} &
无 \\ \hline
ChessPieceView &
\begin{tabular}[c]{@{}l@{}}抽象基类:\\ChessPieceView\\ 派生类:\\ RookView,\\ BishopView等\\具体派生类:\\RookViewConsole,\\ RookViewGUI2D等\end{tabular} &
\begin{tabular}[c]{@{}l@{}}存储关于\\如何绘制\\棋子的信息\end{tabular} &
\begin{tabular}[c]{@{}l@{}}画一个棋子。\end{tabular} &
观察者 \\ \hline
Player &
\begin{tabular}[c]{@{}l@{}}抽象基类:\\Player\\ 具体派生类:\\PlayerConsole,\\ PlayerGUI2D等\end{tabular} &
无 &
\begin{tabular}[c]{@{}l@{}}提示用户走棋,\\检查这一步棋\\是否符合规则,\\然后移动棋子。\end{tabular} &
中介者 \\ \hline
ErrorLogger &
ErrorLogger类 &
\begin{tabular}[c]{@{}l@{}}要记录的\\消息队列\end{tabular} &
\begin{tabular}[c]{@{}l@{}}缓冲消息并将\\其写入日志文件。\end{tabular} &
\begin{tabular}[c]{@{}l@{}}依赖注入\\策略\end{tabular} \\ \hline
\end{longtable}

表中已经提供了一些关于软件设计中的不同类的信息,但并没有清晰地描述它们之间的交互。UML 序列图可以用来模型化这样的交互。图 4.8 展示了这样一个图,可视化了前一个表格中一些类之间的关系。

\myGraphic{0.7}{content/part2/chapter4/images/8.png}{图 4.8}

图 4.8 中的图表只展示了单一的迭代,从 GamePlay 到 Player 的 TakeTurn 调用,这只是一个部分序列图。 TakeTurn 完成后,GamePlay 对象应该请求 ChessBoardView 绘制自身,这反过来又应该请求不同的 ChessPieceViews 进行绘制。此外,应该扩展序列图,以可视化棋子如何吃掉对方的棋子,并包括对王车易位(castling move)的支持,这是一种涉及玩家的王和任意一个车(rook)的移动。王车易位是唯一一种玩家可同时移动两个棋子的走法。

设计文档的这一部分通常会呈现每个类的实际接口,但这个例子将省略那个层次的细节。

设计类并选择数据结构、算法和模式可能会很棘手。应该始终牢记本章前面讨论的抽象和复用原则。对于抽象,关键是要将接口和实现分开考虑。首先,从用户的角度指定接口,决定组件做什么。然后,通过选择数据结构和算法来决定组件将如何做到这一点。对于复用,熟悉标准的数据结构、算法和模式,并确保了解 C++ 的标准库,以及可用的代码。

\mySamllsection{如何进行错误处理}

这一设计步骤中,需要界定每个子系统的错误处理。错误处理应包括系统错误(如网络访问失败)和用户错误(如无效输入)。应该指定每个子系统是否使用异常,可以将这些信息总结在一个表格中。

% Please add the following required packages to your document preamble:
% \usepackage{longtable}
% Note: It may be necessary to compile the document several times to get a multi-page table to line up properly
\begin{longtable}{|l|l|l|}
\hline
\textbf{子系统} &
\textbf{处理系统错误} &
\textbf{处理用户错误} \\ \hline
\endfirsthead
%
\endhead
%
GamePlay &
\begin{tabular}[c]{@{}l@{}}用ErrorLogger记录错误,\\向用户显示一条消息,\\并在发生意外错误时关闭程序。\end{tabular} &
不适用(无直接用户界面)。 \\ \hline
\begin{tabular}[c]{@{}l@{}}ChessBoard\\ ChessPiece\end{tabular} &
\begin{tabular}[c]{@{}l@{}}使用ErrorLogger记录错误,\\并在发生意外错误时抛出异常。\end{tabular} &
不适用(无直接用户界面)。 \\ \hline
\begin{tabular}[c]{@{}l@{}}ChessBoardView\\ ChessPieceView\end{tabular} &
\begin{tabular}[c]{@{}l@{}}用ErrorLogger记录错误,\\并在绘制过程中出现错误时抛出异常。\end{tabular} &
不适用(无直接用户界面)。 \\ \hline
Player &
\begin{tabular}[c]{@{}l@{}}使用ErrorLogger记录错误,\\并在发生意外错误时抛出异常。\end{tabular} &
\begin{tabular}[c]{@{}l@{}}检查用户移动的棋子,\\确保棋子没有离开棋盘;\\然后,提示用户输入另一个条目。\\这个子系统需要在移动棋子前,\\检查每一步棋的合法性;\\如果不合规则,\\会提示用户进行另一次移动。\end{tabular} \\ \hline
ErrorLogger &
\begin{tabular}[c]{@{}l@{}}尝试记录错误;\\当发生意外错误时通知用户。\end{tabular} &
不适用(无直接用户界面)。 \\ \hline
\end{longtable}

错误处理的通用规则是处理所有事情,仔细考虑所有可能的错误情况。如果忘记了一个可能性,它将作为程序中的错误出现!不要将任何东西视为“意外”错误。期待所有可能性:内存分配失败、无效用户输入、磁盘故障和网络故障,仅举几例。然而,正如国际象棋游戏的表格所示,应该以不同的方式处理用户错误和内部错误。用户输入一个无效的移动,不应该导致的国际象棋程序终止。第 14 章会更深入地介绍错误处理。
















