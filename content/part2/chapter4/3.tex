
There are several aspects of the C++ language that you need to keep in mind when designing for C++:

\begin{itemize}
\item
C++ provides object-oriented capabilities. This means your designs can include class hierarchies, class interfaces, and object interactions. Object-oriented design is quite different compared to procedural design as used in languages such as C and others. Chapter 5 focuses on object-oriented design in C++.

\item
C++ is a multi-paradigm programming language. Besides object-oriented capabilities as described in the previous point, C++ supports other paradigms, such as procedural. Which paradigm to choose, object-oriented or procedural, is part of the design process.

\item
C++ has numerous facilities for designing generic and reusable code. Next to object-oriented and procedural capabilities, C++ supports other language facilities such as templates for generic programming. Design techniques for reusable code are discussed in more detail later in this chapter and further in Chapter 6, “Designing for Reuse.”

\item
C++ provides a large Standard Library. This includes a string class, string formatting, I/O facilities, multithreading building blocks, many common data structures and algorithms, and much more. All of these facilitate coding in C++.

\item
C++ readily accommodates many design patterns. In other words, it supports common ways to solve problems.
\end{itemize}

Tackling a design can be overwhelming. I have spent entire days scribbling design ideas on paper, crossing them out, writing more ideas, crossing those out, and repeating the process. Sometimes this process is helpful, and, at the end of those days (or weeks), it leads to a clean, efficient design. Other times it is frustrating and leads nowhere, but it is not a waste of effort. You will most likely waste more time if you have to re-implement a design that turned out to be broken. It’s important to remain aware of whether you are making real progress. If you find that you are stuck, you can take one of the following actions:

\begin{itemize}
\item
Ask for help. Consult a co-worker, mentor, book, newsgroup, or web page.

\item
Work on something else for a while. Come back to this design choice later.

\item
Make a decision and move on. Even if it’s not an ideal solution, decide on something and try to work with it. An incorrect choice will soon become apparent. However, it may turn out to be an acceptable solution. Perhaps there is no clean way to accomplish what you want to with this design. Sometimes you have to accept an “ugly” solution if it’s the only realistic strategy to fulfill your requirements. Whatever you decide, make sure you document your decision so that you and others in the future know why you made it. This includes documenting designs that you have rejected and the rationale behind the rejection.
\end{itemize}

\begin{myNotic}{NOTE}
Keep in mind that good design is hard, and getting it right takes practice. Don’t expect to become an expert overnight—and don’t be surprised if you find it more difficult to master C++ design than C++ coding.
\end{myNotic}











