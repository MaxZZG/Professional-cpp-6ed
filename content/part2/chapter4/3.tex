
设计需要考虑几个方面:

\begin{itemize}
\item
C++提供了面向对象的功能,所以设计可以包括类层次、类接口和对象交互。与C等语言中使用的过程式设计相比,面向对象的设计非常不同。第5章专注于C++中的面向对象设计。

\item
C++是一种多范式编程语言。除了前面提到的面向对象功能外,C++还支持其他范式,如过程式。选择面向对象,还是过程式范式是设计过程的一部分。

\item
C++提供了大量用于设计通用和可重用代码的工具。除了面向对象和过程式能力外,C++还支持其他语言功能,如模板用于泛型编程。关于可重用代码的设计技术将在本章的后面部分,以及第6章中详细介绍。

\item
C++提供了一个大的标准库。这包括一个字符串类、字符串格式化、I/O工具、多线程构建块、许多常见数据结构和算法,以及更多,这些都有助于C++编程。

\item
C++可以轻松适应许多设计模式,支持解决常见问题的通用方法。
\end{itemize}

对于如何进行设计,可能会感到手足无措。我曾经花了整整一天的时间在纸上潦草地写下设计想法,划掉它们,写下更多的想法,再划掉,然后重复这个过程。有时这个过程很有帮助,到了那几天(或几周)的末尾,才有了一个清晰、高效的设计。在其他时候,这个过程令人沮丧,没有结果,但这不是白费力气。若需要重新实现一个设计,结果证明是错误的,可能会浪费更多的时间。重要的是要意识到否取得了真正的进展。若发现自己陷入困境,可以采取以下行动:

\begin{itemize}
\item
寻求帮助。咨询同事、导师、书籍或网页。

\item
暂时做其他事情。稍后再回到这个设计工作。

\item
做出决定并继续前进。即使它不是理想的解决方案,也决定一个并尝试使其工作。若是错误的选择,很快就会显现出来,但它可能证明是一个可接受的解决方案,也许这个设计没有干净的方法来实现想要的东西。有时若这满足要求的最现实策略,则需要接受一个“丑陋”的解决方案。无论决定什么,都要记录下相关的决定,这样你自己和未来的其他人就知道当时为什么做出了这个决定。这包括记录了被拒绝的设计和拒绝的理由。
\end{itemize}

\begin{myNotic}{NOTE}
做出良好的设计非常困难,正确地做好需要练习。不要期望一夜之间成为专家——若发现掌握C++设计比掌握C++编码还困难,这很正常。
\end{myNotic}











