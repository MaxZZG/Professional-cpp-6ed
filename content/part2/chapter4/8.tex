通过解决下面的练习,可以练习本章讨论的内容。所有练习的解决方案都可以在本书的网站\url{www.wiley.com/go/proc++6e}下载到源码。若在练习中卡住了,可以考虑先重读本章的部分内容,试着自己找到答案,再在从网站上寻找解决方案。

\begin{itemize}
\item
\textbf{练习 4-1}:在 C++ 中进行自主设计时,需要遵循哪两个基本设计规则?

\item
\textbf{练习 4-2}: 假设有以下Card类。该类只支持牌组中的普通牌,而不支持小丑牌。

\begin{cpp}
class Card
{
    public:
        enum class Number { Ace, Two, Three, Four, Five, Six, Seven, Eight,
            Nine, Ten, Jack, Queen, King };
        enum class Figure { Diamond, Heart, Spade, Club };
        Card() {}
        Card(Number number, Figure figure)
        : m_number { number }, m_figure { figure } {}
    private:
        Number m_number { Number::Ace };
        Figure m_figure { Figure::Diamond };
};
\end{cpp}

认为下面使用Card类来表示一副牌怎么样?能想到需要改进的地方吗?

\begin{cpp}
int main()
{
    Card deck[52];
    // ...
}
\end{cpp}

\item
\textbf{练习 4-3}: 假设你和一个朋友一起想出了一个制作移动设备3D游戏的好主意。你使用的是Android设备,而你的朋友使用的是苹果iPhone,你们当然希望这款游戏能够同时在两种设备上运行。阐述你将如何处理这两种不同的手机平台,以及将如何为游戏开发做准备。

\item
\textbf{练习 4-4}: 给定以下大O复杂度:O(n), O($n^2$), O(log n)和O(1),能根据复杂度的增加对它们进行排序吗?他们叫什么名字?能想到比这些更复杂的吗?
\end{itemize}








