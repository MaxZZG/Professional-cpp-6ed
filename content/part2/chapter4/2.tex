开始编程之前,跳过分析与设计步骤,或者只是草率地执行真的很诱人。看到代码编译并通过运行,给人一种正在取得进步的印象。当大致知道想要如何组织程序时,正式化设计或写下功能需求似乎是在浪费时间。编写设计文档远不如编码有趣,若整天想写论文,就不可能是计算机程序员!作为一名开发者,我理解这种立即开始编码的诱惑,并曾在某些时候也屈服于这种诱惑。然而,这很可能会导致一些问题。无论作为开发者的经验如何,对常用设计模式的精通程度,以及对C++、问题域和需求的深入理解,设计(思考)都是工作职责的一部分。没有设计好的话,就很难不走弯路。

在一个团队中,每个团队成员都负责项目的不同部分,则为所有团队成员提供一个设计文档至关重要。设计文档也有助于新成员了解项目的设计。若没有设计文档,新加入项目的人都不会知道设计应该是什么样子,并且可能会修改代码,破坏一些未记录的设计,这可能会在项目后期导致问题。

有些公司有专门的功能分析师来编写功能需求,也有专门负责软件设计的软件架构师。这些公司中,开发者通常可以专注于项目编程。在其他公司,开发者必须自己收集需求和设计。有些公司介于这两种极端之间;也许他们只有一位软件架构师做出架构决策,而开发者则自己设计更小的部分。

为了理解编程设计的重要性,想象拥有一块土地,想在那里建一座房子。当建筑商出现时,我们要求看蓝图(建筑图纸)。

“什么蓝图?”他回答。“我知道我要做什么,不需要提前计划每一个小细节。两层楼的房子?没问题。几个月前我做过一栋单层房子——我会从那个模型开始,然后继续。”假设暂且相信他,让他继续。几个月后,注意到管道似乎暴露在房子外面,而不在墙里面。当向建筑商询问这个异常时,他说:“哦。我忘记在墙上为管道留出空间了。我对这种新干墙技术太兴奋了,所以就把这件事忘了。但是在外面效果一样好,功能最重要。”你会对他的方法产生怀疑,但出于更好的判断,你还得让他继续。

当第一次参观完成的建筑时,注意到厨房没有水池。建筑商为自己辩解道:“我们意识到厨房没有空间放水池时,厨房已经完成了三分之二。我们不想从头开始,所以就在隔壁房间增加了一个单独的水池房间。这行得通,对吧?”如果把建筑商的辩解转移到软件领域,听起来是否熟悉?是否曾经发现自己像把管道放在房子外面那样解决了一个“丑陋”的问题?例如,忘记在队列数据结构中包含锁,多个线程共享这个数据结构。当意识到问题时,决定只在队列使用的地方手动上锁。你说,这很丑陋,但能工作。直到有新人加入项目,他们假设锁是数据结构内置的,没有对共享数据的互斥访问,导致了一个需要三周才能追踪到的竞态条件错误。一个专业的C++开发者绝不会决定在每次队列访问时手动上锁,而是直接在队列类内部集成锁,或者以无锁的方式使队列类线程安全。

编写代码之前正式设计,有助于确定如何相互配合。就像房子的蓝图展示了房间如何相互关联,并协同工作以满足房子的要求一样。程序的设计展示了程序的子系统,如何相互关联并协同工作以满足软件需求。没有设计计划,很可能会错过子系统之间的联系、重用或共享信息的可能性,以及完成任务的最简单方式。没有设计给出的“主体的蓝图”,可能会陷入实现细节中,以至于失去了对整体架构和目标的把握。此外,设计为项目所有成员提供了书面文档参考。若使用前面提到的敏捷Scrum方法论这样的迭代过程,需要确保在流程的每个周期中保持设计文档的更新,只要这样做还有价值。敏捷方法论的一个支柱是“工作软件优于全面文档”,至少要确保设计文档在关于项目如何协同工作的部分保持更新。而在我看来,团队是否维护项目小设计文档的价值取决于是否对未来有价值。若没有,那么要么删除这些文档,要么将其标记为过时。

若前面的类比还没有说服各位编写代码之前进行设计,这里有一个例子,直接进入编码无法导致最优设计。假设想写一个棋类游戏程序。而不是在开始编码之前设计整个程序,决定从最容易的部分开始,逐步向更难的部分推进。遵循第1章中介绍的面向对象的观点,并在第5章中详细介绍,决定用类来建模棋子。

兵是棋子中最简单的,所以决定从这里开始。在考虑了兵的特征和行为后,写了一个类,其属性和成员函数如图4.1所示的UML类图所示。

\myGraphic{0.3}{content/part2/chapter4/images/1.png}{图4.1}

这个设计中,m\_color属性表示兵是黑色还是白色,promote()成员函数在到达棋盘对面的地方执行。

实际上并没有制作这个类图,就直接进入了实现阶段。对这个类感到满意后,转向下一个最简单的棋子:象。了解了它的属性和功能后,写了一个类,其属性和成员函数如图4.2所示的类图中所示。

\myGraphic{0.3}{content/part2/chapter4/images/2.png}{图4.2}

因为没有生成一个类图,所以直接跳到了编码阶段。这个阶段,开始怀疑自己可能做错了什么。象和兵看起来很相似,它们的属性完全相同,并且它们共享许多成员函数。尽管兵和主教的move成员函数的实现可能不同,但两者都需要移动的能力。若在开始编码前设计好程序,就会意识到各种棋子实际上非常相似,应该找到一种方法,只编写一次通用功能。第5章解释了如何使用面向对象的设计技术来实现这一点。

此外,棋子的几个方面取决于程序的其他子系统。在棋子类本身中,无法准确地表示棋盘上的位置,除非知道如何建模棋盘。另一方面,可能可以以某种方式,以便棋盘以一种不需要棋子知道它们自己位置的方式来管理棋子。作为另一个例子,如何为棋子编写一个draw成员函数,除非首先确定程序的用户界面?将是图形界面,还是文本界面?棋盘会是什么样子?问题在于,程序的子系统并不孤立存在——它们与其他子系统相关联,确定和定义这些关系需要在设计时完成。




