\noindent
\textbf{内筒概要}

\begin{itemize}
\item
编程设计的定义

\item
编程设计的重要性

\item
C++独有的设计

\item
高效的C++设计有两个基本主题:抽象和重用

\item
可重用的不同类型的代码

\item
代码重用的优点和缺点

\item
选择要库的指导原则

\item
开源库

\item
C++标准库
\end{itemize}

写应用程序的单行代码之前,应该设计你的程序。比如,会使用哪些数据结构?会编写哪些类?这个计划在团队合作编程时尤为重要。想象一下,没有对与你合作同事的计划有任何了解就开始编写程序,是件多么可怕的事情!本章中,将学习如何使用专业的C++方法来进行C++设计。

尽管设计非常重要,但它可能是软件工程过程中最容易误解和最未充分利用的方面。很多情况下,开发者在没有明确计划的情况下开始编写应用程序:边编码边设计。这种方法可能导致设计变得复杂和过于混乱。这也使得开发、调试和维护任务变得更加困难。

虽然这看似违反直觉,但投入更多时间在项目开始时设计实际上是节省项目生命周期中时间的,即所谓“磨刀不误砍柴工”。












