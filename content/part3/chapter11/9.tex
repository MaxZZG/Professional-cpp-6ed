通过解决下面的练习,可以练习本章讨论的内容。所有练习的解决方案都可以在本书的网站\url{www.wiley.com/go/proc++6e}下载到源码。若在练习中卡住了,可以考虑先重读本章的部分内容,试着自己找到答案,再在从网站上寻找解决方案。

\begin{itemize}
\item
\textbf{练习11-1}:编写一个名为simulator的单文件模块,其中包含两个类,CarSimulator和BikeSimulator,位于Simulator命名空间中。这些类的内容对于练习不重要。只需提供一个默认构造函数,该构造函数向标准输出打印一条消息。在main()函数中测试代码。

\item
\textbf{练习11-2}:以练习11-1的解决方案为基础,将模块拆分为几个文件:一个主模块接口文件,没有任何实现,以及两个模块实现文件,一个用于CarSimulator类,另一个用于BikeSimulator类。

\item
\textbf{练习11-3}:以练习11-2的解决方案为基础,将其转换为使用一个主模块接口文件和两个模块接口分区文件,一个用于包含CarSimulator类的simulator:car分区,另一个用于包含BikeSimulator类的simulator:bike分区。

\item
\textbf{练习11-4}:以练习11-3的解决方案为基础,添加一个名为internals的实现分区,其中包含一个名为convertMilesToKm(double miles)的辅助函数,位于Simulator命名空间中。一英里等于1.6公里。为CarSimulator和BikeSimulator类添加一个名为setOdometer(double miles)的成员函数,该函数使用辅助函数将给定的英里转换为公里,然后将其打印到标准输出。在main()函数中确认setOdometer()在两个类上都能正常工作,还要确认main()不能调用convertMilesToKm()。

\item
\textbf{练习11-5}:编写一个源文件,其中包含一个预处理标识符,其值为0或1。使用预处理指令检查这个标识符的值。如果值为1,让编译器输出一条警告。如果为0,则忽略它。如果为其他值,则让编译器报个错。
\end{itemize}













