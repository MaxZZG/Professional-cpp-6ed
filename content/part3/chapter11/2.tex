

第 1 章介绍了 \#include 预处理指令,用于包含头文件的内容。还有一些其他的预处理指令,下表展示了一些最常用的预处理指令:

% Please add the following required packages to your document preamble:
% \usepackage{longtable}
% Note: It may be necessary to compile the document several times to get a multi-page table to line up properly
\begin{longtable}{|l|l|l|}
\hline
\textbf{预处理指令} &
\textbf{功能} &
\textbf{用途} \\ \hline
\endfirsthead
%
\endhead
%
\#include {[}file{]} &
\begin{tabular}[c]{@{}l@{}}将文件名为 {[}file{]} 的内容将插入\\到指令所在位置的代码中。
\end{tabular} &
\begin{tabular}[c]{@{}l@{}}用来包含头文件,以便\\可以在其他地方实现功能。
\end{tabular} \\ \hline
\#define {[}id{]} {[}value{]} &
\begin{tabular}[c]{@{}l@{}}标识符 {[}id{]} 都会替换为 {[}value{]}。
\end{tabular} &
\begin{tabular}[c]{@{}l@{}} C 语言中,经常用来定\\义一个常量值或宏。\\C++ 提供了更好的机制来\\处理常量和大多数类型的\\宏。宏可能很危险,要谨\\慎使用。
\end{tabular} \\ \hline
\#undef {[}id{]} &
\begin{tabular}[c]{@{}l@{}}取消先前使用 \#define 定义的标\\识符 {[}id{]} 的定义。
\end{tabular} &
\begin{tabular}[c]{@{}l@{}}定义的标识符只在有限\\范围内使用。
\end{tabular} \\ \hline
\begin{tabular}[c]{@{}l@{}}\#if {[}expression{]}\\ \#elif {[}expression{]}\\ \#else\\ \#endif\end{tabular} &
\begin{tabular}[c]{@{}l@{}}根据给定表达式的结果有条件\\地包含一段代码。
\end{tabular} &
\begin{tabular}[c]{@{}l@{}}经常用来为特定平台\\提供的特定代码。
\end{tabular} \\ \hline
\begin{tabular}[c]{@{}l@{}}\#ifdef {[}id{]}\\ \#endif\\ \\ \#ifndef {[}id{]}\\ \#endif\end{tabular} &
\begin{tabular}[c]{@{}l@{}}根据指定的标识符是否使用\\ \#define 定义,有条件地包含\\代码。\\\#ifdef {[}id{]} 等同于\\ \#if defined(id),\\而 \#ifndef {[}id{]} 等同于\\ \#if !defined(id)。
\end{tabular} &
\begin{tabular}[c]{@{}l@{}}常用来防止循环包含。\\每个头文件都以 \#ifndef \\开始,检查标识符的缺\\失,然后是一个 \#define \\指令来定义该标识符。\\头文件以一个 \#endif 结\\尾,防止文件多次包含。\end{tabular} \\ \hline
\begin{tabular}[c]{@{}l@{}}\#elifdef {[}id{]}\\ \#elifndef {[}id{]}\\ (C++23)\end{tabular} &
\begin{tabular}[c]{@{}l@{}}\#elifdef {[}id{]} 等同于\\\#elif defined(id),\\而 \#elifndef {[}id{]} \\等同于 \#elif !defined(id)。
\end{tabular} &
\begin{tabular}[c]{@{}l@{}}其他功能的简写。
\end{tabular} \\ \hline
\#pragma {[}xyz{]} &
\begin{tabular}[c]{@{}l@{}}控制编译器特定的行为。\\{[}xyz{]} 依赖于编译器。大多数\\编译器支持 once,以避免头\\文件多次包含。
\end{tabular} &
\begin{tabular}[c]{@{}l@{}}请参见本章后面的\\“头文件”部分的示例。
\end{tabular} \\ \hline
\#error {[}message{]} &
\begin{tabular}[c]{@{}l@{}}编译停止,并显示给定的\\消息。
\end{tabular} &
\begin{tabular}[c]{@{}l@{}}如果用户尝试在不受\\支持的平台编译代码,\\可以用来停止编译。
\end{tabular} \\ \hline
\begin{tabular}[c]{@{}l@{}}\#warning {[}message{]}\\ (C++23)\end{tabular} &
\begin{tabular}[c]{@{}l@{}}编译器将给定的消息作为\\警告发出,但编译继续。
\end{tabular} &
\begin{tabular}[c]{@{}l@{}}用于向用户显示警告,\\而不影响编译结果。
\end{tabular} \\ \hline
\end{longtable}

\mySubsubsection{11.2.1.}{预处理宏}

可以使用C++预处理器来编写宏:

\begin{cpp}
#define SQUARE(x) ((x) * (x)) // No semicolon after the macro definition!

int main()
{
    println("{}", SQUARE(5));
}
\end{cpp}

宏是C语言的遗留物,与内联函数非常相似,不同之处在于它们不进行类型检查,预处理器会愚蠢地将对它们的调用进行替换。预处理不应用真正的函数调用语义,这种行为可能导致意外结果。例如,用2+3而不是5调用SQUARE宏会发生什么:

\begin{cpp}
println("{}", SQUARE(2 + 3));
\end{cpp}

期望SQUARE计算25,确实是这样做的。如果从宏定义中遗漏了一些括号,使其看起来像这样呢?

\begin{cpp}
#define SQUARE(x) (x * x)
\end{cpp}

现在对SQUARE(2+3)的调用生成11,而不是25!记住,宏是直接展开的,不考虑函数调用语义。所以宏体中的x都会替换为2 + 3:

\begin{cpp}
println("{}", (2 + 3 * 2 + 3));
\end{cpp}

按照正确的运算顺序,这行代码首先执行乘法,然后是加法,生成11,而不是25!

宏也可能对性能产生影响。假设按以下方式调用SQUARE宏:

\begin{cpp}
println("{}", SQUARE(veryExpensiveFunctionCallToComputeNumber()));
\end{cpp}

预处理将其替换为以下内容:

\begin{cpp}
println("{}", ((veryExpensiveFunctionCallToComputeNumber()) *
        (veryExpensiveFunctionCallToComputeNumber())));
\end{cpp}

调用了函数两次——这是避免使用宏的另一个原因。

宏还会给调试带来问题,因为编写的代码不是编译器看到的代码,也不会出现在调试器中(因为预处理器有搜索和替换的行为)。出于这些原因,不应该使用宏,而改用内联函数。这里介绍这些,是因为仍然有很多C++代码使用宏,需要理解它们才能阅读和维护这样的代码。

\begin{myNotic}{NOTE}
大多数编译器可以将预处理的源代码输出到文件或标准输出。可以使用这个功能来查看预处理器如何预处理文件。例如,在Microsoft Visual C++中,可以使用/P。在GCC中,可以使用-E。
\end{myNotic}





