

Chapter 1 introduces the \#include preprocessor directive to include the contents of a header file. There are a few more preprocessor directives available. The following table shows some of the most commonly used preprocessor directives:

% Please add the following required packages to your document preamble:
% \usepackage{longtable}
% Note: It may be necessary to compile the document several times to get a multi-page table to line up properly
\begin{longtable}{|l|l|l|}
\hline
\textbf{PREPROCESSOR DIRECTIVE} &
\textbf{FUNCTIONALITY} &
\textbf{COMMON USES} \\ \hline
\endfirsthead
%
\endhead
%
\#include {[}file{]} &
\begin{tabular}[c]{@{}l@{}}The contents of the file with\\ name {[}file{]} is inserted\\ into the code at the location\\ of the directive.\end{tabular} &
\begin{tabular}[c]{@{}l@{}}Almost always used to include\\ header files so that code can\\ make use of functionality defined\\ elsewhere.\end{tabular} \\ \hline
\#define {[}id{]} {[}value{]} &
\begin{tabular}[c]{@{}l@{}}Every occurrence of the\\ identifier {[}id{]} is replaced\\ with {[}value{]}.\end{tabular} &
\begin{tabular}[c]{@{}l@{}}Often used in C to define a constant\\ value or a macro. C++ provides\\ better mechanisms for constants\\ and most types of macros. Macros\\ can be dangerous, so use them\\ cautiously. See the next section for\\ some examples.\end{tabular} \\ \hline
\#undef {[}id{]} &
\begin{tabular}[c]{@{}l@{}}Undefines the identifier\\ {[}id{]} previously defined\\ using \#define.\end{tabular} &
\begin{tabular}[c]{@{}l@{}}Used if a defined identifier is only\\ required within a limited scope\\ of the code.\end{tabular} \\ \hline
\begin{tabular}[c]{@{}l@{}}\#if {[}expression{]}\\ \#elif {[}expression{]}\\ \#else\\ \#endif\end{tabular} &
\begin{tabular}[c]{@{}l@{}}Conditionally include a\\ block of code based on the\\ result of a given expression.\end{tabular} &
\begin{tabular}[c]{@{}l@{}}Often used to provide specific code\\ for specific platforms.\end{tabular} \\ \hline
\begin{tabular}[c]{@{}l@{}}\#ifdef {[}id{]}\\ \#endif\\ \\ \#ifndef {[}id{]}\\ \#endif\end{tabular} &
\begin{tabular}[c]{@{}l@{}}Conditionally include code\\ based on whether the\\ specified identifier has been\\ defined with \#define.\\ \#ifdef {[}id{]} is equivalent\\ to \#if defined(id) and\\ \#ifndef {[}id{]} is equivalent\\ to \#if !defined(id).\end{tabular} &
\begin{tabular}[c]{@{}l@{}}Used most frequently to protect\\ against circular includes. Each\\ header file starts with an \#ifndef\\ checking the absence of an\\ identifier, followed by a \#define\\ directive to define that identifier.\\ The header file ends with an\\ \#endif. This prevents the file from\\ being included multiple times; see\\ the Header Files section later in\\ this chapter.\end{tabular} \\ \hline
\begin{tabular}[c]{@{}l@{}}\#elifdef {[}id{]}\\ \#elifndef {[}id{]}\\ (C++23)\end{tabular} &
\begin{tabular}[c]{@{}l@{}}\#elifdef {[}id{]} is\\ equivalent to\\ \#elif defined(id)\\ and \#elifndef {[}id{]} is\\ equivalent to\\ \#elif !defined(id).\end{tabular} &
\begin{tabular}[c]{@{}l@{}}Shorthand notations for other\\ functionality.\end{tabular} \\ \hline
\#pragma {[}xyz{]} &
\begin{tabular}[c]{@{}l@{}}Controls compiler-specific\\ behavior. {[}xyz{]} is compiler\\ dependent. Most compilers\\ support once to prevent\\ a header file from being\\ included multiple times.\end{tabular} &
\begin{tabular}[c]{@{}l@{}}See the Header Files section later in\\ this chapter for an example.\end{tabular} \\ \hline
\#error {[}message{]} &
\begin{tabular}[c]{@{}l@{}}Causes the compilation\\ to stop with the\\ given message.\end{tabular} &
\begin{tabular}[c]{@{}l@{}}Can be used to stop the compilation\\ if the user tries to compile code on\\ an unsupported platform.\end{tabular} \\ \hline
\begin{tabular}[c]{@{}l@{}}\#warning {[}message{]}\\ (C++23)\end{tabular} &
\begin{tabular}[c]{@{}l@{}}Causes the compiler to\\ emit the given message as\\ a warning, but compilation\\ continues.\end{tabular} &
\begin{tabular}[c]{@{}l@{}}Used to display a warning to\\ the user without affecting the\\ compilation result.\end{tabular} \\ \hline
\end{longtable}

\mySubsubsection{11.2.1.}{Preprocessor Macros}

You can use the C++ preprocessor to write macros, which are like little functions. Here is an example:

\begin{cpp}
#define SQUARE(x) ((x) * (x)) // No semicolon after the macro definition!

int main()
{
    println("{}", SQUARE(5));
}
\end{cpp}

Macros are a remnant from C that are quite similar to inline functions, except that they are not type-checked, and the preprocessor dumbly replaces any calls to them with their expansions. The preprocessor does not apply true function-call semantics. This behavior can cause unexpected results. For example, consider what would happen if you called the SQUARE macro with 2+3 instead of 5, like this:

\begin{cpp}
println("{}", SQUARE(2 + 3));
\end{cpp}

You expect SQUARE to calculate 25, which it does. However, what if you left out some parentheses from the macro definition, so that it looks like this?

\begin{cpp}
#define SQUARE(x) (x * x)
\end{cpp}

Now the call to SQUARE(2+3) generates 11, not 25! Remember that the macro is dumbly expanded without regard to function-call semantics. This means that any x in the macro body is replaced by 2 + 3, leading to this expansion:

\begin{cpp}
println("{}", (2 + 3 * 2 + 3));
\end{cpp}

Following proper order of operations, this line performs the multiplication first, followed by the additions, generating 11 instead of 25!

Macros can also have a performance impact. Suppose you call the SQUARE macro as follows:

\begin{cpp}
println("{}", SQUARE(veryExpensiveFunctionCallToComputeNumber()));
\end{cpp}

The preprocessor replaces this with the following:

\begin{cpp}
println("{}", ((veryExpensiveFunctionCallToComputeNumber()) *
        (veryExpensiveFunctionCallToComputeNumber())));
\end{cpp}

Now you are calling the expensive function twice—another reason to avoid macros.

Macros also cause problems for debugging because the code you write is not the code that the compiler sees or that shows up in your debugger (because of the search-and-replace behavior of the preprocessor). For these reasons, you should avoid macros entirely in favor of inline functions. The details are shown here only because quite a bit of C++ code out there still employs macros. You need to understand them to read and maintain such code.

\begin{myNotic}{NOTE}
Most compilers can output the preprocessed source to a file or to standard output. You can use that feature to see how the preprocessor is preprocessing your file. For example, with Microsoft Visual C++ you can use the /P switch. With GCC you can use the -E switch.
\end{myNotic}





