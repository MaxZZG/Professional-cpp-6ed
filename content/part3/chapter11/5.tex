
You can use feature-test macros to detect which core language features are supported by a compiler. All these macros start with either \_\_cpp\_ or \_\_has\_cpp\_. The following are some examples. Consult your favorite C++ reference for a complete list of all possible core language feature-test macros.

\begin{itemize}
\item
\_\_cpp\_range\_based\_for

\item
\_\_cpp\_binary\_literals

\item
\_\_cpp\_char8\_t

\item
\_\_cpp\_generic\_lambdas

\item
\_\_cpp\_consteval

\item
\_\_cpp\_coroutines

\item
. . .

\item
\_\_has\_cpp\_attribute([attribute\_name])

\item
. . .
\end{itemize}

The value of these macros is a number representing the month and year when a specific feature was added or updated. The date is formatted as YYYYMM. For example, the value of \_\_cpp\_binary\_literals is 201304, i.e., April 2013, which is the date when binary literals were introduced. As another example, the value of \_\_has\_cpp\_attribute(nodiscard) can be 201603, i.e., March 2016, which is the date when the [[nodiscard]] attribute was first introduced. Or it can be 201907, i.e., July 2019, which is the date when the attribute was updated to allow specifying a reason such as [[nodiscard("Reason")]].

All these core language feature-test macros are available without having to include any specific header. Here is an example use:

\begin{cpp}
int main()
{
#ifdef __cpp_range_based_for
    println("Range-based for loops are supported!");
#else
    println("Bummer! Range-based for loops are NOT supported!");
#endif
}
\end{cpp}

Chapter 16, “Overview of the C++ Standard Library,” explains that there are similar feature-test macros for Standard Library features.

\begin{myNotic}{NOTE}
You will rarely need these feature-test macros, unless you are writing crossplatform and cross-compiler code. In that case, you might want to know if certain functionality is supported by a given compiler so that you can provide fallback code in case a feature is missing. Chapter 34, “Developing Cross-Platform and Cross-Language Applications,” discusses cross-platform development.
\end{myNotic}



