
本节描述了C++中的链接概念。如第1章所述,C++源文件首先由预处理处理,预处理处理所有预处理器指令,从而生成翻译单元。所有翻译单元独立编译成目标文件,这些文件包含机器可执行代码,但其中对函数等的引用尚未定义。解析这些引用是由链接器完成,链接器将所有目标文件链接成可执行文件。从技术上讲,编译过程还有更多的阶段,但对于本次讨论,这个简化的流程已经足够了。

C++翻译单元中的每个名称(包括函数和全局变量)要么有链接,要么没有链接,这指定了该名称可以在哪里定义以及可以从哪里访问。这里有四种类型的链接:

\begin{itemize}
\item
无链接:该名称仅能从其定义的作用域内访问。

\item
外部链接:该名称可以从任何翻译单元访问。

\item
内部链接(也称为静态链接):该名称仅能在当前翻译单元访问,但不能在其他翻译单元访问。

\item
模块链接:该名称可以从同一模块的翻译单元访问。
\end{itemize}

\mySubsubsection{11.3.1.}{内部链接}

默认情况下,函数和全局变量具有外部链接。也可以通过使用匿名命名空间来指定内部(或静态)链接。例如,有两个源文件:FirstFile.cpp 和 AnotherFile.cpp。FirstFile.cpp的内容如下所示:

\begin{cpp}
void f();

int main()
{
    f();
}
\end{cpp}

这个文件为 f() 提供了一个声明,但没有显示其定义。

AnotherFile.cpp的内容如下所示:

\begin{cpp}
import std;

void f();

void f()
{
    std::println("f");
}
\end{cpp}

这个文件为 f() 提供了原型和定义,在两个不同的文件中为同一个函数编写原没有问题。这正是将声明放在头文件中,并在每个源文件中 \#include 头文件的原因。这个例子中,我没有使用头文件。使用头文件的原因曾经是更容易维护(并保持同步)一个原型的副本,但现在 C++ 支持模块,更建议使用模块。

每个源文件都可以编译,并且程序可以正常链接:因为 f() 具有外部链接,所以 main() 可以从不同的文件调用它。

然而,假设在 AnotherFile.cpp 中将 f() 函数包装在匿名命名空间中,以给它内部链接:

\begin{cpp}
import std;

namespace
{
    void f();

    void f()
    {
        std::println("f");
    }
}
\end{cpp}

匿名命名空间中的实体具有内部链接,可以在同一翻译单元中的任何进行地方访问,但不能从其他翻译单元访问。因此,每个源文件虽任可以编译,但链接时会失败,因为 f() 具有内部链接,无法从 FirstFile.cpp 进行访问。

使用匿名命名空间给一个名称内部链接的另一种方法是,在声明前加上关键字 static。前面的匿名命名空间示例可以写成如下。注意,不需要在 f() 的定义前重复 static 关键字。只要它位于函数名称的第一个实例之前,就没有重复的必要。

\begin{cpp}
import std;

static void f();

void f()
{
    std::println("f");
}
\end{cpp}

这个版本代码的语义与使用匿名命名空间的版本完全相同。

\begin{myWarning}{WARNING}
如果一个翻译单元需要一个仅在翻译单元内部需要的辅助函数,请将其包装在匿名命名空间中以给它内部链接。不鼓励使用 static 关键字进行此操作。
\end{myWarning}

\mySubsubsection{11.3.2.}{extern 关键字}

关键字 extern 似乎是 static 的对立面,它指定的名称具有外部链接,并且在某些情况下可以这样使用。例如,const 和 typedef 默认具有内部链接,可以使用 extern 给它们外部链接。然而,extern 有一点复杂。将一个名称指定为 extern 时,编译器将其视为声明,而不是定义。对于变量,编译器不会为其分配空间。必须为变量提供一个不带 extern 关键字的独立定义。例如,这是 AnotherFile.cpp 的内容:

\begin{cpp}
extern int x;
int x { 3 };
\end{cpp}

或者,可以在 extern 语句中初始化 x,作为声明和定义:

\begin{cpp}
extern int x { 3 };
\end{cpp}

这时,extern 并不是很有用,因为 x 默认就有外部链接。extern 的真正用途是让另一个源文件可以使用 x ,例如 FirstFile.cpp:

\begin{cpp}
import std;

extern int x;

int main()
{
    std::println("{}", x);
}
\end{cpp}

在这里,FirstFile.cpp 使用 extern 声明以便使用 x。编译器需要 x 的声明才能在 main() 中使用。如果没有使用 extern 关键字声明 x,编译器会认为它是一个定义并为 x 分配空间,导致链接步骤失败(因为全局作用域中会有两个 x 变量)。使用 extern,可以使变量在多个源文件内可访问。

\begin{myWarning}{WARNING}
不推荐使用全局变量。它们令人困惑且容易出错,特别是在大型程序中。请谨慎使用!

唯一的例外是全局常量。不要到处定义相同的常量:定义一次,处处使用。
\end{myWarning}



