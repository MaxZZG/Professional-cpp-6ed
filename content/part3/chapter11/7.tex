

历史遗留代码中,可能会遇到 C 风格的可变长度参数列表的使用。新代码中,应该避免使用这些,而是使用变量模板来实现类型安全的可变长度参数列表,这将在第 26 章中介绍。

为了让您了解 C 风格的可变长度参数列表,看一下来自 <cstdio> 的 C 函数 printf()。可以用任意数量的参数调用它:

\begin{cpp}
printf("int %d\n", 5);
printf("String %s and int %d\n", "hello", 5);
printf("Many ints: %d, %d, %d, %d, %d\n", 1, 2, 3, 4, 5);
\end{cpp}

C/C++ 提供了语法和一些实用宏,可用于编写具有可变数量参数的函数,这些函数看起来很像 printf()。例如,编写一个简单的调试函数,如果设置了调试标志,将字符串打印到 stderr,如果没有设置调试标志,则不做任何事情。就像 printf() 一样,这个函数应该能够打印带有任意数量和任意类型参数的字符串:

\begin{cpp}
import std;
#include <cstdarg>
#include <cstdio>

bool debug { false };

void debugOut(const char* str, ...)
{
    if (debug) {
        va_list ap;
        va_start(ap, str);
        vfprintf(stderr, str, ap);
        va_end(ap);
    }
}
\end{cpp}

这段代码使用了 va\_list()、va\_start() 和 va\_end(),是在 <cstdarg> 中定义的宏,因此需要 \#include <cstdarg>( import std; 不导出任何宏)。同样,stderr 是在 <cstdio> 中定义的宏,需要\#include <cstdio>。

debugOut() 的原型包含一个类型化和命名的参数 str,后面跟着 ...(省略号),省略号代表任意数量和类型的参数。要访问这些参数,必须声明一个类型为 va\_list 的变量,并用对 va\_start 的调用来初始化,va\_start() 的第二个参数必须是参数列表中最右边的命名变量。所有具有可变长度参数列表的函数至少需要一个命名参数。debugOut() 函数只是将这个列表传递给 vfprintf()(<cstdio> 中的一个标准函数)。 vfprintf() 调用返回后,debugOut() 调用 va\_end() 来终止对可变参数列表的访问。调用 va\_start() 之后,必须调用 va\_end(),以确保函数结束时堆栈处于一致状态。

可以用以下方式使用这个函数:

\begin{cpp}
debug = true;
debugOut("int %d\n", 5);

debugOut("String %s and int %d\n", "hello", 5);
debugOut("Many ints: %d, %d, %d, %d, %d\n", 1, 2, 3, 4, 5);
\end{cpp}

\mySubsubsection{11.7.1.}{访问参数}

如果想要自己访问实际的参数,可以使用va\_arg()来实现。它接受一个va\_list作为第一个参数,以及要解释的实参类型,但除非提供一种显式的方法,否则无法知道实参列表的末尾。例如,可以使第一个参数成为实参数量的计数。或者,在有一组指针的情况下,可能要求最后一个指针为nullptr。有很多方法,但对开发者来说都很麻烦。

以下示例演示了调用者如何在第一个命名参数中指定提供了多少个参数。该函数接受任意数量的int,并打印它们。

\begin{cpp}
void printInts(unsigned num, ...)
{
    va_list ap;
    va_start(ap, num);
    for (unsigned i { 0 }; i < num; ++i) {
        int temp { va_arg(ap, int) };
        print("{} ", temp);
    }
    va_end(ap);
    println("");
}
\end{cpp}

可以按以下方式调用printInts()。注意,第一个参数指定了将跟随多少个整数。

\begin{cpp}
printInts(5, 5, 4, 3, 2, 1);
\end{cpp}

\mySubsubsection{11.7.2.}{为什么不使用C风格的可变长度实参列表}

问C风格的可变长度实参列表并不十分安全。从printInts()函数中看,存在几个风险。

\begin{itemize}
\item
不知道参数的数量。使用printInts()时,必须相信调用者传递正确数量的参数作为第一个参数。使用debugOut()时,必须相信调用者传递的参数数量与字符串中的替换字段数量相同。

\item
不知道参数的类型。va\_arg()接受一个类型,可用来解释当前位置的值,也可以告诉va\_arg()将值解释为任何类型,其无法验证类型的正确性。
\end{itemize}

\begin{myWarning}{WARNING}
避免使用C风格的可变长度实参列表。最好传递一个std::array或vector的值,使用第1章中描述的初始化列表,或使用第26章中描述的变长模板,来实现类型安全的可变长度实参列表。
\end{myWarning}













