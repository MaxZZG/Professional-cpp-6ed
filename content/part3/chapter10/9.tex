通过解决下面的练习,可以练习本章讨论的内容。所有练习的解决方案都可以在本书的网站\url{www.wiley.com/go/proc++6e}下载到源码。然而,若在练习中卡住了,在从网站上寻找解决方案之前,可以考虑先重读本章的部分内容,试着自己找到答案。

\begin{itemize}
\item
\textbf{练习 10-1}: 继续使用第 9 章练习 2 中的Person类,添加一个名为 Employee 的派生类。可以省略第 9 章练习 2 中的 operator<=> 重载。Employee 类添加了一个数据成员,即员工 ID。提供一个适当的构造函数。从 Employee 派生出两个更多的类,分别为 Manager 和 Director。

将所有的类,包括 Person 类,放在一个名为 HR 的命名空间中,可以像以下这样从模块导出命名空间中的所有内容:

\begin{cpp}
export namespace HR { /* ... */ }
\end{cpp}

\item
\textbf{练习 10-2}: 继续使用练习 10-1 中的解决方案,为 Person 类添加一个 toString() 成员函数,返回一个字符串表示。在 Employee、Manager 和 Director 类中重写这个成员函数,通过委托部分工作给父类来构建完整的字符串表示。

\item
\textbf{练习 10-3}: 为练习 10-2 中的Person类的层次结构中添加多态行为。定义一个vector,用于混合存储员工、经理和董事,并用一些测试数据进行填充。最后,使用单个基于范围的 for 循环对vector中的所有元素使用 toString()。

\item
\textbf{练习 10-4}: 实际公司中,员工可以晋升为经理或董事职位,经理也可以晋升为董事。如何在类层次结构的实现中,添加晋升方式?
\end{itemize}

