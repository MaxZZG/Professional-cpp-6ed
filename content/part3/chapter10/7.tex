
在 C++ 中,基本类型在第 1 章中进行了回顾,而第 8 章至第 10 章展示了如何使用类编写自己的类型。本节探讨了将一种类型转换为另一种类型的复杂性。

C++ 提供五种特定的类型转换:const\_cast(), static\_cast(), reinterpret\_cast(), dynamic\_cast(), 和 std::bit\_cast。第一种在第1章中介绍过,并且 static\_cast() 可用于在某些原始类型之间进行转换,但在继承的背景下这会更复杂。既然已经熟练地编写自己的类,并理解了类继承,现在是时候来了解一下这些转换了。

请注意,旧式的 C 风格类型转换,如 (int)myFloat,在 C++ 中仍然有效,并且在新代码库中仍然广泛使用。C 风格类型转换涵盖了除 bit\_cast() 之外的所有 C++ 类型转换,因此更易出错,可能会得到意外的结果。我强烈建议只在新的代码中使用 C++ 风格的类型转换,这样会更安全。

\mySubsubsection{10.7.1.}{static\_cast()}

可以使用static\_cast()执行显式转换。例如,编写了一个算术表达式,需要将int转换为double以避免整数除法,请使用static\_cast()。这个例子中,因为这会使两个操作数中的一个成为double,所以只需要对i使用static\_cast(),确保C++执行浮点除法。

\begin{cpp}
int i { 3 };
int j { 4 };
double result { static_cast<double>(i) / j };
\end{cpp}

还可以使用static\_cast()执行由用户定义的构造函数,或转换例程允许的显式转换。例如,类A有一个接受类B对象的构造函数,可以使用static\_cast()将B对象转换为A对象。在大多数情况下,编译器会自动执行转换。

static\_cast()的另一个用途是在继承层次结构中执行向下转换:

\begin{cpp}
class Base
{
    public:
        virtual ~Base() = default;
};

class Derived : public Base
{
    public:
        virtual ~Derived() = default;
};

int main()
{
    Base* b { nullptr };
    Derived* d { new Derived {} };
    b = d; // Don't need a cast to go up the inheritance hierarchy.
    d = static_cast<Derived*>(b); // Need a cast to go down the hierarchy.

    Base base;
    Derived derived;
    Base& br { derived };
    Derived& dr { static_cast<Derived&>(br) };
}
\end{cpp}

这些强制转换适用于指针和引用,但不适用于对象本身。

请注意,使用static\_cast()进行的强制转换不执行运行时类型检查。允许将Base指针转换为Derived指针,或将Base引用转换为Derived引用,即使运行时Base实际上不是Derived。例如,以下代码编译并执行,但使用指针d可能会导致失败,包括访问超出对象边界的内存。

\begin{cpp}
Base* b { new Base {} };
Derived* d { static_cast<Derived*>(b) };
\end{cpp}

为了安全地执行这种带有运行时类型检查的强制转换,请使用dynamic\_cast()。

static\_cast()并不是万能的,不能将一种类型的指针static\_cast()到另一种不相关的类型指针。如果没有可用的转换构造函数,不能直接将一个类型的对象static\_cast()到另一个类型的对象。不能将const类型static\_cast()到非const类型,不能将指针static\_cast()到整数。基本上,只能做符合C++类型规则的事情。

\mySubsubsection{10.7.2.}{reinterpret\_cast()}

reinterpret\_cast()比static\_cast()更加强大,也更不安全。可以使用它执行一些技术上不被C++类型规则允许,但可能在某些情况下对开发者有意义的转换。例如,可以使用reinterpret\_cast()将一个类型的引用转换为另一个不相关类型的引用。同样,可以将一个指针类型转换为其他指针类型,即使它们在继承层次结构中并不相关。然而,将指针转换为void可以隐式进行,不需要显式转换。要将void转换回正确类型的指针,static\_cast()就足够了。void指针只是指向内存中某个位置的一个指针,没有与void指针关联的类型信息。这里有一些例子:

\begin{cpp}
class X {};
class Y {};

int main()
{
    X x;
    Y y;
    X* xp { &x };
    Y* yp { &y };
    // Need reinterpret_cast for pointer conversion from unrelated classes
    // static_cast doesn't work.
    xp = reinterpret_cast<X*>(yp);
    // No cast required for conversion from pointer to void*
    void* p { xp };
    // static_cast is enough for pointer conversion from void*
    xp = static_cast<X*>(p);
    // Need reinterpret_cast for reference conversion from unrelated classes
    // static_cast doesn't work.
    X& xr { x };
    Y& yr { reinterpret_cast<Y&>(x) };
}
\end{cpp}

reinterpret\_cast()并不是万能的,对可以强制转换的内容有很多限制。建议谨慎使用这些类型的强制转换,这些限制在本文中就不进一步讨论了。

通常,应该小心使用reinterpret\_cast(),因为它允许在没有进行类型检查的情况下进行转换。

\begin{myWarning}{WARNING}
还可以使用reinterpret\_cast()将指针转换为整型类型,然后转换回来。但是,只能将指针转换为足够大的整型类型来容纳它。例如,尝试使用reinterpret\_cast()将64位指针转换为32位整数将导致编译错误。
\end{myWarning}

\mySubsubsection{10.7.3.}{dynamic\_cast()}

dynamic\_cast()在继承层次结构中提供了运行时检查,可以使用它来转换指针或引用。动态强制转换在运行时检查基础对象的运行时类型信息。如果转换没有意义,动态强制转换将返回一个空指针(对于指针版本)或抛出一个std::bad\_cast异常(对于引用版本)。

例如,有以下类层次结构:

\begin{cpp}
class Base
{
    public:
        virtual ˜Base() = default;
};

class Derived : public Base
{
    public:
        virtual ˜Derived() = default;
};
\end{cpp}

以下示例显示了正确使用dynamic\_cast():

\begin{cpp}
Base* b;
Derived* d { new Derived {} };
b = d;
d = dynamic_cast<Derived*>(b);
\end{cpp}

以下对引用的 dynamic\_cast()将引发异常:

\begin{cpp}
Base base;
Derived derived;
Base& br { base };
try {
    Derived& dr { dynamic_cast<Derived&>(br) };
} catch (const bad_cast&) {
    println("Bad cast!");
}
\end{cpp}

请注意,可以使用static\_cast()或reinterpret\_cast()在继承层次结构中执行相同的向下转换。dynamic\_cast()与static\_cast()和reinterpret\_cast()的区别在于,执行运行时(动态)类型检查,而static\_cast()和reinterpret\_cast()即使在类型错误的情况下也会执行转换。

记住,运行时类型信息存储在对象的虚表中。因此,为了使用dynamic\_cast(),类必须至少有一个虚成员函数。如果类没有虚表,尝试使用dynamic\_cast()将导致编译错误。例如,Microsoft Visual C++会给报以下错误:

\begin{shell}
error C2683: 'dynamic_cast' : 'Base' is not a polymorphic type.
\end{shell}

\mySubsubsection{10.7.4.}{std::bit\_cast()}

std::bit\_cast() 定义在<bit>中,是唯一属于标准库的类型转换;其他类型转换是 C++ 语言的一部分。bit\_cast() 类似于 reinterpret\_cast(),但它创建一个给定目标类型的新的对象,并将源对象的位复制到这个新对象中,有效地将源对象的位解释为目标对象的位序列。bit\_cast() 要求源对象和目标对象的尺寸相同,并且两者都是简单可复制的。

\begin{myNotic}{NOTE}
一个简单可复制的类型是一个类型的底层字节可以复制到例如 char 数组中。如果该数组的数据再复制回对象,对象将保持其原始值。
\end{myNotic}

举个栗子:

\begin{cpp}
float asFloat { 1.23f };
auto asUint { bit_cast<unsigned int>(asFloat) };
if (bit_cast<float>(asUint) == asFloat) { println("Roundtrip success."); }
\end{cpp}

bit\_cast() 的一个用例是用于二进制 I/O 的简单可复制类型。例如,可以将这类类型的单个字节写入文件。将文件重新读取到内存时,可以使用 bit\_cast() 来正确解释从文件读取的字节。

\mySubsubsection{10.7.5.}{总结}

下表总结了在不同场景下,应该如何使用的强制转换。

% Please add the following required packages to your document preamble:
% \usepackage{longtable}
% Note: It may be necessary to compile the document several times to get a multi-page table to line up properly
\begin{longtable}{|l|l|}
\hline
\textbf{场景}                                                                                                          & \textbf{转换}       \\ \hline
\endfirsthead
%
\endhead
%
移除const                                                                                                           & const\_cast()       \\ \hline
\begin{tabular}[c]{@{}l@{}}显式强制转换(例如,从int到double,从int到bool)\end{tabular} & static\_cast()      \\ \hline
\begin{tabular}[c]{@{}l@{}}由用户定义的构造函数或转换支持的强制转换\end{tabular}               & static\_cast()      \\ \hline
\begin{tabular}[c]{@{}l@{}}将类对象转换为另一个(不相关的)类的对象\end{tabular}                        & bit\_cast()         \\ \hline
\begin{tabular}[c]{@{}l@{}}将指向类对象的指针,指向同一继承层次结构中另一个类对象的指针\end{tabular} &
\begin{tabular}[c]{@{}l@{}}推荐dynamic\_cast()\\或 static\_cast()\end{tabular} \\ \hline
\begin{tabular}[c]{@{}l@{}}同一继承层次结构中,将类对象的引用转换为对另一个类对对象的引用\end{tabular} &
\begin{tabular}[c]{@{}l@{}}推荐dynamic\_cast()\\或 static\_cast()\end{tabular} \\ \hline
将指针到类型转为不相关的指针到类型                                                                                & reinterpret\_cast() \\ \hline
将引用到类型转为不相关的引用到类型                                                                            & reinterpret\_cast() \\ \hline
将函数指针转为函数指针                                                                                  & reinterpret\_cast() \\ \hline
\end{longtable}





















