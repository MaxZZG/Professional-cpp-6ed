
第5章中,了解到is-a关系描述了现实世界事物间的层次结构模式。在编程中,需要编写一个建立在另一个类之上或稍微改变另一个类的类时,可以使用这种模式。实现这一目标的一种方式是,从一个类复制代码并将其粘贴到另一个类中。通过更改相关部分或修改代码,可以实现创建一个与原始类略有不同的新类。然而,这种方法会让面向对象开发者感到沮丧和非常烦恼,原因如下:

\begin{itemize}
\item
两个类包含完全独立的代码,原始类的错误修复不会反映在新类中。

\item
编译器不知道两个类之间的关系,所以它们不是多态的(参见第5章)——不是同一类的变体。

\item
这种方法没有建立真正的is-a关系。新类与原始类相似,只有相似的代码,而不是同一类型的对象。

\item
原始代码可能无法获得,可能只存在于预编译的二进制格式中,所以复制和粘贴代码不可行。
\end{itemize}

毫不奇怪,C++提供了内置支持来定义真正的is-a关系。C++中is-a关系的特征将在下一节中介绍。

\mySubsubsection{10.1.1.}{扩展类}

在C++中编写类定义时,可以告诉编译器类正在继承、派生或扩展一个已存在的类。这样,类会自动包含原始类的数据成员和成员函数,这个原始类称为父类、基类或超类。扩展一个已存在的类让其(称为子类、派生类或子类)有能力仅描述它与父类的不同之处。 要在C++中扩展一个类,需要在编写类定义时指定要扩展的类。

为了展示继承语法,我们使用了两个类,Base和Derived。别担心——后面会有更有趣的例子。首先,看看Base类的定义:

\begin{cpp}
class Base
{
    public:
        void someFunction() {}
    protected:
        int m_protectedInt { 0 };
    private:
        int m_privateInt { 0 };
};
\end{cpp}

如果想要构建一个新类,称为Derived,继承自Base,可以使用以下语法:

\begin{cpp}
class Derived : public Base
{
    public:
        void someOtherFunction() {}
};
\end{cpp}

Derived是一个完整的类,共享了Base类的特性。先别担心public——其含义会在后面进行解释。图10.1展示了Derived和Base之间的简单关系。可以像声明其他对象一样声明Derived类型的对象,可以定义一个从Derived继承的第三个类,形成一个类链,如图10.2所示。

\myGraphic{0.2}{content/part3/chapter10/images/1.png}{图 10.1}

\myGraphic{0.25}{content/part3/chapter10/images/2.png}{图 10.2}

Derived不必是Base的唯一派生类。其他类也可以从Base继承,并成为Derived的兄弟类,如图10.3所示。

\myGraphic{0.35}{content/part3/chapter10/images/3.png}{图 10.3}

内部,派生类包含一个基类的实例作为子对象,关系可以表示为图10.4。

\myGraphic{0.3}{content/part3/chapter10/images/4.png}{图 10.4}

\mySamllsection{客户端对继承的看法}

对于客户端或代码的其他部分来说,因为Derived继承自Base,Derived类型的对象也是一个Base类型的对象。所以Base的所有public成员函数和数据成员,以及Derived的所有public成员函数和数据成员都可用。

使用派生类的代码不需要知道,在继承链中哪个类定义了成员函数就可以调用。以下代码调用了Derived对象的两个成员函数,即使其中一个成员函数由Base类定义:

\begin{cpp}
Derived myDerived;
myDerived.someFunction();
myDerived.someOtherFunction();
\end{cpp}

要理解继承只在一个方向上起作用。Derived类与Base类有一个明确的关系,但Base类,对Derived类一无所知,所以Base类型的对象不能访问Derived的成员函数和数据成员,因为Base不是Derived。

以下代码不能编译,因为Base类不包含一个名为someOtherFunction()的public成员函数:

\begin{cpp}
Base myBase;
myBase.someOtherFunction(); // Error! Base doesn't have a someOtherFunction().
\end{cpp}

\begin{myNotic}{NOTE}
从使用对象的角度来看,对象属于其定义的类,以及基类。
\end{myNotic}

指向类类型的指针或引用,可以引用声明类类型的对象或其任何派生类的对象,这个棘手的问题会在本章后面详细介绍。现在要理解的是,指向Base的指针实际上可以指向Derived对象,引用也是如此。客户端仍然只能访问Base中存在的成员函数和数据成员,但通过这种机制,操作Base的代码也可以操作Derived。

以下代码编译并运行良好,即使看起来类型并不匹配:

\begin{cpp}
Base* base { new Derived {} }; // Create Derived, store in Base pointer.
\end{cpp}

然而,不能通过Base指针调用Derived类的成员函数。以下代码无法工作:

\begin{cpp}
base->someOtherFunction();
\end{cpp}

编译器将其标记为错误,因为尽管对象是Derived类型,确实有someOtherFunction()的定义,但编译器只能将其视为Base类型,而Base类型没有定义someOtherFunction()。

\mySamllsection{派生类对继承的看法}

对于派生类来说,在编写方式或行为方面并没有太大变化。可以在派生类中定义成员函数和数据成员,就像在常规类中一样。前面Derived的定义声明了一个名为someOtherFunction()的成员函数,Derived类可以通过添加成员函数来增强Base类。

派生类可以访问其基类中声明的public和protected的成员函数和数据成员,就像是自己的成员一样,从技术上讲确实是。Derived上someOtherFunction()的实现可以利用在Base中,声明为一部分的数据成员m\_protectedInt。访问基类成员与在派生类中声明成员没有什么不同:

\begin{cpp}
void Derived::someOtherFunction()
{
    println("I can access base class data member m_protectedInt.");
    println("Its value is {}", m_protectedInt);
}
\end{cpp}

如果类将成员声明为protected,派生类可以访问。如果声明为private,派生类则无法访问。因为派生类试图访问基类中的private数据成员,以下someOtherFunction()的实现无法编译:

\begin{cpp}
void Derived::someOtherFunction()
{
    println("I can access base class data member m_protectedInt.");
    println("Its value is {}", m_protectedInt);
    println("The value of m_privateInt is {}", m_privateInt); // Error!
}
\end{cpp}

private访问说明符可以控制派生类如何与基类交互。

第4章给出了以下规则:所有数据成员都应该是private;如果希望从类外部访问数据成员,请提供public getter和setter。现在这个规则可以扩展到protected访问说明符。

\begin{myWarning}{WARNING}
所有数据成员都应该是private。如果希望从类外部访问数据成员,请提供public getter和setter;如果只想让派生类访问,请提供protected的getter和setter。
\end{myWarning}

默认将数据成员设为private的原因是,提供了最高级别的封装。可以更改表示数据的方式,同时保持public和protected接口不变。在不直接访问数据成员的情况下,可以在public和protected的setter中轻松添加对输入数据的检查。成员函数也应该默认为private,只公开那些设计为公共的成员函数。如果只想让派生类访问,请将成员函数设为protected。

\begin{myNotic}{NOTE}
从派生类的角度来看,基类中的所有public和protected的数据成员和成员函数都可用。
\end{myNotic}

下表总结了所有三种访问说明符的含义:

% Please add the following required packages to your document preamble:
% \usepackage{longtable}
% Note: It may be necessary to compile the document several times to get a multi-page table to line up properly
\begin{longtable}{|l|l|l|}
\hline
\textbf{\begin{tabular}[c]{@{}l@{}}访问说明符\end{tabular}} &
\textbf{含义} &
\textbf{何时使用} \\ \hline
\endfirsthead
%
\endhead
%
public &
\begin{tabular}[c]{@{}l@{}}可以调用一个对象的public成员函数或访\\问其public数据成员。
\end{tabular} &
\begin{tabular}[c]{@{}l@{}}希望客户端进行的操作(成员函数)。\\可访问对private和protected数据成\\员进行操作的成员函数(getter和\\setter)。
\end{tabular} \\ \hline
protected &
\begin{tabular}[c]{@{}l@{}}类中的任何成员函数都可以调用protected\\的成员函数和protected的数据成员。派生\\类的成员函数可以访问基类的protected成\\员。
\end{tabular} &
\begin{tabular}[c]{@{}l@{}}不想让客户端使用的成员函数。
\end{tabular} \\ \hline
private &
\begin{tabular}[c]{@{}l@{}}只有类中的成员函数可以调用private成员\\函数和private数据成员。派生类中的成员\\函数不能访问基类的private成员。
\end{tabular} &
\begin{tabular}[c]{@{}l@{}}默认情况下,一切都应该是private,\\尤其是数据成员。如果只允许派生类\\访问,可以提供protected的getter和\\setter;如果希望客户端访问,可以提\\供public的getter和setter。
\end{tabular} \\ \hline
\end{longtable}

\mySamllsection{防止继承}

C++允许将一个类标记为final,当尝试继承时将出现编译错误。一个类可以通过在类名后面直接使用final关键字来标记“不可继承”。一个类尝试从下面的Foo类继承,编译器将报错:

\begin{cpp}
class Foo final { };
\end{cpp}

\mySubsubsection{10.1.2.}{重载成员函数}

继承的主要原因是添加或替换功能。派生类可通过定义成员函数someOtherFunction(),为其父类添加功能。另一个成员函数someFunction()是从Base类继承的,在派生类中的行为与在基类中一样。许多情况下,可能希望通过替换或覆盖成员函数来修改类的行为。

\mySamllsection{ virtual 关键字}

派生类中定义基类的成员函数,并不能正确地覆盖该成员函数。为了正确覆盖成员函数,需要一个新的关键字,称为virtual。只有基类中声明为virtual的成员函数,派生类才能正确覆盖。关键字放在成员函数声明的开头,如下面的Base类的修改版本所示:

\begin{cpp}
class Base
{
    public:
        virtual void someFunction();
        // Remainder omitted for brevity.
};
\end{cpp}

对于Derived类也是如此。如果希望在派生类中覆盖其成员函数,也应该将其成员函数标记为virtual:

\begin{cpp}
class Derived : public Base
{
    public:
        virtual void someOtherFunction();
};
\end{cpp}

成员函数定义前不重复virtual关键字:

\begin{cpp}
void Base::someFunction()
{
    println("This is Base's version of someFunction().");
}
\end{cpp}

\begin{myWarning}{WARNING}
尝试覆盖基类中的非virtual成员函数,会隐藏基类的定义,并且只能在派生类的上下文中使用。
\end{myWarning}

\mySamllsection{覆盖成员函数的语法}

在派生类中覆盖一个成员函数,需要在派生类的定义中重新声明,其方式与基类中的声明一样,但是要添加override关键字,并移除virtual关键字。在Derived类中为someFunction()提供一个新的定义,首先需要在Derived的类定义中添加声明:

\begin{cpp}
class Derived : public Base
{
    public:
        void someFunction() override; // Overrides Base's someFunction()
        virtual void someOtherFunction();
};
\end{cpp}

someFunction()的新定义与Derived的其他成员函数一起指定,成员函数定义中不需要重复override关键字:

\begin{cpp}
void Derived::someFunction()
{
    println("This is Derived's version of someFunction().");
}
\end{cpp}

可以在覆盖的成员函数前添加virtual关键字,但这很多余:

\begin{cpp}
class Derived : public Base
{
    public:
        virtual void someFunction() override; // Overrides Base's someFunction()
};
\end{cpp}

当成员函数或析构函数标记为虚函数,其对所有派生类来说都是虚函数,即使在派生类中移除了virtual关键字。

\mySamllsection{客户端对覆写成员函数的看法}

在前面更改的基础上,其他代码仍然可以以之前相同的方式调用someFunction(),成员函数可以在Base类的对象或Derived类对象上调用该函数,而someFunction()的行为会根据对象的类型的变化而变化。

以下代码的工作方式与之前相同,调用Base的someFunction():

\begin{cpp}
Base myBase;
myBase.someFunction(); // Calls Base's version of someFunction().
\end{cpp}

输出为:

\begin{shell}
This is Base's version of someFunction().
\end{shell}

如果代码声明为Derived类对象,则会自动调用另一个:

\begin{cpp}
Derived myDerived;
myDerived.someFunction(); // Calls Derived's version of someFunction()
\end{cpp}

这次的输出:

\begin{shell}
This is Derived's version of someFunction().
\end{shell}

Derived类对象其他部分保持不变,从Base继承的其他成员函数。

指针或引用可以引用一个类的对象或其任何派生类的对象。对象本身“知道”它实际是哪个类的成员,所以只要声明为虚函数,就会调用适当的成员函数。如果有一个Base引用,引用了一个Derived的对象,someFunction()实际上会调用派生类的版本,如下所示。如果省略了基类中的virtual关键字,就无法覆盖的这个函数。

\begin{cpp}
Derived myDerived;
Base& ref { myDerived };
ref.someFunction(); // Calls Derived's version of someFunction()
\end{cpp}

即使Base引用或指针知道其正在引用一个Derived实例,也无法访问在Base中未定义的Derived类成员。以下代码无法编译,因为Base引用没有名为someOtherFunction()的成员函数:

\begin{cpp}
Derived myDerived;
Base& ref { myDerived };
myDerived.someOtherFunction(); // This is fine.
ref.someOtherFunction(); // Error
\end{cpp}

这种派生类特性对于非指针或非引用对象不可用,因为Derived也是Base类型,所以可以将Derived赋值或转换为Base,对象将会失去Derived类的所有特性。

\begin{cpp}
Derived myDerived;
Base assignedObject { myDerived }; // Assigns a Derived to a Base.
assignedObject.someFunction(); // Calls Base's version of someFunction()
\end{cpp}

可以将Base对象想象成一个占据一定内存的盒子,因为Derived拥有Base的所有内容,所以Derived对象是一个稍大的盒子。无论是通过Base引用或指针,还是Derived引用或指针访问Derived,盒子都不会改变——只是有了一种新的访问方式。将Derived转换为Base时,就会丢弃Derived类的所有“特性”,以使其成为一个更小的盒子。

\begin{myNotic}{NOTE}
通过基类指针或引用引用派生类时,派生类保留所有数据成员和成员函数。当将派生类强制转换为基类对象时,会失去其独特性。派生类数据成员和成员函数的丢失的现象,称为切片。
\end{myNotic}

\mySamllsection{override关键字}

可以选择性使用override关键字,但这里强烈推荐。没有这个关键字,可能会不小心在派生类中创建一个新的(虚)成员函数,而不是覆写基类中的成员函数,从而有效地隐藏了基类中的成员函数。以下Base和Derived类,其中Derived正确地覆写了someFunction(),但没有使用override关键字:

\begin{cpp}
class Base
{
    public:
    virtual void someFunction(double d);
};

class Derived : public Base
{
    public:
    virtual void someFunction(double d);
};
\end{cpp}

可以通过引用调用someFunction():

\begin{cpp}
Derived myDerived;
Base& ref { myDerived };
ref.someFunction(1.1); // Calls Derived's version of someFunction()
\end{cpp}

正确地调用了来自Derived类的覆写的someFunction()。假设在覆写someFunction()时,不小心使用了整数参数,而不是双精度浮点数:

\begin{cpp}
class Derived : public Base
{
    public:
        virtual void someFunction(int i);
};
\end{cpp}

这段代码并没有覆写Base的someFunction(),而是创建了新的虚成员函数。如果通过Base引用调用someFunction(),调用的是Base的someFunction(),而不是Derived的someFunction()!

\begin{cpp}
Derived myDerived;
Base& ref { myDerived };
ref.someFunction(1.1); // Calls Base's version of someFunction()
\end{cpp}

开始修改Base类,但忘记更新所有派生类时,可能会发生这种问题。Base类的第一个版本可能有一个名为someFunction()的成员函数,接受一个整数。然后编写了Derived类,覆盖了这个接受整数的someFunction()。后来决定Base中的someFunction()需要接受双精度浮点数而不是整数,所以更新了Base类中的someFunction()。可能在更新的时候,忘记更新派生类中的someFunction(),也改为接受双精度浮点数的参数。由于疏忽,实际上是创建了一个新的虚成员函数,而不是覆写基成员函数。

可以通过使用override关键字来避免这种情况:

\begin{cpp}
class Derived : public Base
{
    public:
        void someFunction(int i) override;
};
\end{cpp}

使用override关键字意味着someFunction()应该覆盖基类中的一个成员函数,所以这个Derived的定义会产生编译错误,但Base类中的someFunction()只接受双精度浮点数,不接受整数。

当在基类中重命名一个成员函数,并忘记在派生类中重命名覆写的成员函数时,也可能发生意外创建新成员函数的问题。

\begin{myWarning}{WARNING}
始终在将要覆写基类成员函数的成员函数上使用override关键字。
\end{myWarning}

\mySamllsection{虚函数的真相}

如果一个成员函数不是虚函数,尝试在派生类中进行覆写后,将隐藏基类中相应的成员函数。本节探讨了编译器如何实现虚成员函数,以及它们对性能的影响,同时还讨论了虚析构函数的重要性。

\mySamllsection{如何实现虚函数}

要理解如何避免隐藏成员函数,需要更多地了解virtual关键字实际上做了什么。在C++中编译一个类时,会创建一个二进制对象,其中包含该类的所有成员函数。函数非虚时,根据编译时类型,直接在调用成员函数的地方硬编码。这称为静态绑定,也称为早期绑定。

如果成员函数声明为虚函数,则通过使用一个特殊的内存区域,称为vtable(“虚表”)来调用正确的实现。有一个或多个虚成员函数的类都有一个vtable,该类的每个对象都包含一个指向该vtable的指针。这个vtable包含指向虚成员函数实现的指针。当在指向或引用对象的指针或引用上调用成员函数时,会跟随其vtable指针,并根据运行时对象的实际类型调用适当的成员函数。这称为动态绑定,也称为晚期绑定。重要的是,这种动态绑定仅在使用指向对象的指针或引用时有效。如果直接在对象上调用虚成员函数,该调用将使用编译时解析的静态绑定。

为了更好地理解vtable如何使成员函数的覆写成为可能,以Base和Derived类为例:

\begin{cpp}
class Base
{
    public:
        virtual void func1();
        virtual void func2();
        void nonVirtualFunc();
};

class Derived : public Base
{
    public:
        void func2() override;
        void nonVirtualFunc();
};
\end{cpp}

假设有以下两个实例:

\begin{cpp}
Base myBase;
Derived myDerived;
\end{cpp}

图10.5展示了两个实例的vtable的鸟瞰视图,myBase对象包含一个指向其vtable的指针。这个vtable有两个条目,一个用于func1(),一个用于func2()。这些条目指向Base::func1()和Base::func2()的实现。

\myGraphic{0.7}{content/part3/chapter10/images/5.png}{图 10.5}

myDerived也包含一个指向其vtable的指针,该vtable也有两个条目,一个用于func1(),一个用于func2()。Derived没有覆盖func1(),所以func1()指向Base::func1(),func2()指向Derived::func2()。

注意,两个vtable都不包含针对nonVirtualFunc()成员函数,因为该成员函数不是虚函数。

\mySamllsection{使用virtual的理由}

一些语言中,比如Java,所有成员函数都是自动虚化的,因此可以正确地覆写。C++并非如此,反对在C++中使所有内容虚化的论据,以及创建关键字的原因,与vtable的开销有关。要调用虚拟成员函数,程序需要通过取消引用指向要执行的正确代码的指针来执行一个操作。大多数情况下,这是一个微不足道的性能损失,但C++的设计者认为,让开发者决定是否需要性能损失会更好。如果成员函数永远不会覆写,就没有必要使其虚拟并承担性能损失。随着当今的CPU的发展,性能损失以纳秒甚至更小的数值来衡量,并且随着未来CPU的发展,这个数值将不断减小。大多数应用程序中,使用虚成员函数和不使用相比较,不会有可观的性能差异。

某些特定的用例中,性能开销会很高,可能需要有一个选项来避免这种开销。假设有一个带有虚成员函数的Point类,如果有另一个数据结构存储了数百万甚至数十亿个Point,在每个点上调用虚成员函数会产生显著的开销。这种情况下,避免在Point类中使用虚成员函数可能是明智之举。

每个对象的内存使用也会有一点增加。除了成员函数的实现之外,每个对象还需要一个指向其vtable的指针,这会占用一点空间。但这并不是问题,但有时确实很重要。再次以Point类和存储数十亿个Point的容器为例。在这种情况下,所需的内存变得很可观。

\mySamllsection{虚析构函数的必要性}

析构函数应该为虚。如果析构函数不为虚,很容易导致对象销毁时内存没有释放。只有在标记为final的类中,才可以让其析构函数设置为非虚。

派生类在构造函数中动态分配内存,并在析构函数中删除它。如果析构函数从未调用,则内存将永远不会释放。同样,如果派生类有在类实例销毁时自动删除的成员,比如std::unique\_ptr,那么如果析构函数从未调用,这些成员也不会删除。

正如以下代码所示,如果析构函数不为虚,很容易“欺骗”编译器跳过对析构函数的调用:

\begin{cpp}
class Base
{
    public:
        Base() = default;
        ~Base() {}
};

class Derived : public Base
{
    public:
        Derived()
        {
            m_string = new char[30];
            println("m_string allocated");
        }

        ~Derived()
        {
            delete[] m_string;
            println("m_string deallocated");
        }
    private:
        char* m_string;
};

int main()
{
    Base* ptr { new Derived {} }; // m_string is allocated here.
    delete ptr; // ~Base is called, but not ˜Derived because the destructor
                // is not virtual!
}
\end{cpp}

从以下输出中可以看出,Derived对象的析构函数从未调用,也就是说,“m\_string deallocated”消息从未显示:

\begin{cpp}
m_string allocated
\end{cpp}

从技术上讲,前面代码中的delete调用的行为在标准中是未定义的。C++编译器在这种未定义的情况下可以做任何事情。然而,大多数编译器只是调用基类的析构函数,而不调用派生类的析构函数。

修复这个问题的方法是在基类中标记析构函数为虚。如果不在那个析构函数中做其他工作,但想要使其为虚,可以标记为显式默认:

\begin{cpp}
class Base
{
    public:
        Base() = default;
        virtual ~Base() = default;
};
\end{cpp}

有了这个改变,输出就和预期一样了:

\begin{shell}
m_string allocated
m_string deallocated
\end{shell}

自从C++11以来,如果一个类有用户声明的析构函数,那么生成复制构造函数和复制赋值操作符是弃用的。基本上,当有了一个用户声明的析构函数,规则五就会生效。所以需要声明一个复制构造函数、复制赋值操作符、移动构造函数和移动赋值操作符,可以显式默认它们。不过,为了保持本章中的例子简洁明了,没有这样做。

\begin{myWarning}{WARNING}
除非有特殊的理由不这么做,或者类标记为final,否则析构函数应该为虚。构造函数不能,也不需要为虚,因为创建对象时,需要对确切具体的类进行构造。
\end{myWarning}

本章前面建议在打算覆写基类成员函数的成员函数上使用override关键字,也可以在析构函数上使用override关键字。如果基类中的析构函数不为虚,编译器将报错。也可以组合使用virtual、override和default:

\begin{cpp}
class Derived : public Base
{
    public:
        virtual ~Derived() override = default;
};
\end{cpp}

\mySamllsection{防止覆写}

除了将整个类标记为final之外,C++可以将单个成员函数标记为final。这样的成员函数不能在进一步的派生类中覆写。例如,从下面的Derived类覆盖someFunction()到DerivedDerived将导致编译错误:

\begin{cpp}
class Base
{
    public:
        virtual ~Base() = default;
        virtual void someFunction();
};
class Derived : public Base
{
    public:
        void someFunction() override final;
};
class DerivedDerived : public Derived
{
    public:
        void someFunction() override; // Compilation error.
};
\end{cpp}






