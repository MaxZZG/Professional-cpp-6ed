\noindent
\textbf{内筒概要}

\begin{itemize}
\item
通过继承扩展类

\item
使用继承重用代码

\item
建立基类和派生类之间的交互

\item
使用继承来实现多态

\item
处理多重继承

\item
处理继承中的异常

\item
将一种类型转换为另一种类型
\end{itemize}

本章的所有代码示例都可以在\url{https://github.com/Professional-CPP/edition-6}获得。

如果没有继承,类将仅仅是带有相关行为的数据结构。这本身就已经比过程式语言强得多了,但继承又为类增加了一个全新的维度。通过继承,可以基于现有的类构建新的类,类就变成了可重用和可扩展的组件。本章将介绍如何利用继承的力量,将了解继承的具体语法,以及如何充分利用继承的技术。

本章关于多态部分的讨论,大量引用了在第 8 章和第 9 章中讨论的电子表格示例。本章还引用了在第 5 章中描述的面向对象方法论。如果没有阅读那一章,并且不熟悉继承的理论,在继续本章之前最好回顾一下第 5 章。






























