
扩展一个类会引发各种问题。可以改变类的哪些特性?不能改变哪些特性?什么是非公有继承?什么是虚基类?这些问题将在这里得到解答。

\mySubsubsection{10.6.1.}{改变覆写成员函数的返回类型}

通常,覆写一个成员函数的原因是改变其实现。然而,有时可能想改变成员函数的其他特性,比如返回类型。

经验法则是使用与基类中,使用的完全相同的成员函数声明或成员函数原型来覆写成员函数。实现可以改变,但原型保持不变。

然而,这并非必须。在C++中,一个覆写的成员函数可以改变其返回类型,只要基类成员函数的返回类型是指向类的指针或引用,而在派生类中,返回类型是指向派生类的指针或引用,即更具体的类。这样的类型称为协变返回类型。当基类和派生类在并行层次结构中处理对象时,这种特性有时非常有用——即另一组与第一个类层次结构相关但稍显次要的类。

例如,一个基本的汽车模拟器。可能有两个类层次结构来模拟不同但明显相关的真实对象。第一个是Car层次结构,基类Car有派生类GasolineCar和ElectricalCar。同样,还有一个类层次结构,基类称为PowerSource,派生类有GasolinePowerSource和ElectricalPowerSource。图10.11显示了这两个类层次结构。

\myGraphic{0.8}{content/part3/chapter10/images/11.png}{图 10.11}

假设电源可以打印自己的类型,而汽油电源有一个成员函数fillTank(),而电力电源有一个成员函数chargeBatteries():

\begin{cpp}
class PowerSource
{
    public:
        virtual void printType() = 0;
};

class GasolinePowerSource : public PowerSource
{
    public:
        void printType() override { println("GasolinePowerSource"); }
        virtual void fillTank() { println("Gasoline tank filled up."); }
};

class ElectricalPowerSource : public PowerSource
{
    public:
        void printType() override { println("ElectricalPowerSource"); }
        virtual void chargeBatteries() { println("Batteries charged."); }
};
\end{cpp}

Car有一个名为getFilledUpPowerSource()的虚成员函数,该函数返回特定汽车的“加满”电源的引用:

\begin{cpp}
class Car
{
    public:
        virtual PowerSource& getFilledUpPowerSource() = 0;
};
\end{cpp}

这是一个纯虚、抽象的成员函数,只有在具体的派生类中提供实现才有意义。由于GasolinePowerSource是PowerSource,所以GasolineCar类可以按以下方式实现这个成员函数:

\begin{cpp}
class GasolineCar : public Car
{
    public:
        PowerSource& getFilledUpPowerSource() override
        {
            m_engine.fillTank();
            return m_engine;
        }
    private:
        GasolinePowerSource m_engine;
};
\end{cpp}

ElectricalCar可以按以下方式实现:

\begin{cpp}
class ElectricalCar : public Car
{
    public:
        PowerSource& getFilledUpPowerSource() override
        {
            m_engine.chargeBatteries();
            return m_engine;
        }
    private:
        ElectricalPowerSource m_engine;
};
\end{cpp}

这些类可以按以下方式进行测试:

\begin{cpp}
GasolineCar gc;
gc.getFilledUpPowerSource().printType();
println("");
ElectricalCar ev;
ev.getFilledUpPowerSource().printType();
\end{cpp}

输出如下:

\begin{shell}
Gasoline tank filled up.
GasolinePowerSource

Batteries charged.
ElectricalPowerSource
\end{shell}

这种实现是可行的,但我们知道GasolineCar的getFilledUpPowerSource()成员函数总是返回GasolinePowerSource,而ElectricalCar总是返回ElectricalPowerSource,可以通过改变返回类型来向使用这些类的用户表明这一点:

\begin{cpp}
class GasolineCar : public Car
{
    public:
    GasolinePowerSource& getFilledUpPowerSource() override
    { /* omitted for brevity */ }
};

class ElectricalCar : public Car
{
    public:
    ElectricalPowerSource& getFilledUpPowerSource() override
    { /* omitted for brevity */ }
};
\end{cpp}

确定是否可以更改重写的成员函数的返回类型的一个好方法是考虑现有代码是否仍然有效,这称为Liskov替换原则(LSP)。在前面的例子中,因为假设getFilledUpPowerSource()成员函数总是返回PowerSource的代码仍然可以编译并正确工作,所以可以更改返回类型。因为ElectricalPowerSource和GasolinePowerSource都是PowerSources,所以可以对getFilledUpPowerSource()返回的PowerSource调用的成员函数,同样可以对getFilledUpPowerSource()返回的ElectricalPowerSource或GasolinePowerSource调用相应的成员函数。

例如,不能将返回类型更改为完全无关的类型,如int\&。以下代码无法编译:

\begin{cpp}
class ElectricalCar : public Car
{
    public:
        int& getFilledUpPowerSource() override // Error!
        { /* omitted for brevity */ }
};
\end{cpp}

这会产生一个编译错误:

\begin{shell}
'ElectricalCar::getFilledUpPowerSource': overriding virtual function return type
differs and is not covariant from 'Car::getFilledUpPowerSource'
\end{shell}

这个例子使用了PowerSource的引用,而不是智能指针。使用,例如shared\_ptr作为返回类型时,更改返回类型不起作用。假设Car::getFilledUpPowerSource()返回shared\_ptr<PowerSource>。因为shared\_ptr是一个类模板,所以不能将ElectricalCar::getFilledUpPowerSource()的返回类型改为shared\_ptr。需要创建了两个shared\_ptr类模板的实例,shared\_ptr<PowerSource>和shared\_ptr<ElectricalPowerSource>。这两个实例化是完全不同的类型,并且彼此没有任何关系。不能将覆写成员函数的返回类型,更改为完全不同的类型。

\mySubsubsection{10.6.2.}{向派生类添加基类虚成员函数的重载}

可以在派生类中添加基类虚成员函数的新重载,可以在派生类中添加一个虚成员函数的新重载,具有新的原型,但继续继承基类的版本。可以使用using声明,明确地在派生类中包含基类成员函数的定义。以下是一个例子:

\begin{cpp}
class Base
{
    public:
        virtual void someFunction();
};

class Derived : public Base
{
    public:
        using Base::someFunction; // Explicitly inherits the Base version.
        virtual void someFunction(int i); // Adds a new overload of someFunction().
        virtual void someOtherFunction();
};
\end{cpp}

\begin{myNotic}{NOTE}
在派生类中找到与基类中成员函数同名,但参数列表不同的成员函数的情况很罕见。
\end{myNotic}

\mySubsubsection{10.6.3.}{继承的构造函数}

前一节中,看到了使用using声明来明确包含基类中成员函数定义的示例。这适用于普通类成员函数,也适用于构造函数,允许从基类继承构造函数。看看Base和Derived类的定义:

\begin{cpp}
class Base
{
    public:
        virtual ~Base() = default;
        Base() = default;
        explicit Base(int i) {}
};

class Derived : public Base
{
    public:
        explicit Derived(int i) : Base(i) {}
};
\end{cpp}

Derived构造函数所做的唯一事情是,将其参数传递给Base构造函数。

只能使用Base提供的构造函数来构造Base对象,要么是默认构造函数,要么是接受一个int的构造函数。另一方面,构造Derived对象只能通过提供的Derived构造函数,需要一个整数作为参数。不能使用Base类的默认构造函数来构造Derived对象:

\begin{cpp}
Base base { 1 }; // OK, calls integer Base ctor.
Derived derived1 { 2 }; // OK, calls integer Derived ctor.
Derived derived2; // Error, Derived does not have a default ctor.
\end{cpp}

由于Derived构造函数只是将其参数传递给Base构造函数,并没有做其他事情,所以可以在Derived类中通过using声明显式继承Base构造函数:

\begin{cpp}
class Derived : public Base
{
    public:
        using Base::Base;
};
\end{cpp}

using声明继承了Base的所有构造函数,所以可以这样构造Derived对象:

\begin{cpp}
Derived derived1 { 2 }; // OK, calls inherited integer Base ctor.
Derived derived2; // OK, calls inherited default Base ctor.
\end{cpp}

派生类中的继承构造函数,具有与基类构造函数相同的访问说明符(public、protected或private)。在基类中显式删除的继承构造函数,也会在派生类中删除。

\mySamllsection{隐藏继承的构造函数}

派生类可以定义一个参数列表与基类中继承的构造函数相同。这种情况下,派生类的构造函数优先于继承的构造函数。以下示例中,Derived类继承了Base类的所有构造函数,使用using声明。然而,由于Derived类定义了自己的一个单参数类型为float的构造函数,隐藏了Base类中参数类型为float的继承构造函数。

\begin{cpp}
class Base
{
    public:
        virtual ~Base() = default;
        Base() = default;
        explicit Base(std::string_view str) {}
        explicit Base(float f) {}
};

class Derived : public Base
{
    public:
        using Base::Base;
        explicit Derived(float f) {} // Hides inherited float Base ctor.
};
\end{cpp}

有了这个定义,可以创建Derived对象:

\begin{cpp}
Derived derived1 { "Hello" }; // OK, calls inherited string_view Base ctor.
Derived derived2 { 1.23f }; // OK, calls float Derived ctor.
Derived derived3; // OK, calls inherited default Base ctor.
\end{cpp}

使用using声明从基类继承构造函数时,有一些限制。

\begin{itemize}
\item
当从基类继承一个构造函数时,就继承了所有构造函数。不可能只继承基类构造函数的一个子集。

\item
当继承构造函数时,其与基类中访问权限相同(直接继承),不受using声明下的访问权限的限制。
\end{itemize}

\mySamllsection{继承构造函数和多重继承}

继承构造函数的另一个限制与多重继承有关。如果另一个基类有一个与第一个基类参数列表相同的构造函数,那么不可能从第一个基类继承构造函数,这会导致歧义。为了解决这个问题,派生类需要显式定义冲突的构造函数。例如,以下Derived类尝试从Base1和Base2继承所有构造函数,这导致了浮点数构造函数的歧义。

\begin{cpp}
class Base1
{
    public:
        virtual ~Base1() = default;
        Base1() = default;
        explicit Base1(float f) {}
};

class Base2
{
    public:
        virtual ~Base2() = default;
        Base2() = default;
        explicit Base2(std::string_view str) {}
        explicit Base2(float f) {}
};

class Derived : public Base1, public Base2
{
    public:
        using Base1::Base1;
        using Base2::Base2;
        explicit Derived(char c) {}
};

int main()
{
    Derived d { 1.2f }; // Error, ambiguity.
}
\end{cpp}

Derived中的第一个using声明继承了Base1的所有构造函数,所以Derived有以下构造函数:

\begin{cpp}
Derived(float f); // Inherited from Base1.
\end{cpp}

Derived中的第二个using声明尝试从Base2继承所有构造函数,所以Derived得到了第二个Derived(float)构造函数。问题通过在Derived类中显式声明冲突的构造函数来解决,如下所示:

\begin{cpp}
class Derived : public Base1, public Base2
{
    public:
        using Base1::Base1;
        using Base2::Base2;
        explicit Derived(char c) {}
        explicit Derived(float f) {}
};
\end{cpp}

Derived类现在显式声明了一个单参数类型为float的构造函数,解决了歧义。Derived类中也接受浮点数参数的显式构造函数仍然可以在初始化器中转发调用Base1和Base2的构造函数,如下所示:

\begin{cpp}
Derived::Derived(float f) : Base1 { f }, Base2 { f } {}
\end{cpp}

\mySamllsection{数据成员的初始化}

使用继承的构造函数时,请确保所有数据成员都正确初始化。例如,以下Base和Derived的新定义。这些定义在所有情况下,都没有正确初始化m\_int数据成员。

\begin{cpp}
class Base
{
    public:
        virtual ~Base() = default;
        explicit Base(std::string_view str) : m_str { str } {}
    private:
        std::string m_str;
};

class Derived : public Base
{
    public:
        using Base::Base;
        explicit Derived(int i) : Base { "" }, m_int { i } {}
    private:
        int m_int;
};
\end{cpp}

可以像这样创建一个Derived对象:

\begin{cpp}
Derived s1 { 2 };
\end{cpp}

这调用了Derived(int)构造函数,初始化Derived类的m\_int数据成员,并用空字符串调用Base构造函数初始化了Base类的m\_str数据成员。

因为Base构造函数在Derived类中继承,也可以像这样构造一个Derived对象:

\begin{cpp}
Derived s2 { "Hello World" };
\end{cpp}

这调用Derived类中的继承Base构造函数,这个继承的Base构造函数只初始化Base类的m\_str,并没有初始化Derived类的m\_int,使其处于未初始化状态。这种情况下,解决方案是在类中使用成员初始化器,这在第8章中介绍过。

以下代码使用类内成员初始化器将m\_int初始化为0。当然,Derived(int)构造函数仍然可以改变这个值,并初始化m\_int为构造参数i。

\begin{cpp}
class Derived : public Base
{
    public:
        using Base::Base;
        explicit Derived(int i) : Base { "" }, m_int { i } {}
    private:
        int m_int { 0 };
};
\end{cpp}

\mySubsubsection{10.6.4.}{重写成员函数的特殊情况}

重写一个成员函数时,有几个特殊情况需要注意。本节概述了可能遇到的情况。

\mySamllsection{静态基类成员函数}

在C++中,不能重写一个静态成员函数。大部分情况下,这就是需要知道的全部。

首先,一个成员函数不能同时是静态和虚的。如果在派生类中有与基类中静态成员函数同名,且具有相同参数列表的静态成员函数,这样其实是两个独立的成员函数。

以下代码展示了两个类都有名为beStatic()的静态成员函数。这两个成员函数没有任何关系。

\begin{cpp}
class BaseStatic
{
    public:
    static void beStatic() { println("BaseStatic being static."); }
};

class DerivedStatic : public BaseStatic
{
    public:
    static void beStatic() { println("DerivedStatic keepin' it static."); }
};
\end{cpp}

因为静态成员函数属于其类,所以使用这两个不同类名调用同名的成员函数会调用各自的成员函数。

\begin{cpp}
BaseStatic::beStatic();
DerivedStatic::beStatic();
\end{cpp}

输出的结果是:

\begin{shell}
BaseStatic being static.
DerivedStatic keepin' it static.
\end{shell}

只要通过类名访问成员函数,一切都非常合理。当涉及到对象时,行为就不那么清晰了。在C++中,可以使用对象调用静态成员函数,但由于成员函数是静态的,没有this指针,也没有访问对象本身,所以等价于通过其类名调用。参考前面的示例类,可以写如下代码,但结果可能令人惊讶。

\begin{cpp}
DerivedStatic myDerivedStatic;
BaseStatic& ref { myDerivedStatic };
myDerivedStatic.beStatic();
ref.beStatic();
\end{cpp}

第一次调用beStatic()显然是DerivedStatic的版本,因为明确地在声明为DerivedStatic的对象上调用。第二次调用可能不会按预期的工作。对象是一个BaseStatic引用,但它引用一个DerivedStatic对象,调用的是BaseStatic的beStatic()版本。原因是C++在调用静态成员函数时不关心对象实际上是什么,只关心编译时的类型。这种情况下,类型是一个对BaseStatic的引用。

前面的示例输出的结果:

\begin{shell}
DerivedStatic keepin' it static.
BaseStatic being static.
\end{shell}

\begin{myNotic}{NOTE}
静态成员函数按定义它们的类名进行作用域限定,但不适用于特定对象的成员函数。当调用静态成员函数时,由正常的名称解析可以确定相应的函数版本。当成员函数通过对象调用时,对象实际上并不参与调用,除了在编译时确定类型。
\end{myNotic}

\mySamllsection{重载基类成员函数}

当通过指定名称和一组参数来覆写一个成员函数时,编译器隐式隐藏了基类中具有相同名称的其他实例。如果已经用给定名称覆写了一个成员函数,就可能有意覆写所有具有该名称的成员函数,但这应该视为是一个错误。考虑一下——为什么要改变一个成员函数的部分重载版本呢?考虑以下Derived类,它覆写了一个成员函数,但而没有覆盖其相关联的重载:

\begin{cpp}
class Base
{
    public:
        virtual ~Base() = default;
        virtual void overload() { println("Base's overload()"); }
        virtual void overload(int i) { println("Base's overload(int i)"); }
};

class Derived : public Base
{
    public:
        void overload() override { println("Derived's overload()"); }
};
\end{cpp}

如果尝试在Derived对象上调用接受int参数的overload(),代码将无法编译,因为它没有被显式覆写。

\begin{cpp}
Derived myDerived;
myDerived.overload(2); // Error! No matching member function for overload(int).
\end{cpp}

然而,可以从Derived对象访问这个成员函数,只需要一个指向Base对象的指针或引用。

\begin{cpp}
Derived myDerived;
Base& ref { myDerived };
ref.overload(7);
\end{cpp}

C++中隐藏未实现的重载成员函数只是表面现象。明确声明为派生类实例的对象不会使这些成员函数可用,但简单的类型转换到基类就可以使它们可用。

当只想改变其中一个时,使用using声明可以节省覆写所有重载的麻烦。在以下代码中,Derived类定义使用Base的一个overload(),并显式覆写了另一个:

\begin{cpp}
class Derived : public Base
{
    public:
        using Base::overload;
        void overload() override { println("Derived's overload()"); }
};
\end{cpp}

using声明存在一些风险。假设Base中添加了第三个overload()成员函数,应该在Derived中覆写。因为有using声明,这些现在不会检测为错误,Derived类的开发者已经明确表示,“我愿意接受父类中这个成员函数的所有重载。”

\begin{myNotic}{NOTE}
为了在覆写基类成员函数时避免难以发现的错误,最好也覆写该成员函数的所有重载。
\end{myNotic}

\mySamllsection{private基类成员函数}

覆写private成员函数绝对没有问题,成员函数的访问说明符决定了谁能调用该成员函数。仅仅因为派生类不能调用其父类的私有成员函数,并不意味着不能覆写它们。实际上,模板成员函数模式在C++中是一种常见模式,通过覆写private成员函数来实现。其允许派生类定义自己的函数,可在基类中引用。注意,Java和C\#只允许覆写public和protected的成员函数,而不允许覆写private成员函数。

例如,以下类是汽车模拟器的一部分,根据汽车的油耗和剩余燃料量估计汽车可以行驶的里程数。getMilesLeft()成员函数是模板成员函数,模板成员函数非虚。通常会在基类中定义一些算法骨架,调用虚成员函数来查询信息。派生类可以覆写这些虚成员函数以改变算法,而无需修改基类算法本身。

\begin{cpp}
export class MilesEstimator
{
    public:
        virtual ~MilesEstimator() = default;
        int getMilesLeft() const { return getMilesPerGallon() * getGallonsLeft(); }
        virtual void setGallonsLeft(int gallons) { m_gallonsLeft = gallons; }
        virtual int getGallonsLeft() const { return m_gallonsLeft; }
    private:
        int m_gallonsLeft { 0 };
        virtual int getMilesPerGallon() const { return 20; }
};
\end{cpp}

getMilesLeft()成员函数根据自己的两个成员函数的结果进行计算:getGallonsLeft()是public,getMilesPerGallon()是private。以下代码使用MilesEstimator来计算两加仑汽油可以行驶多少英里:

\begin{cpp}
MilesEstimator myMilesEstimator;
myMilesEstimator.setGallonsLeft(2);
println("Normal estimator can go {} more miles.",
    myMilesEstimator.getMilesLeft());
\end{cpp}

代码的输出如下:

\begin{shell}
Normal estimator can go 40 more miles.
\end{shell}

为了使模拟器更有趣,会引入不同类型的车辆,也许是一辆更高效的汽车。现有的MilesEstimator,假设所有的汽车每加仑可以行驶20英里,但这个值从一个特定的成员函数返回,派生类就可以覆写它:

\begin{cpp}
export class EfficientCarMilesEstimator : public MilesEstimator
{
    private:
        int getMilesPerGallon() const override { return 35; }
};
\end{cpp}

通过覆盖这个private成员函数,新类完全改变了基类中现有、未修改的public成员函数的行为。基类中的getMilesLeft()成员函数调用private getMilesPerGallon()成员函数的覆写版本。使用新类的示例如下:

\begin{cpp}
EfficientCarMilesEstimator myEstimator;
myEstimator.setGallonsLeft(2);
println("Efficient estimator can go {} more miles.",
        myEstimator.getMilesLeft());
\end{cpp}

这次的输出:

\begin{shell}
Efficient estimator can go 70 more miles.
\end{shell}

\begin{myNotic}{NOTE}
覆写private和protected的成员函数,是改变类的某些特性,而不进行重大修改的好方法。
\end{myNotic}


\mySamllsection{基类成员函数的默认参数}

派生类中,重写的成员函数可以具有与基类不同的默认参数。使用的参数取决于变量的声明类型,而不是底层对象。以下是一个派生类提供不同默认参数的重写成员函数的简单示例:

\begin{cpp}
class Base
{
    public:
        virtual ~Base() = default;
        virtual void go(int i = 2) { println("Base's go with i={}", i); }
};

class Derived : public Base
{
    public:
        void go(int i = 7) override { println("Derived's go with i={}", i); }
};
\end{cpp}

如果在派生对象上调用 go(),则执行派生类的 go() ,并使用默认参数 7。如果在基对象上调用 go(),则调用基类的 go() ,并使用默认参数 2。然而(这是奇怪的部分),实际上指向派生对象的基指针或基引用上调用 go(),则调用派生类的 go() ,但使用基类的默认参数 2。以下示例展示了这种行为:

\begin{cpp}
Base myBase;
Derived myDerived;
Base& myBaseReferenceToDerived { myDerived };
myBase.go();
myDerived.go();
myBaseReferenceToDerived.go();
\end{cpp}

代码的输出如下:

\begin{shell}
Base's go with i=2
Derived's go with i=7
Derived's go with i=2
\end{shell}

这种行为的原因是 C++ 使用表达式的编译时类型,来绑定默认参数,而不是运行时类型。在 C++ 中,默认参数不“继承”。如果此示例中的派生类没有为其父类提供 go() 的默认参数,则在不对派生对象传递参数的情况下无法调用 go()。

\begin{myNotic}{NOTE}
重写具有默认参数的成员函数时,应提供默认参数,并且它可能是相同的值。建议使用常量作为默认值,以便在派生类中使用相同的常量。
\end{myNotic}

\mySamllsection{基类成员函数具有不同的访问规格}

可以以两种方式更改成员函数的访问规格:尝试使其有更多或少的限制性。在 C++ 中,这两种情况都没有太大意义,但有一些理由可以尝试这样做。

为了对成员函数(或数据成员)实施更严格的限制,可以采取两种方法。一种方法是更改整个基类的访问说明符。这种方法在本章后面描述。另一种方法是在派生类中简单地重新定义访问权限,如下面的 Shy 类所示:

\begin{cpp}
class Gregarious
{
    public:
        virtual void talk() { println("Gregarious says hi!"); }
};

class Shy : public Gregarious
{
    protected:
        void talk() override { println("Shy reluctantly says hello."); }
};
\end{cpp}

Shy 中的 talk() 的protected版本正确地覆盖了 Gregarious::talk() 成员函数。尝试在 Shy 对象上调用 talk() 的客户端代码都会收到编译错误:

\begin{cpp}
Shy myShy;
myShy.talk(); // Error! Attempt to access protected member function.
\end{cpp}

然而,成员函数并非完全受到保护。只需获取一个 Gregarious 引用或指针,即可访问protected的成员函数:

\begin{cpp}
Shy myShy;
Gregarious& ref { myShy };
ref.talk();
\end{cpp}

代码的输出如下:

\begin{shell}
Shy reluctantly says hello.
\end{shell}

这证明在派生类中将成员函数设置为protected确实覆写了成员函数(因为正确调用了派生类的版本),但也证明如果基类将其设置为 public,则无法对protected进行访问。

\begin{myNotic}{NOTE}
没有合理的方法(或好的理由)来限制对public基类成员函数的访问。
\end{myNotic}

\begin{myNotic}{NOTE}
因为它想要显示不同的消息,所以前面的示例在派生类中重新定义了成员函数。如果不想更改实现,而只想更改成员函数的访问规格,首选方法是在派生类定义中简单地添加一个具有所需访问规格的 using 声明。
\end{myNotic}

派生类中放宽访问限制要容易得多(也更有意义)。最简单的方法是在派生类中提供一个public成员函数,该函数调用基类中的protected成员函数:

\begin{cpp}
class Secret
{
    protected:
        virtual void dontTell() { println("I'll never tell."); }
};

class Blabber : public Secret
{
    public:
        virtual void tell() { dontTell(); }
};
\end{cpp}

客户端调用 Blabber 对象的public tell() 成员函数实际上访问了 Secret 类的protected成员函数。当然,这实际上并没有改变 dontTell() 的访问规格,只是提供了访问相应函数的方式。

也可以在 Blabber 中显式覆写 dontTell(),并为其提供新的公共访问行为。这比减少访问规格更有意义,因为通过基类的引用或指针会发生什么完全清楚。例如,假设 Blabber 实际上使 dontTell() 成员函数成为public:

\begin{cpp}
class Blabber : public Secret
{
    public:
        void dontTell() override { println("I'll tell all!"); }
};
\end{cpp}

现在你可以在 Blabber 对象上调用 dontTell():

\begin{cpp}
myBlabber.dontTell(); // Outputs "I'll tell all!"
\end{cpp}

如果不想更改覆盖的成员函数的实现,而只想更改访问规格,可以使用 using 声明。这里有一个例子:

\begin{cpp}
class Blabber : public Secret
{
    public:
        using Secret::dontTell;
};
\end{cpp}

这也允许在 Blabber 对象上调用 dontTell(),但这次输出将是“I’ll never tell”:

\begin{cpp}
myBlabber.dontTell(); // Outputs "I'll never tell."
\end{cpp}

在两种先前的情况下,基类中的protected成员函数仍然受保护,任何尝试通过 Secret 指针或引用,调用 Secret 的 dontTell() 成员函数的方法都无法编译:

\begin{cpp}
Blabber myBlabber;
Secret& ref { myBlabber };
Secret* ptr { &myBlabber };
ref.dontTell(); // Error! Attempt to access protected member function.
ptr->dontTell(); // Error! Attempt to access protected member function.
\end{cpp}

\begin{myNotic}{NOTE}
更改成员函数访问规格的唯一有用方法是,提供一个对protected成员函数较少限制的访问器。
\end{myNotic}

\mySubsubsection{10.6.5.}{派生类中的复制构造函数和赋值运算符}

第9章解释了在类中拥有动态分配内存时,提供复制构造函数和赋值运算符的必要性。定义派生类时,需要谨慎处理复制构造函数和operator=。

如果派生类不需要非默认复制构造函数或operator=的特殊数据(通常是指针),则不需要有它们。如果派生类省略了复制构造函数或operator=,将为其在派生类中指定的数据成员,提供默认复制构造函数或operator=,并使用基类中的复制构造函数或operator=为其在基类中指定的数据成员。

另一方面,如果在派生类中指定了复制构造函数,则需要显式调用父复制构造函数,如下面的代码所示。如果不这样做,将使用默认构造函数(不是复制构造函数!)来构造对象的父类。

\begin{cpp}
class Base
{
    public:
        virtual ~Base() = default;
        Base() = default;
        Base(const Base& src) { }
};

class Derived : public Base
{
    public:
        Derived() = default;
        Derived(const Derived& src) : Base { src } { }
};
\end{cpp}

同样,如果派生类覆盖了operator=,需要调用父类的operator=。唯一不需要这样做的情况是,希望在赋值时只赋值对象的一部分。以下代码显示了如何使用派生类调用父类的赋值运算符:

\begin{cpp}
Derived& Derived::operator=(const Derived& rhs)
{
    if (&rhs == this) { return *this; }
    Base::operator=(rhs); // Calls parent's operator=.
    // Do necessary assignments for derived class.
    return *this;
}
\end{cpp}

\begin{myWarning}{WARNING}
如果派生类没有指定自己的复制构造函数或operator=,基类的功能将继续有效。但是,如果派生类提供了自己的复制构造函数或operator=,则需要显式调用基类的。
\end{myWarning}


\begin{myNotic}{NOTE}
当需要在继承层次结构中具有复制功能时,因为依赖标准的复制构造函数和复制赋值运算符并不够,所以专业C++开发者常用的一个惯用法是实现多态的clone()成员函数。多态clone()惯用法将在第12章中介绍。
\end{myNotic}

\mySubsubsection{10.6.6.}{运行时的类型工具}

相对于其他面向对象语言,C++ 非常注重编译时。比如,覆写成员函数之所以有效,是因为成员函数及其实现之间的间接级别。

然而,C++ 中有一些特性提供了对象的运行时视图,这些特性通常称为运行时类型信息(RTTI)的特性集。RTTI 提供了许多特性,用于处理有关对象类成员资格的信息。这样的一个特性是 dynamic\_cast(),允许在面向对象层次结构内安全地在类型之间转换。在没有 vtable 的类上使用 dynamic\_cast(),若没有虚成员函数的类,则会导致编译错误。

第二个 RTTI 特性是 typeid 运算符,允许在运行时查询类型。应用该运算符的结果是对 中定义的 std::type\_info 对象的引用。type\_info 类有一个名为 name() 的成员函数,返回编译器依赖的类型名称。typeid 运算符的行为如下所示:

\begin{itemize}
\item
typeid(type): 结果是对表示给定类型的 type\_info 对象的引用。

\item
typeid(expression)
\begin{itemize}
\item
如果计算 expression 会导致类型多态化,则计算 expression,并且 typeid 运算符的结果是对表示对表达式动态类型的 type\_info 对象的引用。

\item
否则,不计算expression,并且结果是静态类型 type\_info 对象的引用。
\end{itemize}
\end{itemize}

通常,不需要使用 typeid,而使用基于对象类型的条件执行代码。

下面的代码使用 typeid 来基于对象的类型打印消息:

\begin{cpp}
class Animal { public: virtual ~Animal() = default; };
class Dog : public Animal {};
class Bird : public Animal {};

void speak(const Animal& animal)
{
    if (typeid(animal) == typeid(Dog)) {
        println("Woof!");
    } else if (typeid(animal) == typeid(Bird)) {
        println("Chirp!");
    }
}
\end{cpp}

每当看到这样的代码,应该立即考虑将功能重新实现为虚成员函数。这个例子中,更好的实现方法是在 Animal 基类中声明一个名为 speak() 的虚成员函数。Dog 将覆写该成员函数以打印 “Woof!”,而 Bird 将覆写以打印 “Chirp!”。这种方法更适合面向对象编程,其中将与对象相关的功能赋予对象。

\begin{myWarning}{WARNING}
typeid 运算符仅在类至少有一个虚成员函数时,即类有 vtable 时,才能正确工作。此外,typeid 运算符会从其参数中去除引用和 const 限定符。
\end{myWarning}

typeid 运算符的一个可能用例是用于日志记录和调试,下面的代码使用 typeid 进行日志记录。logObject() 函数接受一个“可记录”的对象作为参数。设计是这样的:可记录的对象都继承自 Loggable 类,并支持一个名为 getLogMessage() 的成员函数。

\begin{cpp}
class Loggable
{
    public:
        virtual ~Loggable() = default;
        virtual string getLogMessage() const = 0;
};

class Foo : public Loggable
{
    public:
        string getLogMessage() const override { return "Hello logger."; }
};

void logObject(const Loggable& loggableObject)
{
    print("{}: ", typeid(loggableObject).name());
    println("{}", loggableObject.getLogMessage());
}
\end{cpp}

logObject() 首先向控制台打印对象的类名,然后是其日志消息。 这样,稍后阅读日志时,可以看到每个写入行的对象。以下是使用 Microsoft Visual C++ 2022 调用 logObject() 时,Foo 实例生成的输出:

\begin{shell}
class Foo: Hello logger.
\end{shell}

typeid 运算符返回的名称是“class Foo”。然而,这个名称取决于你的编译器。例如,如果使用 GCC 编译和运行相同的代码,输出如下:

\begin{cpp}
3Foo: Hello logger.
\end{cpp}

\begin{myNotic}{NOTE}
如果使用 typeid 的目的不是日志记录和调试,请考虑修改你的设计。例如,使用虚成员函数。
\end{myNotic}

\mySubsubsection{10.6.7.}{非public继承}

之前的示例中,父类总是使用public关键字列出。读者们会想知道父类是否可以是private的protected的吗?实际上可以,尽管这两种方式都不像public继承那样常见。如果没有为父类指定访问说明符,那么对于类来说它是private继承,对于结构体来说是public继承。

将父类的关系声明为protected,则基类中的所有public成员函数和数据成员在派生类的上下文中变为protected。类似地,指定private继承则在基类中的所有public和protected成员函数和数据成员在派生类中变为private。

因为一些原因,想要以这种方式统一降低父类的访问级别,但大多数原因暗示了层次结构设计的缺陷。一些开发者滥用这种语言特性,通常与多重继承结合使用,来实现类的“组件”。他们不是创建一个包含引擎数据成员和机身数据成员的Airplane类,而是创建一个protected的引擎和protected的机身的Airplane类。通过这种方式,Airplane对客户端代码来说看起来不像引擎或机身(因为一切都是protected),但它能够使用所有内部功能。

\begin{myNotic}{NOTE}
非public继承很少见,我建议谨慎使用。如果没有其他原因,仅仅是因为大多数开发者对其不熟悉。
\end{myNotic}

\mySubsubsection{10.6.8.}{虚基类}

本章之前的内容中,已经提及了关于歧义基类的问题,当多个父类中每个都具有一个共同的基类时,会出现这种情况,如图 10.12 所示。之前推荐的方法是确保共享的基类没有相同的功能,则其成员函数就永远不会调用,也就没有歧义问题。

\myGraphic{0.3}{content/part3/chapter10/images/12.png}{图 10.12}

C++ 有一个机制,称为虚基类,用于解决如果确实希望共享基类的功能问题。如果共享的基类标记为虚基类,就不会有歧义。下面的代码在 Animal 基类中添加了一个 sleep() 成员函数及其实现,并修改了 Dog 和 Bird 类,使其从 Animal 虚继承。如果没有使用虚基类,对 DogBird 对象调用 sleep() 将会产生歧义,并且会导致编译错误,因为 DogBird 会有两个 Animal 子对象,一个来自 Dog,一个来自 Bird。然而,当 Animal 虚继承时,DogBird 只有一个 Animal 子对象,因此 sleep() 就不会有歧义。

\begin{cpp}
class Animal
{
    public:
        virtual void eat() = 0;
        virtual void sleep() { println("zzzzz...."); }
};

class Dog : public virtual Animal
{
    public:
        virtual void bark() { println("Woof!"); }
        void eat() override { println("The dog ate."); }
};

class Bird : public virtual Animal
{
    public:
        virtual void chirp() { println("Chirp!"); }
        void eat() override { println("The bird ate."); }
};

class DogBird : public Dog, public Bird
{
    public:
        void eat() override { Dog::eat(); }
};

int main()
{
    DogBird myConfusedAnimal;
    myConfusedAnimal.sleep(); // Not ambiguous because of virtual base class.
}
\end{cpp}

对于这样的类层次结构,要小心构造函数。例如,下面的代码在不同的类中添加了一些数据成员,添加了构造函数来初始化这些数据成员。并且,在 Animal 类中添加了一个protected的默认构造函数。

\begin{cpp}
class Animal
{
    public:
        explicit Animal(double weight) : m_weight { weight } {}
        virtual double getWeight() const { return m_weight; }
    protected:
        Animal() = default;
    private:
        double m_weight { 0.0 };
};
class Dog : public virtual Animal
{
    public:
        explicit Dog(double weight, string name)
            : Animal { weight }, m_name { move(name) } {}
    private:
        string m_name;
};
class Bird : public virtual Animal
{
    public:
        explicit Bird(double weight, bool canFly)
            : Animal { weight }, m_canFly { canFly } {}
    private:
        bool m_canFly { false };
};

class DogBird : public Dog, public Bird
{
    public:
        explicit DogBird(double weight, string name, bool canFly)
            : Dog { weight, move(name) }, Bird { weight, canFly } {}
};

int main()
{
    DogBird dogBird { 22.33, "Bella", true };
    println("Weight: {}", dogBird.getWeight());
}
\end{cpp}

运行此代码时,输出为:

\begin{cpp}
Weight: 0
\end{cpp}

main() 中构造 DogBird 时,给出的 22.33 重量丢失的原因是,这段代码使用了虚 Animal 基类。因此,DogBird 实例只有一个 Animal 子对象。DogBird 构造函数调用了 Dog 和 Bird 的构造函数,这两个构造函数转发给了 Animal 基类的构造函数,所以 Animal 构造了两次。这时,编译器禁用了在派生类的构造函数中,对 Animal 构造函数的调用,而是调用了 Animal 基类的默认构造函数,因此 Animal 需要一个protected的默认构造函数,所以派生类本身负责调用共享基类的构造函数。看看正确的实现:

\begin{cpp}
class Animal { /* Same as before. */ };

class Dog : public virtual Animal
{
    public:
        explicit Dog(double weight, string name)
            : Animal { weight }, m_name { move(name) } {}
    protected:
        explicit Dog(string name) : m_name { move(name) } {}
    private:
        string m_name;
};

class Bird : public virtual Animal
{
    public:
        explicit Bird(double weight, bool canFly)
            : Animal { weight }, m_canFly { canFly } {}
    protected:
        explicit Bird(bool canFly) : m_canFly { canFly } {}
    private:
        bool m_canFly { false };
};

class DogBird : public Dog, public Bird
{
    public:
        explicit DogBird(double weight, string name, bool canFly)
            : Animal { weight }, Dog { move(name) }, Bird { canFly } {}
};
\end{cpp}

这个实现中,Dog 和 Bird 都添加了protected的单参数构造函数,因为它们应该只在派生类中使用。客户端代码只能使用两个参数的构造函数来构造 Dog 和 Bird。

修改之后,输出正确了:

\begin{shell}
Weight: 22.33
\end{shell}

\begin{myNotic}{NOTE}
虚基类是避免类层次结构中歧义的一个很好的方法,但许多 C++ 开发者对这个概念不熟悉。
\end{myNotic}










