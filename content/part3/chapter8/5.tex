通过解决下面的练习,可以练习本章讨论的内容。所有练习的解决方案都可以在本书的网站\url{www.wiley.com/go/proc++6e}下载到源码。然而,若在练习中卡住了,在从网站上寻找解决方案之前,可以考虑先重读本章的部分内容,试着自己找到答案。

\begin{itemize}
\item
\textbf{练习 8-1}: 实现一个 Person 类,存储一个首字母和姓作为数据成员。添加一个接受两个参数的单构造函数,首字母和姓。提供适当的 getters 和 setters。编写一个小的 main() 函数来测试您的实现,通过在栈上和堆上创建 Person 对象。

\item
\textbf{练习 8-2}: 基于练习 8-1 中的实现,以下代码行不能编译:

\begin{cpp}
Person persons[3];
\end{cpp}

能解释为什么这行代码不能编译吗?修改 Person 类的实现,使其可以工作。

\item
\textbf{练习 8-3}: 在 Person 类实现中添加以下成员函数:一个复制构造函数、一个复制赋值运算符和一个析构函数。在这些所有成员函数中,实现您认为必要的部分,并添加输出一行文本到控制台的代码,以便追踪它们何时执行,修改 main() 函数来测试这些新的成员函数。注意:因为编译器生成的版本已经足够好,这些新的成员函数对于这个 Person 类并不必需,这里只是为了练习。

\item
\textbf{练习 8-4}: 从 Person 类中删除复制构造函数、复制赋值运算符和析构函数,因为默认的编译器生成的版本正好是这个简单类所需要的。接下来,添加一个新数据成员来存储一个人的缩写名,并提供一个 getter 和 setter。添加一个新的构造函数,接受三个参数,一个首字母和姓,以及一个人的缩写名。修改原始的两个参数构造函数,为给定的首字母和姓自动生成缩写名,并将实际的构造工作委托给新的三个参数构造函数。在 main() 函数中测试这个新功能。
\end{itemize}