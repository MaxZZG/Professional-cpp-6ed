This chapter covered the fundamental aspects of C++’s facilities for object-oriented programming: classes and objects. It first reviewed the basic syntax for writing classes and using objects, including access control. Then, it covered object life cycles: when objects are constructed, destructed, and assigned to, and what member functions those actions invoke. The chapter included details of the constructor syntax, including ctor-initializers and initializer-list constructors, and introduced the notion of copy assignment operators. It also specified exactly which constructors the compiler writes for you and under what circumstances, and it explained that default constructors require no arguments.

You may have found this chapter to be mostly review. Or, it may have opened your eyes to the world of object-oriented programming in C++. In any case, now that you are proficient with objects and classes, read Chapter 9 to learn more about their tricks and subtleties.