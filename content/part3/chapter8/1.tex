Both this chapter and the next present a runnable example of a simple spreadsheet application. A spreadsheet is a two-dimensional grid of “cells,” and each cell contains a number or a string. Professional spreadsheets such as Microsoft Excel provide the ability to perform mathematical operations, such as calculating the sum of the values of a set of cells. The spreadsheet example in these chapters does not attempt to challenge Microsoft in the marketplace, but it is useful for illustrating the issues of classes and objects.

The spreadsheet application uses two basic classes: Spreadsheet and SpreadsheetCell. Each Spreadsheet object contains SpreadsheetCell objects. In addition, a SpreadsheetApplication class manages a collection of Spreadsheets. This chapter focuses on the SpreadsheetCell class. Chapter 9, “Mastering Classes and Objects,” develops the Spreadsheet and SpreadsheetApplication classes.

\begin{myNotic}{NOTE}
This chapter shows several different versions of the SpreadsheetCell class in order to introduce concepts gradually. Thus, the various attempts at the class throughout the chapter do not always illustrate the “best” way to do every aspect of class writing. In particular, the early examples omit important features that would normally be included but have not yet been introduced. You can download the final version of the class as described in the beginning of this chapter.
\end{myNotic}











