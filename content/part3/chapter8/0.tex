\noindent
\textbf{内筒概要}

\begin{itemize}
\item
如何编写类,包括成员函数和数据成员

\item
如何控制对成员函数和数据成员的访问

\item
如何在栈上和堆上使用对象

\item
对象的生命周期是什么

\item
当对象创建或销毁时如何执行代码

\item
如何编写复制或赋值对象的代码
\end{itemize}

本章的所有代码示例都可以在\url{https://github.com/Professional-CPP/edition-6}获得。

作为一门面向对象的语言,C++ 提供了一系列用于使用对象和编写类蓝图(称为类)的功能。当然可以在不使用类和对象的情况下编写 C++ 程序,但是这样做,就没有利用到语言最基本和最有用的方面;编写不使用类的 C++ 程序就像是去巴黎却只在麦当劳吃饭。为了有效地使用类和对象,必须理解它们的语法和能力。

第 1 章回顾了类定义的基本语法,第 5 章介绍了 C++ 中的面向对象编程方法,并提出了具体的类和对象设计策略。这一章描述了使用类和对象的基本概念,包括编写类定义、定义成员函数、在栈上和堆上使用对象、编写构造函数、默认构造函数、编译器生成的构造函数、构造函数初始化器(也称为 ctor-initializer)、复制构造函数、初始化列表构造函数、析构函数和赋值运算符。即使已经对类和对象感到舒适,也应该浏览一下这一章,因为很可能包含了你还不熟悉的信息。
























