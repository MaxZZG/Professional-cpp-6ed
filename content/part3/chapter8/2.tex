
编写一个类时,需要指定将应用于该类对象的行为(或成员函数),以及每个对象将包含的属性(或数据成员)。

编写类的过程包含两个部分:定义类本身和其成员函数。

\mySubsubsection{8.2.1.}{类的定义}

这里是对一个名为spreadsheet\_cell的模块中一个简单的SpreadsheetCell类的进行尝试,每个单元格只能存储一个数字:

\begin{cpp}
export module spreadsheet_cell;

export class SpreadsheetCell
{
    public:
        void setValue(double value);
        double getValue() const;
    private:
        double m_value;
};
\end{cpp}

如第1章所述,第一行指定这是一个名为spreadsheet\_cell的模块的定义。每个类定义都以关键字class开始,后跟类名。如果类在模块中定义,并且必须对导入模块的客户端可见,那么类关键字前缀为export。类定义是一个声明,以分号结束。

类定义通常放在与类同名的文件中。例如,SpreadsheetCell类定义放在名为SpreadsheetCell.cppm的文件中,会一些编译器要求使用特定的扩展名。

\mySamllsection{类成员}

类可以有多个成员。成员可以是成员函数(函数、构造函数或析构函数)、成员变量(也称为数据成员)、成员枚举、类型别名、嵌套类等。

下面函数原型的两行,声明了这个类支持的成员函数:

\begin{cpp}
void setValue(double value);
double getValue() const;
\end{cpp}

第1章指出,将不会改变对象的成员函数声明为const,比如getValue()成员函数。

变量声明的一行声明了这个类的数据成员:

\begin{cpp}
double m_value;
\end{cpp}

一个类定义了将应用于其对象的成员函数和数据成员,这仅适用于类的特定实例,即对象。唯一的例外是静态成员,这在第9章中解释。类定义了概念;对象包含实际成员。所以,每个对象都包含自己的m\_value数据成员的值,对象共享成员函数的实现。类可以包含任意数量的成员函数和数据成员,但不能给数据成员和成员函数起相同的名字。

\mySamllsection{访问控制}

类中的每个成员都受以下三种访问说明符之一:public、private或protected。受保护访问说明符在继承的上下文中在第10章中解释。访问说明符适用于其之后的所有成员声明,直到下一个访问说明符。SpreadsheetCell类中,setValue()和getValue()成员函数为public,而m\_value数据成员问为private。

类的默认访问说明符是private:第一个访问说明符之前的所有成员声明都为private。例如,将public访问说明符移动到setValue()成员函数声明之后,则setValue()成员函数问private,而非public:

\begin{cpp}
export class SpreadsheetCell
{
        void setValue(double value); // now has private access
    public:
        double getValue() const;
    private:
        double m_value;
};
\end{cpp}

在C++中,struct可以有与class相同的成员函数。只有一个区别:对于struct,默认访问说明符是public,而class是private。

例如,SpreadsheetCell类可以改写为使用struct:

\begin{cpp}
export struct SpreadsheetCell
{
    void setValue(double value);
        double getValue() const;
    private:
        double m_value;
};
\end{cpp}

然而,这样做是不合常规的。struct通常用于,需要一组公开可访问的数据成员(没有成员函数)。以下是一个简单的struct示例,用于存储2D点坐标:

\begin{cpp}
export struct Point
{
    double x;
    double y;
};
\end{cpp}

\mySamllsection{声明的顺序}

可以按任何顺序声明成员和访问控制说明符:C++不强制任何限制,比如成员函数在前,数据成员在后,或者public在前,private在后。此外,可以重复访问说明符。例如,SpreadsheetCell定义可以如下所示:

\begin{cpp}
export class SpreadsheetCell
{
    public:
        void setValue(double value);
    private:
        double m_value;
    public:
        double getValue() const;
};
\end{cpp}

然而,为了清晰起见,最好根据访问说明符对声明进行分组,并在那些声明内对成员函数和数据成员进行分组。

\mySamllsection{类内成员初始化器}

数据成员可以直接在类定义中进行初始化。例如,SpreadsheetCell类可以直接在类定义中将m\_value默认初始化为0:

\begin{cpp}
export class SpreadsheetCell
{
    // Remainder of the class definition omitted for brevity
    private:
        double m_value { 0 };
};
\end{cpp}

\begin{myNotic}{NOTE}
建议始终初始化类的数据成员。
\end{myNotic}

\mySubsubsection{8.2.2.}{定义成员函数}

前面的SpreadsheetCell类定义足以创建该类的对象。然而,如果尝试调用setValue()或getValue()成员函数,链接器会抱怨成员函数尚未定义。这是因为这些成员函数目前只有原型,而没有实现。通常,类定义放在模块接口文件中。对于成员函数的定义,可以选择将其放在模块接口文件中或模块实现文件中。

以下是在类内实现成员函数的SpreadsheetCell类:

\begin{cpp}
export module spreadsheet_cell;

export class SpreadsheetCell
{
    public:
        void setValue(double value) { m_value = value; }
        double getValue() const { return m_value; }
    private:
        double m_value { 0 };
};
\end{cpp}

与头文件不同,在C++模块中,将成员函数定义放在模块接口文件中并无害处。这在第11章中有更详细的讨论。然而,为了保持模块接口文件的清洁和不含任何实现细节,本书通常将成员函数定义放在模块实现文件中。

模块实现文件的第一行指定实现属于哪个模块。以下是spreadsheet\_cell模块,SpreadsheetCell类中两个成员函数的定义:

\begin{cpp}
module spreadsheet_cell;

void SpreadsheetCell::setValue(double value)
{
    m_value = value;
}

double SpreadsheetCell::getValue() const
{
    return m_value;
}

\end{cpp}

注意,每个成员函数名前都跟着类名和两个冒号:

\begin{cpp}
void SpreadsheetCell::setValue(double value)
\end{cpp}

双冒号称为作用域解析运算符。在这个上下文中,这个语法告诉编译器即将到来的setValue()成员函数的定义属于SpreadsheetCell类。还要注意,定义成员函数时,不需要访问说明符。

\mySamllsection{访问数据成员}

类的非静态成员函数,如setValue()和getValue(),总是代表该类的特定对象执行。成员函数体内,可以访问该对象的所有数据成员。前面setValue()的定义中,修改了对象内的m\_value变量:

\begin{cpp}
m_value = value;
\end{cpp}

如果为两个不同的对象调用setValue(),相同的代码行(每个对象执行一次)会更改两个不同对象中的变量。

\mySamllsection{调用其他成员函数}

可以从一个成员函数内部调用类的其他成员函数。例如,考虑对SpreadsheetCell类进行扩展,将单元格的值设置为字符串或数字。尝试将字符串设置为单元格的值时,单元格会尝试将字符串转换为数字。如果字符串不代表有效的数字,则忽略单元格值。这个程序中,不是数字的字符串将生成单元格值为0。以下是对SpreadsheetCell类定义的首次尝试:

\begin{cpp}
export module spreadsheet_cell;
import std;
export class SpreadsheetCell
{
    public:
        void setValue(double value);
        double getValue() const;

        void setString(std::string_view value);
        std::string getString() const;
    private:
        std::string doubleToString(double value) const;
        double stringToDouble(std::string_view value) const;
        double m_value { 0 };
};
\end{cpp}

这个版本的类只将数据存储为double。如果客户端将数据设置为字符串,则会将其转换为double。如果文本不是有效的数字,则double值设置为0。类定义展示了两个新的成员函数,用于设置和检索单元格的文本表示,以及两个新的私有辅助成员函数,用于将double转换为字符串以及反之。以下是所有成员函数的实现:

\begin{cpp}
module spreadsheet_cell;
import std;
using namespace std;

void SpreadsheetCell::setValue(double value)
{
    m_value = value;
}

double SpreadsheetCell::getValue() const
{
    return m_value;
}

void SpreadsheetCell::setString(string_view value)
{
    m_value = stringToDouble(value);
}

string SpreadsheetCell::getString() const
{
    return doubleToString(m_value);
}

string SpreadsheetCell::doubleToString(double value) const
{
    return to_string(value);
}

double SpreadsheetCell::stringToDouble(string_view value) const
{
    double number { 0 };
    from_chars(value.data(), value.data() + value.size(), number);
    return number;
}
\end{cpp}

std::to\_string()和from\_chars()函数在第2章中进行了解释。

使用此doubleToString()成员函数的实现,例如6.1的值转换为6.100000。因为它是一个私有的辅助成员函数,可以自由修改实现,而无需修改任何客户端(调用)代码。

\mySubsubsection{8.2.3.}{使用对象}

前面的类定义说明,SpreadsheetCell包含一个数据成员、四个公共成员函数和两个私有成员函数。类定义实际上并没有创建任何SpreadsheetCell对象,只是指定了其外貌和行为。类就像建筑蓝图,蓝图指定房子应该是什么样子,但是绘制蓝图的行为并不会建造房子,而房子必须根据制定的蓝图来建造。

C++中可以从SpreadsheetCell类定义构造一个SpreadsheetCell“对象”,通过声明一个类型为SpreadsheetCell的变量。就像一个建筑商可以根据一套给定的蓝图建造多套房子一样,开发者也可以使用一个SpreadsheetCell类,创建多个SpreadsheetCell对象。有两种创建和使用对象的方式:栈和堆。

\mySamllsection{栈上对象}

以下是一些在栈上创建和使用SpreadsheetCell对象的代码:

\begin{cpp}
SpreadsheetCell myCell, anotherCell;
myCell.setValue(6);
anotherCell.setString("3.2");
println("cell 1: {}", myCell.getValue());
println("cell 2: {}", anotherCell.getValue());
\end{cpp}

创建对象的方式和声明简单变量的方式一样,只不过变量的类型是类名。像myCell.setValue(6);中的点称为“点”运算符,也称为成员访问运算符;允许调用对象的公共成员函数。如果对象中有任何公共数据成员,也可以用点运算符访问。记住,不推荐使用公共数据成员。

程序的输出如下:

\begin{shell}
cell 1: 6
cell 2: 3.2
\end{shell}

\mySamllsection{堆上对象}

也可以通过使用new来动态分配对象:

\begin{cpp}
SpreadsheetCell* myCellp { new SpreadsheetCell { } };
myCellp->setValue(3.7);
println("cell 1: {} {}", myCellp->getValue(), myCellp->getString());
delete myCellp;
myCellp = nullptr;
\end{cpp}

当创建一个堆区对象时,可通过“箭头”运算符:->来访问它的成员。箭头运算符结合了解引用(*)和成员访问(.)。也可以使用这两个运算符,但这在风格上会显得笨拙:

\begin{cpp}
SpreadsheetCell* myCellp { new SpreadsheetCell { } };
(*myCellp).setValue(3.7);
println("cell 1: {} {}", (*myCellp).getValue(), (*myCellp).getString());
delete myCellp;
myCellp = nullptr;
\end{cpp}

必须释放在堆区上分配的内存,通过调用delete来释放在堆区上为对象分配的内存。为了确保安全并避免内存问题,应该使用智能指针,如下例所示:

\begin{cpp}
auto myCellp { make_unique<SpreadsheetCell>() };
// Equivalent to:
// unique_ptr<SpreadsheetCell> myCellp { new SpreadsheetCell { } };
myCellp->setValue(3.7);
println("cell 1: {} {}", myCellp->getValue(), myCellp->getString());
\end{cpp}

使用智能指针,不需要手动释放内存。

\begin{myWarning}{WARNING}
使用new分配一个对象时,在完成使用后,要么用delete释放,要么使用智能指针自动管理内存!
\end{myWarning}

\begin{myNotic}{NOTE}
如果不使用智能指针,在删除指向的对象后,将指针重置为nullptr是一个好主意。虽然这不是必需的,但这将使调试更容易。
\end{myNotic}

\mySubsubsection{8.2.4.}{this指针}

每个正常的成员函数调用都会隐式传递一个指向调用对象的指针,作为名为this的“隐藏”参数。可以使用this指针来访问数据成员或调用成员函数,也可以将其传递给其他成员函数或函数。有时它也有助于消除名称歧义。例如,可以定义SpreadsheetCell类使用value数据成员而不是m\_value。这时,setValue()看起来如下:

\begin{cpp}
void SpreadsheetCell::setValue(double value)
{
    value = value; // Confusing!
}
\end{cpp}

这行代码令人困惑。指的是作为参数传递的value,还是对象成员的value?

\begin{myNotic}{NOTE}
使用某些编译器或编译器设置,上述混淆的行可能会编译,而没有任何警告或错误,但不会产生预期的结果。
\end{myNotic}

为了消除名称歧义,你可以使用this指针:

\begin{cpp}
void SpreadsheetCell::setValue(double value)
{
    this->value = value;
}
\end{cpp}

然而,如果使用第3章中描述的命名约定,就永远不会遇到这种名称冲突。

还可以使用this指针从对象的成员函数中调用一个接受对象指针作为参数的函数。例如,编写了一个独立的printCell()函数(不是成员函数):

\begin{cpp}
void printCell(const SpreadsheetCell& cell)
{
    println("{}", cell.getString());
}
\end{cpp}

如果想从setValue()成员函数中调用printCell(),必须传递*this作为参数,以便printCell()引用setValue()操作的SpreadsheetCell:

\begin{cpp}
void SpreadsheetCell::setValue(double value)
{
    this->value = value;
    printCell(*this);
}
\end{cpp}

\begin{myNotic}{NOTE}
与其编写printCell()函数,不如编写自定义格式化器,如第2章所述。然后,可以使用以下行来打印名为myCell的SpreadsheetCell:

\begin{cpp}
std::println("{}", myCell);
\end{cpp}

或者,可以重载<{}<运算符。然后,可以编写如下内容:

\begin{cpp}
cout << myCell << endl;
\end{cpp}
\end{myNotic}

\mySubsubsection{8.2.5.}{明确的对象参数}

\CXXTwentythreeLogo{-40}{15}

从C++23开始,可以使用显式的对象参数,而不是依赖于编译器提供隐式的this参数。通常,这个显式的对象参数是一个引用类型。以下代码段使用显式对象参数,实现了前面部分的SpreadsheetCell的setValue()成员函数:

\begin{cpp}
void SpreadsheetCell::setValue(this SpreadsheetCell& self, double value)
{
    self.m_value = value;
    printCell(self);
}
\end{cpp}

现在,setValue()的第一个参数是显式的对象参数,通常称为self,但也可以使用其他名称。self的类型前缀为this关键字。这个显式的对象参数必须是成员函数的第一个参数。使用显式的对象参数,函数就不再有隐式定义的this;因此,在setValue()的函数体内,必须明确使用self来访问SpreadsheetCell。

调用使用显式对象参数的成员函数与调用使用隐式this参数的成员函数没有区别。现在setValue()指定了两个参数,self和value,但仍然可以通过传递一个参数(想要设置的值)来调用:

\begin{cpp}
SpreadsheetCell myCell;
myCell.setValue(6);
\end{cpp}

如本节所示,使用显式对象参数,会使代码更加冗长。然而,在以下情况下它们很有用:

\begin{itemize}
\item
提供更明确的语法来编写返回类型引用成员函数。

\item
对于成员函数模板,其中显式对象参数的类型是模板类型参数。这在实现成员函数的const和非const重载时避免代码重复,如第12章所述。

\item
用于编写第19章中解释的递归Lambda表达式。
\end{itemize}










