已经熟练地处理了异常,现在是讨论异常安全保证的时候了。可以为编写代码提供几种级别的保证,以便代码的用户了解处理异常的情况。以下是可以为函数指定的异常安全保证:

\begin{itemize}
\item
nothrow(或nofail)异常保证:该函数永远不会抛出异常。

\item
强异常保证:如果抛出异常,所有涉及的实例都会回滚到函数调用之前的状态。提供这种保证的代码的例子是在第9章介绍的复制和交换模式。

\item
基本异常保证:如果抛出异常,所有涉及的实例都保持有效状态,没有资源泄漏。实例的状态可能与函数调用之前的状态不同。

\item
无保证:当抛出异常时,应用程序可能处于任何无效状态,资源可能泄漏,内存可能破坏等。
\end{itemize}

\begin{myNotic}{NOTE}
如果函数可以抛出异常,应该提供基本的异常保证。
\end{myNotic}