
当一段代码抛出异常时,就会在栈上搜索捕获处理程序,这个捕获处理程序可能是执行栈上零个或多个函数调用。找到时,栈将回滚到定义捕获处理程序的栈级别,通过展开所有中间栈帧来实现。栈展开意味着调用所有局部作用域变量的析构函数,并跳过每个函数中当前执行点之后的所有代码。

栈展开期间,指针变量显然不会释放,也不会执行其他清理工作。这种行为可能会导致问题。例如,以下代码有内存泄漏:

\begin{cpp}
void funcOne();
void funcTwo();

int main()
{
    try {
        funcOne();
    } catch (const exception& e) {
        println(cerr, "Exception caught!");
        return 1;
    }
}

void funcOne()
{
    string str1;
    string* str2 { new string {} };
    funcTwo();
    delete str2;
}

void funcTwo()
{
    ifstream fileStream;
    fileStream.open("filename");
    throw exception {};
    fileStream.close();
}
\end{cpp}

当funcTwo()抛出异常时,最近的异常处理程序在main()中。然后控制从funcTwo()中的这一行:

\begin{cpp}
throw exception {};
\end{cpp}

跳转到main()中的这一行:

\begin{cpp}
println(cerr, "Exception caught!");
\end{cpp}

在funcTwo()中,控制留在抛出异常的那一行,所以后续的这行代码永远不会运行:

\begin{cpp}
fileStream.close();
\end{cpp}

然而,幸运的是,ifstream析构函数会调用,因为fileStream是栈上的局部变量。ifstream析构函数关闭文件,所以这里没有资源泄漏。如果动态分配了fileStream,它就不会销毁,文件也不会关闭。

在funcOne()中,控制位于对funcTwo()的调用,所以后续的这行代码也永远不会运行:

\begin{cpp}
delete str2;
\end{cpp}

这种情况下,真的存在内存泄漏。栈展开不会自动为你调用str2的delete。另一方面,str1可正确销毁,因为它是一个栈上的局部变量,所以栈展开正确地销毁了所有局部变量。

\begin{myWarning}{WARNING}
粗心的异常处理可能导致内存和资源的泄漏。
\end{myWarning}

这就是为什么不应该将旧的C语言模型中的分配(即使你调用new,看起来像C++)与现代编程方法(如异常)混合使用的原因之一。在C++中,这种情况应该通过使用基于栈的分配来处理,如果不可,则使用接下来两节中讨论的方式。

\mySubsubsection{14.5.1.}{使用智能指针}

如果基于栈的分配不可能,那么使用智能指针,其允许编写在异常处理期间自动防止内存或资源泄漏的代码。每当销毁智能指针对象时,都会释放底层资源。这里是一个使用<memory>中定义的unique\_ptr智能指针的funcOne()实现,该指针在第7章中引入:

\begin{cpp}
void funcOne()
{
    string str1;
    auto str2 { make_unique<string>("hello") };
    funcTwo();
}
\end{cpp}

当从funcOne()返回或抛出异常时,str2指针将自动删除。

当然,只有在有充分的理由时,才进行动态分配。例如,在funcOne()中,没有充分的理由让str2成为一个动态分配的字符串,应该只是一个基于栈的字符串变量。这里只是作为一个示例,来展示抛出异常的后果。

\begin{myNotic}{NOTE}
使用智能指针或其他资源获取即初始化(RAII)对象时,永远不需要记住释放底层资源:RAII对象的析构函数会为你完成这项工作,无论是正常离开函数,还是通过异常。这是一种在第32章中才会介绍的技术。
\end{myNotic}

\mySubsubsection{14.5.2.}{捕获、清理和重新抛出}

避免内存和资源泄漏的另一种方式是每个函数捕获可能的异常,执行必要的清理工作,然后为栈上更高层的函数重新抛出异常。这里是使用这种方法的funcOne():

\begin{cpp}
void funcOne()
{
    string str1;
    string* str2 { new string {} };
    try {
        funcTwo();
    } catch (...) {
        delete str2;
        throw; // Rethrow the exception.
    }
    delete str2;
}
\end{cpp}

这个函数用一个异常处理程序包装了对funcTwo()的调用,该处理程序执行清理工作(str2上调用delete)然后重新抛出异常,关键字throw会重新抛出最近捕获的异常。注意,catch语句使用了…语法来捕获所有异常。

这种方法虽然可行,但会使代码变得混乱且容易出错。特别是要注意,现在有两行完全相同的代码调用了delete str2:一行是在处理异常时,另一行是在函数正常退出时。

\begin{myWarning}{WARNING}
首选的解决方案是使用基于栈的分配,如果不行,则使用智能指针或其他RAII类,而不是捕获、清理和重新抛出技术。
\end{myWarning}















