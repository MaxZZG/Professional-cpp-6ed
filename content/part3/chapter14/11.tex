通过解决下面的练习,可以练习本章讨论的内容。所有练习的解决方案都可以在本书的网站\url{www.wiley.com/go/proc++6e}下载到源码。然而,若在练习中卡住了,在从网站上寻找解决方案之前,可以考虑先重读本章的部分内容,试着自己找到答案。

\begin{itemize}
\item
\textbf{练习14-1}:在不编译和执行的情况下,找出并纠正以下代码中的错误:

\begin{cpp}
// Throws a logic_error exception if the number of elements
// in the given dataset is not even.
void verifyDataSize(const vector<int>& data)
{
    if (data.size() % 2 != 0)
        throw logic_error { "Number of data points must be even." };
}

// Throws a logic_error exception if the number of elements
// in the given dataset is not even.
// Throws a domain_error exception if one of the datapoints is negative.
void processData(const vector<int>& data)
{
    // Verify the size of the given dataset.
    try {
        verifyDataSize(data);
    } catch (const logic_error& caughtException) {
        // Write message on standard output.
        println(cerr, "Invalid number of data points in dataset. Aborting.");
        // And rethrow the exception.
        throw caughtException;
    }
    // Verify for negative datapoints.
    for (auto& value : data) {
        if (value < 0)
            throw domain_error { "Negative datapoints not allowed." };
    }
    // Process data ...
}

int main()
{
    try {
        vector data { 1, 2, 3, -5, 6, 9 };
        processData(data);
    } catch (const logic_error& caughtException) {
        println(cerr, "logic_error: {}", caughtException.what());
    } catch (const domain_error& caughtException) {
        println(cerr, "domain_error: {}", caughtException.what());
    }
}
\end{cpp}

\item
\textbf{练习14-2}:使用第13章中的双向I/O示例代码。可以在这个下载源代码存档的Ch13\verb|\|22\_Bidirectional文件夹中找到这个示例。这个示例实现了一个changeNumberForID()函数,将代码修改为在所有适当的地方使用异常。当代码开始使用异常,是否可以对changeNumberForID()函数声明进行修改?

\item
\textbf{练习14-3}:为练习13-3中的人员数据库解决方案添加适当的错误处理,使用异常。

\item
\textbf{练习14-4}:查看第9章中的Spreadsheet示例代码,该代码支持使用swap()的移动语义。可以在这个下载源代码存档的Ch09\verb|\|07\_SpreadsheetMoveSemantics–WithSwap文件夹中找到完整的代码。为代码添加适当的错误处理,包括处理内存分配失败。在类中添加最大宽度和高度,并包括适当的验证检查。编写自己的异常类InvalidCoordinate,存储无效的坐标和允许的坐标范围,并在verifyCoordinate()成员函数中使用它。在main()中编写几个测试来测试各种错误条件。
\end{itemize}














