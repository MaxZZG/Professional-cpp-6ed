\noindent
\textbf{WHAT’S IN THIS CHAPTER?}

\begin{itemize}
\item
How to handle errors in C++, including pros and cons of exceptions

\item
The syntax of exceptions

\item
Exception class hierarchies and handling exceptions polymorphically

\item
How to rethrow caught exceptions

\item
Stack unwinding and cleanup

\item
How to embed, inside a custom exception, the exact source code location where an exception occurred

\item
How to embed the entire stack trace, also known as call stack, in a custom exception

\item
Common error-handling situations
\end{itemize}

\noindent
\textbf{WILEY.COM DOWNLOADS FOR THIS CHAPTER}

Please note that all the code examples for this chapter are available as part of this chapter’s code download on the book’s website at \url{www.wiley.com/go/proc++6e} on the Download Code tab.

Inevitably, your C++ programs will encounter errors during execution. The program might be unable to open a file, the network connection might go down, or the user might enter an incorrect value, to name a few possibilities. The C++ language provides a feature called exceptions to handle these exceptional but not unexpected situations.

Most code examples in this book so far have generally ignored error conditions for brevity. This chapter rectifies that simplification by teaching you how to incorporate error handling into your programs from their beginnings. It focuses on C++ exceptions, including the details of their syntax, and describes how to employ them effectively to create well-designed error-handling programs.

The chapter also discusses how you can write your own exception classes. This includes discussing how to automatically embed both, the exact location in your source code where an exception occurred, as well as the full stack trace at the moment the exception was raised. Both of these help tremendously in diagnosing any errors.











