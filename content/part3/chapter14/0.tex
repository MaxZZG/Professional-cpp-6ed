\noindent
\textbf{内筒概要}

\begin{itemize}
\item
处理C++中的错误,包括异常的优缺点

\item
异常的语法

\item
异常类的层次结构和多态异常处理

\item
重新抛出捕获的异常

\item
堆栈展开,以及清理

\item
自定义异常中嵌入异常发生的确切源代码位置

\item
自定义异常中嵌入堆栈跟踪(也称为调用堆栈)

\item
常见的错误处理
\end{itemize}

本章的所有代码示例都可以在\url{https://github.com/Professional-CPP/edition-6}获得。

C++程序在执行过程中不可避免地会遇到错误。程序可能无法打开文件,网络连接可能中断,或者用户可能输入了错误的值等。C++语言提供了一个称为异常处理的功能来处理这些特殊,但并不意外的情况。

目前为止,本书中的大多数代码示例通常为了简洁而忽略了错误条件。本章去掉了这个简化,并在程序开始时如何将错误处理融入其中,专注于C++异常,包括语法细节。并描述了如何有效地使用它们,来创建设计良好的错误处理程序。

本章还讨论了如何编写自己的异常类。这包括讨论如何自动嵌入异常发生时在源代码中的确切位置,以及异常抛出时记录完整堆栈跟踪。这两者对于诊断错误都有很大的帮助。












