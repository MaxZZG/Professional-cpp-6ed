使用throw关键字还可以重新抛出当前的异常,而不进行复制,如下例所示:

\begin{cpp}
void g() { throw invalid_argument { "Some exception" }; }

void f()
{
    try {
        g();
    } catch (const invalid_argument& e) {
        println("caught in f(): {}", e.what());
        throw; // rethrow
    }
}

int main()
{
    try {
        f();
    } catch (const invalid_argument& e) {
        println("caught in main(): {}", e.what());
    }
}
\end{cpp}

此示例产生以下输出:

\begin{shell}
caught in f(): Some exception
caught in main(): Some exception
\end{shell}

如果使用throw e;来重新抛出捕获的异常e,是不行的,因为这可能导致异常对象切片。例如,将f()修改为捕获std::exception,并将main()修改为捕获exception和invalid\_argument异常:

\begin{cpp}
void g() { throw invalid_argument { "Some exception" }; }
void f()
{
    try {
        g();
        } catch (const exception& e) {
        println("caught in f(): {}", e.what());
        throw; // rethrow
        }
    }

int main()
{
    try {
        f();
    } catch (const invalid_argument& e) {
        println("invalid_argument caught in main(): {}", e.what());
    } catch (const exception& e) {
        println("exception caught in main(): {}", e.what());
    }
}
\end{cpp}

请记住,invalid\_argument派生自exception,因此必须首先捕获。此代码的输出和期望一致:

\begin{shell}
caught in f(): Some exception
invalid_argument caught in main(): Some exception
\end{shell}

现在,尝试将f()中的throw;语句替换为throw e;。然后输出变为:

\begin{shell}
caught in f(): Some exception
exception caught in main(): Some exception
\end{shell}

main()似乎捕获了一个异常对象,而不是一个invalid\_argument对象。这是因为throw e;语句会进行切片,将invalid\_argument退化为exception。

\begin{myWarning}{WARNING}
使用throw;来重新抛出异常。永远不要使用throw e;来重新抛出捕获的异常!
\end{myWarning}














