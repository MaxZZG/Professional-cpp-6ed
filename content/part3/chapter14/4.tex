The throw keyword can also be used to rethrow the current exception without copying it, as in the following example:

\begin{cpp}
void g() { throw invalid_argument { "Some exception" }; }

void f()
{
    try {
        g();
    } catch (const invalid_argument& e) {
        println("caught in f(): {}", e.what());
        throw; // rethrow
    }
}

int main()
{
    try {
        f();
    } catch (const invalid_argument& e) {
        println("caught in main(): {}", e.what());
    }
}
\end{cpp}

This example produces the following output:

\begin{shell}
caught in f(): Some exception
caught in main(): Some exception
\end{shell}

You might think you could rethrow a caught exception e with throw e;. However, that’s wrong, because it can cause slicing of your exception object. For example, suppose f() is modified to catch std::exceptions, and main() is modified to catch both exception and invalid\_argument exceptions:

\begin{cpp}
void g() { throw invalid_argument { "Some exception" }; }
void f()
{
    try {
        g();
        } catch (const exception& e) {
        println("caught in f(): {}", e.what());
        throw; // rethrow
        }
    }

int main()
{
    try {
        f();
    } catch (const invalid_argument& e) {
        println("invalid_argument caught in main(): {}", e.what());
    } catch (const exception& e) {
        println("exception caught in main(): {}", e.what());
    }
}
\end{cpp}

Remember that invalid\_argument derives from exception, hence it must be caught first. The output of this code is as you would expect, shown here:

\begin{shell}
caught in f(): Some exception
invalid_argument caught in main(): Some exception
\end{shell}

Now, try replacing the throw; statement in f() with throw e;. The output then becomes as follows:

\begin{shell}
caught in f(): Some exception
exception caught in main(): Some exception
\end{shell}

main() seems to be catching an exception object, instead of an invalid\_argument object. That’s because the throw e; statement causes slicing, reducing the invalid\_argument to an exception.

\begin{myWarning}{WARNING}
Always use throw; to rethrow an exception. Never do something like throw e; to rethrow a caught exception e!
\end{myWarning}














