
无约束算法的“魔力”在于在迭代器上工作,而不是直接在容器本身。这样,就与特定的容器实现无关。所有标准库算法都作为函数模板实现,其中模板类型参数通常是迭代器类型,迭代器作为函数的参数指定。函数模板可以从函数参数中推导出模板类型,所以可以像调用普通一样使用这些算法。

大多数算法需要一个源序列来应用算法。对于无约束算法,源序列指定为一个迭代器对,即开始和结束迭代器,称为共同范围。如第17章所述,对于大多数容器,共同范围是半开放的,即它们包括范围内的第一个元素,但不包括最后一个。结束迭代器实际上是“超出末尾”的标记。

算法对给它们传递的迭代器有一定的要求。例如,copy\_backward(),从最后一个元素开始将元素从一个序列复制到另一个序列,是一个需要双向迭代器的算法的例子。同样,stable\_sort()就地排序元素,同时保持重复元素的原顺序,是一个需要随机访问迭代器的例子。forward\_list是一个只支持前向迭代器的容器的例子,没有双向或随机访问迭代器。因此,copy\_backward()和stable\_sort()不能在forward\_list上工作。

大多数算法在<algorithm>中定义,一些数值算法在<numeric>中定义,都在std命名空间中。

大多数算法是constexpr的,可以用于constexpr函数的实现。可以查阅标准库手册(见附录B),以了解哪些算法是constexpr。

理解算法的最佳方式是查看一些示例。了解几个算法的使用方式后,其他算法就很容易掌握了。本节详细描述了find()、find\_if()和accumulate()算法。后续讨论了每个算法类具有代表性的示例。

\mySubsubsection{20.1.1.}{find和find\_if算法}

find()在共同范围内查找特定元素,可以在容器类型中的元素上使用。其返回指向找到的元素的迭代器,或者在未找到元素时返回范围的结束迭代器。注意,find()调用中指定的范围不必是容器中所有元素的完整范围,可能是一个子集。

\begin{myWarning}{WARNING}
如果find()未能找到元素,会返回一个等于函数调用范围中的结束迭代器,而不是底层容器的结束迭代器。
\end{myWarning}

了解find()之前,让我们定义一个函数模板,用于用整数填充容器。这个函数模板在整个章节中都会使用。它是一个参数化在容器类型的函数模板,并且有一个约束,强制给定的容器类型支持push\_back(int)。

\begin{cpp}
template <typename Container>
    requires requires(Container& c, int i) { c.push_back(i); }
void populateContainer(Container& cont)
{
    while (true) {
        print("Enter a number (0 to stop): ");
        int value;
        cin >> value;
        if (value == 0) { break; }
        cont.push_back(value);
    }
}
\end{cpp}

看看如何使用std::find()。这个例子和populateContainer()函数模板假设用户输入有效数字,没有对用户输入进行错误检查。对流输入进行错误检查在第13章中已经介绍过了。

\begin{cpp}
vector<int> myVector;
populateContainer(myVector);

while (true) {
    print("Enter a number to lookup (0 to stop): ");
    int number;
    cin >> number;
    if (number == 0) { break; }
    auto endIt { cend(myVector) };
    auto it { find(cbegin(myVector), endIt, number) };
    if (it == endIt) {
        println("Could not find {}", number);
    } else {
        println("Found {}", *it);
    }
}
\end{cpp}

要搜索vector的所有元素,调用find(),其中cbegin(myVector)和endIt作为迭代器参数,其中endIt定义为cend(myVector)。如果想在子范围内搜索,可以改变这两个迭代器。

以下是程序的示例运行:

\begin{shell}
Enter a number (0 to stop): 3
Enter a number (0 to stop): 4
Enter a number (0 to stop): 5
Enter a number (0 to stop): 6
Enter a number (0 to stop): 0
Enter a number to lookup (0 to stop): 5
Found 5
Enter a number to lookup (0 to stop): 8
Could not find 8
Enter a number to lookup (0 to stop): 0
\end{shell}

使用if语句的初始化器,find()的调用和检查结果可以一次完成:

\begin{cpp}
if (auto it { find(cbegin(myVector), endIt, number) }; it == endIt) {
    println("Could not find {}", number);
} else {
    println("Found {}", *it);
}
\end{cpp}

一些容器,如map和set,提供了自己的find()作为类成员函数。

\begin{myWarning}{WARNING}
如果一个容器提供了与泛型算法相同功能的成员函数,应该使用成员函数,因为它更快。例如,泛型find()算法在map上运行的时间复杂度为线性,而map上的find()成员函数运行的时间复杂度为对数。
\end{myWarning}

find\_if()与find()类似,不同之处在于它接受一个返回真或假的谓词函数,而不是一个简单的要匹配的元素。find\_if()算法对范围内的每个元素调用谓词,直到谓词返回true,这时find\_if()返回一个指向该元素的迭代器。以下程序从用户那里读取测试分数,然后检查是否有“完美”的分数,完美分数是100分或以上的分数。程序与前面的示例类似:

\begin{cpp}
bool perfectScore(int num) { return num >= 100; }
int main()
{
    vector<int> myVector;
    populateContainer(myVector);

    auto endIt { cend(myVector) };
    auto it { find_if(cbegin(myVector), endIt, perfectScore) };
    if (it == endIt) {
        println("No perfect scores");
    } else {
        println("Found a \"perfect\" score of {}", *it);
    }
}
\end{cpp}

这个程序将perfectScore()函数的指针传递给find\_if(),算法然后对每个元素调用它,直到它返回true。

\CXXTwentythreeLogo{-40}{-50}

从 C++23 开始,如果不需要访问函数对象类中的非静态数据成员和成员函数,可以将函数调用运算符标记为static。因此,perfectScore() 函数可以重写为一个名为 PerfectScore 的函数对象,其函数调用运算符可标记为static:

\begin{cpp}
class PerfectScore
{
    public:
        static bool operator()(int num) { return num >= 100; }
};
\end{cpp}

find\_if()的调用需要相应地修改:

\begin{cpp}
auto it { find_if(cbegin(myVector), endIt, &PerfectScore::operator()) };
\end{cpp}

使用Lambda表达式。

\begin{cpp}
auto it { find_if(cbegin(myVector), endIt, [](int i){ return i >= 100; }) };
\end{cpp}


\mySubsubsection{20.1.2.}{accumulate 算法}

计算容器中元素的总和,或者某些其他算术量。accumulate() 函数——定义在 <numeric> 中,而不是在 <algorithm> 中——正是为此目的而设计的。最基本的形式中,它计算指定范围内的元素的总和。以下函数计算一组整数的算术平均值,算术平均值就是所有元素之和除以元素的数量:

\begin{cpp}
double arithmeticMean(span<const int> values)
{
    double sum { accumulate(cbegin(values), cend(values), 0.0) };
    return sum / values.size();
}
\end{cpp}

accumulate() 算法将总和的初始值作为第三个参数,这个例子中应该是 0.0(加法的元单位)来开始一个新的总和。

accumulate() 的第二个重载允许调用者指定一个操作来执行,而不是默认的加法。这个操作以二进制回调的形式出现。假设想要计算几何平均值,它是序列中所有数字的乘积的逆数次方。这时,希望使用 accumulate() 来计算乘积,而不是加和:

\begin{cpp}
int product(int value1, int value2) { return value1 * value2; }

double geometricMean(span<const int> values)
{
    int mult { accumulate(cbegin(values), cend(values), 1, product) };
    return pow(mult, 1.0 / values.size()); // pow() is defined in <cmath>
}
\end{cpp}

product() 函数作为回调传递给 accumulate(),并且总和的初始值是 1(乘法的元单位)。

也可以使用 Lambda 表达式:

\begin{cpp}
double geometricMeanLambda(span<const int> values)
{
    int mult { accumulate(cbegin(values), cend(values), 1,
        [](int value1, int value2) { return value1 * value2; }) };
    return pow(mult, 1.0 / values.size());
}
\end{cpp}

还可以使用在第 19 章中讨论的透明 multiplies<> 函数对象来实现 geometricMean() 函数:

\begin{cpp}
double geometricMeanFunctor(span<const int> values)
{
    int mult { accumulate(cbegin(values), cend(values), 1, multiplies<>{}) };
    return pow(mult, 1.0 / values.size());
}
\end{cpp}

\mySubsubsection{20.1.3.}{算法的移动语义}

与标准库容器一样,标准库算法也在适当时候会使用移动语义。可以移动对象,而不是执行复制操作。这可以加速某些算法,例如后面章节中详细讨论的 remove()。强烈建议在想要存储在容器中的自定义元素类中实现移动语义,通过移动构造函数和移动赋值运算符,可以为类添加移动语义。正如第 18 章所讨论的,两者都应该标记为 noexcept,否则标准库容器和算法不会使用它们。有关如何向类添加移动语义的详细信息,请参阅第 9 章。

\mySubsubsection{20.1.4.}{算法的回调}

\begin{myWarning}{WARNING}
算法允许为给定的回调(如函数对象和 Lambda 表达式)创建多个副本,并为不同的元素调用不同的副本。
\end{myWarning}

由于可以为回调创建多个副本,限制了这些回调的副作用。基本上,回调必须是状态无关的。对于函数对象,函数调用运算符必须是 const;因此,不能编写依赖于对象内部状态在调用之间保持一致的函数对象。对于 Lambda 表达式,不能标记为 mutable。

generate() 和 generate\_n() 算法算是例外,可以接受有状态的回调,但即使这些,也会创建回调的一个副本。此外,在算法完成后不返回那个副本,无法访问对状态所做的更改。唯一例外的是 for\_each(),将给定的谓词复制一次到 for\_each() 算法中,并在完成后返回那个副本,可以通过这个返回值访问改变的状态。

为了防止算法复制回调,可以使用 std::ref() 辅助函数将回调引用传递给算法,这样确保算法总是使用相同的回调。以下代码片段基于本章早些时候的一个示例,但使用了存储在变量 isPerfectScore 中的 Lambda 表达式。Lambda 表达式计数器记录了调用的次数,并将那个值写入标准输出。isPerfectScore 会传递给 find\_if() 算法,但未使用 ref()。代码的最后一条语句明确调用 isPerfectScore 一次。

\begin{cpp}
auto isPerfectScore { [tally = 0] (int i) mutable {
        println("{}", ++tally); return i >= 100; } };

auto endIt { cend(myVector) };
auto it { find_if(cbegin(myVector), endIt, isPerfectScore) };
if (it != endIt) { println("Found a \"perfect\" score of {}", *it); }
println("---");
isPerfectScore(1);
\end{cpp}

输出可能如下所示:

\begin{shell}
Enter a number (0 to stop): 11
Enter a number (0 to stop): 22
Enter a number (0 to stop): 33
Enter a number (0 to stop): 0
1
2
3
---
1
\end{shell}

输出显示 find\_if() 算法三次调用 isPerfectScore 产生输出 1, 2, 3。最后一行显示了显式调用 isPerfectScore 发生在不同的 isPerfectScore 实例上(从 1 开始)。

现在,更改对 find\_if() 的调用:

\begin{cpp}
auto it { find_if(cbegin(myVector), endIt, ref(isPerfectScore)) };
\end{cpp}

现在的输出是 1, 2, 3, 4,没有对 isPerfectScore 进行复制。





