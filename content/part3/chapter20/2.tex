
第 16 章列出了所有可用的标准库算法,并将其分为不同的类别。大多数算法定义在 <algorithm> 中,但有些位于 <numeric> 中,并且都在 std 命名空间中。

本章的目标不是提供一个参考风格的概述,而是挑选了一些类别并提供示例。当知道如何使用这些算法,其他算法的使用方式也很相近。有关所有算法的完整参考,请查阅标准库手册(见附录 B)。

\mySubsubsection{20.2.1.}{非修改序列算法}

非修改序列算法是不修改操作元素序列的算法。这些包括在范围内搜索元素和比较两个范围的一系列算法,还包括一些计数算法。

\mySamllsection{搜索算法}

之前,已经看到了两个搜索算法示例:find() 和 find\_if()。标准库提供了基于 find() 算法的一些版本,可以工作在元素序列上。第 16 章的部分描述了可用的不同搜索算法,包括它们的复杂性。

所有这些算法都使用 operator== 或 < 作为默认比较操作符,也提供可以指定不同比较的重载版本。

以下是几种搜索算法示例:

\begin{cpp}
// The list of elements to be searched.
vector myVector { 5, 6, 9, 8, 8, 3 };
auto beginIter { cbegin(myVector) };
auto endIter { cend(myVector) };

// Find the first element that does not satisfy the given lambda expression.
auto it { find_if_not(beginIter, endIter, [](int i){ return i < 8; }) };
if (it != endIter) {
    println("First element not < 8 is {}", *it);
}

// Find the first pair of matching consecutive elements.
it = adjacent_find(beginIter, endIter);
if (it != endIter) {
    println("Found two consecutive equal elements with value {}", *it);
}

// Find the first of two values.
vector targets { 8, 9 };
it = find_first_of(beginIter, endIter, cbegin(targets), cend(targets));
if (it != endIter) {
    println("Found one of 8 or 9: {}", *it);
}

// Find the first subsequence.
vector sub { 8, 3 };
it = search(beginIter, endIter, cbegin(sub), cend(sub));
if (it != endIter) {
    println("Found subsequence {{8,3}}");
} else {
    println("Unable to find subsequence {{8,3}}");
}

// Find the last subsequence (which is the same as the first in this example).
auto it2 { find_end(beginIter, endIter, cbegin(sub), cend(sub)) };
if (it != it2) {
    println("Error: search and find_end found different subsequences "
        "even though there is only one match.");
}

// Find the first subsequence of two consecutive 8s.
it = search_n(beginIter, endIter, 2, 8);
if (it != endIter) {
    println("Found two consecutive 8s");
} else {
    println("Unable to find two consecutive 8s");
}
\end{cpp}

输出为:

\begin{shell}
First element not < 8 is 9
Found two consecutive equal elements with value 8
Found one of 8 or 9: 9
Found subsequence {8,3}
Found two consecutive 8s
\end{shell}

\begin{myNotic}{NOTE}
容器具有与泛型算法等效的成员函数,因为成员函数更高效,建议使用那些成员函数,而不是泛型算法。一个例子是 std::set 和 std::map 的 find() 成员函数。
\end{myNotic}

\mySamllsection{专用搜索器}

search() 算法的可选参数允许指定要使用的搜索算法。有三个选项——default\_searcher、boyer\_moore\_searcher 或 boyer\_moore\_horspool\_searcher,这些都在 <functional> 中定义。后两个选项实现了著名的 Boyer-Moore 和 Boyer-Moore-Horspool 搜索算法,这些算法比默认搜索器更高效,可以用于在更大的文本中查找子字符串。BoyerMoore 搜索器的复杂性如下(N 是要搜索的序列的大小,即 haystack,M 是要查找的模式的大小,即 needle):

\begin{itemize}
\item
如果模式未找到,最坏情况下的时间复杂度为 O(N+M)。

\item
如果找到模式,最坏情况下的复杂度是 O(N*M)。
\end{itemize}

这些都是理论上的最坏情况复杂度,这些专用搜索器是亚线性的,比 O(N) 更好,所以比默认搜索器快得多!能够跳过字符,而不是查看堆中的每个单个字符。它们还有一个有趣的特性,即模式越长,工作速度越快,因为能够在堆中跳过更多的字符。Boyer-Moore 和 Boyer-Moore-Horspool 算法之间的区别在于,后者在初始化和算法每次迭代中的常数开销更小;最坏情况下的复杂度可能比 Boyer-Moore 算法高得多,而选择哪个取决于具体用例。

以下是一个使用 Boyer-Moore 搜索器的示例:

\begin{cpp}
string text { "This is the haystack to search a needle in." };
string toSearchFor { "needle" };
boyer_moore_searcher searcher { cbegin(toSearchFor), cend(toSearchFor) };
auto result { search(cbegin(text), cend(text), searcher) };
if (result != cend(text)) {
    println("Found the needle.");
} else {
    println("Needle not found.");
}
\end{cpp}

\mySamllsection{比较算法}

可以以几种不同的方式比较整个元素范围:equal(), mismatch(), lexicographical\_compare()和 lexicographical\_compare\_three\_way(). 这些算法的一个优点,可以比较来自不同容器的序列。例如,可以比较 vector 和 list 的内容。一般来说,这些算法最适合与顺序容器一起工作,通过比较两个集合中对应位置的值来工作。以下列表描述了每个算法的工作方式:

\begin{itemize}
\item
equal() 如果所有对应的元素都相等,则返回 true。最初,equal() 接受三个迭代器:第一个范围的开头和结束迭代器,以及第二个范围的开头迭代器,这个版本要求两个范围具有相同数量的元素。从 C++14 开始,有一个接受四个迭代器的重载:第一个范围的开头和结束迭代器,以及第二个范围的开头和结束迭代器,这个版本可以处理不同大小的范围。因为更安全,更推荐使用四迭代器的版本!

\item
mismatch() 返回每个范围的迭代器,以指示范围中对应元素不匹配的位置。有三个迭代器和四个迭代器的版本可供选择,就像 equal() 一样。

\item
lexicographical\_compare() 按顺序比较两个提供的范围内具有相同位置的元素(依次)。如果第一个范围中第一个不等的元素,小于第二个范围中对应的元素,或者第一个范围中的元素少于第二个范围,并且所有元素都与第二个范围中对应的初始子序列相等,则返回 true。lexicographical\_compare() 有这个命名的原因是,其类似于字典中比较字符串的规则,但将此规则扩展到处理的类型对象上。

\item
lexicographical\_compare\_three\_way() 与 lexicographical\_compare() 类似,只是它执行三向比较,并返回比较类别类型(strong\_ordering, weak\_ordering 或 partial\_ordering,在第 1 章中介绍过)而不是布尔值。
\end{itemize}

\begin{myNotic}{NOTE}
如果想比较两个相同类型容器的所有元素,可以直接使用操作符operator== 或 operator< ,而不是这些算法。这些算法适用于比较子范围、C 风格数组、不同容器类型中的元素序列等。
\end{myNotic}

以下是这些算法的一些示例:

\begin{cpp}
vector<int> myVector;
list<int> myList;

println("Populate the vector:");
populateContainer(myVector);
println("Populate the list:");
populateContainer(myList);

// Compare the two containers
if (equal(cbegin(myVector), cend(myVector),
          cbegin(myList), cend(myList))) {
    println("The two containers have equal elements");
} else {
    // If the containers were not equal, find out why not
    auto miss { mismatch(cbegin(myVector), cend(myVector),
        cbegin(myList), cend(myList)) };
    println("The following initial elements are the same in "
        "the vector and the list:");
    for (auto iter { cbegin(myVector) }; iter != miss.first; ++iter) {
        print("{}\t", *iter);
    }
    println("");
}

// Now order them.
if (lexicographical_compare(cbegin(myVector), cend(myVector),
cbegin(myList), cend(myList))) {
    println("The vector is lexicographically first.");
} else {
    println("The list is lexicographically first.");
}
\end{cpp}

运行结果为:

\begin{shell}
Populate the vector:
Enter a number (0 to stop): 5
Enter a number (0 to stop): 6
Enter a number (0 to stop): 7
Enter a number (0 to stop): 0
Populate the list:
Enter a number (0 to stop): 5
Enter a number (0 to stop): 6
Enter a number (0 to stop): 9
Enter a number (0 to stop): 8
Enter a number (0 to stop): 0
The following initial elements are the same in the vector and the list:
5 6
The vector is lexicographically first.
\end{shell}

此外,以下比较算法适用于单个范围:all\_of()、any\_of()和none\_of():

\begin{cpp}
// all_of()
vector vec2 { 1, 1, 1, 1 };
if (all_of(cbegin(vec2), cend(vec2), [](int i){ return i == 1; })) {
    println("All elements are == 1");
} else {
    println("Not all elements are == 1");
}

// any_of()
vector vec3 { 0, 0, 1, 0 };
if (any_of(cbegin(vec3), cend(vec3), [](int i){ return i == 1; })) {
    println("At least one element == 1");
} else {
    println("No elements are == 1");
}

// none_of()
vector vec4 { 0, 0, 0, 0 };
if (none_of(cbegin(vec4), cend(vec4), [](int i){ return i == 1; })) {
    println("All elements are != 1");
} else {
    println("Some elements are == 1");
}
\end{cpp}

输出为:

\begin{shell}
All elements are == 1
At least one element == 1
All elements are != 1
\end{shell}

\mySamllsection{计数算法}

非修改计数算法包括 count() 和 count\_if()。以下示例使用 count\_if() 算法来计算一个vector中满足特定条件的元素的个数。条件以 Lambda 表达式的形式给出,通过值捕获了外层作用域中的 value 变量:

\begin{cpp}
vector values { 1, 2, 3, 4, 5, 6, 7, 8, 9 };
int value { 3 };
auto tally { count_if(cbegin(values), cend(values),
    [value](int i){ return i > value; }) };
println("Found {} values > {}.", tally, value);
\end{cpp}

输出为:

\begin{shell}
Found 6 values > 3
\end{shell}

示例可以扩展以演示通过引用捕获变量。以下 Lambda 表达式通过引用捕获外层作用域中的一个变量,并计算其调用的次数:

\begin{cpp}
vector values { 1, 2, 3, 4, 5, 6, 7, 8, 9 };
int value { 3 };
int callCounter { 0 };
auto tally { count_if(cbegin(values), cend(values),
    [value, &callCounter](int i){ ++callCounter; return i > value; }) };
println("The lambda expression was called {} times.", callCounter);
println("Found {} values > {}.", tally, value);
\end{cpp}

输出为:

\begin{shell}
The lambda expression was called 9 times.
Found 6 values > 3
\end{shell}

\mySubsubsection{20.2.2.}{修改序列算法}

标准库提供了一系列修改序列算法,将元素从一个范围复制到另一个范围、删除元素或反转一个范围内元素的顺序。

一些修改算法使用源范围和目标范围的概念,元素从源范围读取并在目标范围内修改。例如,copy() 就是这样的一个算法。

其他算法在其工作位置上进行修改,只需要一个范围,例如 generate() 算法。

\begin{myWarning}{WARNING}
修改算法不能在目标中插入元素,只能覆盖/修改目标中已有的元素。第 17 章描述了如何使用迭代器适配器,在目标中插入元素。
\end{myWarning}

\begin{myNotic}{NOTE}
不能使用map和multimap的范围作为修改算法的目标。这些算法会覆盖整个元素,而在map中,这包括键/值对。map和multimap将键标记为 const,因此它不能进行赋值。同样适用于set和multiset。替代方案是使用在第 17 章中描述的插入迭代器。
\end{myNotic}

第 16 章中列出了所有可用的修改算法,并对每个算法进行了描述,本节为这些算法中的一个选择提供了代码示例。如果理解了这个部分中解释的算法,其他算法应该也能轻松使用。

\mySamllsection{generate}

generate() 算法需要一个共同的范围,并替换该范围内返回的值,这些值是从作为第三个参数给出的函数回调中得到。以下示例使用 generate() 算法与 Lambda 表达式一起,将数字 2, 4, 8, 16等,放入一个vector中:

\begin{cpp}
vector<int> values(10); // Create a vector of 10 elements.
int value { 1 };
generate(begin(values), end(values), [&value]{ value *= 2; return value; });
println("{:n}", values);
\end{cpp}

输出为:

\begin{shell}
2, 4, 8, 16, 32, 64, 128, 256, 512, 1024
\end{shell}

\mySamllsection{transform}

transform() 算法有多个重载,其中一个重载对范围内的每个元素应用回调,并期望回调生成一个新元素,将其存储在指定的目标范围内。源范围和目标范围可以相同,如果希望 transform() 在原地工作。参数是源序列的开头和结束迭代器、目标序列的开头迭代器。例如,以下代码段将vector中的每个元素增加 100:

\begin{cpp}
vector<int> myVector;
populateContainer(myVector);
println("The vector contains: {:n}", myVector);
transform(begin(myVector), end(myVector), begin(myVector),
    [](int i){ return i + 100;});
println("The vector contains: {:n}", myVector);
\end{cpp}

可能的输出为:

\begin{shell}
Enter a number (0 to stop): 1
Enter a number (0 to stop): 11
Enter a number (0 to stop): 0
The vector contains: 1, 11
The vector contains: 101, 111
\end{shell}

transform() 的另一个重载对两个范围内的元素对调用二元函数,需要第一个范围的开头和结束迭代器、第二个范围的开头迭代器,以及目标范围的开头迭代器。以下示例创建两个vector,并使用 transform() 来计算元素对的和,并将结果存储回第一个vector:

\begin{cpp}
vector<int> vec1, vec2;
println("Vector1:"); populateContainer(vec1);
println("Vector2:"); populateContainer(vec2);
if (vec2.size() < vec1.size())
{
    println("Vector2 should be at least the same size as vector1.");
    return 1;
}

println("Vector1: {:n}", vec1);
println("Vector2: {:n}", vec2);
transform(begin(vec1), end(vec1), begin(vec2), begin(vec1),
    [](int a, int b){ return a + b; });
println("Vector1: {:n}", vec1);
println("Vector2: {:n}", vec2);
\end{cpp}

输出可能如下所示:

\begin{shell}
Vector1:
Enter a number (0 to stop): 1
Enter a number (0 to stop): 2
Enter a number (0 to stop): 0
Vector2:
Enter a number (0 to stop): 11
Enter a number (0 to stop): 22
Enter a number (0 to stop): 33
Enter a number (0 to stop): 0
Vector1: 1, 2
Vector2: 11, 22, 33
Vector1: 12, 24
Vector2: 11, 22, 33
\end{shell}

\begin{myNotic}{NOTE}
transform() 和其他修改算法通常返回一个指向目标范围末尾值的迭代器。本书中的示例通常忽略这个返回值。
\end{myNotic}

\mySamllsection{copy}

copy() 算法允许从一个范围复制元素到另一个范围,从范围中的第一个元素开始,直到最后一个元素。源范围和目标范围必须不同,但在某些限制下,可以重叠。这些限制如下:对于 copy(b,e,d),如果 d 在 b 之前,重叠是可以的;但如果 d 在 [b,e) 范围内,行为将不确定。与所有修改算法一样,copy() 不能在目标中插入元素,只能覆盖已存在的元素。第 17 章描述了如何使用迭代器适配器在 copy() 中,向容器或流中插入元素。

以下是一个简单的 copy() 示例,使用 vector 的 resize() 成员函数确保目标容器有足够的空间,从 vec1 复制所有元素到 vec2。

\begin{cpp}
vector<int> vec1, vec2;
populateContainer(vec1);
vec2.resize(size(vec1));
copy(cbegin(vec1), cend(vec1), begin(vec2));
println("{:n}", vec2);
\end{cpp}

还有一个 copy\_backward() 算法,将元素从源向后复制到目标。换句话说,从源范围的最后一个元素开始,将其放入目标范围的最后位置,然后每次复制后向后移动。对于 copy\_backward(),源范围和目标范围也必须不同,但它们可以重叠,条件与 copy() 相同。前面的示例可以修改为使用 copy\_backward(),而不是 copy()。注意,需要将 end(vec2) 作为第三个参数,而不是 begin(vec2),其输出与 copy() 相同。

\begin{cpp}
copy_backward(cbegin(vec1), cend(vec1), end(vec2));
\end{cpp}

copy\_if() 通过输入范围由两个迭代器指定,输出目标由一个迭代器指定,以及一个谓词(一个函数或 Lambda 表达式)来工作,该算法将满足给定谓词的所有元素复制到目标中。copy() 不创建或扩展容器,只是替换现有的元素,所以目标应该足够大以容纳所有要复制的元素。复制元素之后,可能希望从目标容器中移除“超出”最后复制元素位置的空间。为了实现这一点,copy\_if() 返回一个指向目标范围中,最后一个复制元素之后的迭代器。可以用来确定从目标容器中应该移除多少元素,以下示例通过复制 vec1 中的偶数到 vec2 来展示这一点:

\begin{cpp}
vector<int> vec1, vec2;
populateContainer(vec1);
vec2.resize(size(vec1));
auto endIterator { copy_if(cbegin(vec1), cend(vec1),
    begin(vec2), [](int i){ return i % 2 == 0; }) };
vec2.erase(endIterator, end(vec2));
println("{:n}", vec2);
\end{cpp}

copy\_n() 复制源中的 n 个元素到目标,copy\_n() 的第一个参数是开始迭代器。第二个参数是一个整数,指定要复制的元素数量,第三个参数是目标迭代器。copy\_n() 算法不执行边界检查,必须确保开始迭代器,加上要复制的元素数量,不超过源集合的 end(),否则程序将具有未定义的行为。以下是一个示例:

\begin{cpp}
vector<int> vec1, vec2;
populateContainer(vec1);
size_t tally { 0 };
print("Enter number of elements you want to copy: ");
cin >> tally;
tally = min(tally, size(vec1));
vec2.resize(tally);
copy_n(cbegin(vec1), tally, begin(vec2));
println("{:n}", vec2);
\end{cpp}

\mySamllsection{move}

有两个与移动相关的算法:move() 和 move\_backward(),都使用移动语义,这在第 9 章中有详细讨介绍。如果想使用这些算法在具有自定义类型的容器上,元素的类型必须提供移动赋值操作符,如以下示例所示。main() 函数创建一个包含三个 MyClass 对象的vector,将这些元素从 vecSrc 移动到 vecDst。代码包括了 move() 的两种不同用法。接受单个参数的 move() 函数将左值转换为右值,定义在 <utility> 中;而接受三个参数的 move() 是标准库的 move() 算法,用于在容器之间移动元素。关于实现移动赋值操作符和单参数版本 std::move() 的使用详情,请参阅第 9 章。

\begin{cpp}
class MyClass
{
    public:
        MyClass() = default;
        MyClass(const MyClass& src) = default;
        explicit MyClass(string str) : m_str { move(str) } {}
        virtual ~MyClass() = default;

        // Move assignment operator
        MyClass& operator=(MyClass&& rhs) noexcept {
            if (this == &rhs) { return *this; }
            m_str = move(rhs.m_str);
            println("Move operator= (m_str={})", m_str);
            return *this;
        }

        void setString(string str) { m_str = move(str); }
        const string& getString() const { return m_str; }
    private:
        string m_str;
};

int main()
{
    vector<MyClass> vecSrc { MyClass { "a" }, MyClass { "b" }, MyClass { "c" } };
    vector<MyClass> vecDst(vecSrc.size());
    move(begin(vecSrc), end(vecSrc), begin(vecDst));
    for (const auto& c : vecDst) { print("{} ", c.getString()); }
}
\end{cpp}

输出为:

\begin{shell}
Move operator= (m_str=a)
Move operator= (m_str=b)
Move operator= (m_str=c)
a b c
\end{shell}

\begin{myNotic}{NOTE}
第 9 章中,移动操作中的源对象处于一种有效,但不确定的状态。对于前面的示例,在移动操作之后,不应再使用 vecSrc 中的元素,除非将其恢复到一种确定的状态。通过调用它们的一个成员函数,而不需要前提条件,例如 setString()。
\end{myNotic}

move\_backward() 与 move() 有相同的移动机制,但会从最后一个元素开始向前移动。对于 move() 和 move\_backward(),源范围和目标范围可以重叠,但受与 copy() 和 copy\_backward() 相同的限制。

\mySamllsection{replace}

replace() 和 replace\_if() 算法分别替换匹配一个值或谓词的范围内的元素。以 replace\_if() 为例,前两个参数指定了要处理的元素范围。第三个参数是一个回调,返回 true 或 false。如果返回 true,容器中的值将用第四个参数中给出的新值替换;如果返回 false,则保留原始值。

例如,想要从一个容器中替换所有奇数值为零:

\begin{cpp}
vector<int> values;
populateContainer(values);
replace_if(begin(values), end(values), [](int i){ return i % 2 != 0; }, 0);
println("{:n}", values);
\end{cpp}

replace() 和 replace\_if() 也有称为 replace\_copy() 和 replace\_copy\_if() ,将结果复制到不同的目标范围。与 copy() 类似,目标范围必须足够大,以容纳复制的元素。

\mySamllsection{erase}

如第 18 章介绍的,std::erase() 和 std::erase\_if() 支持几乎所有的标准库容器,这些操作称为统一容器删除操作。erase() 函数从容器中删除与给定值匹配的所有元素,而 erase\_if() 删除与给定谓词匹配的所有元素。这些算法需要一个容器的引用,而不是一个公共范围,并且是删除容器中元素的首选方法。

以下代码片段从一个字符串vector中删除所有空字符串,并使用 erase\_if() 完成所有工作:

\begin{cpp}
vector<string> values {"", "one", "", "two", "three", "four"};
println("{:n}", values);
erase_if(values, [](const string& str){ return str.empty(); });
println("{:n}", values);
\end{cpp}

输出为:

\begin{shell}
"", "one", "", "two", "three", "four"
"one", "two", "three", "four"
\end{shell}

\begin{myNotic}{NOTE}
std::erase() 不适用于有序和无序关联容器,这些容器有一个名为 erase(key) 的成员函数,性能更好,应该优先使用。另一方面,erase\_if() 函数与所有容器都兼容。
\end{myNotic}

\mySamllsection{remove}

前一节中讨论的 erase() 和 erase\_if() 算法,自 C++20 以来一直可用。先来看看 C++20 之前的方式,因为会在历史代码中遇到。首先想到的解决方案是查看文档,看看容器是否有一个 erase() 成员函数,然后遍历所有元素并调用 erase() 对每个匹配条件的元素。vector就是一个具有此类 erase() 成员函数的容器的例子。如果应用于vector容器,这个解决方案效率极低,会需要大量的内存操作,来保持vector在内存中的连续性,从而导致复杂度上升为二次方。这个解决方案也容易出错,需要小心,调用 erase() 后需要确保迭代器的有效性。例如,这是一个不使用算法的函数,从字符串向量中删除所有空字符串。注意,在for 循环内部如何操作迭代器 iter。

\begin{cpp}
void removeEmptyStringsWithoutAlgorithms(vector<string>& strings)
{
    for (auto iter { begin(strings) }; iter != end(strings); ) {
        if (iter->empty()) {
            iter = strings.erase(iter);
        } else {
            ++iter;
        }
    }
}
\end{cpp}

\begin{myNotic}{NOTE}
二次复杂度意味着运行时间是输入大小的平方,$O(n^2)$。
\end{myNotic}

这个解决方案效率低下,不推荐使用。对于这个问题,更好的解决方案是使用 remove-erase 模式,其运行时间为线性,接下来将解释。

remove 算法只能访问迭代器抽象,而不能访问容器,实际上无法从底层容器中删除元素。相反,算法通过使用移动赋值来替换匹配给定值或谓词的元素,可通过移动赋值来完成。结果是所有要保留的元素都移动到范围的前部,范围划分为两组:要保留的元素和要删除的元素。返回一个迭代器,指向要删除的元素的范围内的第一个元素。不要使用该范围内的元素,它们可能已经移动,必须做的唯一事情是实际从容器中删除这些元素。首先使用 remove() 或 remove\_if() 算法,然后对容器调用 erase() 来删除从返回的迭代器到范围末尾的所有元素,这个过程称为 remove-erase 模式。以下是使用这种模式实现的 removeEmptyStrings() 函数:

\begin{cpp}
void removeEmptyStrings(vector<string>& strings)
{
    auto it { remove_if(begin(strings), end(strings),
        [](const string& str){ return str.empty(); }) };
    // Erase the removed elements.
    strings.erase(it, end(strings));
}
\end{cpp}

\begin{myWarning}{WARNING}
使用 remove-erase 模式时,不要忘记 erase() 的第二个参数!如果忘记了这个第二个参数,erase() 只会从容器中删除一个元素,即作为第一个参数传递的迭代器所指向的元素。
\end{myWarning}

remove\_copy() 和 remove\_copy\_if() 是 remove() 和 remove\_if() 一样,不会改变源范围。相反,将所有保留的元素复制到不同的目标范围。与 copy() 类似,目标范围必须足够大,以容纳新元素。

\begin{myNotic}{NOTE}
remove() 函数族是稳定的。即使在移动保留元素到范围的前部时,也会保持容器中剩余元素的顺序。
\end{myNotic}

\begin{myNotic}{NOTE}
建议使用 std::erase\_if()、std::erase() 算法(或关联容器的 erase(key) 成员函数),而不是使用 remove-erase 模式,当然也不推荐使用循环。
\end{myNotic}

\mySamllsection{unique}

unique() 算法是 remove() 算法的一个特殊情况,删除所有连续的重复元素。列表容器提供自己的 unique() 成员函数,实现相同的语义。应该在对排序序列上使用 unique(),也可以对未排序的序列使用。

unique() 的工作方式与 remove() 类似:将要保留的元素移动到范围的前部,并返回一个迭代器,该迭代器指向要删除元素范围内的第一个元素。与 remove-erase习惯一样,调用 unique() 必须紧跟一个 erase()。

unique() 的基本形式就地运行,也有一个算法的称为 unique\_copy(),将结果复制到新的目标范围。

第 18 章在部分展示了 list::unique() 算法的一个示例,因此这里省略了通用形式的示例。

\mySamllsection{shuffle}

shuffle() 以线性复杂度重新排列范围中元素的无序顺序,适用于实现像洗牌一副扑克牌这样的任务。shuffle() 需要一个开始和结束迭代器来指定要混洗的范围,以及一个指定如何生成随机数的统一随机数生成器对象。以下是示例(关于如何使用随机数生成引擎,以及如何“设置种子”的细节在第 23 章中介绍):

\begin{cpp}
vector values { 1, 2, 3, 4, 5, 6, 7, 8, 9 };
random_device seeder;
default_random_engine generator { seeder() };
for (int i { 0 }; i < 6; ++i) {
    shuffle(begin(values), end(values), generator);
    println("{:n}", values);
}
\end{cpp}

这里是一些可能的输出:

\begin{shell}
8, 6, 7, 5, 4, 1, 2, 9, 3
4, 1, 6, 2, 3, 7, 5, 9, 8
1, 4, 2, 5, 6, 8, 7, 3, 9
8, 4, 2, 7, 5, 9, 1, 6, 3
8, 9, 1, 7, 4, 5, 2, 6, 3
1, 7, 8, 5, 4, 3, 9, 6, 2
\end{shell}


\mySamllsection{sample}

sample() 算法从一个给定的源范围中返回 n 个随机选择的元素,并将它们存储在目标范围内,需要五个参数:

\begin{itemize}
\item
采样范围的开始和结束迭代器

\item
存储随机选择的元素的目标范围的开始迭代器

\item
元素数量

\item
随机数生成引擎
\end{itemize}

以下是示例:

\begin{cpp}
vector values { 1, 2, 3, 4, 5, 6, 7, 8, 9 };
const size_t numberOfSamples { 5 };
vector<int> samples(numberOfSamples);

random_device seeder;
default_random_engine generator { seeder() };

for (int i { 0 }; i < 6; ++i) {
    sample(cbegin(values), cend(values), begin(samples),
        numberOfSamples, generator);
    println("{:n}", samples);
}
\end{cpp}

这里是一些可能的输出:

\begin{shell}
1, 4, 7, 8, 9
1, 3, 4, 7, 9
2, 3, 4, 5, 7
3, 5, 6, 7, 9
1, 2, 3, 6, 7
1, 2, 4, 5, 8
\end{shell}

\mySamllsection{reverse}

reverse() 算法反转范围内元素的顺序。范围内的第一个元素与最后一个元素交换,第二个元素与倒数第二个元素交换,依此类推。

reverse() 的基本形式就地运行,需要两个参数:一个开始迭代器和结束迭代器。还有一个算法变体称为 reverse\_copy(),将结果复制到新的目标范围,需要三个参数:源范围的开始和结束迭代器,以及目标范围的开头迭代器。目标范围必须足够大以容纳新元素。

以下是使用 reverse() 的示例:

\begin{cpp}
vector<int> values;
populateContainer(values);
reverse(begin(values), end(values));
println("{:n}", values);
\end{cpp}

\mySamllsection{移动元素}

shift\_left() 和 shift\_right() 算法在给定范围内移动元素到新位置。shift\_left() 返回一个指向新范围末尾的迭代器,而 shift\_right() 返回一个指向新范围开头的迭代器。调用任一算法之后,必须在erase() 中使用返回的迭代器,删除从范围两端的元素:

\begin{cpp}
vector values { 11, 22, 33, 44, 55 };
println("{:n}", values);

// Shift elements to the left by 2 positions.
auto newEnd { shift_left(begin(values), end(values), 2) };
// Resize the vector to its proper size.
values.erase(newEnd, end(values));
println("{:n}", values);

// Shift elements to the right by 2 positions.
auto newBegin { shift_right(begin(values), end(values), 2) };
// Resize the vector to its proper size.
values.erase(begin(values), newBegin);
println("{:n}", values);
\end{cpp}

输出如下:

\begin{shell}
11, 22, 33, 44, 55
33, 44, 55
33
\end{shell}

\mySubsubsection{20.2.3.}{操作算法}

这个类中只有两个算法:for\_each() 和 for\_each\_n()。在范围内对每个元素执行一个回调,for\_each() 或者在范围内对前 n 个元素执行一个回调,for\_each\_n()。如果给定的迭代器类型是非常量,回调可以修改范围内的元素。虽然在这里提到这些算法,但通常使用简单的基于范围的 for 循环更容易且更易读。

\mySamllsection{for\_each}

以下是一个使用通用 Lambda 表达式的示例,打印map中的元素:

\begin{cpp}
map<int, int> myMap { { 4, 40 }, { 5, 50 }, { 6, 60 } };
for_each(cbegin(myMap), cend(myMap), [](const auto& p)
    { println("{} -> {}", p.first, p.second); });
\end{cpp}

p 的类型是 const pair<int, int>\&。输出如下:

\begin{shell}
4 -> 40
5 -> 50
6 -> 60
\end{shell}

以下示例展示了如何使用 for\_each() 算法,并且Lambda表达式同时计算范围元素的和与积。Lambda 表达式显式捕获它需要的变量,通过引用捕获这些变量;否则,在 Lambda 表达式中修改 sum 和 product,并且在 Lambda 外部可见。

\begin{cpp}
vector<int> myVector;
populateContainer(myVector);

int sum { 0 };
int product { 1 };
for_each(cbegin(myVector), cend(myVector),
[&sum, &product](int i){
    sum += i;
    product *= i;
});
println("The sum is {}", sum);
println("The product is {}", product);
\end{cpp}

这个例子也可以写成一个函数对象,在其中累积信息,然后在 for\_each() 完成处理所有元素后检索信息。例如,可以通过一个同时跟踪两者(sum 和 product)的 SumAndProduct 函数对象,在一次遍历中计算元素的 sum 和 product:

\begin{cpp}
class SumAndProduct
{
    public:
        void operator()(int value)
        {
            m_sum += value;
            m_product *= value;
        }
        int getSum() const { return m_sum; }
        int getProduct() const { return m_product; }
    private:
        int m_sum { 0 };
        int m_product { 1 };
};
int main()
{
    vector<int> myVector;
    populateContainer(myVector);
    SumAndProduct calculator;
    calculator = for_each(cbegin(myVector), cend(myVector), calculator);
    println("The sum is {}", calculator.getSum());
    println("The product is {}", calculator.getProduct());
}
\end{cpp}

可能想忽略 for\_each() 的返回值,尝试在算法完成后从 calculator 中读取信息。但这样做是行不通的,for\_each() 复制了函数对象,并在调用结束时返回这个副本,必须捕获返回值以确保正确的行为。

另一种方法是通过引用传递 calculator 使用 std::ref(),如本章前面所示:

\begin{cpp}
for_each(cbegin(myVector), cend(myVector), ref(calculator));
\end{cpp}

关于 for\_each()(也适用于下一节中讨论的 for\_each\_n())的最后一个要点,回调允许有非常量引用作为参数并修改它。这导致在实际范围内改变值,以下是修改范围内每个元素的一个示例:

\begin{cpp}
vector values { 11, 22, 33, 44 };
// Double each element in the values vector.
for_each(begin(values), end(values), [](auto& value) { value *= 2; });
println("{:n}", values);
\end{cpp}

\mySamllsection{for\_each\_n}

for\_each\_n() 算法需要一个范围的开始迭代器,要迭代的元素数量 n,以及一个回调。返回一个 begin + n 的迭代器。并且,不执行边界检查。以下是一个只遍历映射中前两个元素的示例:

\begin{cpp}
map<int, int> myMap { { 4, 40 }, { 5, 50 }, { 6, 60 } };
for_each_n(cbegin(myMap), 2, [](const auto& p)
    { println("{} -> {}", p.first, p.second); });
\end{cpp}

\mySubsubsection{20.2.4.}{分区算法}

partition\_copy() 从源复制元素到两个不同的目标,每个元素的复制目的地是根据谓词结果选择的,即 true 或 false。partition\_copy() 返回值是一个迭代器对的引用:指向第一个和第二个目标范围中最后一个元素的迭代器。这些返回的迭代器可以与 erase() 一起使用,从两个目标范围中删除多余的元素,就像之前的 copy\_if() 示例一样。以下代码要求用户输入一系列整数,将它们分区到两个目标vector中:一个存偶数,另一个存奇数:

\begin{cpp}
vector<int> values, vecOdd, vecEven;
populateContainer(values);
vecOdd.resize(size(values));
vecEven.resize(size(values));

auto pairIters { partition_copy(cbegin(values), cend(values),
    begin(vecEven), begin(vecOdd),
    [](int i){ return i % 2 == 0; }) };

vecEven.erase(pairIters.first, end(vecEven));
vecOdd.erase(pairIters.second, end(vecOdd));

println("Even numbers: {:n}", vecEven);
println("Odd numbers: {:n}", vecOdd);
\end{cpp}

输出可能如下所示:

\begin{shell}
Enter a number (0 to stop): 11
Enter a number (0 to stop): 22
Enter a number (0 to stop): 33
Enter a number (0 to stop): 44
Enter a number (0 to stop): 0
Even numbers: 22, 44
Odd numbers: 11, 33
\end{shell}

partition() 算法对一个序列进行排序,使得所有使谓词返回 true 的元素都排在所有返回 false 的元素之前,而不保留每个分区内元素的原始顺序。以下示例,演示了如何将一个vector分区为所有偶数后跟所有奇数:

\begin{cpp}
vector<int> values;
populateContainer(values);
partition(begin(values), end(values), [](int i){ return i % 2 == 0; });
println("Partitioned result: {:n}", values);
\end{cpp}

输出可能如下所示:

\begin{shell}
Enter a number (0 to stop): 55
Enter a number (0 to stop): 44
Enter a number (0 to stop): 33
Enter a number (0 to stop): 22
Enter a number (0 to stop): 11
Enter a number (0 to stop): 0
Partitioned result: 22, 44, 33, 55, 11
\end{shell}

还有一些其他分区算法可供使用。请参阅第 16 章以获取列表。

\mySubsubsection{20.2.5.}{排序算法}

标准库提供了几种排序算法。排序算法重新排列容器的内容,以维护集合中连续元素之间的顺序,只适用于顺序集合。排序对有序关联容器不适用,已经保持元素以排序顺序排列。排序对无序关联容器也不适用,其没有排序的概念。一些容器,如list和forward\_list,提供自己的排序成员函数,这些成员函数可以比通用的排序机制更有效地实现,所以通用的排序算法对于vector、deque、array和 C 风格数组更有用。

sort() 算法通常以 O(N log N) 的时间复杂度对一个范围内的元素进行排序。在范围应用sort()之后,范围内的元素会按非递减顺序(从最低到最高)排列。如果不喜欢这个顺序,可以指定一个比较器,例如 greater。

sort() 的一个版本,称为 stable\_sort(),在范围内保持相等元素的相对顺序,但它不如 sort() 高效。

以下是一个使用透明 greater<> 比较器的 sort() 示例:

\begin{cpp}
vector<int> values;
populateContainer(values);
sort(begin(values), end(values), greater<>{});
\end{cpp}

还有一个 is\_sorted() 函数,如果给定的范围是排序的,则返回 true;还有一个 is\_sorted\_until() 函数,返回一个迭代器,该迭代器之前的内容是排序的。

nth\_element() 是一个强大的选择算法。给定一个范围内的元素和一个指向该范围内第 n 个元素的迭代器,该算法重新排列范围内的元素,指向第 n 个位置的元素。如果整个范围已排序,迭代器将指向那个位置的元素。此外,它重新排列所有元素,使得第 n 个元素之前的所有元素都小于新的第 n 个元素,而之后的元素都大于新的第 n 个元素。这个算法有趣的地方在于,可以在线性时间内完成所有这些操作,即 O(n)。与其使用 nth\_element(),也可以直接对整个范围进行排序,然后获取感兴趣的数据,但这会导致复杂度为线性对数,即 O(n log n)。

来看看这个算法是如何工作,一个示例是找到给定范围内的第三个最大元素。假设用户输入至少三个值。

\begin{cpp}
vector<int> values;
populateContainer(values);
// Find the third largest value.
nth_element(begin(values), begin(values) + 2, end(values), greater<>{});
println("3rd largest value: {}", values[2]);
\end{cpp}

另一个示例是从一个范围内获取五个最大的元素,并以排序顺序,假设用户输入至少五个值。

\begin{cpp}
vector<int> values;
populateContainer(values);
// Get the 5 largest elements in sorted order.
nth_element(begin(values), begin(values) + 4, end(values), greater<>{});
// nth_element() has partitioned the elements, now sort the first subrange.
sort(begin(values), begin(values) + 5);
// And finally, output the sorted subrange.
for_each_n(begin(values), 5, [](const auto& element) { print("{} ", element); });
\end{cpp}

\mySubsubsection{20.2.6.}{二分查找算法}

有几种搜索算法只能用于排序的序列,或者至少对要搜索的元素进行了分区。这些算法包括 binary\_search()、lower\_bound()、upper\_bound() 和 equal\_range()。关联容器(如 map 和 set)具有等效的成员函数,应该使用这些成员函数。关于如何在这些容器上使用这些成员函数的示例,请参阅第 18 章。

lower\_bound() 算法在排序范围内找到第一个大于或等于给定值的元素,通常用于在排序vector中找到新值应该插入的位置,以保持vector的顺序。以下是一个示例:

\begin{cpp}
vector<int> values;
populateContainer(values);

// Sort the container
sort(begin(values), end(values));
println("Sorted vector: {:n}", values);

while (true) {
    int number;
    print("Enter a number to insert (0 to stop): ");
    cin >> number;
    if (number == 0) { break; }
    auto iter { lower_bound(begin(values), end(values), number) };
    values.insert(iter, number);
    println("New vector: {:n}", values);
}
\end{cpp}

binary\_search() 算法在对数时间内,而不是线性时间内找到匹配的元素。需要指定要搜索的范围的开头和结束迭代器、要搜索的值,以及可选的比较器回调。如果在范围内找到值,则返回 true,否则返回 false。二分搜索要求范围已排序,并通过比较范围的中间元素来工作。根据中间元素是否大于或小于要搜索的值,会继续比较左半部分或右半部分范围的中间元素,依此类推。这样持续进行,直到找到元素。基本上,每次迭代都将范围减半,因此具有对数复杂度。以下示例演示了这个算法:

\begin{cpp}
vector<int> values;
populateContainer(values);

// Sort the container
sort(begin(values), end(values));

while (true) {
    print("Enter a number to find (0 to stop): ");
    int number;
    cin >> number;
    if (number == 0) { break; }
    if (binary_search(cbegin(values), cend(values), number)) {
        println("That number is in the vector.");
    } else {
        println("That number is not in the vector.");
    }
}
\end{cpp}

\mySubsubsection{20.2.7.}{set算法}

set算法适用于任何已排序的范围。includes() 算法实现了标准的子集确定,检查一个已排序的范围,是否完全包含在另一个已排序的范围中。

set\_union()、set\_intersection()、set\_difference() 和 set\_symmetric\_difference() 算法实现了这些操作的标准语义。在set中,并集的结果是两个set中所有的元素,交集的结果是两个set中都有的元素,差集的结果是第一个set中有但第二个set中没有的所有元素。对称差集的结果是set的“异或”:在一个set中,但不在另一个set中的所有元素。

\begin{myWarning}{WARNING}
确保结果范围足够大,以容纳操作的结果。对于 set\_union() 和 set\_symmetric\_difference(),结果最多是两个输入范围大小的和。对于 set\_intersection(),结果最多是输入范围中最小的大小,对于 set\_difference() 是最大是第一个范围的大小。
\end{myWarning}

\begin{myWarning}{WARNING}
不能使用关联容器(包括set)的常见范围来存储结果,不允许更改键。
\end{myWarning}

了解一下这些set算法是如何工作。定义了一个受约束的 DumpRange() 函数模板,用于将给定范围的内容写入标准输出流;实现如下所示。ranges::subrange() 将由迭代器给出的常见范围转换为范围,可以传递给 println()。

\begin{cpp}
template <forward_iterator Iterator>
void DumpRange(string_view message, Iterator begin, Iterator end)
{
    println("{}{:n}", message, ranges::subrange(begin, end));
}
\end{cpp}

有了这个辅助函数定义后,这里是一些使用集合算法的示例:

\begin{cpp}
vector<int> vec1, vec2, result;
println("Enter elements for set 1:");
populateContainer(vec1);
println("Enter elements for set 2:");
populateContainer(vec2);

// set algorithms require sorted ranges
sort(begin(vec1), end(vec1));
sort(begin(vec2), end(vec2));

println("Set 1: {:n}", vec1);
println("Set 2: {:n}", vec2);

if (includes(cbegin(vec1), cend(vec1), cbegin(vec2), cend(vec2))) {
    println("The second set is a subset of the first.");
}
if (includes(cbegin(vec2), cend(vec2), cbegin(vec1), cend(vec1))) {
    println("The first set is a subset of the second");
}

result.resize(size(vec1) + size(vec2));
auto newEnd { set_union(cbegin(vec1), cend(vec1), cbegin(vec2),
    cend(vec2), begin(result)) };
DumpRange("The union is: ", begin(result), newEnd);

newEnd = set_intersection(cbegin(vec1), cend(vec1), cbegin(vec2),
cend(vec2), begin(result));
DumpRange("The intersection is: ", begin(result), newEnd);

newEnd = set_difference(cbegin(vec1), cend(vec1), cbegin(vec2),
cend(vec2), begin(result));
DumpRange("The difference between set 1 and 2 is: ", begin(result), newEnd);

newEnd = set_symmetric_difference(cbegin(vec1), cend(vec1),
cbegin(vec2), cend(vec2), begin(result));
DumpRange("The symmetric difference is: ", begin(result), newEnd);
\end{cpp}

这是程序运行的一个示例结果:

\begin{shell}
Enter elements for set 1:
Enter a number (0 to stop): 5
Enter a number (0 to stop): 6
Enter a number (0 to stop): 7
Enter a number (0 to stop): 8
Enter a number (0 to stop): 0
Enter elements for set 2:
Enter a number (0 to stop): 8
Enter a number (0 to stop): 9
Enter a number (0 to stop): 10
Enter a number (0 to stop): 0
Set 1: 5, 6, 7, 8
Set 2: 8, 9, 10
The union is: 5, 6, 7, 8, 9, 10
The intersection is: 8
The difference between set 1 and set 2 is: 5, 6, 7
The symmetric difference is: 5, 6, 7, 9, 10
\end{shell}

merge() 算法允许合并两个已排序的范围,同时保持排序顺序。结果是一个包含两个源范围所有元素的已排序范围,它以线性时间工作。以下是必需参数:

\begin{itemize}
\item
第一个源范围的开始和结束迭代器

\item
第二个源范围的开始和结束迭代器

\item
目的范围的开始迭代器

\item
可选,比较器的回调
\end{itemize}

如果没有 merge(),也可以将两个范围连接起来,并使用 sort() 来实现相同的效果,但这比 merge()的效率低,时间复杂度为O(N log N)。

\begin{myWarning}{WARNING}
确保提供足够大的目的范围来存储 merge() 的结果!
\end{myWarning}

以下示例演示了如何使用 merge():

\begin{cpp}
vector<int> vectorOne, vectorTwo, vectorMerged;
println("Enter values for first vector:");
populateContainer(vectorOne);
println("Enter values for second vector:");
populateContainer(vectorTwo);

// Sort both containers
sort(begin(vectorOne), end(vectorOne));
sort(begin(vectorTwo), end(vectorTwo));

// Make sure the destination vector is large enough to hold the values
// from both source vectors.
vectorMerged.resize(size(vectorOne) + size(vectorTwo));

merge(cbegin(vectorOne), cend(vectorOne),
      cbegin(vectorTwo), cend(vectorTwo), begin(vectorMerged));

println("Merged vector: {:n}", vectorMerged);
\end{cpp}

\mySubsubsection{20.2.8.}{最小/最大算法}

min() 和 max() 算法比较两个或多个类型的元素,使用 operator< 或用户提供的二元谓词,分别返回最小或最大元素的有常引用的引用。minmax() 算法返回一个包含两个或多个元素最小值和最大值的元组,这些算法不适用于常见范围。

min\_element() 和 max\_element() 算法适用于常见范围,并分别返回范围内最小或最大元素的迭代器。minmax\_element() 算法也适用于常见范围,并返回一个包含范围内最小和最大元素迭代器的元组。

以下程序提供了一些示例:

\begin{cpp}
int x { 4 }, y { 5 };
println("x is {} and y is {}", x, y);
println("Max is {}", max(x, y));
println("Min is {}", min(x, y));

// Using max() and min() on more than two values.
int x1 { 2 }, x2 { 9 }, x3 { 3 }, x4 { 12 };
println("Max of 4 elements is {}", max({ x1, x2, x3, x4 }));
println("Min of 4 elements is {}", min({ x1, x2, x3, x4 }));

// Using minmax().
auto p2 { minmax({ x1, x2, x3, x4 }) }; // p2 is of type pair<int, int>.
println("Minmax of 4 elements is <{},{}>", p2.first, p2.second);

// Using minmax() + structured bindings.
auto [min1, max1] { minmax({ x1, x2, x3, x4 }) };
println("Minmax of 4 elements is <{},{}>", min1, max1);

// Using minmax_element() + structured bindings.
vector values { 11, 33, 22 };
auto [min2, max2] { minmax_element(cbegin(values), cend(values)) };
println("minmax_element() result: <{},{}>", *min2, *max2);
\end{cpp}

程序输出如下:

\begin{shell}
x is 4 and y is 5
Max is 5
Min is 4
Max of 4 elements is 12
Min of 4 elements is 2
Minmax of 4 elements is <2,12>
Minmax of 4 elements is <2,12>
minmax_element() result: <11,33>
\end{shell}

\begin{myNotic}{NOTE}
有时,可能会遇到用于查找最小值和最大值的非标准宏。例如,GNU C 库(glibc)有宏 MIN() 和 MAX(),而 Windows.h 头文件定义了 min() 和 max() 宏。由于这些是宏,它们有可能计算其参数之一两次,而 std::min() 和 std::max() 则精确地计算每个参数一次。确保始终使用 C++ 版本,即 std::min() 和 std::max()。

更糟糕的是,这样的 min() 和 max() 宏可能会与使用 std::min() 和 std::max() 发生冲突。这种情况下,有三种解决方案:

\begin{itemize}
\item
在包含有问题的头文件(例如 Windows.h)之后,使用 \#undef min 和 \#undef max。

\item
包含 Windows.h 之前,添加 \#define NOMINMAX。

\item
为 std::min() 和 std::max() 使用括号,如下所示:

\begin{cpp}
auto maxValue { (std::max)(1, 2) };
\end{cpp}
\end{itemize}
\end{myNotic}

std::clamp() 是一个辅助函数,定义在 <algorithm> 中,可以使用它来确保一个值(v)在给定的最小值(lo)和最大值(hi)之间。如果 v < lo,则返回 lo 的引用;如果 v > hi,返回 hi 的引用;否则,返回 v 的引用。以下是一个示例:

\begin{cpp}
println("{}", clamp(-3, 1, 10));
println("{}", clamp(3, 1, 10));
println("{}", clamp(22, 1, 10));
\end{cpp}

输出为:

\begin{shell}
1
3
10
\end{shell}

\mySubsubsection{20.2.9.}{并行算法}

C++ 支持并行执行超过 60 个标准库迭代器基础算法,以提高它们的性能。这些包括 std::for\_each()、all\_of()、copy()、count\_if()、find()、replace()、search()、sort()、transform() 等。

支持并行执行的算法有一个可选的执行策略,作为它们的第一个参数。执行策略允许指定算法是否允许向量化并/或并行执行。当编译器向量化代码时,用一个向量 CPU 指令替换几个 CPU 指令。向量指令对多个数据片段执行某些操作,只需一条硬件指令,这些也称为单指令多数据(SIMD)指令。有四种标准的执行策略类型,以及这些类型的相应全局实例,都在 <execution> 中定义在 std::execution 命名空间中:

% Please add the following required packages to your document preamble:
% \usepackage{longtable}
% Note: It may be necessary to compile the document several times to get a multi-page table to line up properly
\begin{longtable}{|l|l|l|}
\hline
\textbf{执行策略类型} &
\textbf{全局实例} &
\textbf{描述} \\ \hline
\endfirsthead
%
\endhead
%
sequenced\_policy &
seq &
\begin{tabular}[c]{@{}l@{}}算法不允许并行化或向量化其执行。\end{tabular} \\ \hline
parallel\_policy &
par &
\begin{tabular}[c]{@{}l@{}}算法允许并行化但不允许向量化其执行。
\end{tabular} \\ \hline
parallel\_unsequenced\_policy &
par\_unseq &
\begin{tabular}[c]{@{}l@{}}算法允许并行化和向量化其执行,也允许\\在多个线程之间迁移执行。
\end{tabular} \\ \hline
unsequenced\_policy &
unseq &
\begin{tabular}[c]{@{}l@{}}算法允许向量化,但不允许并行化其执行。
\end{tabular} \\ \hline
\end{longtable}

标准库实现可以添加执行策略。

如何为算法指定一个执行策略。以下是使用并行策略对vector内容进行排序的示例:

\begin{cpp}
sort(execution::par, begin(values), end(values));
\end{cpp}

\begin{myWarning}{WARNING}
传递给并行算法的回调,不允许抛出未捕获的异常。这样将触发将调用 std::terminate(),从而终止运行中的应用程序。
\end{myWarning}

使用 parallel\_unsequenced\_policy 或 unsequenced\_policy 执行的算法,允许回调函数交错执行,也就是无序执行。这有助于编译器向量化代码,但这也意味着对回调函数可以做什么有大量的限制。例如,不能分配/释放内存,不能获取互斥锁,不能使用非锁定的 std::atomic(请参见第 27 章)等。对于其他标准策略,函数调用是有序的,但顺序不确定。这样的策略不会对回调函数可以做什么施加限制。

并行算法不防止数据竞争和死锁,当以并行方式执行算法时,避免数据竞争和死锁是开发者的责任。数据竞争和死锁的预防在第 27 章中详细介绍,背景是多线程编程。

并行版本的算法的返回类型与非并行版本可能略有不同。非并行版本的 for\_each() 返回提供的回调,而并行版本则不返回任何内容。请查阅标准库手册,以获取所有算法的完整概述,包括参数和返回类型,以及并行和非并行版本。

\begin{myNotic}{NOTE}
当处理大型数据集或必须对数据集中的每个单独元素执行大量工作时,使用并行版本的算法可提高性能。
\end{myNotic}

算法的并行版本并不保证执行速度会比非并行版本快。当处理少量元素时,并行版本可能实际上会更慢,因为并行化带来了开销。另一个例子是当容器不支持随机访问迭代器时。要决定是否为特定用例使用并行或顺序版本,必须对两者进行基准测试,并选择性能最佳的版本。第 29 章介绍了基准测试。

\mySubsubsection{20.2.10.}{数值处理算法}

之前有一个数值处理算法的示例:accumulate(),接下来的部分将提供一些更多数值算法的示例。

\mySamllsection{iota}

iota() 算法,定义在 <numeric> 中,生成一个指定范围内的值序列,从指定的值开始,并应用 operator++ 来生成每个后续的值。以下示例展示了如何使用此算法对整数vector进行操作,适用于实现 operator++ 的元素类型:

\begin{cpp}
vector<int> values(10);
iota(begin(values), end(values), 5);
println("{:n}", values);
\end{cpp}

输出如下所示:

\begin{shell}
5, 6, 7, 8, 9, 10, 11, 12, 13, 14
\end{shell}

\mySamllsection{Reduce 算法}

标准库有四个 reduce 算法:accumulate(), reduce(), inner\_product()和 transform\_reduce(), 都定义在 <numeric> 中。accumulate() 算法在本章前面已经讨论过了。reduce 算法都反复应用一个操作符来组合给定范围内的两个元素或两个给定的范围,直到只剩下一个值。这些也称为累积、聚合、压缩、注入或折叠算法。

\mySamllsection{reduce}

std::accumulate() 是少数几个不支持并行执行的算法之一。相反,需要使用 std::reduce() 来计算一个广义的和,并可以选择以并行方式执行。

以下代码两次计算相同的和,一次使用 accumulate(),一次使用 reduce()。后者运行一个并行和向量化的版本,因此在大输入范围内可以快得多。需要一个范围的开头和结束迭代器,以及一个初始值,例如 0。

\begin{cpp}
vector values { 1, 3, 6, 4, 6, 9 };
int result1 { accumulate(cbegin(values), cend(values), 0) };
int result2 { reduce(execution::par_unseq, cbegin(values), cend(values), 0) };
\end{cpp}

accumulate() 和 reduce() 都计算以下和,对于范围 [$x_0$, $x_{n}$),给定的初始值 Init,和给定的二元操作符 Ѳ:

Init Ѳ $x_0$ Ѳ $x_1$ Ѳ . . . Ѳ $x_{n−1}$

默认情况下,accumulate() 的二元操作符是 operator+,而 reduce() 的二元操作符是 std::plus。

\mySamllsection{inner\_product}

inner\_product() 计算两个序列的内积。以下示例的内积计算为 (1x9)+(2x8)+(3x7)+(4x6),结果是 70:

\begin{cpp}
vector v1 { 1, 2, 3, 4 };
vector v2 { 9, 8, 7, 6 };
println("{}", inner_product(cbegin(v1), cend(v1), cbegin(v2), 0));
\end{cpp}

inner\_product() 可以接受两个参数,这两个参数是计算中使用的两个二元操作符,默认分别是 operator+ 和 operator*。

inner\_product() 是另一个不支持并行执行的算法。如果需要并行执行,请使用 transform\_reduce(),将在下一节介绍。

\mySamllsection{transform\_reduce}

transform\_reduce() 支持并行执行,可以对单个范围或两个范围进行执行。在其第一个版本中,它计算给定范围 [$x_0$, $x_n$),给定的初始值 Init,给定的单函数 f,以及给定的二元操作符 Ѳ(默认是 std::plus)的以下和:

Init Ѳ $f(x_0)$ Ѳ $f(x_1)$ Ѳ . . . Ѳ $f(x_{n−1})$

当对两个范围执行时,与 inner\_product() 行为相同,只是默认使用 std::plus 和 std::multiplies 代替 operator+ 和 operator*。

\mySamllsection{扫描算法}

扫描算法也称为前缀和、累积和或部分和算法。应用于一个范围的此类算法的结果是另一个范围,其中包含源范围元素的和。

有五个扫描算法:exclusive\_scan(), inclusive\_scan()/partial\_sum(), transform\_exclusive\_scan(), 和 transform\_inclusive\_scan(), 都定义在 <numeric> 中。

以下表格显示了 exclusive\_scan() 和 inclusive\_scan()/partial\_sum() 对于一个范围 [$x_0$, $x_n$) 的元素,使用给定的初始值 Init(对于 partial\_sum() 为 0),以及给定的二元操作符 Ѳ,计算哪些和 [$y_0$, $y_n$):

% Please add the following required packages to your document preamble:
% \usepackage{longtable}
% Note: It may be necessary to compile the document several times to get a multi-page table to line up properly
\begin{longtable}{|l|l|}
\hline
\textbf{exclusive\_scan()} &
\textbf{inclusive\_scan()/partial\_sum()} \\ \hline
\endfirsthead
%
\endhead
%
\begin{tabular}[c]{@{}l@{}}$y_0$ = Init\\ $y_1$ = Init Ѳ $x_0$\\ $y_2$ = Init Ѳ $x_0$ Ѳ $x_1$\\ . . .\\ $y_{n-1}$ = Init Ѳ $x_0$ Ѳ $x_1$ Ѳ . . . Ѳ $x_{n-2}$\end{tabular} &
\begin{tabular}[c]{@{}l@{}}$y_0$ = Init Ѳ $x_0$\\ $y_1$ = Init Ѳ $x_0$ Ѳ $x_1$\\ . . .\\ $y_{n-1}$ = Init Ѳ $x_0$ Ѳ $x_1$ Ѳ . . . Ѳ $x_{n-1}$\end{tabular} \\ \hline
\end{longtable}

transform\_exclusive\_scan() 和 transform\_inclusive\_scan() 都首先应用一个单函数到元素上,然后再计算广义的和,这与 transform\_reduce() 在减少元素之前应用单函数的方式类似。

这些扫描算法(除了 partial\_sum())可以接受一个可选的执行策略来并行化它们的执行。计算的顺序不确定,而对于 partial\_sum() 和 accumulate(),计算的顺序是从左到右。这也是为什么 partial\_sum() 和 accumulate() 不能并行化的原因。

\mySubsubsection{20.2.11.}{约束算法}

大多数算法在 std::ranges 命名空间中都有约束版本。查阅标准手册,了解可用的确切约束算法。

这些算法也在 <algorithm> 和 <numeric> 中定义,但与 std 命名空间中相应的无约束算法不同,约束变体使用概念(请参阅第 12 章)来约束模板类型参数。如果传递无效的参数,编译器会提供更好的错误消息。例如,sort() 算法需要随机访问迭代器。将一对 std::list 迭代器作为参数传递给 std::sort(),可能导致编译器抛出一大堆难以理解的错误。使用约束的 ranges::sort() 算法,编译器会抱怨,传递的迭代器不能随机访问。

这些约束算法的另一个好处是,可以处理作为一对开始和结束迭代器或范围给出的序列的元素。此外,它们支持投影。范围和投影在第 17 章中讨论。

\begin{myNotic}{NOTE}
约束算法目前还不支持并行执行,因此接受并行执行策略作为参数。
\end{myNotic}

来看看这些约束算法的一些示例。

\mySamllsection{约束版find}

与所有约束算法一样,std::ranges::find() 约束算法可以用迭代器对或范围作为参数调用。使用迭代器对调用它与无约束 std::find() 的方式相同:

\begin{cpp}
vector values {1, 2, 3};
auto result { ranges::find(cbegin(values), cend(values), 2) };
if (result != cend(values)) { println("{}", *result); }
\end{cpp}

如果想对容器中的所有元素应用一个算法,必须指定一个开始/结束迭代器对来定义序列,这相当繁琐。有了范围支持,只需指定一个范围作为单个参数,前面的 find() 可以写得更简单:

\begin{cpp}
auto result { ranges::find(values, 2) };
\end{cpp}

\mySamllsection{约束版 generate}

这里是一个使用约束版 std::ranges::generate() 算法的示例。代码首先创建了一个简单的 Lambda 表达式,只是返回下一个数字。创建了一个包含 10 个整数的vector,并使用 generate() 算法与 nextNumber Lambda 表达式一起填充vector,以包含递增的整数。vector的内容打印到控制台,接着是对 nextNumber Lambda 表达式的四次或更多次调用。

\begin{cpp}
auto nextNumber { [counter = 0] () mutable { return ++counter; } };
vector<int> values(10);
ranges::generate(values, nextNumber);
println("Vector contains {:n}", values);
print("Four more next numbers: ");
for (unsigned i { 0 }; i < 4; ++i) { print("{}, ", nextNumber()); }
\end{cpp}

输出为:

\begin{shell}
Vector contains 1, 2, 3, 4, 5, 6, 7, 8, 9, 10
Four more next numbers: 1, 2, 3, 4,
\end{shell}

如输出所示,generate() 复制了 Lambda 表达式。为了避免这种情况,可以使用 std::ref(),如本章前面所解释的,以传递 Lambda 表达式的引用:

\begin{cpp}
ranges::generate(values, ref(nextNumber));
\end{cpp}

输出现在为:

\begin{shell}
Vector contains 1, 2, 3, 4, 5, 6, 7, 8, 9, 10
Four more next numbers: 11, 12, 13, 14,
\end{shell}

\mySamllsection{约束版 for\_each}

以下示例展示了如何使用 std::ranges::for\_each() 算法对一个使用 std::ranges::views::filter(定义在 <ranges> 中)创建的过滤视图进行操作。只保留vector中的偶数值在视图中,这个过滤视图随后会传递给 for\_each(),将值乘以 10。输出vector的内容,确认了vector中只有偶数值乘以了 10。

\begin{cpp}
vector values { 1, 2, 3, 4, 5, 6, 7, 8, 9, 10 };
println("Before: {:n}", values);
ranges::for_each(values | views::filter([](int value) { return value % 2 == 0; }),
    [](int& value) { value *= 10; });
println("After: {:n}", values);
\end{cpp}

输出如下所示:

\begin{shell}
Before: 1, 2, 3, 4, 5, 6, 7, 8, 9, 10
After: 1, 20, 3, 40, 5, 60, 7, 80, 9, 100
\end{shell}

\CXXTwentythreeLogo{-40}{-50}

\mySamllsection{仅约束算法}

C++23 引入了新的算法,这些算法仅作为约束算法可用,都在 std::ranges 命名空间中定义。这些包括非修改序列算法 contains(), contains\_subrange(), starts\_with(), ends\_with(), find\_last(), find\_last\_if()和 find\_last\_if\_not(), 以及折叠算法 fold\_left(), fold\_left\_first(), fold\_right(), fold\_right\_last(), fold\_left\_with\_iter()和 fold\_left\_first\_with\_iter()。

以下是某些非修改序列算法的示例:

\begin{cpp}
vector values { 11, 22, 33, 44, 55 };
vector v { 11, 22 };
println("{} contains 33 = {}", values, ranges::contains(values, 33));
println("{} contains {} = {}", values, v, ranges::contains_subrange(values, v));
println("{} starts with {} = {}", values, v, ranges::starts_with(values, v));
\end{cpp}

会产生以下输出:

\begin{shell}
[11, 22, 33, 44, 55] contains 33 = true
[11, 22, 33, 44, 55] contains [11, 22] = true
[11, 22, 33, 44, 55] starts with [11, 22] = true
\end{shell}

以下是一个折叠算法的示例。fold\_left() 和 fold\_right() 接受一个初始值作为它们的参数之一,而 fold\_left\_first() 使用给定范围中的第一个元素作为起始值,fold\_right\_last() 使用给定范围中的最后一个元素作为起始值。这个示例展示了左折叠和右折叠的区别,fold\_left\_first() 和 fold\_right\_last() 算法返回一个可选值,可以使用 value\_or() 来处理空结果。

\begin{cpp}
vector values { 500.0, 10.0, 2.0 };
auto foldedLeft { ranges::fold_left_first(values, divides<>{}) };
auto foldedRight { ranges::fold_right_last(values, divides<>{}) };
println("foldedLeft = {}", foldedLeft.value_or(0.0));
println("foldedRight = {}", foldedRight.value_or(0.0));
\end{cpp}

输出为:

\begin{shell}
foldedLeft = 25
foldedRight = 100
\end{shell}

左折叠操作计算 ((500.0 / 10.0) / 2.0),而右折叠操作计算 (500.0 / (10.0 / 2.0))。

请参阅标准库手册,以获取这些约束算法的详细信息。











