
std::invoke(),定义在 <functional> 头文件中,可以用来调用可调用对象,并传递一系列参数。以下示例使用了 invoke() 三次:一次调用普通函数,一次调用 lambda 表达式,以及一次调用字符串实例上的成员函数:

\begin{cpp}
int main()
{
    invoke(printMessage, "Hello invoke.");
    invoke([](const auto& msg) { println("{}", msg); }, "Hello invoke.");
    string msg { "Hello invoke." };
    println("{}", invoke(&string::size, msg));
}
\end{cpp}

这段代码的输出如下:

\begin{shell}
Hello invoke.
Hello invoke.
13
\end{shell}

\CXXTwentythreeLogo{-40}{-50}

C++23 增加了 std::invoke\_r(),也定义在 <functional> 中,允许指定返回类型。以下是一个示例:

\begin{cpp}
int sum(int a, int b) { return a + b; }

int main()
{
    auto res1 { invoke(sum, 11, 22) }; // Type of res1 is int.
    auto res2 { invoke_r<double>(sum, 11, 22) }; // Type of res2 is double.
}
\end{cpp}

\begin{myNotic}{NOTE}
单独来看,invoke() 和 invoke\_r() 并不是那么有用,因为可以直接调用函数或 Lambda 表达式。但当编写泛型模板代码,需要调用一些可调用对象时,它们就非常有用了。
\end{myNotic}






