如前节所述,可以创建和使用独立函数的指针,也可以处理独立变量的指针。现在,考虑指向类数据成员和成员函数的指针。C++中获取类数据成员和成员函数的地址以获得指向它们的指针完全合法。然而,不能在没有对象的情况下,访问非静态数据成员或调用非静态成员函数。类数据成员和成员函数的意义在于基于对象而存在,当想通过指针调用成员函数或访问数据成员时,必须在对象的上下文中取消指针的引用。这里使用第1章的Employee类:

\begin{cpp}
int (Employee::*functionPtr) () const { &Employee::getSalary };
Employee employee { "John", "Doe" };
println("{}", (employee.*functionPtr)());
\end{cpp}

不要对语法感到恐慌。第一行声明了一个名为functionPtr的变量,其类型是指向Employee类的非静态const成员函数的指针,该函数不接受参数并返回一个int。同时,将这个变量初始化为指向Employee类的getSalary()成员函数。这个语法与声明一个简单函数指针非常相似,只是增加了Employee::在*functionPtr之前。注意,这种情况下需要\&。

第三行调用employee对象的getSalary()成员函数(通过functionPtr指针),注意使用括号包围employee.*functionPtr。这些是必需的,因为operator()的优先级高于.*。

如果有一个指向对象的指针,可以使用->*代替.*:

\begin{cpp}
int (Employee::*functionPtr) () const { &Employee::getSalary };
Employee johnD { "John", "Doe" };
Employee* employee { &johnD };
println("{}", (employee->*functionPtr)());
\end{cpp}

可以使用类型别名使functionPtr的定义更易理解:

\begin{cpp}
using PtrToGet = int (Employee::*) () const;
PtrToGet functionPtr { &Employee::getSalary };
Employee employee { "John", "Doe" };
println("{}", (employee.*functionPtr)());
\end{cpp}

最后,可以使用auto进一步简化:

\begin{cpp}
auto functionPtr { &Employee::getSalary };
Employee employee { "John", "Doe" };
println("{}", (employee.*functionPtr)());
\end{cpp}

\begin{myNotic}{NOTE}
可以使用std::mem\_fn(),这在本书后面介绍函数对象时会解释,以摆脱.*或->*。
\end{myNotic}

指向成员函数和数据成员的指针在日常编程中不会经常出现,但不能在没有对象的情况下,取消引用指向非静态成员函数或数据成员的指针。有时,可能想尝试将指向非静态成员函数的指针,传递给需要函数指针的函数,如qsort(),但这根本行不通。

\begin{myNotic}{NOTE}
C++允许在没有对象的情况下,取消引用指向静态数据成员或静态成员函数的指针。
\end{myNotic}













