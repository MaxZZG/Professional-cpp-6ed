\noindent
\textbf{内筒概要}

\begin{itemize}
\item
如何使用函数指针

\item
如何使用指针类成员函数

\item
什么是函数对象

\item
有哪些标准函数对象可用,以及如何编写自己的函数对象

\item
如何使用多态函数包装器

\item
什么是Lambda表达式
\end{itemize}

本章的所有代码示例都可以在\url{https://github.com/Professional-CPP/edition-6}获得。

C++中,函数是一等函数,因为函数可以像普通变量一样使用,例如:将它们作为参数传递给其他函数,从其他函数返回它们,以及将它们赋值给变量。经常出现的一个术语是回调(callback),代表可以调用的对象。它可以是一个函数指针,或者像函数指针一样的行为,例如:重载了operator()的对象,或者一个内联的Lambda表达式。重载了operator()的类称为函数对象,或者简称为仿函数。方便的是,标准库提供了一系列类,可以用来创建回调对象,以及适配现有的回调对象。Lambda表达式允许在需要的地方创建小的内联回调,这可以提高代码的可读性和可维护性。现在是时候仔细地了解回调的概念,下一章中的许多算法接受这样的回调进行行为定制。
