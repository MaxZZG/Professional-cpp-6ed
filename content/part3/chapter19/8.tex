通过解决下面的练习,可以练习本章讨论的内容。所有练习的解决方案都可以在本书的网站\url{www.wiley.com/go/proc++6e}下载到源码。然而,若在练习中卡住了,在从网站上寻找解决方案之前,可以考虑先重读本章的部分内容,试着自己找到答案。

\begin{itemize}
\item
\textbf{练习 19-1}:使用 Lambda 表达式重写本章中的 IsLargerThan 函数对象示例。可以在下载源代码档案中找到代码,位置在 Ch19\verb|\|03\_FunctionObjects\verb|\|01\_IsLargerThan.cpp。

\item
\textbf{练习 19-2}:使用 Lambda 表达式重写 bind() 的示例。可以在下载源代码档案中找到代码,位置在 Ch19\verb|\|03\_FunctionObjects\verb|\|07\_bind.cpp。

\item
\textbf{练习 19-3}:\\使用 Lambda 表达式重写绑定类成员函数 Handler::handleMatch() 的示例。可以在下载源代码档案中找到代码,位置在 Ch19\verb|\|03\_FunctionObjects\verb|\|10\_FindMatchesWithMemberFunctionPointer.cpp。

\item
\textbf{练习 19-4}:第 18 章介绍了 std::erase\_if() 函数,用于从容器中删除满足特定谓词返回 true 的元素。现在了解了回调的知识,请编写一个小型程序,创建一个整数vector,然后使用 erase\_if() 删除向量中的所有奇数值。需要传递给 erase\_if() 的谓词应该接受一个单一参数,并返回布尔值。

\item
\textbf{练习 19-5}:实现一个名为 Processor 的类,构造函数应该接受一个接受单个整数并返回整数的回调,将这个回调存储在类的数据成员中。接下来,为函数调用运算符添加一个重载,接受一个整数并返回整数。实现简单地将工作转发给存储的回调,用不同的回调测试这个类。

\item
\textbf{练习 19-6}:编写一个递归 Lambda 表达式来计算一个数的幂。例如,4 的 3 次幂,写作 $4^3$,等于 4×4×4。确保它适用于负指数。为了帮助您,$4^{-3}$ 等于 $1/4^3$。任何数的 0 次幂等于 1。通过生成 -10 到 10 之间的所有 2的幂来测试 Lambda 表达式。
\end{itemize}