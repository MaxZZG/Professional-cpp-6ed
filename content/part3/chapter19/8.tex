通过解决下面的练习,可以练习本章讨论的内容。所有练习的解决方案都可以在本书的网站\url{www.wiley.com/go/proc++6e}下载到源码。然而,若在练习中卡住了,在从网站上寻找解决方案之前,可以考虑先重读本章的部分内容,试着自己找到答案。

\begin{itemize}
\item
Exercise 19-1: Rewrite the IsLargerThan function object example from this chapter using a lambda expression. You can find the code in the downloadable source code archive in Ch19\verb|\|03\_FunctionObjects\verb|\|01\_IsLargerThan.cpp.

\item
Exercise 19-2: Rewrite the example given for bind() using lambda expressions instead. You can find the code in the downloadable source code archive in Ch19\verb|\|03\_FunctionObjects\verb|\|07\_bind.cpp.

\item
Exercise 19-3: Rewrite the example given for binding the class member function Handler::handleMatch() using lambda expressions instead. You can find the code in the downloadable source code archive in Ch19\verb|\|03\_FunctionObjects\verb|\|10\_FindMatchesWithMemberFunctionPointer.cpp.

\item
Exercise 19-4: Chapter 18 introduces the std::erase\_if() function to remove elements from a container for which a certain predicate returns true. Now that you know everything about callbacks, write a small program that creates a vector of integers, and then uses erase\_if() to remove all odd values from the vector. The predicate you need to pass to erase\_if() should accept a single value and return a Boolean.

\item
Exercise 19-5: Implement a class called Processor. The constructor should accept a callback accepting a single integer and returning an integer. Store this callback in a data member of the class. Next, add an overload for the function call operator accepting an integer and returning an integer. The implementation simply forwards the work to the stored callback. Test your class with different callbacks.

\item
Exercise 19-6: Write a recursive lambda expression to calculate the power of a number. For example, 4 to the power 3, written as 4\^{}3, equals 4×4×4. Make sure it works with negative exponents. To help you, 4\^{}-3 equals 1/(4\^{}3). Any number to the power 0 equals 1. Test your lambda expression by generating all powers of two with exponents between -10 and 10.
\end{itemize}