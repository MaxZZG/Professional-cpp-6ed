\noindent
\textbf{WHAT’S IN THIS CHAPTER?}

\begin{itemize}
\item
How to work with compile-time rational numbers

\item
How to work with time

\item
How to work with dates and calendars

\item
How to convert time points between different time zones
\end{itemize}

\noindent
\textbf{WILEY.COM DOWNLOADS FOR THIS CHAPTER}

Please note that all the code examples for this chapter are available as part of this chapter’s code download on the book’s website at \url{www.wiley.com/go/proc++6e} on the Download Code tab.

This chapter discusses the time-related functionality provided by the C++ Standard Library, known collectively as the chrono library. It is a collection of classes and functions to work with time and dates. The library consists of the following components:

\begin{itemize}
\item
Durations

\item
Clocks

\item
Time points

\item
Dates

\item
Time zones
\end{itemize}

Everything is defined in <chrono> in the std::chrono namespace. However, before we can start the discussion of each of these chrono library components, we need a small digression to look at the compile-time rational number support available in C++, as this is heavily used by the chrono library.

