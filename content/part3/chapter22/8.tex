通过解决下面的练习,可以练习本章讨论的内容。所有练习的解决方案都可以在本书的网站\url{www.wiley.com/go/proc++6e}下载到源码。然而,若在练习中卡住了,在从网站上寻找解决方案之前,可以考虑先重读本章的部分内容,试着自己找到答案。

\begin{itemize}
\item
\textbf{练习 22-1}: 创建一个精确到秒的时间段 d1,初始化为 42 秒。创建第二个时间段 d2,其精度为分钟,初始化为 1.5 分钟。计算 d1 和 d2 的和。将结果以秒和分钟的形式写入标准输出。

\item
\textbf{练习 22-2}: 要求用户输入日期格式为 yyyy-mm-dd,例如 2020-06-22。使用正则表达式(请参阅第 21 章)提取年、月和日组件,并最终使用 year\_month\_day 来验证日期。

\item
\textbf{练习 22-3}: 编写一个 getNumberOfDaysBetweenDates() 函数,计算两个给定日期之间相差的天数,并在 main() 函数中测试实现。

\item
\textbf{练习 22-4}: 编写一个程序,输出 2020 年 6 月 22 日是星期几。

\item
\textbf{练习 22-5}: 构建一个 UTC 时间,将这个时间转换为日本东京的本地时间。进一步将转换后的时间转换为纽约时间。最后将转换后的时间转换为 GMT。验证原始 UTC 时间与最终 GMT 时间是否相等。提示:东京的时区标识符是 Asia/Tokyo,纽约是 America/New\_York,而 GMT 是 GMT。

\item
\textbf{练习 22-6}: 编写一个 getDurationSinceMidnight() 函数,该函数返回从午夜到当前本地时间的时间段(以秒为单位)。使用函数将自午夜以来的秒数,打印到标准输出控制台。最后,使用 hh\_mm\_ss 类将函数返回的时间段转换为小时、分钟和秒,并将结果打印到标准输出。
\end{itemize}











