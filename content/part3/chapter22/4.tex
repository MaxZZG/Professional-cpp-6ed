A point in time is represented by the time\_point class and stored as a duration relative to an epoch, representing the beginning of time. A time\_point is always associated with a certain clock, and the epoch is the origin of this associated clock. For example, the epoch for the classic Unix/Linux time is January 1, 1970, and durations are measured in seconds. The epoch for Windows is January 1, 1601, and durations are measured in 100-nanosecond units. Other operating systems have different epoch dates and duration units.

The time\_point class has a function called time\_since\_epoch(), which returns a duration representing the time between the epoch of the associated clock and the stored point in time.

Arithmetic operations of time\_points and durations that make sense are supported. The following table lists those operations. tp is a time\_point, and d is a duration:

% Please add the following required packages to your document preamble:
% \usepackage{longtable}
% Note: It may be necessary to compile the document several times to get a multi-page table to line up properly
\begin{longtable}{|l|l|}
\hline
tp + d = tp & tp – d = tp \\ \hline
\endfirsthead
%
\endhead
%
d + tp = tp & tp – tp = d \\ \hline
tp += d     & tp -= d     \\ \hline
\end{longtable}

An example of an operation that is not supported is tp+tp.

Comparison operators == and <=> to compare two time points are supported. Two static member functions are provided: min() and max() returning the smallest and largest possible point in time, respectively.

The time\_point class has three constructors:

\begin{itemize}
\item
time\_point(): Constructs a time\_point initialized with duration::zero(). The resulting time\_point represents the epoch of the associated clock.

\item
time\_point(const duration\& d): Constructs a time\_point initialized with the given duration. The resulting time\_point represents epoch + d.

\item
template<class Duration2> time\_point(const time\_point<clock, Duration2>\& t): Constructs a time\_point initialized with t.time\_since\_epoch().
\end{itemize}

Each time\_point is associated with a clock. To create a time\_point, you specify the clock as the template parameter:

\begin{cpp}
time_point<steady_clock> tp1;
\end{cpp}

Each clock also knows its time\_point type, so you can also write it as follows:

\begin{cpp}
steady_clock::time_point tp1;
\end{cpp}

The following code snippet demonstrates some operations with time\_points:

\begin{cpp}
// Create a time_point representing the epoch of the associated steady clock.
time_point<steady_clock> tp1;
// Add 10 minutes to the time_point.
tp1 += minutes { 10 };
// Store the duration between epoch and time_point.
auto d1 { tp1.time_since_epoch() };
// Convert the duration to seconds and output to the console.
duration<double> d2 { d1 };
println("{}", d2);
\end{cpp}

The output is as follows:

\begin{shell}
600s
\end{shell}

Converting time\_points can be done implicitly or explicitly, similar to duration conversions. Here is an example of an implicit conversion. The output is 42000ms:

\begin{cpp}
time_point<steady_clock, seconds> tpSeconds { 42s };
// Convert seconds to milliseconds implicitly.
time_point<steady_clock, milliseconds> tpMilliseconds { tpSeconds };
println("{}", tpMilliseconds.time_since_epoch());
\end{cpp}

If the implicit conversion can result in a loss of data, then you need an explicit conversion using time\_point\_cast(), similar to using duration\_cast() for explicit duration casts as discussed earlier in this chapter. The following example outputs 42000ms, even though you start from 42,424ms:

\begin{cpp}
time_point<steady_clock, milliseconds> tpMilliseconds { 42'424ms };
// Convert milliseconds to seconds explicitly.
time_point<steady_clock, seconds> tpSeconds {
    time_point_cast<seconds>(tpMilliseconds) };
// Or:
// auto tpSeconds { time_point_cast<seconds>(tpMilliseconds) };
// Convert seconds back to milliseconds and output the result.
milliseconds ms { tpSeconds.time_since_epoch() };
println("{}", ms);
\end{cpp}

The library supports floor(), ceil(), and round() operations for time\_points that behave just as they behave with numerical data.
