
时间点由 time\_point 类表示,并作为与某个时钟相关的持续时间的起点存储。时间点总是与某个时钟的纪元相关联,这个纪元表示这个时钟的起点。例如,Unix/Linux 时间点的纪元是 1970 年 1 月 1 日,持续时间以秒为单位。Windows 时间点的纪元是 1601 年 1 月 1 日,持续时间以 100 纳秒为单位。其他操作系统有不同的纪元日期和持续时间单位。

time\_point 类有一个名为 time\_since\_epoch() 的函数返回一个时间段,表示存储的时间点与关联时钟的纪元之间的时间差。

time\_point 类支持一些有意义的算术运算。以下表格列出了这些运算:

% Please add the following required packages to your document preamble:
% \usepackage{longtable}
% Note: It may be necessary to compile the document several times to get a multi-page table to line up properly
\begin{longtable}{|l|l|}
\hline
tp + d = tp & tp – d = tp \\ \hline
\endfirsthead
%
\endhead
%
d + tp = tp & tp – tp = d \\ \hline
tp += d     & tp -= d     \\ \hline
\end{longtable}

不支持的运算示例是 tp+tp。

time\_point 类支持比较运算符 == 和 <=>,用于比较两个时间点。有两个静态成员函数 min() 和 max(),分别返回可能的最小和最大时间点。

time\_point 类有三个构造函数:

\begin{itemize}
\item
time\_point(): 使用duration::zero()构造一个初始化的时间点,得到的time\_point代表关联时钟的纪元。

\item
time\_point(const duration\& d): 使用给定的持续时间d构造一个初始化的time\_point。得到的time\_point代表epoch + d。

\item
template<class Duration2> time\_point(const time\_point<clock, Duration2>\& t): 使用t.time\_since\_epoch()构造一个初始化的time\_point。
\end{itemize}

每个time\_point都与一个时钟相关联。要创建一个time\_point,需要将时钟指定为模板参数:

\begin{cpp}
time_point<steady_clock> tp1;
\end{cpp}

每个时钟也知道其time\_point类型,可以按以下方式使用:

\begin{cpp}
steady_clock::time_point tp1;
\end{cpp}

下代码段演示了使用time\_point的一些操作:

\begin{cpp}
// Create a time_point representing the epoch of the associated steady clock.
time_point<steady_clock> tp1;
// Add 10 minutes to the time_point.
tp1 += minutes { 10 };
// Store the duration between epoch and time_point.
auto d1 { tp1.time_since_epoch() };
// Convert the duration to seconds and output to the console.
duration<double> d2 { d1 };
println("{}", d2);
\end{cpp}

输出如下:

\begin{shell}
600s
\end{shell}

time\_points的转换可以隐式或显式地进行,类似于时间段的转换。以下是一个隐式转换的例子,输出是42000ms:

\begin{cpp}
time_point<steady_clock, seconds> tpSeconds { 42s };
// Convert seconds to milliseconds implicitly.
time_point<steady_clock, milliseconds> tpMilliseconds { tpSeconds };
println("{}", tpMilliseconds.time_since_epoch());
\end{cpp}

如果隐式转换可能导致数据丢失,那么需要使用time\_point\_cast()进行显式转换,类似于本章前面讨论的duration\_cast()用于显式时间段转换。以下示例即使从42,424ms开始,输出的也是42000ms:

\begin{cpp}
time_point<steady_clock, milliseconds> tpMilliseconds { 42'424ms };
// Convert milliseconds to seconds explicitly.
time_point<steady_clock, seconds> tpSeconds {
    time_point_cast<seconds>(tpMilliseconds) };
// Or:
// auto tpSeconds { time_point_cast<seconds>(tpMilliseconds) };
// Convert seconds back to milliseconds and output the result.
milliseconds ms { tpSeconds.time_since_epoch() };
println("{}", ms);
\end{cpp}

库支持floor()、ceil()和round()操作,这些操作对time\_point的行为和对数字数据的行为一样。
