
时间段是两个时间点之间的间隔,由 duration 类模板表示,该模板存储了时间戳的数量和时间戳周期。时间戳周期是两个时间戳之间的时间,以秒为单位,并且作为编译时的有理数常量表示,所以可以是秒的分数(上一节中进行了讨论)。duration 模板接受两个模板类型参数,定义如下:

\begin{cpp}
template <class Rep, class Period = ratio<1>> class duration {...}
\end{cpp}

第一个模板参数Rep,是存储时间戳数量的变量的类型,应该是一个算术类型,例如 long、double 等。第二个模板参数Period,是有理常数,代表时间戳的周期。如果没有指定时间戳周期,将使用默认值 ratio<1> ,这表示时间周期为一秒钟。

提供了三个构造函数:默认构造函数;一个接受单个值的构造函数,该值是时间戳的数量;以及一个接受另一个时间段的构造函数。后一个构造函数可以用来将一个时间段转换为另一个时间段,例如:将分钟转换为秒。本节稍后将给出一个示例。

时间段支持算术运算,如 +、-、*、/、\%、++、--、+=、-=、*=、/= 和 \%=,并且支持比较运算符 == 和 <=>。该类还包含以下表中的成员函数:

% Please add the following required packages to your document preamble:
% \usepackage{longtable}
% Note: It may be necessary to compile the document several times to get a multi-page table to line up properly
\begin{longtable}{|l|l|}
\hline
\textbf{成员函数} & \textbf{描述}                                         \\ \hline
\endfirsthead
%
\endhead
%
Rep count() const &
\begin{tabular}[c]{@{}l@{}}返回时间段值作为刻度的数量。返回类型是作为时间段模板的第一个模板\\类型参数指定的类型。
\end{tabular} \\ \hline
static duration zero()   & 返回一个值为零的时间段。
 \\ \hline
\begin{tabular}[c]{@{}l@{}}static duration min()\\ static duration max()\end{tabular} &
\begin{tabular}[c]{@{}l@{}}返回一个时间段,其值为由时间段模板的第一个模板类型参数指定类型的\\最小/最大值。
\end{tabular} \\ \hline
\end{longtable}

库支持对时间段进行 floor()、ceil()、round() 和 abs() 操作,这些操作的行为与数字数据相同。

现在,来看看如何定义时间段。一个时间戳周期为一秒钟的时间段,定义如下:

\begin{cpp}
duration<long> d1;
\end{cpp}

因为 ratio<1> 是默认的时间戳周期,这等同于:

\begin{cpp}
duration<long, ratio<1>> d1;
\end{cpp}

下一个语句定义了一个以分钟为单位(时间周期 = 60 秒)的时间段:

\begin{cpp}
duration<long, ratio<60>> d2;
\end{cpp}

下面是一个时间戳周期为每秒六十分之一的时间段:

\begin{cpp}
duration<double, ratio<1, 60>> d3;
\end{cpp}

<ratio> 定义了一系列 SI 有理常数,这些预定义的常数在定义时间戳周期时非常有用。例如,下一个语句定义了一个时间戳周期为每毫秒的时间段:

\begin{cpp}
duration<long long, milli> d4;
\end{cpp}

\mySubsubsection{22.2.1.}{转换时间段和一些示例}

让我们看看时间段是如何工作的。以下示例演示了时间段的几种方式,展示了如何定义它们,如何对它们执行算术运算,如何将它们打印到屏幕上,以及如何将一个具有不同时间周期的时间段转换为另一个时间段:

\begin{cpp}
// Specify a duration where each tick is 60 seconds.
duration<long, ratio<60>> d1 { 123 };
println("{} ({})", d1, d1.count());

// Specify a duration represented by a double with each tick
// equal to 1 second and assign the largest possible duration to it.
auto d2 { duration<double>::max() };
println("{}", d2);

// Define 2 durations:
// For the first duration, each tick is 1 minute.
// For the second duration, each tick is 1 second.
duration<long, ratio<60>> d3 { 10 }; // = 10 minutes
duration<long, ratio<1>> d4 { 14 }; // = 14 seconds

// Compare both durations.
if (d3 > d4) { println("d3 > d4"); }
else { println("d3 <= d4"); }

// Increment d4 with 1 resulting in 15 seconds.
++d4;

// Multiply d4 by 2 resulting in 30 seconds.
d4 *= 2;

// Add both durations and store as minutes.
duration<double, ratio<60>> d5 { d3 + d4 };

// Add both durations and store as seconds.
duration<long, ratio<1>> d6 { d3 + d4 };
println("{} + {} = {} or {}", d3, d4, d5, d6);

// Create a duration of 30 seconds.
duration<long> d7 { 30 };

// Convert the seconds of d7 to minutes.
duration<double, ratio<60>> d8 { d7 };
println("{} = {}", d7, d8);
println("{} seconds = {} minutes", d7.count(), d8.count());
\end{cpp}

输出为:

\begin{shell}
123min (123)
1.79769e+308s
d3 > d4
10min + 30s = 10.5min or 630s
30s = 0.5min
30 seconds = 0.5 minutes
\end{shell}

\begin{myNotic}{NOTE}
输出中的第二行代表了使用 double 类型的最大可能时间段,具体值可能因编译器而异。
\end{myNotic}

请特别注意以下两行代码:

\begin{cpp}
duration<double, ratio<60>> d5 { d3 + d4 };
duration<long, ratio<1>> d6 { d3 + d4 };
\end{cpp}

都计算了 d3 + d4,其中 d3 以分钟为单位,d4 以秒为单位,但第一句将结果存储为表示分钟的浮点值,而第二句将结果存储为表示秒的整数值。从分钟到秒,或反之,转换会自动发生。

以下两行代码示例展示了,如何显式地在不同时间单位之间进行转换:

\begin{cpp}
duration<long> d7 { 30 }; // seconds
duration<double, ratio<60>> d8 { d7 }; // minutes
\end{cpp}

第一句定义了一个表示 30 秒的时间段。第二句将这些 30 秒转换为分钟,结果为 0.5 分钟。这种转换可能会导致非整数值,因此需要使用浮点类型的时间段;否则,将得到一些神秘的编译错误。例如,以下代码不会编译,因为 d8 使用 long 而不是浮点类型:

\begin{cpp}
duration<long> d7 { 30 }; // seconds
duration<long, ratio<60>> d8 { d7 }; // minutes // Error!
\end{cpp}

然而,可以使用 duration\_cast() 强制进行这种转换:

\begin{cpp}
duration<long> d7 { 30 }; // seconds
auto d8 { duration_cast<duration<long, ratio<60>>>(d7) }; // minutes
\end{cpp}

这种情况下,d8 将是 0 分钟,因为使用了整数除法将 30 秒转换为分钟。

如果源是整数类型,则不需要浮点类型即可进行反向转换,因为如果从整数开始,结果总是整数。例如,以下语句将十分钟转换为秒,两者都使用整数类型 long 表示:

\begin{cpp}
duration<long, ratio<60>> d9 { 10 }; // minutes
duration<long> d10 { d9 }; // seconds
\end{cpp}

\mySubsubsection{22.2.2.}{预定义时间段}

库在 std::chrono 命名空间中提供了以下标准时间段类型:

\begin{cpp}
using nanoseconds = duration<X 64 bits, nano>;
using microseconds = duration<X 55 bits, micro>;
using milliseconds = duration<X 45 bits, milli>;
using seconds = duration<X 35 bits>;
using minutes = duration<X 29 bits, ratio<60>>;
using hours = duration<X 23 bits, ratio<3'600>>;
using days = duration<X 25 bits, ratio_multiply<ratio<24>, hours::period>>;
using weeks = duration<X 22 bits, ratio_multiply<ratio<7>, days::period>>;
using years = duration<X 17 bits,
              ratio_multiply<ratio<146'097, 400>, days::period>>;
using months = duration<X 20 bits, ratio_divide<years::period, ratio<12>>>;
\end{cpp}

X 的确切类型取决于编译器,但 C++ 标准要求至少为指定大小的有符号整数类型。前面的类型别名,使用了在本章早些时候描述的预定义 SI 有理数类型别名。有了这些预定义的类型,就可以这样:

\begin{cpp}
duration<long, ratio<60>> d9 { 10 }; // minutes
\end{cpp}

而不是:

\begin{cpp}
minutes d9 { 10 }; // minutes
\end{cpp}

以下代码是另一个如何使用这些预定义时间段的示例。代码首先定义了一个变量 t,是 1 小时 + 23 分钟 + 45 秒,使用 auto 关键字让编译器自动确定 t 的确切类型。第二句使用预定义的 seconds 时间段构造函数将 t 的值转换为秒,并将其写入控制台:

\begin{cpp}
auto t { hours { 1 } + minutes { 23 } + seconds { 45 } };
println("{}", seconds { t });
\end{cpp}

由于标准要求预定义的时间段使用整数类型,如果转换可能导致非整数值,则可能会出现编译错误。尽管整数除法通常截断,但对于使用比率类型的时间段实现,编译器将可能将非零余数的计算声明为编译时错误。例如,以下代码不会编译,将 90 秒转换为分钟会得到 1.5 分钟:

\begin{cpp}
seconds s { 90 };
minutes m { s };
\end{cpp}

然而,以下代码也不会编译,尽管 60 秒正好等于 1 分钟。其标记为编译时错误,因为从秒转换为分钟可能会导致非整数值:

\begin{cpp}
seconds s { 60 };
minutes m { s };
\end{cpp}

反向转换则完全没问题,因为分钟时间段使用整数类型,并且将其转换为秒总是得到整数值:

\begin{cpp}
minutes m { 2 };
seconds s { m };
\end{cpp}

\mySubsubsection{22.2.3.}{标准字面量}

可以使用标准字面量 h、min、s、ms、us 和 ns 来创建时间段。这些定义在 std::literals::chrono\_literals 命名空间中,但就像在第 2 章中讨论的标准字符串字面量一样,chrono\_literals 命名空间是一个内联命名空间。因此,可以通过以下 using 指令使 chrono 字面量可用:

\begin{cpp}
using namespace std;
using namespace std::literals;
using namespace std::chrono_literals;
using namespace std::literals::chrono_literals;
\end{cpp}

此外,这些字面量也以 std::chrono 命名空间的形式提供。以下是一个示例:

\begin{cpp}
using namespace std::chrono;
// ...
auto myDuration { 42min }; // 42 minutes
\end{cpp}

\mySubsubsection{22.2.4.}{hh\_mm\_ss}

chrono 库提供了 hh\_mm\_ss 类模板,该模板接受一个时间段,并将给定的时间段分割为小时、分钟、秒和亚秒。它有 getter 方法 hours()、minutes()、seconds() 和 subseconds() 来检索数据,返回非负值。如果时间段是负的,is\_negative() 成员函数则返回 true,否则返回 false。在本章末尾的一个练习中会使用到 hh\_mm\_ss 类模板。
