分数库在编译时可以精确表示有限的有理数,其内容都在 <ratio> 中定义,位于 std 命名空间中。有理数的分子和分母以 std::intmax\_t 类型的编译时常量表示,这是一种由编译器支持的,带有最大宽度的有符号整数类型。由于这些有理数是编译时的,使用时看起来与习惯的方式不同。不能像定义普通对象那样定义一个有理数对象,也不能调用其成员函数。而有理数是一个类模板,ratio 类模板的具体实例化代表一个特定的有理数。为了命名这样的特定实例化,可以使用类型别名。以下定义了一个编译时有理数,表示分数 1/60:

\begin{cpp}
using r1 = ratio<1, 60>;
\end{cpp}

r1 有理数的分子(num)和分母(den)是编译时常量,可以按以下方式访问:

\begin{cpp}
intmax_t num { r1::num };
intmax_t den { r1::den };
\end{cpp}

有理数代表一个编译时有理数,多以分子和分母需要在编译时已知。以下代码会有编译错误:

\begin{cpp}
intmax_t n { 1 }; // Numerator
intmax_t d { 60 }; // Denominator
using r1 = ratio<n, d>; // Error
\end{cpp}

将 n 和 d 定义为常量就可以正常工作:

\begin{cpp}
const intmax_t n { 1 }; // Numerator
const intmax_t d { 60 }; // Denominator
using r1 = ratio<n, d>; // Ok
\end{cpp}

有理数总是规范化。对于一个有理数 ratio<n, d>,计算最大公约数 gcd,然后定义分子 num 和分母 den 如下:

\begin{cpp}
num = sign(n)*sign(d)*abs(n)/gcd
\end{cpp}

\begin{cpp}
den = abs(d)/gcd
\end{cpp}

库支持添加、减、乘和除有理数,不能使用标准的算术运算符,所有这些操作都是在编译时进行。需要使用特定的算术有理数类模板,有以下算术类模板可用:ratio\_add、ratio\_subtract、ratio\_multiply 和 ratio\_divide,分别执行加法、减法、乘法和除法。这些模板计算结果作为新的有理数类型,这个类型可以通过称为 type 的嵌入类型别名访问。以下代码首先定义了两个有理数,一个表示 1/60,另一个表示 1/30。

ratio\_add 模板将这两个有理数相加,以产生结果有理数,该结果在规范化后为 1/20:

\begin{cpp}
using r1 = ratio<1, 60>;
using r2 = ratio<1, 30>;
using result = ratio_add<r1, r2>::type;
\end{cpp}

标准还定义了一系列有理数比较类模板:ratio\_equal、ratio\_not\_equal、ratio\_less、ratio\_less\_equal、ratio\_greater 和 ratio\_greater\_equal。与算术有理数类模板一样,比较有理数类模板也在编译时计算。这些比较模板定义了一个新的类型——std::bool\_constant,表示结果。bool\_constant 是一个 struct 模板,用于存储类型和编译时常量值。integral\_constant<int, 15> 存储值为 15 的整数,bool\_constant 是一个 bool 类型的 integral\_constant。bool\_constant<true> 是 integral\_constant<bool, true>,存储值为 true 的布尔值。比较模板的结果是 bool\_constant<true> 或 bool\_constant<false>。与 bool\_constant 或 integral\_constant 关联的值可以使用 value 数据成员访问。以下示例演示了如何使用 ratio\_less:

\begin{cpp}
using r1 = ratio<1, 60>;
using r2 = ratio<1, 30>;
using res = ratio_less<r2, r1>;
println("{}", res::value); // false
\end{cpp}

代码结合了前面讨论的所有内容。由于ratio不是对象,而是类型,不能println("\{\}", r1); 直接打印ratio,需要分别获取分子和分母,并单独打印它们。

\begin{cpp}
// Define a compile-time rational number.
using r1 = ratio<1, 60>;

// Get numerator and denominator.
intmax_t num { r1::num };
intmax_t den { r1::den };
println("1) r1 = {}/{}", num, den);

// Add two rational numbers.
using r2 = ratio<1, 30>;
println("2) r2 = {}/{}", r2::num, r2::den);
using result = ratio_add<r1, r2>::type;
println("3) sum = {}/{}", result::num, result::den);

// Compare two rational numbers.
using res = ratio_less<r2, r1>;
println("4) r2 < r1: {}", res::value);
\end{cpp}

输出为:

\begin{shell}
1) r1 = 1/60
2) r2 = 1/30
3) sum = 1/20
4) r2 < r1: false
\end{shell}

库提供了一系列 SI (Système International) 类型别名,以便于使用:

\begin{cpp}
using yocto = ratio<1, 1'000'000'000'000'000'000'000'000>; // *
using zepto = ratio<1, 1'000'000'000'000'000'000'000>; // *
using atto = ratio<1, 1'000'000'000'000'000'000>;
using femto = ratio<1, 1'000'000'000'000'000>;
using pico = ratio<1, 1'000'000'000'000>;
using nano = ratio<1, 1'000'000'000>;
using micro = ratio<1, 1'000'000>;
using milli = ratio<1, 1'000>;
using centi = ratio<1, 100>;
using deci = ratio<1, 10>;
using deca = ratio<10, 1>;
using hecto = ratio<100, 1>;
using kilo = ratio<1'000, 1>;
using mega = ratio<1'000'000, 1>;
using giga = ratio<1'000'000'000, 1>;
using tera = ratio<1'000'000'000'000, 1>;
using peta = ratio<1'000'000'000'000'000, 1>;
using exa = ratio<1'000'000'000'000'000'000, 1>;
using zetta = ratio<1'000'000'000'000'000'000'000, 1>; // *
using yotta = ratio<1'000'000'000'000'000'000'000'000, 1>; // *
\end{cpp}

带有星号结尾的 SI 单位只有在编译器能够表示那些类型时,别名的常数分子和分母值作为 intmax\_t 时才可定义。下一节讨论时间段,将给出使用这些预定义 SI 单位的示例。




