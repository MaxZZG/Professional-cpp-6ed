
时钟是一个由time\_point和时间段组成的类。下一节将详细讨论time\_point类型,但是要理解时钟的工作原理并不需要这些细节,time\_point本身依赖于时钟,所以先来了解一下时钟。

该标准定义了几种时钟,如下表所示。时钟的历元是时钟开始计数的时间。

% Please add the following required packages to your document preamble:
% \usepackage{longtable}
% Note: It may be necessary to compile the document several times to get a multi-page table to line up properly
\begin{longtable}{|l|l|l|}
\hline
\textbf{时钟} &
\textbf{描述} &
\textbf{历元} \\ \hline
\endfirsthead
%
\endhead
%
system\_clock &
\begin{tabular}[c]{@{}l@{}}表示来自系统范围的实时时钟,\\UTC壁挂时钟时间。
\end{tabular} &
1970-01-01 00:00:00 \\ \hline
steady\_clock &
\begin{tabular}[c]{@{}l@{}}确保其time\_point永远不\\会减少,因为系统时钟可以随\\时调整,所以system\_clock\\不保证。实际上,该时钟不要\\求与壁挂时钟时间相关;例如,\\可以是自操作系统启动以来的\\时间。
\end{tabular} &
未指定 \\ \hline
high\_resolution\_clock &
\begin{tabular}[c]{@{}l@{}}具有最短的可能的刻度周期。\\根据编译器,这个时钟可能是\\steady\_clock或\\system\_clock的同义词。\end{tabular} &
未指定 \\ \hline
utc\_clock &
\begin{tabular}[c]{@{}l@{}}表示协调世界时(UTC)的壁\\挂时钟时间。
\end{tabular} &
1970-01-01 00:00:00 \\ \hline
tai\_clock &
\begin{tabular}[c]{@{}l@{}}代表国际原子时(TAI),使用\\多个原子钟的加权平均值。
\end{tabular} &
1958-01-01 00:00:00 \\ \hline
gps\_clock &
\begin{tabular}[c]{@{}l@{}}代表全球定位系统(GPS)时\\间,即由GPS卫星的原子钟\\维护的时间。
\end{tabular} &
1980-01-06 00:00:00 \\ \hline
file\_clock &
\begin{tabular}[c]{@{}l@{}}代表文件时间,是\\std::filesystem::\\file\_time\_type\\的别名。
\end{tabular} &
\begin{tabular}[c]{@{}l@{}}未指定,\\Unix通常是1970-01-01,\\Windows中是1601-01-01。
\end{tabular} \\ \hline
\end{longtable}

utc\_clock是唯一一个跟踪闰秒的时钟,闰秒是在 UTC 时间与真太阳时间之间不匹配时,偶尔添加到或从 UTC 时间中减去的秒。其他时钟不跟踪闰秒,而file\_clock对此没有指定。

\begin{myNotic}{NOTE}
不建议使用high\_resolution\_clock,其实现在不同编译器之间不一致。对于某些编译器,可能是 steady\_clock 的别名,而对于其他编译器,可能是 system\_clock 的别名。对于某些编译器,high\_resolution\_clock可能会倒退。

建议使用 system\_clock 来处理挂壁时钟时间,并使用 steady\_clock 来测量时间段。
\end{myNotic}

每个时钟都有一个静态的 now() 成员函数,用来获取当前时间作为时间点,以及一个 is\_steady() 成员函数,如果时钟是稳定的(即永远不会倒退),则返回 true,否则返回 false。

system\_clock 还定义了两个静态帮助成员函数,用于将时间点转换为和从 time\_t C 风格时间表示转换。第一个称为 to\_time\_t(),将给定的时间点转换为 time\_t;第二个,from\_time\_t(),执行相反的转换。time\_t 类型在 <ctime> 中定义。

\mySubsubsection{22.3.1.}{打印当前时间}

以下示例演示了如何获取当前的 UTC 时间,并以人类可读的格式打印到控制台:

\begin{cpp}
// Set the global locale to the user's local (see Chapter 21).
locale::global(locale { "" });
// Print the current UTC time.
println("UTC: {:L}", system_clock::now());
println("UTC: {:L%c}", system_clock::now());
\end{cpp}

这个代码片段首先将全局区域设置为用户区域(参见第 21 章),这确保了所有内容都按照用户的偏好进行打印。println() 语句使用 L 格式说明符根据配置的全局区域设置格式化日期和时间。\%c 格式说明符的效果也得到了展示。支持更多的格式说明符。有关更多信息,请参阅标准库手册。以下是前面代码的输出:

\begin{shell}
UTC: 2023-07-19 11:38:44,5521944
UTC: 2023-07-19 11:38:44
\end{shell}

\mySubsubsection{22.3.2.}{进行计时}

要计时代码执行的时间,应该使用不会倒退的时钟,应该使用 steady\_clock。以下代码段中,start 和 end 变量的实际类型是 steady\_clock::time\_point,而 diff 的实际类型是时间段。

\begin{cpp}
// Get the start time.
auto start { steady_clock::now() };
// Execute code that you want to time.
const int numberOfIterations { 10'000'000 };
double d { 0 };
for (int i { 0 }; i < numberOfIterations; ++i) { d += sqrt(abs(sin(i) * cos(i))); }
// Get the end time and calculate the difference.
auto end { steady_clock::now() };
auto diff { end - start };
// Use the calculated result, otherwise the compiler might
// optimize away the entire loop!
println("d = {}", d);
// Convert the difference into milliseconds and output to the console.
println("Total: {}", duration<double, milli> { diff });
// Use duration_cast() if you don't need fractional milliseconds.
println("Total: {}", duration_cast<milliseconds>(diff));
// Print the time per iteration in nanoseconds.
println("{} per iteration", duration<double, nano> { diff / numberOfIterations });
\end{cpp}

这是在我测试系统上运行的输出:

\begin{shell}
d = 5393526.082683575
Total: 78.7931ms
Total: 78ms
7ns per iteration
\end{shell}

这个例子中的循环执行了一些算术运算,包括 sqrt()、abs()、sin() 和 cos(),以确保循环不会结束得太快。如果在系统上得到的差异毫秒值非常小,这些值将不准确,应该增加循环迭代的次数,使其持续更长时间。小时间值将不准确,因为计时器通常具有毫秒级的分辨率,但在大多数操作系统上,这个计时器更新不频繁,每 10 毫秒或 15 毫秒。这会导致一种称为“门控误差”的现象,其中小于一个计时器节拍的事件就似乎花费了零个时间单位,而发生在两个计时器节拍之间的事件似乎花费了一个计时器单位。在一个具有 15 毫秒计时器更新的系统上,一个需要 44 毫秒的循环将只显示为 30 毫秒。当使用此类计时器计时计算时,重要的是要确保整个计算跨越相当多的基本计时器节拍单位,以便误差最小化。































