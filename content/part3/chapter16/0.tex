\noindent
\textbf{内筒概要}

\begin{itemize}
\item
标准库中使用的编码原则

\item
标准库功能的高级概述
\end{itemize}

本章的所有代码示例都可以在\url{https://github.com/Professional-CPP/edition-6/}获得。

对于 C++ 开发者,最重要的库就是 C++ 标准库。这个库是 C++ 标准的一部分,所以符合标准的编译器都应该包括它。标准库并不是一个单实体:它包括几个不相关的组件,其中一些可能已经使用过了,可能甚至认为它们是语言核心的一部分。所有标准库的类和函数都在 std 命名空间中声明,或者在 std 的子命名空间中声明。

C++ 标准库的核心是其泛型容器和算法。有些人仍然称这个库的子集为标准模板库(STL),因为最初它是基于标准模板库的第三方库,这个库大量使用了模板。然而,STL 并不是 C++ 标准本身定义的术语,所以本书不使用这个术语。标准库的力量在于以一种方式提供了泛型容器和泛型算法,使得大多数算法都能在大多数容器上工作,无论容器存储什么类型的数据。性能是标准库的一个重要方面,目标是使标准库的容器和算法与手写代码一样快,或者更快。

C++ 标准库还包括 C11 标准中的大多数 C 头文件,但名称不同。可以通过包含 <cstdio> 来访问 C 头文件<stdio.h> 的功能。前者将所有内容放在全局命名空间中,而后者将所有内容放在 std 命名空间中。尽管从技术上讲,前者允许将内容放在 std 命名空间中,而后者允许将内容放在全局命名空间中。C11 头文件 <stdnoreturn.h>、<threads.h> 以及 <c...> 的等价物并未包含在 C++ 标准中。从 C++23 开始,<stdatomic.h> 头文件一直可用,但未提供等价的 <cstdatomic>。此外,C++17 已经弃用,C++20 已经移除了以下 C 头文件:

\begin{itemize}
\item
<ccomplex> 和 <ctgmath>:用 <complex> 和/或 <cmath> 替换它们。

\item
<ciso646>、<cstdalign> 和 <cstdbool>:这些头文件在 C++ 中是无用的,它们要么是空的,要么定义了在 C++ 中是关键字的宏。
\end{itemize}

C 头文件并不保证可以导入。要访问它们定义的功能,请使用 \#include,而不是 import。

\begin{myNotic}{NOTE}
如果有一个 C++ 头文件提供了 C 头文件提供的功能,建议使用 C++ 头文件。
\end{myNotic}

一个声称掌握 C++ 语言的 C++ 开发者应该熟悉标准库。通过将标准库的容器和算法纳入程序,而不是编写和调试自己的版本,这样可以节省大量的时间和精力。现在是掌握这个标准库的时候了。

关于标准库的第1章提供了可用功能的概述。接下来的几章将详细介绍标准库的几个方面,包括容器、迭代器、泛型算法、预定义的函数对象类、正则表达式、随机数生成等。此外,第25章会专门介绍如何使用自己符合标准库规范的算法和数据结构,来定制和扩展标准库。

尽管接下来的章节中有大量信息,标准库太大了,这本书无法全面覆盖。应该阅读这些章节来了解标准库,但它们并没有提到各个类提供的所有成员函数和数据成员,也没有展示所有算法的原型。附录C总结了标准库中的所有头文件。要了解所有提供的功能的完整参考,请查阅标准库手册。




















