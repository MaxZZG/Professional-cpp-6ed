通过解决下面的练习,可以练习本章讨论的内容。所有练习的解决方案都可以在本书的网站\url{www.wiley.com/go/proc++6e}下载到源码。若在练习中卡住了,可以考虑先重读本章的部分内容,试着自己找到答案,再在从网站上寻找解决方案。

\begin{itemize}
\item
\textbf{练习 17-1}: 编写一个程序,惰性构造元素序列 10-100,对每个数字求平方,删除所有能被5整除的数字,并将剩余的值转换为字符串使用 std::to\_string()。

\item
\textbf{练习 17-2}: 编写一个程序,创建一个包含 Person 类的实例及其年龄的 vector 的 pair。接下来,使用范围库构建一个单一的管道,从 vector 中提取所有人员的年龄,并删除所有低于 12 岁和高于 65 岁的年龄。最后,使用本章早些时候介绍的 sum() 算法计算剩余年龄的平均值。由于将向 sum() 算法传递一个范围,必须处理一个共同的范围。

\item
\textbf{练习 17-3}: 在 练习17-2 的基础上,为 Person 类添加 operator<{}< 实现。

接下来,创建一个管道来提取 vector 中每个 pair 的 Person,只保留前四个 Person。使用本章早些时候介绍的 myCopy() 算法将这四个人的名字打印到标准输出;一行一个名字。

最后,创建一个类似的管道,但这次它还投影所有过滤后的 Person 到他们的姓氏。这次,使用一个单 println() 语句将姓氏打印到标准输出。

\item
\textbf{练习 17-4}: 编写一个程序,使用基于范围的 for 循环和 ranges::istream\_view() 从标准输入读取整数,直到输入 -1。将读取的整数存储在 vector 中,然后打印 vector 的内容到控制台以验证它包含正确的值。

\item
\textbf{额外练习}: 能否找到几种改变练习17-4的方法,使其不使用任何显式循环?提示:一个选项可能是使用 std::ranges::copy() 算法将源范围复制到目标,可以接受范围作为第一个参数和输出迭代器作为第二个参数。
\end{itemize}




