This chapter explained the ideas behind iterators, which are an abstraction that allows you to navigate the elements of a container without the need to know the structure of the container. You have seen that output stream iterators can use standard output as a destination for iterator-based algorithms, and similarly that input stream iterators can use standard input as the source of data for algorithms. The chapter also discussed the insert-, reverse-, and move iterator adapters that can be used to adapt other iterators.

The last part of this chapter discussed the ranges library, part of the C++ Standard Library. It allows you to write more functional-style code, by specifying what you want to accomplish instead of how. You can construct pipelines consisting of a combination of operations applied to the elements of a range. Such pipelines are executed lazily; that is, they don’t do anything until you iterate over the resulting view.