
标准库提供了许多迭代器适配器,这些特殊迭代器都在 <iterator> 中定义,分为两组。第一组适配器由容器创建,通常用作输出迭代器:

\begin{itemize}
\item
back\_insert\_iterator: 使用 push\_back() 将元素插入容器

\item
front\_insert\_iterator: 使用 push\_front() 将元素插入容器

\item
insert\_iterator: 使用 insert() 将元素插入容器
\end{itemize}

其他适配器由另一个迭代器创建,通常用作输入迭代器。常见的两个适配器是:

\begin{itemize}
\item
reverse\_iterator: 反转另一个迭代器的迭代顺序。

\item
move\_iterator: move\_iterator的解引用操作符,自动将值转换为右值引用,因此可以将其移动到新的位置。
\end{itemize}

也可以自定义迭代器适配器,本书中没有介绍。相关信息,请参阅附录 B 中列出的标准库手册。

\mySubsubsection{17.3.1.}{插入迭代器}

myCopy() 函数模板,并不将元素插入到容器中,只是用新元素替换范围中的旧元素。为了使这类算法更有用,标准库提供了三种插入迭代器适配器,可以将元素插入到容器中:insert\_iterator、back\_insert\_iterator 和 front\_insert\_iterator。都针对容器类型参数化,并在构造函数中接收实际容器的引用。提供了必要的迭代器接口,这些适配器可以用作像 myCopy() 这样的算法的目标迭代器。但它们不会替换容器中的元素,而是通过对其容器调用适当的函数来实际插入新元素。

insert\_iterator 调用容器的 insert(position, element) ,back\_insert\_iterator 调用容器的 push\_back(element),front\_insert\_iterator 调容器的用 push\_front(element)。

以下示例使用 back\_insert\_iterator 和 myCopy() 来用 vectorOne 中所有元素的副本填充 vectorTwo。vectorTwo 没有先调整大小以拥有足够的元素,不过插入迭代器依旧会正确地插入新元素。

\begin{cpp}
vector vectorOne { 1, 2, 3, 4, 5, 6, 7, 8, 9, 10 };
vector<int> vectorTwo;

back_insert_iterator<vector<int>> inserter { vectorTwo };
myCopy(cbegin(vectorOne), cend(vectorOne), inserter);

println("{:n}", vectorTwo);
\end{cpp}

使用插入迭代器时,不需要提前调整目标容器的尺寸。

还可以使用 std::back\_inserter() 实用函数来创建 back\_insert\_iterator。之前的例子中,可以删除定义 inserter 变量的那一行,并重写 myCopy() 调用如下,结果不变。

\begin{cpp}
myCopy(cbegin(vectorOne), cend(vectorOne), back_inserter(vectorTwo));
\end{cpp}

利用类模板参数推断(CTAD),这也可以写成:

\begin{cpp}
myCopy(cbegin(vectorOne), cend(vectorOne), back_insert_iterator { vectorTwo });
\end{cpp}

front\_insert\_iterator 和 insert\_iterator 的工作方式类似,只是 insert\_iterator 还接受一个初始迭代器位置作为其构造函数的参数,将这个位置传递给第一个 insert(position, element),后续的迭代器位置会根据每次 insert() 调用的返回值生成。

使用 insert\_iterator 的好处是,可以将关联容器作为修改算法的目标。第 20 章会介绍,关联容器的问题在于不允许修改迭代过程中的键。通过使用 insert\_iterator,插入元素而不是修改现有的元素。关联容器有一个 insert() 成员函数,接受一个迭代器位置并可以使用该位置作为“提示”,也可以选择忽略。在关联容器上使用 insert\_iterator 时,可以将容器的 begin() 或 end() 迭代器作为提示传递。insert\_iterator 在每次调用 insert() 后修改,并传递给 insert(),使得迭代器指向刚插入元素之后的位置。

以下是将目标容器更改为集合的示例:

\begin{cpp}
vector vectorOne { 1, 2, 3, 4, 5, 6, 7, 8, 9, 10 };
set<int> setOne;

insert_iterator<set<int>> inserter { setOne, begin(setOne) };
myCopy(cbegin(vectorOne), cend(vectorOne), inserter);

println("{:n}", setOne);
\end{cpp}

类似于 back\_insert\_iterator 示例,也可以使用 std::inserter() 函数来创建 insert\_iterator:

\begin{cpp}
myCopy(cbegin(vectorOne), cend(vectorOne), inserter(setOne, begin(setOne)));
\end{cpp}

或者,使用类模板参数进行推断:

\begin{cpp}
myCopy(cbegin(vectorOne), cend(vectorOne),
    insert_iterator { setOne, begin(setOne) });
\end{cpp}

\mySubsubsection{17.3.2.}{反向迭代器}

标准库提供了 std::reverse\_iterator 类模板,通过一个双向或随机访问迭代器在反向方向上遍历。标准库中的每个可逆容器(标准库中除 forward\_list 和无序关联容器之外的所有容器)都提供 reverse\_iterator 类型别名和名为 rbegin() 和 rend() 的成员函数。这些 reverse\_iterator 类型别名是类型 std::reverse\_iterator<T> 的别名,其中 T 等于容器的迭代器类型别名。成员函数 rbegin() 返回一个指向容器最后一个元素的 reverse\_iterator,而 rend() 返回一个指向容器第一个元素之前元素的 reverse\_iterator。对 reverse\_iterator 应用 operator++ 会调用底层容器迭代器的 operator-{}-,反之亦然。例如,遍历一个集合从开始到结束可以这样做:

\begin{cpp}
for (auto iter { begin(collection) }; iter != end(collection); ++iter) {}
\end{cpp}

从集合的末尾到开始遍历元素,可以使用 reverse\_iterator 通过调用 rbegin() 和 rend()。这里仍使用 ++iter。

\begin{cpp}
for (auto iter { rbegin(collection) }; iter != rend(collection); ++iter) {}
\end{cpp}

std::reverse\_iterator 主要用于标准库中的算法或自定义函数,这些函数没有可以反向工作的等效函数。本章前面介绍的 myFind() 函数在序列中搜索第一个元素。如果想找到序列中的最后一个元素,可以使用 reverse\_iterator。当使用 reverse\_iterator 调用像 myFind() 这样的算法时,也会返回一个 reverse\_iterator。可以通过调用其 base() 成员函数从 reverse\_iterator 获得底层迭代器,由于 reverse\_iterator 的实现方式,base() 返回的迭代器指向 reverse\_iterator 所指向的元素之后的一个元素。为了得到相同的元素,必须减去一。

以下是一个使用 reverse\_iterator 的 myFind() 示例:

\begin{cpp}
vector myVector { 11, 22, 33, 22, 11 };
auto it1 { myFind(begin(myVector), end(myVector), 22) };
auto it2 { myFind(rbegin(myVector), rend(myVector), 22) };
if (it1 != end(myVector) && it2 != rend(myVector)) {
    println("Found at position {} going forward.",
             distance(begin(myVector), it1));
    println("Found at position {} going backward.",
             distance(begin(myVector), --it2.base()));
} else {
    println("Failed to find.");
}
\end{cpp}

这个程序的输出如下所示:

\begin{shell}
Found at position 1 going forward.
Found at position 3 going backward.
\end{shell}

\mySubsubsection{17.3.3.}{移动迭代器}

第9章介绍了移动语义,知道了源对象在赋值操作或复制构造之后会销毁的情况下,或者在使用 std::move() 时,可以用来避免不必要的复制。标准库提供了一个迭代器适配器,称为 std::move\_iterator。move\_iterator 的解引用操作符自动将值转换为右值引用,所以值可以不经过复制而移动到新的目的地。在使用移动语义之前,需要确保你的对象支持移动语义,以下 MoveableClass 支持移动语义。详细信息,参见第9章。

\begin{cpp}
class MoveableClass
{
    public:
        MoveableClass() {
            println("Default constructor");
        }
        MoveableClass(const MoveableClass& src) {
            println("Copy constructor");
        }
        MoveableClass(MoveableClass&& src) noexcept {
            println("Move constructor");
        }
        MoveableClass& operator=(const MoveableClass& rhs) {
            println("Copy assignment operator");
            return *this;
        }
        MoveableClass& operator=(MoveableClass&& rhs) noexcept {
            println("Move assignment operator");
            return *this;
        }
};
\end{cpp}

这里,构造函数和赋值操作符没有做任何事情,只是打印一条消息,以便了解调用了哪个函数。现在有了这个类,可以定义一个 vector 并将其中的几个 MoveableClass 实例存储在 vector 中:

\begin{cpp}
vector<MoveableClass> vecSource;
MoveableClass mc;
vecSource.push_back(mc);
vecSource.push_back(mc);
\end{cpp}

输出可能是这样的,每行后面的数字不是输出的一部分,为什么这样做后面有说明:

\begin{cpp}
Default constructor // [1]
Copy constructor // [2]
Copy constructor // [3]
Move constructor // [4]
\end{cpp}

代码的第二行通过使用默认构造函数创建 MoveableClass 实例,[1]。第一个 push\_back() 调用触发了复制构造函数来复制 mc 到 vector 中,[2]。之后,vector 有空间存放一个元素,即 mc 的第一个副本。这个是基于 Microsoft Visual C++ 2022 对 vector 的增长策略和初始大小实现的。C++ 标准没有指定 vector 的初始容量或其增长策略,所以输出可能会因不同的编译器而不同。

第二个 push\_back() 调用触发了 vector 自身的大小调整,为第二个元素分配空间。这个调整会调用移动构造函数,将每个元素从旧 vector 移动到新的调整大小后的 vector 中,[4]。将调用复制构造函数,将 mc 第二次复制到 vector 中,[3]。移动和复制的顺序未定义,所以 [3] 和 [4] 的顺序可能会颠倒。

可以创建一个新的 vector,名为 vecOne,它包含 vecSource 中的元素副本:

\begin{cpp}
vector<MoveableClass> vecOne { cbegin(vecSource), cend(vecSource) };
\end{cpp}

不使用 move\_iterators,这段代码会触发复制构造函数两次,其中一次对 vecSource 中的每个元素:

\begin{shell}
Copy constructor
Copy constructor
\end{shell}

通过使用 std::make\_move\_iterator() 创建 move\_iterator,调用MoveableClass 的移动构造函数:

\begin{cpp}
vector<MoveableClass> vecTwo { make_move_iterator(begin(vecSource)),
                               make_move_iterator(end(vecSource)) };
\end{cpp}

会产生以下输出:

\begin{shell}
Move constructor
Move constructor
\end{shell}

也可以使用类模板参数推断(CTAD)与 move\_iterator:

\begin{cpp}
vector<MoveableClass> vecTwo { move_iterator { begin(vecSource) },
                               move_iterator { end(vecSource) } };
\end{cpp}

\begin{myWarning}{WARNING}
当一个对象移动到为另一个对象,就不能再使用这个源对象了。
\end{myWarning}














