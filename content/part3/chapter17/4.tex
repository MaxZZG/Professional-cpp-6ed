

C++ 标准库的迭代器支持算法独立于实际容器工作,因为它们抽象了遍历容器元素的方法。如您所见的所有迭代器示例,大多数算法需要一个迭代器对,包括一个指向序列中第一个元素的 begin 迭代器,以及一个指向最后一个元素之后的 end 迭代器。这使得算法能够处理各种容器,但始终需要提供两个迭代器来指定一个元素序列,并确保提供匹配的迭代器,这里会有些繁琐。范围库提供的范围是对迭代器之上的抽象层,消除了不匹配迭代器的错误,并添加了额外的功能,例如允许范围适配器延迟过滤和转换底层元素序列。范围库,定义在 <ranges> 中,由以下组件组成:

\begin{itemize}
\item
范围:范围是一个概念(参见第 12 章),定义了一个类型,允许迭代其元素。任何支持 begin() 和 end() 的数据结构都是有效的范围。例如,std::array、vector、string\_view、span、固定大小的 C 风格数组等,都是有效的范围。

\item
受约束的算法:第 16 章和第 20 章讨论了接受迭代器对的可用标准库算法,以执行其工作。对于这些算法中的大多数,都有等效的范围基础和受约束的变体,它们接受迭代器对或范围。

\item
投影:许多受约束的算法接受一个投影回调。这个回调对范围内的每个元素进行调用,并可以将一个元素转换为另一个值,然后再传递给算法。

\item
视图:视图可以用来转换或过滤底层范围中的元素。视图可以组合在一起,形成应用于范围的操作流水线。

\item
工厂:范围工厂用于构造生成值的视图。
\end{itemize}

对范围内的元素进行迭代可以使用可以从访问器(如 ranges::begin()、end()、rbegin() 等)检索到的迭代器。范围还支持 ranges::empty()、data()、cdata() 和 size()。后者返回范围内的元素数量,但仅在可以以常数时间检索大小时才有效。否则,使用 std::distance() 计算范围内的 begin 和 end 迭代器之间的元素数量。所有这些访问器都不是成员函数,而是独立函数,都需要一个范围作为参数。

此外,std::format()、print() 和 println() 支持格式化和打印范围,这一点在本节中会通过许多示例进行展示。

\mySubsubsection{17.4.1.}{Constrained Algorithms}

The std::sort() algorithm is an example of an algorithm that requires a sequence of elements specified as a begin and end iterator. Algorithms are introduced in Chapter 16 and discussed in detail in Chapter 20. The sort() algorithm is straightforward to use. For example, the following code sorts all the elements of a vector:

\begin{cpp}
vector data { 33, 11, 22 };
sort(begin(data), end(data));
\end{cpp}

This code sorts all the elements in the data container, but you have to specify the sequence as a begin/ end iterator pair. Wouldn’t it be nicer to more accurately describe in your code what you really want to do? That’s where the range-based and constrained algorithms, simply called constrained algorithms in this book, come in. These algorithms live in the std::ranges namespace and are defined in the same header file as the corresponding unconstrained variants. With those, you can simply write the following:

\begin{cpp}
ranges::sort(data);
\end{cpp}

This code clearly describes your intent, that is, sorting all elements of the data container. Since you are not specifying iterators anymore, these constrained algorithms eliminate the possibility of accidentally supplying mismatching begin and end iterators. The constrained algorithms have proper type constraints (see Chapter 12) for their template type parameters. This allows compilers to provide clearer error messages in case you supply a container to a constrained algorithm that does not provide the type of iterator the algorithm requires. For example, calling the ranges::sort() algorithm on an std::list will result in a compiler error stating more clearly that sort() requires a randomaccess range, which list isn’t. Similar to iterators, you have input-, output-, forward-, bidirectional-, random-access-, and contiguous ranges, with corresponding concepts such as ranges::contiguous\_range, ranges::random\_access\_range, and so on.

\begin{myNotic}{NOTE}
Most Standard Library algorithms, introduced in Chapter 16 and discussed in detail in Chapter 20, have constrained equivalents in the std::ranges namespace. It’s recommended to always use these constrained algorithms, if possible, as the compiler can provide better error messages when such algorithms are used with the wrong types, thanks to their type constraints.
\end{myNotic}

\mySamllsection{Projection}

A lot of the constrained algorithms have a projection parameter, a callback used to transform each element before it is handed over to the algorithm. Let’s look at an example. Suppose you have a simple class representing a person:

\begin{cpp}
class Person
{
    public:
        explicit Person(string first, string last)
            : m_firstName { move(first) }, m_lastName { move(last) } { }
        const string& getFirstName() const { return m_firstName; }
        const string& getLastName() const { return m_lastName; }
    private:
        string m_firstName;
        string m_lastName;
};
\end{cpp}

The following code stores a couple of Person objects in a vector:

\begin{cpp}
vector persons { Person {"John", "White"}, Person {"Chris", "Blue"} };
\end{cpp}

Since the Person class does not implement operator<, you cannot sort this vector using the normal std::sort() algorithm, as it compares elements using operator<. So, the following does not compile:

\begin{cpp}
sort(begin(persons), end(persons)); // Error: does not compile.
\end{cpp}

Switching to the constrained ranges::sort() algorithm doesn’t help much at first sight. The following still doesn’t compile as the algorithm still doesn’t know how to compare elements in the range:

\begin{cpp}
ranges::sort(persons); // Error: does not compile.
\end{cpp}

However, you can sort persons based on their first name, by specifying a projection function for the sort algorithm to project each person to their first name. The projection parameter is the third one, so we have to specify the second parameter as well, which is the comparator to use, by default std::ranges::less. In the following call, the \{\} specifies to use the default comparator, and the projection function is specified as a lambda expression, see upcoming note.

\begin{cpp}
ranges::sort(persons, {},
    [](const Person& person) { return person.getFirstName(); });
\end{cpp}

Or even shorter:

\begin{cpp}
ranges::sort(persons, {}, &Person::getFirstName);
\end{cpp}

\begin{myNotic}{NOTE}
The discussion on ranges in this chapter uses a few basic lambda expressions. Lambda expressions are discussed in detail in Chapter 19, “Function Pointers, Function Objects, and Lambda Expressions,” but all those details are not important for the current discussion. For now, it’s enough to know only basic use. A lambda expression, as used in this chapter, has the following syntax:

\begin{cpp}
[](const Person& person) { return person.getFirstName(); }
\end{cpp}

The [] denotes the start of a lambda expression. Next is a comma-separated list of parameters, just as for functions. Finally, the body of the lambda expression is between a set of curly brackets.

Basically, lambda expressions allow you to write small, unnamed inline functions at the place where you need them. The previous lambda expression could be replaced with the following standalone function:

\begin{cpp}
auto getFirstName(const Person& person) {
    return person.getFirstName(); }
\end{cpp}

The type of lambda expression parameters can also be auto. Here is an example:

\begin{cpp}
[](const auto& person) { return person.getFirstName(); }
\end{cpp}
\end{myNotic}

\mySubsubsection{17.4.2.}{Views}

A view allows you to perform operations on an underlying range’s elements, such as filtering and transforming. Views can be chained/composed together to form a pipeline performing multiple operations on the elements of a range. Composing views is easy, you just combine different operations using the bitwise OR operator, operator|. For example, you can easily filter the elements of a range first and then transform the remaining elements. In contrast, if you want to do something similar, filtering followed by transforming, using the unconstrained algorithms, your code will be much less readable and possibly less performant, as you’ll have to create temporary containers to store intermediate results.

A view has the following important properties:

\begin{itemize}
\item
Lazily evaluated: Just constructing a view doesn’t perform any operations yet. The operations of a view are applied only at the moment you iterate over the elements of the view and dereference such an iterator.

\item
Nonowning[C++23 slightly modifies the definition of a view. It allows for a view to own its elements, but only if it guarantees that it’s either non-copyable, or copyable in constant time, O(1). Most views will be nonowning, so owning views are not further discussed in this text.]: A view doesn’t own any elements. As the name suggests, it’s a view over a range’s elements that could be stored in some container, and it’s that container that’s the owner of the data. A view just allows you to view that data in different ways. As such, the number of elements in a view does not influence the cost of copying, moving, or destroying a view. This is similar to std::string\_view discussed in Chapter 2, “Working with Strings and String Views,” and std::span discussed in Chapter 18, “Standard Library Containers.”

\item
Nonmutating: A view never modifies the data in the underlying range.
\end{itemize}

A view itself is also a range, but not every range is a view. A container is a range but not a view, as it owns its elements.

Views can be created using range adapters. A range adapter accepts an underlying sequence of elements, and optionally some arguments, and creates a new view. The following table lists the range adapters provided by the Standard Library. If none of the Standard Library adapters suits your needs, it’s possible to write your own range adapters that properly interoperate with existing adapters. However, writing a full-fledged production-quality range adapter is not trivial and would take us a bit too far for the scope of this book. See your favorite Standard Library reference for more details.

% Please add the following required packages to your document preamble:
% \usepackage{longtable}
% Note: It may be necessary to compile the document several times to get a multi-page table to line up properly
\begin{longtable}{|l|l|}
\hline
\textbf{RANGE ADAPTER} &
\textbf{DESCRIPTION} \\ \hline
\endfirsthead
%
\endhead
%
views::all &
Creates a view that includes all elements of a range. \\ \hline
\begin{tabular}[c]{@{}l@{}}filter\_view\\ views::filter\end{tabular} &
\begin{tabular}[c]{@{}l@{}}Filters the elements of an underlying sequence based on a given\\ predicate. If the predicate returns true, the element is kept, otherwise\\ it is skipped.\end{tabular} \\ \hline
\begin{tabular}[c]{@{}l@{}}transform\_view\\ views::transform\end{tabular} &
\begin{tabular}[c]{@{}l@{}}Applies a callback to each element of an underlying sequence to\\ transform the element to some other value, possibly of a different type.\end{tabular} \\ \hline
\begin{tabular}[c]{@{}l@{}}take\_view\\ views::take\end{tabular} &
Creates a view of the first n elements of another view. \\ \hline
\begin{tabular}[c]{@{}l@{}}take\_while\_view\\ views::take\_while\end{tabular} &
\begin{tabular}[c]{@{}l@{}}Creates a view of the initial elements of an underlying sequence until an\\ element is reached for which a given predicate returns false.\end{tabular} \\ \hline
\begin{tabular}[c]{@{}l@{}}drop\_view\\ views::drop\end{tabular} &
Creates a view by dropping the first n elements of another view. \\ \hline
\begin{tabular}[c]{@{}l@{}}drop\_while\_view\\ views::drop\_while\end{tabular} &
\begin{tabular}[c]{@{}l@{}}Creates a view by dropping all initial elements of an underlying\\ sequence until an element is reached for which a given predicate\\ returns false.\end{tabular} \\ \hline
\begin{tabular}[c]{@{}l@{}}join\_view\\ views::join\end{tabular} &
\begin{tabular}[c]{@{}l@{}}Flattens a view of ranges into a view. For example, flatten a\\ vector\textless{}vector\textless{}int\textgreater{}\textgreater into a vector\textless{}int\textgreater{}.\end{tabular} \\ \hline
\begin{tabular}[c]{@{}l@{}}lazy\_split\_view\\ views::lazy\_split\\ split\_view\\ views::split\end{tabular} &
\begin{tabular}[c]{@{}l@{}}Given a delimiter, splits a given view into subranges on the delimiter.\\ The delimiter can be a single element or a view of elements.\end{tabular} \\ \hline
\begin{tabular}[c]{@{}l@{}}reverse\_view\\ views::reverse\end{tabular} &
\begin{tabular}[c]{@{}l@{}}Creates a view that iterates over the elements of another view in\\ reverse order. The view must be a bidirectional view.\end{tabular} \\ \hline
\begin{tabular}[c]{@{}l@{}}elements\_view\\ views::elements\end{tabular} &
\begin{tabular}[c]{@{}l@{}}Requires a view of tuple-like elements, creates a view of the nth\\ elements of the tuple-like elements.\end{tabular} \\ \hline
\begin{tabular}[c]{@{}l@{}}keys\_view\\ views::keys\end{tabular} &
\begin{tabular}[c]{@{}l@{}}Requires a view of pair-like elements, creates a view of the first element\\ of each pair.\end{tabular} \\ \hline
\begin{tabular}[c]{@{}l@{}}values\_view\\ views::values\end{tabular} &
\begin{tabular}[c]{@{}l@{}}Requires a view of pair-like elements, creates a view of the second\\ element of each pair.\end{tabular} \\ \hline
\begin{tabular}[c]{@{}l@{}}common\_view\\ views::common\end{tabular} &
\begin{tabular}[c]{@{}l@{}}Depending on the type of range, begin() and end() might return\\ different types, such as a begin iterator and an end sentinel. This means\\ that you cannot, for example, pass such an iterator pair to functions\\ that expect them to be of the same type. common\_view can be used\\ to convert such a range to a common range which is a range for which\\ begin() and end() return the same type. You will use this range\\ adapter in one of the exercises.\end{tabular} \\ \hline
\end{longtable}

\CXXTwentythreeLogo{-40}{-50}

C++23 adds the following range adapters:

% Please add the following required packages to your document preamble:
% \usepackage{longtable}
% Note: It may be necessary to compile the document several times to get a multi-page table to line up properly
\begin{longtable}{|l|l|}
\hline
\textbf{RANGE ADAPTER} &
\textbf{DESCRIPTION} \\ \hline
\endfirsthead
%
\endhead
%
\begin{tabular}[c]{@{}l@{}}as\_const\_view\\ views::as\_const\end{tabular} &
\begin{tabular}[c]{@{}l@{}}Creates a view through which the elements of an underlying\\ sequence cannot be modified.\end{tabular} \\ \hline
\begin{tabular}[c]{@{}l@{}}as\_rvalue\_view\\ views::as\_rvalue\end{tabular} &
\begin{tabular}[c]{@{}l@{}}Creates a view of rvalues of all elements of an\\ underlying sequence.\end{tabular} \\ \hline
\begin{tabular}[c]{@{}l@{}}enumerate\_view\\ views::enumerate\end{tabular} &
\begin{tabular}[c]{@{}l@{}}Creates a view where each element represents the position\\ and value of all elements of an underlying sequence.\end{tabular} \\ \hline
\begin{tabular}[c]{@{}l@{}}zip\_view\\ views::zip\end{tabular} &
\begin{tabular}[c]{@{}l@{}}Creates a view consisting of tuples of reference to\\ corresponding elements of all given views.\end{tabular} \\ \hline
\begin{tabular}[c]{@{}l@{}}zip\_transform\_view\\ views::zip\_transform\end{tabular} &
\begin{tabular}[c]{@{}l@{}}Creates a view whose ith element is the result of applying a\\ given callable to the ith elements of all given views.\end{tabular} \\ \hline
\begin{tabular}[c]{@{}l@{}}adjacent\_view\\ views::adjacent\end{tabular} &
\begin{tabular}[c]{@{}l@{}}For a given n, creates a view whose $i^{th}$ element is a tuple\\ of references to the $i^{th}$ through $(i + n - 1)^{th}$ elements of a\\ given view.\end{tabular} \\ \hline
\begin{tabular}[c]{@{}l@{}}adjacent\_transform\_view\\ views::adjacent\_transform\end{tabular} &
\begin{tabular}[c]{@{}l@{}}For a given n, creates a view whose $i^{th}$ element is the result\\ of applying a given callable to the $i^{th}$ through $(i + n - 1)^{th}$\\ elements of a given view.\end{tabular} \\ \hline
\begin{tabular}[c]{@{}l@{}}views::pairwise\\ views::pairwise\_transform\end{tabular} &
\begin{tabular}[c]{@{}l@{}}Helper types representing views::adjacent\textless{}2\textgreater and\\ views::adjacent\_transform\textless{}2\textgreater respectively.\end{tabular} \\ \hline
\begin{tabular}[c]{@{}l@{}}join\_with\_view\\ views::join\_with\end{tabular} &
\begin{tabular}[c]{@{}l@{}}Given a delimiter, flattens the elements of a given view,\\ inserting every element of the delimiter in between elements\\ of the view. The delimiter can be a single element or a view\\ of elements.\end{tabular} \\ \hline
\begin{tabular}[c]{@{}l@{}}stride\_view\\ views::stride\end{tabular} &
\begin{tabular}[c]{@{}l@{}}For a given n, creates a view of an underlying sequence,\\ advancing over n elements at a time, instead of one by one.\end{tabular} \\ \hline
\begin{tabular}[c]{@{}l@{}}slide\_view\\ views::slide\end{tabular} &
\begin{tabular}[c]{@{}l@{}}For a given n, creates a view whose $i^{th}$ element is a view over\\ the $i^{th}$ through $(i + n - 1)^{th}$ elements of the original view.\\ Similar to views::adjacent, but the window size, n, is a\\ runtime parameter for slide, while it’s a template argument\\ for adjacent.\end{tabular} \\ \hline
\begin{tabular}[c]{@{}l@{}}chunk\_view\\ views::chunk\end{tabular} &
\begin{tabular}[c]{@{}l@{}}For a given n, creates a range of views that are n-sized non\\ overlapping successive chunks of the elements of the original\\ view, in order.\end{tabular} \\ \hline
\begin{tabular}[c]{@{}l@{}}chunk\_by\_view\\ views::chunk\_by\end{tabular} &
\begin{tabular}[c]{@{}l@{}}Splits a given view into subranges between each pair of\\ adjacent elements for which a given predicate returns false.\end{tabular} \\ \hline
\begin{tabular}[c]{@{}l@{}}cartesian\_product\_view\\ views::cartesian\_product\end{tabular} &
\begin{tabular}[c]{@{}l@{}}Given a number of ranges, n, creates a view of tuples\\ calculated by the n-ary cartesian product of the\\ provided ranges.\end{tabular} \\ \hline
\end{longtable}

The range adapters in the first column of both tables show both the class name in the std::ranges namespace and a corresponding range adapter object from the std::ranges::views namespace. The Standard Library provides a namespace alias called std::views equal to std::ranges::views.

Each range adapter can be constructed by calling its constructor and passing any required arguments.
The first argument is always the range on which to operate, followed by zero or more additional arguments, as follows:

\begin{cpp}
std::ranges::operation_view { range, arguments... }
\end{cpp}

Usually, you will not create these range adapters using their constructors, but instead use the range adapter objects from the std::ranges::views namespace in combination with the bitwise OR operator, |, as follows:

\begin{cpp}
range | std::ranges::views::operation(arguments...)
\end{cpp}

Let’s see some of these range adapters in action. The following example first defines an abbreviated function template called printRange() to print a message followed by all the elements in a given range. Next, the main() function starts by creating a vector of integers, 1. . .10, and subsequently applies several range adapters on it, each time calling printRange() on the result so you can follow what’s happening. Afterward, it demonstrates several of the new C++23 range adapters. The example uses the myCopy() function introduced earlier in this chapter.

\begin{cpp}
void printRange(string_view msg, auto&& range) { println("{}{:n}", msg, range); }
int main()
{
    vector values { 1, 2, 3, 4, 5, 6, 7, 8, 9, 10 };
    printRange("Original sequence: ", values);

    // Filter out all odd values, leaving only the even values.
    auto result1 { values
        | views::filter([](const auto& value) { return value % 2 == 0; }) };
    printRange("Only even values: ", result1);

    // Transform all values to their double value.
    auto result2 { result1
        | views::transform([](const auto& value) { return value * 2.0; }) };
    printRange("Values doubled: ", result2);

    // Drop the first 2 elements.
    auto result3 { result2 | views::drop(2) };
    printRange("First two dropped: ", result3);

    // Reverse the view.
    auto result4 { result3 | views::reverse };
    printRange("Sequence reversed: ", result4);

    // C++23: views::zip
    vector v1 { 1, 2 };
    vector v2 { 'a', 'b', 'c' };
    auto result5 { views::zip(v1, v2) };
    printRange("views::zip: ", result5);

    // C++23: views::adjacent
    vector v3 { 1, 2, 3, 4, 5 };
    auto result6 { v3 | views::adjacent<2> };
    printRange("views::adjacent: ", result6);

    // C++23: views::chunk
    auto result7 { v3 | views::chunk(2) };
    printRange("views::chunk: ", result7);

    // C++23: views::stride
    auto result8 { v3 | views::stride(2) };
    printRange("views::stride: ", result8);

    // C++23: views::enumerate + views::split
    string lorem { "Lorem ipsum dolor sit amet" };
    for (auto [index, word] : lorem | views::split(' ') | views::enumerate) {
        print("{}:'{}' ", index, string_view { word });
    }
    println("");

    // C++23: views::as_rvalue
    vector<string> words { "Lorem", "ipsum", "dolor", "sit", "amet" };
    vector<string> movedWords;
    auto rvalueView { words | views::as_rvalue };
    myCopy(begin(rvalueView), end(rvalueView), back_inserter(movedWords));
    printRange("movedWords: ", movedWords);

    // C++23: Cartesian product of vector v with itself.
    vector v { 0, 1, 2 };
    for (auto&&[a, b] : views::cartesian_product(v, v)) {print("({},{}) ", a, b);}
}
\end{cpp}

The output of this program is as follows:

\begin{shell}
Original sequence: 1, 2, 3, 4, 5, 6, 7, 8, 9, 10
Only even values: 2, 4, 6, 8, 10
Values doubled: 4, 8, 12, 16, 20
First two dropped: 12, 16, 20
Sequence reversed: 20, 16, 12
views::zip: (1, 'a'), (2, 'b')
views::adjacent: (1, 2), (2, 3), (3, 4), (4, 5)
views::chunk: (1, 2), (3, 4), (5)
views::stride: 1, 3, 5
0:'Lorem' 1:'ipsum' 2:'dolor' 3:'sit' 4:'amet'
movedWords: Lorem, ipsum, dolor, sit, amet
(0,0) (0,1) (0,2) (1,0) (1,1) (1,2) (2,0) (2,1) (2,2)
\end{shell}

It’s worth repeating that views are lazily evaluated. In this example, the construction of the result1 view does not do any actual filtering yet. The filtering happens at the time when the printRange() function iterates over the elements of result1.

The code snippet uses the range adapter objects from std::ranges::views. You can also construct range adapters using their constructors. For example, the result1 view can be constructed as follows:

\begin{cpp}
auto result1 { ranges::filter_view { values,
        [](const auto& value) { return value % 2 == 0; } } };
\end{cpp}

This example is creating several intermediate views, result1, result2, result3, and result4, to be able to output their elements to make it easier to follow what’s happening in each step. If you don’t need these intermediate views, you can chain them all together in a single pipeline as follows:

\begin{cpp}
vector values { 1, 2, 3, 4, 5, 6, 7, 8, 9, 10 };
printRange("Original sequence: ", values);
auto result { values
    | views::filter([](const auto& value) { return value % 2 == 0; })
    | views::transform([](const auto& value) { return value * 2.0; })
    | views::drop(2)
    | views::reverse };
printRange("Final sequence: ", result);
\end{cpp}

The output is as follows. The last line shows that the final sequence is the same as the earlier result4 view.

\begin{shell}
Original sequence: 1, 2, 3, 4, 5, 6, 7, 8, 9, 10
Final sequence: 20, 16, 12
\end{shell}

\mySamllsection{Modifying Elements Through a View}

Some ranges are read-only. For example, the result of views::transform is a read-only view, because it creates a view with transformed elements but without transforming the actual values in the underlying range. If a range is not read-only, then you can modify the elements of that range through a view. Let’s look an example.

The following example constructs a vector of ten elements. It then creates a view over the even values, drops the first two even values, and finally reverses the elements. The range-based for loop then multiplies the elements in the resulting view with 10. The last line outputs the elements in the original values vector to confirm that some elements have been changed through the view.

\begin{cpp}
vector values { 1, 2, 3, 4, 5, 6, 7, 8, 9, 10 };
printRange("Original sequence: ", values);

// Filter out all odd values, leaving only the even values.
auto result1 { values
    | views::filter([](const auto& value) { return value % 2 == 0; }) };
printRange("Only even values: ", result1);

// Drop the first 2 elements.
auto result2 { result1 | views::drop(2) };
printRange("First two dropped: ", result2);

// Reverse the view.
auto result3 { result2 | views::reverse };
printRange("Sequence reversed: ", result3);

// Modify the elements using a range-based for loop.
for (auto& value : result3) { value *= 10; }
printRange("After modifying elements through a view, vector contains:\n", values);
\end{cpp}

The output of this program is as follows:

\begin{shell}
Original sequence: 1, 2, 3, 4, 5, 6, 7, 8, 9, 10
Only even values: 2, 4, 6, 8, 10
First two dropped: 6, 8, 10
Sequence reversed: 10, 8, 6
After modifying elements through a view, vector contains:
1, 2, 3, 4, 5, 60, 7, 80, 9, 100
\end{shell}

\mySamllsection{Mapping Elements}

Transforming elements of a range doesn’t need to result in a range with elements of the same type. Instead, you can map elements to another type. The following example starts with a range of integers, filters out all odd elements, keeps only the first three even values, and transforms those to strings using std::format():

\begin{cpp}
vector values { 1, 2, 3, 4, 5, 6, 7, 8, 9, 10 };
printRange("Original sequence: ", values);

auto result { values
    | views::filter([](const auto& value) { return value % 2 == 0; })
    | views::take(3)
    | views::transform([](const auto& v) { return format("{}", v); }) };
printRange("Result: ", result);
\end{cpp}

The output is as follows:

\begin{shell}
Original sequence: 1, 2, 3, 4, 5, 6, 7, 8, 9, 10
Result: "2", "4", "6"
\end{shell}

\mySubsubsection{17.4.3.}{Range Factories}

The ranges library provides the following range factories to construct views that produce elements lazily on demand:

% Please add the following required packages to your document preamble:
% \usepackage{longtable}
% Note: It may be necessary to compile the document several times to get a multi-page table to line up properly
\begin{longtable}{|l|l|}
\hline
\textbf{RANGE FACTORY} & \textbf{DESCRIPTION}                        \\ \hline
\endfirsthead
%
\endhead
%
empty\_view            & Creates an empty view.                      \\ \hline
single\_view           & Creates a view with a single given element. \\ \hline
iota\_view &
\begin{tabular}[c]{@{}l@{}}Creates an infinite or a bounded view containing elements starting with\\ an initial value, and where each subsequent element has a value equal\\ to the value of the previous element incremented by one.\end{tabular} \\ \hline
repeat\_view(C++23) &
\begin{tabular}[c]{@{}l@{}}Creates a view that repeats a given value. The resulting view can be\\ unbounded (infinite) or bounded by a given number of values.\end{tabular} \\ \hline
\begin{tabular}[c]{@{}l@{}}basic\_istream\_view\\ istream\_view\end{tabular} &
\begin{tabular}[c]{@{}l@{}}Creates a view containing elements retrieved by calling the extraction\\ operator, operator\textgreater{}\textgreater{}, on an underlying input stream.\end{tabular} \\ \hline
\end{longtable}

Just as with the range adapters from the previous section, the names in the range factories table are class names living in the std::ranges namespace, which you can directly create using their constructor. Alternatively, you can use the factory functions available in the std::ranges:views namespace. For example, the following two statements are equivalent and create an infinite view with elements 10, 11, 12, . . .:

\begin{cpp}
std::ranges::iota_view { 10 }
std::ranges::views::iota(10)
\end{cpp}

Let’s look at a range factory in practice. The following example is loosely based on an earlier example, but instead of constructing a vector with 10 elements in it, this code uses the iota range factory to create a lazy infinite sequence of numbers starting at 10. It then removes all odd values, doubles the remaining elements, and finally only keeps the first ten elements that are subsequently output to the console using printRange().

\begin{cpp}
// Create an infinite sequence of the numbers 10, 11, 12, ...
auto values { views::iota(10) };
// Filter out all odd values, leaving only the even values.
auto result1 { values
    | views::filter([](const auto& value) { return value % 2 == 0; }) };
// Transform all values to their double value.
auto result2 { result1
    | views::transform([](const auto& value) { return value * 2.0; }) };
// Take only the first ten elements.
auto result3 { result2 | views::take(10) };
printRange("Result: ", result3);
\end{cpp}

The output is as follows:

\begin{shell}
Result: 20, 24, 28, 32, 36, 40, 44, 48, 52, 56
\end{shell}

The values range represents an infinite range, which is subsequently filtered and transformed. Working with infinite ranges is possible because all these operations are lazily evaluated only at the time when printRange() iterates over the elements of the view. This also means that in this example you cannot call printRange() to output the contents of values, result1, or result2 because that would trigger an infinite loop in printRange() as those are infinite ranges.

Of course, you can get rid of those intermediate views and simply construct one big pipeline. The following produces the same output as before:

\begin{cpp}
auto result { views::iota(10)
    | views::filter([](const auto& value) { return value % 2 == 0; })
    | views::transform([](const auto& value) { return value * 2.0; })
    | views::take(10) };
printRange("Result: ", result);
\end{cpp}

Another range factory example demonstrates how to use a repeat\_view:

\begin{cpp}
printRange("Repeating view: ", views::repeat(42, 5));
\end{cpp}

This outputs the following:

\begin{shell}
Repeating view: 42, 42, 42, 42, 42
\end{shell}

\mySamllsection{Input Streams as Views}

The basic\_istream\_view/istream\_view range factory can be used to construct a view over the elements read from an input stream, such as the standard input. Elements are read using operator>{}>.

For example, the following code snippet keeps reading integers from standard input. For each read number that is less than 5, the number is doubled and printed on standard output. Once you enter a number 5 or higher, the loop stops.

\begin{cpp}
println("Type integers, an integer >= 5 stops the program.");
for (auto value : ranges::istream_view<int> { cin }
    | views::take_while([](const auto& v) { return v < 5; })
    | views::transform([](const auto& v) { return v * 2; })) {
    println("> {}", value);
}
println("Terminating...");
\end{cpp}

The following is a possible output sequence:

\begin{shell}
Type integers, an integer >= 5 stops the program.
1 2
> 2
> 4
3
> 6
4
> 8
5
Terminating...
\end{shell}

\CXXTwentythreeLogo{-40}{-50}

\mySubsubsection{17.4.4.}{Converting a Range into a Container}

Before C++23, it was not easy to convert a range into a container. C++23 introduces std::ranges::to() to make such conversions straightforward. This can also be used to convert the elements of a view into a container, as a view is a range. Even more, since a container is a range as well, you can use ranges::to() to convert one container into a different container, even with different element types.

The following code snippet demonstrates several uses of ranges::to(). The example also demonstrates special constructors for set and string, which accept the std::from\_range tag as the first parameter and convert a given range into a set or string. All Standard Library containers now include such constructors.

\begin{cpp}
// Convert a vector to a set with the same element type.
vector vec { 33, 11, 22 };
auto s1 { ranges::to<set>(vec) };
println("{:n}", s1);

// Convert a vector of integers to a set of doubles, using the pipe operator.
auto s2 { vec | ranges::to<set<double>>() };
println("{:n}", s2);

// Convert a vector of integers to a set of doubles, using from_range constructor.
set<double> s3 { from_range, vec };
println("{:n}", s3);

// Lazily generate the integers from 10 to 14, divide these by 2,
// and store the result in a vector of doubles.
auto vec2 { views::iota(10, 15)
    | views::transform([](const auto& v) { return v / 2.0; })
    | ranges::to<vector<double>>() };
println("{:n}", vec2);

// Use views::split() and views::transform() to create a view
// containing individual words of a string, and then convert
// the resulting view to a vector of strings containing all the words.
string lorem { "Lorem ipsum dolor sit amet" };
auto words { lorem | views::split(' ')
    | views::transform([](const auto& v) { return string { from_range, v }; })
    | ranges::to<vector>() };
println("{:n:?}", words);
\end{cpp}

The output is as follows:

\begin{shell}
11, 22, 33
11, 22, 33
11, 22, 33
5, 5.5, 6, 6.5, 7
"Lorem", "ipsum", "dolor", "sit", "amet"
\end{shell}

C++23 also introduces a number of new member functions for Standard Library containers, providing for interoperability between containers and ranges. These member functions are of the form xyz\_range(...), where xyz can be insert, append, prepend, assign, replace, push, push\_front, or push\_back. Chapter 18 discusses all Standard Library containers in detail, but consult a Standard Library reference to learn exactly which member functions are supported by which container. Here is an example demonstrating the append\_range() and insert\_range() member functions of vector:

\begin{cpp}
vector<int> vec3;
vec3.append_range(views::iota(10, 15));
println("{:n}", vec3);
vec3.insert_range(begin(vec3), views::iota(10, 15) | views::reverse);
println("{:n}", vec3);
\end{cpp}

The output is:

\begin{shell}
10, 11, 12, 13, 14
14, 13, 12, 11, 10, 10, 11, 12, 13, 14
\end{shell}








