\noindent
\textbf{内筒概要}

\begin{itemize}
\item
随机数引擎和引擎适配器的概念

\item
生成随机数

\item
改变随机数的分布
\end{itemize}

本章的所有代码示例都可以在\url{https://github.com/Professional-CPP/edition-6}获得。

本章讨论了如何在 C++ 中生成随机数,软件中生成良好的随机数是一个复杂的话题。本章没有讨论生成实际随机数所涉及的复杂数学公式,但确实介绍了如何使用标准库提供的功能来生成随机数。

C++ 的随机数生成库可以通过不同的算法和分布生成随机数,该库由 <random> 定义在 std 命名空间中。有三个主要组件:引擎、引擎适配器,以及分布。随机数引擎负责生成实际的随机数,并存储用于生成后续随机数的内部状态。分布决定了生成的随机数的范围,以及在该范围内数学分布的方式。随机数引擎适配器会修改与它关联的随机数引擎的结果。

深入探讨C++ 随机数生成库之前,先简要解释一下旧式的 C 风格随机数生成机制,及其存在的问题。










