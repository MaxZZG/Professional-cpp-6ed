By solving the following exercises, you can practice the material discussed in this chapter. Solutions to all exercises are available with the code download on the book’s website at \url{www.wiley.com/go/proc++6e}. However, if you are stuck on an exercise, first reread parts of this chapter to try to find an answer yourself before looking at the solution from the website.

\begin{itemize}
\item
Exercise 23-1: Write a loop asking the user if dice should be thrown or not. If yes, throw a die twice using a uniform distribution and print the two numbers on the screen. If no, stop the program. Use the standard mt19937 Mersenne twister engine. Do not create your random number generator directly in the function where you need it. Instead, write a function createDiceValueGenerator() that creates the correct random number generator object and returns it.

\item
Exercise 23-2: Modify your solution to Exercise 23-1 to use a ranlux48 engine instead of the Mersenne twister.

\item
Exercise 23-3: Modify your solution to Exercise 23-1. Instead of directly using the mt19937 Mersenne twister engine, adapt the engine with a shuffle\_order\_engine adapter.

\item
Exercise 23-4: Take the source code from earlier in this chapter used to generate histograms to make graphs of a distribution and experiment a bit with different distributions. Try to plot the graphs in a spreadsheet application to see the effects of a distribution. The code can be found in the downloadable source code archive in the folder Ch23\verb|\|01\_Random. You can take either the 07\_uniform\_int\_distribution.cpp or the 08\_normal\_distribution.cpp file depending on whether your distribution uses integers or doubles.

\item
Bonus: Besides exporting the data to a CSV file, draw the histogram on the standard output console using characters.
\end{itemize}













