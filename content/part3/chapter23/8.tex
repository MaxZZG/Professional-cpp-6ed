通过解决下面的练习,可以练习本章讨论的内容。所有练习的解决方案都可以在本书的网站\url{www.wiley.com/go/proc++6e}下载到源码。若在练习中卡住了,可以考虑先重读本章的部分内容,试着自己找到答案,再在从网站上寻找解决方案。

\begin{itemize}
\item
\textbf{练习 23-1}:编写一个循环,询问用户是否应该掷骰子。如果是,使用均匀分布掷骰子两次并在屏幕上打印两个数字。如果不是,停止程序。使用标准的  Mersenne twister 引擎mt19937,不要在需要它的函数中直接创建随机数生成器。编写一个 createDiceValueGenerator() 函数,创建正确的随机数生成器对象并返回它。

\item
\textbf{练习 23-2}:修改你的练习 23-1 的解决方案,使用 ranlux48 引擎。

\item
\textbf{练习 23-3}:修改你的练习 23-1 的解决方案。不要直接使用  Mersenne twister 引擎mt19937,而是使用 shuffle\_order\_engine 适配器来适配引擎。

\item
\textbf{练习 23-4}:使用本章早些时候用于生成直方图的源代码来制作分布图,并尝试使用不同的分布。尝试在电子表格应用程序中绘制图表,以查看分布的影响。代码可以在源代码库的 Ch23\verb|\|01\_Random 文件夹中找到。可以选择使用整数或双精度浮点数的分布的 07\_uniform\_int\_distribution.cpp 或 08\_normal\_distribution.cpp 文件。

\item
\textbf{附加题}:除了将数据导出到 CSV 文件外,同时在标准输出控制台上使用字符绘制直方图。
\end{itemize}













