随机数引擎适配器会修改与之关联的随机数引擎的结果,这个关联的引擎称为基础引擎。这是适配器模式的一个例子(请参见第33章)。以下定义了三个适配器模板:

\begin{cpp}
template<class Engine, size_t p, size_t r> class discard_block_engine {...}
template<class Engine, size_t w, class UIntT> class independent_bits_engine {...}
template<class Engine, size_t k> class shuffle_order_engine {...}
\end{cpp}

discard\_block\_engine适配器通过丢弃基础引擎生成的某些值来生成随机数。它需要三个参数:要适配的引擎、块大小p,以及使用的块大小r。基础引擎用于生成p个随机数。然后适配器丢弃p-r个这些数字,并返回剩余的r个数字。

independent\_bits\_engine适配器通过组合基础引擎生成的几个随机数,来生成具有给定位数w的随机数。

shuffle\_order\_engine适配器生成与基础引擎相同的随机数,但以不同的顺序提供它们,模板参数k是适配器使用的内部表的大小。根据请求,从该表中随机选择一个随机数,然后用基础引擎生成的新随机数进行替换。

与随机数引擎一样,也有一些预定义的引擎适配器可供使用。下一节将概述预定义的引擎和引擎适配器。


























