生成随机数之前,首先需要创建一个引擎的实例。如果使用基于软件的引擎,还需要定义一个分布。分布是一个数学公式,描述了在特定范围内数字是如何分布的。推荐创建引擎的方式是,使用前一部分讨论的预定义引擎。

以下示例使用了名为mt19937的预定义引擎,使用Mersenne twister引擎。这是一个基于软件的生成器。就像旧的rand()生成器一样,基于软件的引擎必须用一个种子进行初始化。srand()使用的种子通常是当前时间。现代C++中,建议使用random\_device生成种子:

\begin{cpp}
random_device seeder;
mt19937 engine { seeder() };
\end{cpp}

大多数现代系统都有一个具有适当熵的random\_device实现。如果不确定代码将运行的系统上,random\_device的实现是否具有适当的熵,可以使用基于时间的种子作为后备:

\begin{cpp}
random_device seeder;
const auto seed { seeder.entropy() ? seeder() : time(nullptr) };
mt19937 engine { static_cast<mt19937::result_type>(seed) };
\end{cpp}

定义一个分布。此示例使用均匀整数分布,范围为1到99。下一节将详细解释分布,这个均匀分布对于此示例来说足够简单:

\begin{cpp}
uniform_int_distribution<int> distribution { 1, 99 };
\end{cpp}

定义了引擎和分布,就可以通过调用分布的函数调用操作符,并传递引擎作为参数来生成随机数。对于此示例,这写作distribution(engine):

\begin{cpp}
println("{}", distribution(engine));
\end{cpp}

要使用基于软件的引擎生成随机数,需要指定引擎和分布。在第19章中引入的std::bind()工具可以用来在生成随机数时,去除需要指定分布和引擎的需求。以下示例使用与上一个示例相同的mt19937引擎和均匀分布,通过使用std::bind()将engine绑定到distribution()的第一个参数来定义generator。可以在不传递参数的情况下,调用generator()来生成随机数。然后演示了将generator()与受限范围的generate()算法结合使用,来填充一个包含十个随机数元素的vector。第20章中已经介绍过generate()算法。

\begin{cpp}
auto generator { bind(distribution, engine) };

vector<int> values(10);
ranges::generate(values, generator);

println("{:n}", values);
\end{cpp}

\begin{myNotic}{NOTE}
generate()算法会覆盖现有元素,而不会插入新元素。首先需要将vector的大小调整为所需元素的数量,然后调用generate()算法。上一个示例通过在构造函数中指定大小,来调整vector的大小。
\end{myNotic}

即使不知道generator的确切类型,仍然可以将generator传递给想要使用该generator的另一个函数。这里有几种选择:使用类型为std::function<int()>的参数,或使用函数模板。上一个示例可以适应一个名为fillVector()的函数来生成随机数。这里使用std::function进行实现:

\begin{cpp}
void fillVector(vector<int>& values, const function<int()>& generator)
{
    ranges::generate(values, generator);
}
\end{cpp}

这是一个受限函数模板变体:

\begin{cpp}
template <invocable T>
void fillVector(vector<int>& values, const T& generator)
{
    ranges::generate(values, generator);
}
\end{cpp}

可以使用简化的函数模板语法进一步简化:

\begin{cpp}
void fillVector(vector<int>& values, const auto& generator)
{
    ranges::generate(values, generator);
}
\end{cpp}

最后,函数可以这样使用:

\begin{cpp}
vector<int> values(10);
fillVector(values, generator);
\end{cpp}

\begin{myWarning}{WARNING}
随机数生成器不是线程安全的。如果需要在多个线程中生成随机数,应该在每个线程中创建一个生成器,而不是在多个线程间共享一个生成器。第27章会来介绍多线程。
\end{myWarning}








