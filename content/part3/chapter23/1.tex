C++11之前,可以使用C风格的srand()和rand()函数生成随机数。srand()函数需要在应用程序中调用一次,用于初始化随机数生成器,这个过程也称为设置种子。通常,会选择当前系统时间会作为种子。

\begin{myWarning}{WARNING}
需要确保为基于软件的随机数生成器使用高质量的种子。如果每次都使用相同的种子初始化随机数生成器,将创建相同的随机数序列。这就是为什么种子通常是当前系统时间的原因。
\end{myWarning}

生成器初始化后,就可以使用rand()生成随机数。以下示例展示了如何使用srand()和rand()。time()函数定义在<ctime>中,返回系统时间,通常编码为自系统纪元以来的秒数,纪元代表时间的开始。

\begin{cpp}
srand(static_cast<unsigned int>(time(nullptr)));
println("{}", rand());
\end{cpp}

可以使用以下函数生成一定范围内的随机数:

\begin{cpp}
int getRandom(int min, int max)
{
    return static_cast<int>(rand() % (max + 1UL - min)) + min;
}
\end{cpp}

旧的C风格rand()函数生成0到RAND\_MAX范围内的随机数,标准定义为至少为32,767。不能生成大于RAND\_MAX的随机数。某些系统上,例如GCC,RAND\_MAX是2,147,483,647,等于有符号整数的最大值。为了避免这样的系统上的算术溢出,getRandom()中的公式使用了无符号长整型(UL)计算。

rand()的低阶位通常不是很随机,所以使用先前的getRandom()函数在较小的范围内生成随机数,例如1到6,将无法提供好的随机性。

\begin{myNotic}{NOTE}
基于软件的随机数生成器永远不能生成真正的随机数,所以称为伪随机数生成器(PRNG),因为它们依赖于数学公式,从而给人一种随机的感觉。
\end{myNotic}

除了生成低质量的随机数之外,旧的srand()和rand()函数在灵活性方面也没有提供太多,例如:不能更改生成的随机数的分布。总之,强烈建议停止使用srand()和rand(),并开始使用<random>中的类。















