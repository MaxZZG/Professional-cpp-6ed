字符串流提供了一种使用流语义与字符串相结合的方式,这样就可以有一个代表文本数据的内存流。例如,在GUI应用程序中,可能会想要使用流来构建文本数据,但不是将文本输出到控制台或文件,而是想在GUI(如消息框或编辑控件)中显示结果。另一个例子是,想要将字符串流传递到不同的函数中,同时保持当前的读取位置,以便每个函数都可以处理流的一部分。字符串流在解析文本时也非常有用,因为流具有内置的标记化功能。

std::ostringstream类用于将数据写入字符串,而std::istringstream类用于从字符串读取数据。ostringstream中的o代表输出,而istringstream中的i代表输入,都定义在<sstream>中。由于ostringstream和istringstream继承了ostream和istream的行为,因此使用的方式也非常相似。

以下程序请求用户输入单词,并将它们输出到一个单独的ostringstream中,单词之间用逗号分隔,并用双引号包围。程序结束时,整个流通过str()成员函数转换为字符串对象,并写入控制台。

通过输入标记“done”或通过使用Control+D(Unix)或Control+Z(Windows)关闭输入流来停止输入。

\begin{cpp}
println("Enter tokens. "
        "Control+D (Unix) or Control+Z (Windows) followed by Enter to end.");
ostringstream outStream;
bool firstLoop { true };
while (cin) {
    string nextToken;
    print("Next token: ");
    cin >> nextToken;

    if (!cin || nextToken == "done") { break; }

    if (!firstLoop) { outStream << ", "; }
    outStream << '"' << nextToken << '"';
    firstLoop = false;
}
println("The end result is: {}", outStream.str());
\end{cpp}

从字符串流中读取数据也同样熟悉,以下函数从字符串输入流创建并填充一个Muffin对象(参见之前的示例)。流数据以固定格式表示,因此函数可以轻松地将其值转换为Muffin的setter调用。这个固定格式是Muffin的描述(在双引号之间),后面跟着大小,后面跟着真或假,取决于Muffin是否有巧克力碎片。例如,以下字符串是一个有效的Muffin:

\begin{shell}
"Raspberry Muffin" 12 true
\end{shell}

这里是实现。注意使用quoted操作符,从输入流读取带引号的字符串。

\begin{cpp}
Muffin createMuffin(istringstream& stream)
{
    Muffin muffin;
    // Assume data is properly formatted:
    // "Description" size chips

    string description;
    int size;
    bool hasChips;

    // Read all three values. Note that chips is represented
    // by the strings "true" and "false".
    stream >> quoted(description) >> size >> boolalpha >> hasChips;
    if (stream) { // Reading was successful.
        muffin.setSize(size);
        muffin.setDescription(description);
        muffin.setHasChocolateChips(hasChips);
    }
    return muffin;
}
\end{cpp}

\begin{myNotic}{NOTE}
将对象转换为“扁平化”类型,如字符串,通常称为编组。编组对于将对象保存到磁盘或通过网络发送很有用。
\end{myNotic}

字符串流相对于标准C++字符串的优势是,除了数据,还知道下一次读或写操作将发生的位置,也称为当前位置。

另一个优势是字符串流支持操作符和区域设置,以实现比字符串更强大的格式化。

最后,如果需要通过连接几个较小的字符串来构建一个字符串,使用字符串流将比直接连接字符串对象更为高效。
