
因为读写文件总是涉及位置和数据,所以文件非常适合流。在C++中,std::ofstream和ifstream类为文件提供了输出和输入功能,其定义在<fstream>中。

处理文件系统时,特别重要的是检测和处理错误情况。正在处理的文件可能位于刚刚脱机的网络文件存储上,或者可能会尝试写入位于已满磁盘上的文件,也许正在尝试打开一个当前用户没有权限的文件。可以通过使用前面描述的标准错误处理机制,来检测错误。

输出文件流与其他输出流之间的主要区别是,文件流构造函数可以接受文件名,以及期望以哪种模式打开它。默认模式是写入,ios\_base::out,从文件开始写入,覆盖现有数据。也可以通过在文件流构造函数中指定常量ios\_base::app作为第二个参数,以附加模式打开输出文件流。以下表格列出了有关读写模式的常量:


% Please add the following required packages to your document preamble:
% \usepackage{longtable}
% Note: It may be necessary to compile the document several times to get a multi-page table to line up properly
\begin{longtable}{|l|l|}
\hline
\textbf{常量} & \textbf{描述}                                    \\ \hline
\endfirsthead
%
\endhead
%
ios\_base::app    & 打开文件,并在每次写操作前跳转到文件末尾。
    \\ \hline
ios\_base::ate    & 打开文件,并在立即打开后跳转到文件末尾。
 \\ \hline
ios\_base::binary           & \begin{tabular}[c]{@{}l@{}}以二进制模式进行输入输出,而不是文本模式。
\end{tabular} \\ \hline
ios\_base::in     & 以只读模式打开文件,从开始处开始读取。
         \\ \hline
ios\_base::out              & \begin{tabular}[c]{@{}l@{}}以写入模式打开文件,从开始处开始写入,覆盖现有数据。
\end{tabular}            \\ \hline
ios\_base::trunc  & 输出选项。删除所有现有数据(截断)。
    \\ \hline
ios\_base::noreplace(C++23) & \begin{tabular}[c]{@{}l@{}}输出选项。以独占模式打开。如果文件已存在,打开将失败。
\end{tabular}     \\ \hline
\end{longtable}

模式可以组合使用。例如,如果想以二进制模式打开文件用于输出,同时截断现有数据,指定得打开模式如下所示:

\begin{cpp}
ios_base::out | ios_base::binary | ios_base::trunc
\end{cpp}

ifstream包含ios\_base::in模式,而ofstream包含ios\_base::out模式。

以下程序打开test.txt文件并将程序的参数写入其中。ifstream和ofstream析构函数会自动关闭底层的文件,没有必要显式调用close()。


\begin{cpp}
int main(int argc, char* argv[])
{
    ofstream outFile { "test.txt", ios_base::trunc };
    if (!outFile.good()) {
        println(cerr, "Error while opening output file!");
        return -1;
    }
    outFile << "There were " << argc << " arguments to this program." << endl;
    outFile << "They are: " << endl;
    for (int i { 0 }; i < argc; i++) {
        outFile << argv[i] << endl;
    }
}
\end{cpp}

\mySubsubsection{13.4.1.}{文本模式和二进制模式}

默认情况下,文件流以文本模式打开。如果指定了ios\_base::binary,则文件以二进制模式打开。

二进制模式下,请求流写入的的字节会写入文件。读取时,字节会以文件中的确切形式返回。

文本模式下,有些隐藏的转换正在发生:写入或从文件读取的每行都以\verb|\|n结尾。然而,文件中行的结束如何编码取决于操作系统。例如,在Windows上,一行以\verb|\|r\verb|\|n而不是单个\verb|\|n字符结束。因此,当文件以文本模式打开时,写入文件的以\verb|\|n结尾的行,底层实现会自动将\verb|\|n转换为\verb|\|r\verb|\|n后再写入文件。类似地,从文件中读取一行时,从文件中读取的\verb|\|r\verb|\|n会自动转换回\verb|\|n后再返回。

\mySubsubsection{13.4.2.}{使用seek()和tell()跳转}

seekx()和tellx()成员函数存在于所有输入和输出流中,seekx()成员函数允许在输入或输出流中的任意位置移动。seekx()有几种形式,对于输入流,成员函数称为seekg()(g代表get),对于输出流,它称为seekp()(p代表put)。可能会想知道为什么既有seekg()又有seekp()成员函数,而不是一个seek()成员函数。原因是,有既是输入又是输出的流,例如文件流。这种情况下,流需要记住一个读取位置和一个单独的写入位置,这也称作双向I/O,本章后面将详细介绍。

seekg()和seekp()各有两个重载。一个重载接受一个单一参数,即绝对位置,并移动到这个绝对位置。第二个重载接受一个偏移量和位置,并在给定位置上寻找相对偏移量。位置的类型是std::streampos,而偏移量的类型是std::streamoff;两者都按字节进行。有三个预定义的位置可用,如下表所示:

% Please add the following required packages to your document preamble:
% \usepackage{longtable}
% Note: It may be necessary to compile the document several times to get a multi-page table to line up properly
\begin{longtable}{|l|l|}
\hline
\textbf{位置} & \textbf{描述}               \\ \hline
\endfirsthead
%
\endhead
%
ios\_base::beg    & 流的起始       \\ \hline
ios\_base::end    & 流的末尾              \\ \hline
ios\_base::cur    & 流的当前位置 \\ \hline
\end{longtable}

例如,要在输出流中移动到绝对位置,可以使用seekp()的一个参数重载。如下例所示,使用constant ios\_base::beg将位置移动到流的开始:

\begin{cpp}
outStream.seekp(ios_base::beg);
\end{cpp}

输入流中移动与在输出流中移动相同,只是换成使用seekg()成员函数了:

\begin{cpp}
inStream.seekg(ios_base::beg);
\end{cpp}

两个参数的重载用于在流中移动相对位置。第一个参数规定要移动多少位置,第二个参数提供起始点。要相对于文件的开头移动,使用constant ios\_base::beg。要相对于文件末尾移动,使用ios\_base::end。要相对于当前位置移动,使用ios\_base::cur。例如,以下行将输出流的位置移动到开头后的第二个字节,注意整数会隐式转换为streampos和streamoff类型。

\begin{cpp}
outStream.seekp(2, ios_base::beg);
\end{cpp}

下一个例子将输入流的位置移动到,输入流的倒数第三个字节:

\begin{cpp}
inStream.seekg(-3, ios_base::end);
\end{cpp}

还可以使用tellx()成员函数查询流的当前位置,该函数返回一个streampos,指示当前位置。可以使用这个结果在执行seekx()之前记住当前标记位置,或者查询是否处于特定位置。对于输入流和输出流,tellx()也有独立版本。输入流使用tellg(),输出流使用tellp()。

以下代码检查输入流的位置,以确定是否位于开头:

\begin{cpp}
streampos curPos { inStream.tellg() };
if (ios_base::beg == curPos) {
    println("We're at the beginning.");
}
\end{cpp}

以下是一个示例程序,将所有内容整合在一起。这个程序写入一个名为test.out的文件,并执行以下测试:

\begin{enumerate}
\item
向文件输出字符串54321

\item
验证标记是否位于流中的位置5

\item
输出流中移动到位置2

\item
在位置2输出一个0,并关闭输出流

\item
打开一个输入流,用于test.out文件

\item
读取第一个标记作为整数

\item
确认值是54021
\end{enumerate}

\begin{cpp}
ofstream fout { "test.out" };
if (!fout) {
    println(cerr, "Error opening test.out for writing.");
    return 1;
}

// 1. Output the string "54321".
fout << "54321";

// 2. Verify that the marker is at position 5.
streampos curPos { fout.tellp() };
if (curPos == 5) {
    println("Test passed: Currently at position 5.");
} else {
    println("Test failed: Not at position 5!");
}

// 3. Move to position 2 in the output stream.
fout.seekp(2, ios_base::beg);

// 4. Output a 0 in position 2 and close the output stream.
fout << 0;
fout.close();

// 5. Open an input stream on test.out.
ifstream fin { "test.out" };
if (!fin) {
    println(cerr, "Error opening test.out for reading.");
    return 1;
}

// 6. Read the first token as an integer.
int testVal;
fin >> testVal;
if (!fin) {
    println(cerr, "Error reading from file.");
    return 1;
}

// 7. Confirm that the value is 54021.
const int expected { 54021 };
if (testVal == expected) {
    println("Test passed: Value is {}.", expected);
} else {
    println("Test failed: Value is not {} (it was {}).", expected, testVal);
}
\end{cpp}

\mySubsubsection{13.4.3.}{链接流}

可以在输入流和输出流之间建立链接,以实现访问时自动刷新的行为。换句话说,当从输入流请求数据时,其链接的输出流将自动刷新。这种行为对所有流都可用,但对于可能相互依赖的文件流尤其有用。

流链接通过tie()成员函数实现。要将输出流链接到输入流,请在输入流上调用tie(),并传递输出流的地址。要断开链接的话,传递nullptr即可。

以下程序将一个文件的输入流链接到另一个完全不同文件的输出流,也可以将链接到同一文件的输出流,但双向I/O(下一节介绍)可能是同时读写同一文件更优雅的方式。

\begin{cpp}
ifstream inFile { "input.txt" }; // Note: input.txt must exist.
ofstream outFile { "output.txt" };
// Set up a link between inFile and outFile.
inFile.tie(&outFile);
// Output some text to outFile. Normally, this would
// not flush because std::endl is not send.
outFile << "Hello there!";
// outFile has NOT been flushed.
// Read some text from inFile. This will trigger flush() on outFile.
string nextToken;
inFile >> nextToken;
// outFile HAS been flushed.
\end{cpp}

flush()成员函数定义在ostream基类中,也可以将一个输出流链接到另一个输出流中。例如:

\begin{cpp}
outFile.tie(&anotherOutputFile);
\end{cpp}

这样,每次你向一个文件写入数据时,已发送到另一个文件的缓冲数据将刷新。可以使用这种机制来保持两个文件内容同步。

这种流链接的一个例子是cout和cin之间的链接。每当尝试从cin输入数据时,cout会自动刷新。cerr和cout之间也有链接,所以cerr的输出都会刷新cout。另一方面,clog流没有链接到cout。这些流宽字符版本之间也有类似的链接。

\mySubsubsection{13.4.4.}{读取整个文件}

可以使用getline()读取整个文件的内容,通过指定\verb|\|0作为分隔符。只要文件内容中不包含\verb|\|0字符,这就可以工作。例如:

\begin{cpp}
ifstream inputFile { "some_data.txt" };
if (inputFile.fail()) {
    println(cerr, "Unable to open file for reading.");
    return 1;
}
string fileContents;
getline(inputFile, fileContents, '\0');
println("\"{}\"", fileContents);
\end{cpp}














