通过解决下面的练习,可以练习本章讨论的内容。所有练习的解决方案都可以在本书的网站\url{www.wiley.com/go/proc++6e}下载到源码。若在练习中卡住了,可以考虑先重读本章的部分内容,试着自己找到答案,再在从网站上寻找解决方案。

\begin{itemize}
\item
\textbf{练习13-1}: 回顾在之前章节中开发的Person类。使用练习9-2中的实现,并添加一个output()成员函数,将一个人的详细信息写入标准输出控制台。

\item
\textbf{练习13-2}: 从上一个练习的output()成员函数,将一个Person的详细信息写入标准输出控制台。将output()成员函数改为接受一个输出流作为参数,并将一个Person的详细信息写入该流。在main()中测试新实现,通过将一个Person写入标准输出控制台、字符串流和文件。注意,如何使用流成员函数将一个Person输出到各种不同的目标(输出控制台、字符串流、文件等)。

\item
\textbf{练习13-3}: 开发一个名为Database的类,用于存储Person(来自练习13-2)。提供一个add()成员函数,用于向数据库添加一个Person。还提供一个save()成员函数,接受一个文件名作为参数,将数据库中的所有人在该文件中保存。现有的文件内容都会删除,提供一个load()成员函数,接受一个文件名作为参数,从该文件加载数据库中的所有Person。提供一个clear()成员函数,用于从数据库中删除所有Person。最后,提供一个outputAll()成员函数,对数据库中的所有Person都调用output()。确保实现即使在人名中包含空格也能正常工作。

\item
\textbf{练习13-4:} 练习113-3中的Database将所有人的信息存储在一个文件中。为了练习文件系统库,这里需要将每个人的信息单独存储在一个文件中。修改save()和load()成员函数,接受一个目录作为参数,文件应存储到或加载自该目录。save()成员函数将数据库中的每个人保存到自己的文件中。每个文件的名称是该人的姓前名后,以下划线分隔,扩展名为.person。如果文件已存在,则覆盖。load()成员函数需要遍历给定目录中的所有.person文件,并加载它们。
\end{itemize}

