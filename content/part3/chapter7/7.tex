By solving the following exercises, you can practice the material discussed in this chapter. Solutions to all exercises are available with the code download on the book’s website at www.wiley.com/go/ proc++6e. However, if you are stuck on an exercise, first reread parts of this chapter to try to find an answer yourself before looking at the solution from the website.


\begin{itemize}
\item
Exercise 7-1: Analyze the following code snippet. Can you list any problems you find with it? You don’t need to fix the problems in this exercise; that will be for Exercise 7-2.

\begin{cpp}
const size_t numberOfElements { 10 };
int* values { new int[numberOfElements] };
// Set values to their index value.
for (int index { 0 }; index < numberOfElements; ++index) {
    values[index] = index;
}
// Set last value to 99.
values[10] = 99;
// Print all values.
for (int index { 0 }; index <= numberOfElements; ++index) {
    print("{} ", values[index]);
}
\end{cpp}

\item
Exercise 7-2: Rewrite the code snippet from Exercise 7-1 to use modern and safe C++ constructs.

\item
Exercise 7-3: Write a basic class to store a 3-D point with x, y, and z coordinates. Include a constructor accepting x, y, and z arguments. Write a function that accepts a 3-D point and prints out its coordinates using std::print(). In your main() function, dynamically allocate an instance of your class and subsequently call your function.

\item
Exercise 7-4: Earlier in this chapter, the following function is shown in the context of out-of-bounds memory access. Can you modernize this function using safe C++ alternatives? Test your solution in your main() function.

\begin{cpp}
void fillWithM(char* text)
{
    int i { 0 };
    while (text[i] != '\0') {
        text[i] = 'm';
        ++i;
    }
}
\end{cpp}
\end{itemize}