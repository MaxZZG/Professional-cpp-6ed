
C++中的内存管理是持续错误和bug的来源,这些bug源于动态内存分配和指针的使用。当程序中广泛使用动态内存分配,并将许多指针传递给对象时,很难记住在每个指针上精确地调用一次 delete 并恰当地执行。

做错事情的后果很严重:当多次释放动态分配的内存,或使用一个指向已经释放的内存的指针时,可能会导致内存损坏或致命的运行时错误;当忘记释放动态分配的内存时,会导致内存泄漏。

智能指针可以管理动态分配内存,并且是避免内存泄漏推荐的方式。智能指针可以持有动态分配的资源,如内存。当智能指针超出作用域或重置时,可以自动释放它所持有的资源。智能指针可以在函数的作用域内管理动态分配的资源,或者作为类的数据成员,还可以用于通过函数参数传递动态分配资源的拥有权。

C++为智能指针提供了几个极具吸引力的特性。首先,可以使用模板为任何指针类型编写一个类型安全的智能指针类(请参阅第12章)。其次,可以通过重载操作符(请参阅第15章)为智能指针对象提供接口,允许代码像使用普通原始指针一样使用智能指针对象。具体来说,可以重载 *, –>, 和 [] 操作符,使得客户端代码像使用普通指针一样使用智能指针对象。

有几种不同类型的智能指针,最简单的类型对资源具有独占/唯一的所有权。作为一个资源的唯一所有者,智能指针可以在超出作用域或重置时自动释放所引用的资源。标准库提供了 std::unique\_ptr,这是一个具有唯一所有权语义的智能指针。

稍微高级一点的智能指针类型允许共享所有权,几个这样的智能指针可以指向同一个资源。当这样的智能指针超出作用域或重置时,只有在最后一个指向该资源的智能指针释放时,才释放所引用的资源。标准库提供了支持共享所有权的 std::shared\_ptr。

这两种标准智能指针,unique\_ptr 和 shared\_ptr,都定义在<memory>头文件中,并在下一节中进行了详细讨论。

\begin{myNotic}{NOTE}
默认使用的智能指针应该是 unique\_ptr,只有在需要共享时才使用 shared\_ptr。
\end{myNotic}

\begin{myWarning}{WARNING}
永远不要将资源分配的结果分配给原始指针!无论使用什么资源分配方法,都要立即将资源指针存储在一个智能指针中,无论是 unique\_ptr 还是 shared\_ptr,或者使用其他 RAII 类(RAII 代表资源获取是初始化)。一个 RAII 类可以持有某种资源,并在适当的时候处理其释放,会在第32章中详细讨论。
\end{myWarning}

\mySubsubsection{7.5.1.}{unique\_ptr}

unique\_ptr 对资源具有独占所有权。当 unique\_ptr 销毁或重置时,将自动释放资源。其优点是,内存和资源总是在执行 return 语句或抛出异常时释放。当一个函数有多个 return 语句时,这简化了编码,因为不需要记住在每个 return 语句之前释放资源。

作为一个一般规则,总是将具有单一所有者的动态分配资源存储在 unique\_ptr 实例中。

\mySamllsection{创建 unique\_ptr}

以下函数,通过在堆上分配一个 Simple 对象并忽视释放,来演示泄漏内存:

\begin{cpp}
void leaky()
{
    Simple* mySimplePtr { new Simple{} }; // BUG! Memory is never released!
    mySimplePtr->go();
}
\end{cpp}

有时可能认为代码已经正确地释放了动态分配的内存,但这很可能不正确。考虑以下函数:

\begin{cpp}
void couldBeLeaky()
{
    Simple* mySimplePtr { new Simple{} };
    mySimplePtr->go();
    delete mySimplePtr;
}
\end{cpp}

这个函数动态分配了一个 Simple 对象,使用了该对象,然后正确地调用了 delete。然而,仍然可以在这个例子中看到内存泄漏!如果 go() 成员函数抛出异常,调用 delete 就不会执行,导致内存泄漏。

相反,应该使用 unique\_ptr,它使用 std::make\_unique() 工厂函数创建。unique\_ptr 是一个泛型的智能指针,可以指向任何类型的内存。这就是为什么它是一个类模板,而 make\_unique() 是一个函数模板。两者都需要在尖括号 < > 之间指定一个模板参数,指定 unique\_ptr 所需指向的内存类型。

以下函数使用 unique\_ptr 而不是原始指针。Simple 对象没有显式删除,但当 unique\_ptr 实例超出作用域(函数结束时,或者因为异常抛出)时,会在其析构函数中自动释放 Simple 对象。

\begin{cpp}
void notLeaky()
{
    auto mySimpleSmartPtr { make_unique<Simple>() };
    mySimpleSmartPtr->go();
}
\end{cpp}

这段代码使用了 make\_unique(),结合了 auto 关键字,因此只需要指定指针的类型,在本例中是 Simple。这是创建 unique\_ptr 的推荐方法。如果 Simple 构造函数需要参数,可以将它们作为 make\_unique() 的参数传递。

make\_unique() 使用值初始化。基本类型初始化为零,而对象默认构造。如果不需要这种值初始化,需要默认初始值,则可以跳过值初始化,通过使用 make\_unique\_for\_overwrite() 函数来提高性能,该函数会使用默认初始化。对于基本类型,则根本不初始化,而对象仍会默认构造。

也可以通过直接调用其构造函数来创建 unique\_ptr。请注意,必须两次提到 Simple:

\begin{cpp}
unique_ptr<Simple> mySimpleSmartPtr { new Simple{} };
\end{cpp}

正如本书前面所讨论的,类模板参数推断(CTAD)可以让编译器根据传递给类模板构造函数的参数来推断类模板的模板类型参数。使用vector v\{1,2\} 而替换 vector<int> v\{1,2\}。CTAD 对 unique\_ptr 不起作用,不能省略模板类型参数。

C++17 之前,必须使用 make\_unique() 不仅是因为其只指定类型一次,还因为安全原因!考虑以下调用 foo() 函数的示例:

\begin{cpp}
foo(unique_ptr<Simple> { new Simple{} }, unique_ptr<Bar> { new Bar { data() } });
\end{cpp}

如果 Simple 或 Bar 的构造函数,或者 data() 函数抛出异常,根据编译器优化,可能会有一个 Simple 或 Bar 对象泄露。而使用 make\_unique() 就不会有泄露:

\begin{cpp}
foo(make_unique<Simple>(), make_unique<Bar>(data()))
\end{cpp}

自 C++17 之后,对 foo() 的两次调用都是安全的,但我仍然推荐使用 make\_unique(),因为它生成的代码更容易阅读。

\begin{myNotic}{NOTE}
始终使用 make\_unique() 来创建 unique\_ptr。
\end{myNotic}

\mySamllsection{使用 unique\_ptr}

智能指针最伟大的特性是,无需用户学习大量新的语法。智能指针仍然可以像标准指针一样解引用(使用 * 或 ->)。在之前的示例中,使用 -> 操作符调用了 go() 成员函数:

\begin{cpp}
mySimpleSmartPtr->go();
\end{cpp}

就像标准指针一样,也可以这样:

\begin{cpp}
(*mySimpleSmartPtr).go();
\end{cpp}

get() 成员函数可以用来直接访问底层指针,需要将指针传递给需要原始指针的函数时非常有用。有一个这样的函数:

\begin{cpp}
void processData(Simple* simple) { /* Use the simple pointer... */ }
\end{cpp}

可以这样调用:

\begin{cpp}
processData(mySimpleSmartPtr.get());
\end{cpp}

可以使用 reset() 方法释放 unique\_ptr 底层的指针,并可以对指针进行设置:

\begin{cpp}
mySimpleSmartPtr.reset(); // Free resource and set to nullptr
mySimpleSmartPtr.reset(new Simple{}); // Free resource and set to a new
// Simple instance
\end{cpp}

使用 release() 方法将 unique\_ptr 底层的指针与 unique\_ptr 分离,这会返回资源的底层指针并设置智能指针为 nullptr。这样做实际上会使智能指针失去对资源的拥有权,因此调用者负责在完成使用后释放资源!这里有一个例子:

\begin{cpp}
Simple* simple { mySimpleSmartPtr.release() }; // Release ownership
// Use the simple pointer...
delete simple;
simple = nullptr;
\end{cpp}

由于 unique\_ptr 表示唯一所有权,不能复制!但有一个小提示,可以使用移动语义将一个 unique\_ptr 移动到另一个 unique\_ptr 上,这在第 9 章中有详细介绍。std::move() 可以用来显式地移动 unique\_ptr 的所有权,如以下代码片段所示。现在不要担心语法问题,第 9 章会解释清楚。

\begin{cpp}
class Foo
{
    public:
    Foo(unique_ptr<int> data) : m_data { move(data) } { }
    private:
    unique_ptr<int> m_data;
};

auto myIntSmartPtr { make_unique<int>(42) };
Foo f { move(myIntSmartPtr) };
\end{cpp}

\mySamllsection{unique\_ptr和 C 风格数组}

unique\_ptr 可以存储一个动态分配的旧 C 风格数组。以下创建了一个 unique\_ptr,持有10个整数的动态分配的 C 风格数组:

\begin{cpp}
auto myVariableSizedArray { make_unique<int[]>(10) };
\end{cpp}

myVariableSizedArray 的类型是 unique\_ptr<int[]>,并支持使用数组语法访问其元素:

\begin{cpp}
myVariableSizedArray[1] = 123;
\end{cpp}

就像非数组情况一样,make\_unique() 对数组中的所有元素使用值初始化,这与 std::vector 一样。对于基本类型,初始化为零。可以使用 make\_unique\_for\_overwrite() 函数创建一个带有默认初始化值的数组,这样基本类型就未初始化。应该尽量避免使用未初始化的数据,所以要谨慎地使用这个功能。

尽管可以使用 unique\_ptr 来存储一个动态分配的 C 风格数组,但还是建议使用标准库容器,如 std::array 或 std::vector。

\mySamllsection{自定义删除器}

默认情况下,unique\_ptr 使用标准 new 和 delete 操作符来分配和释放内存。可以改变这种行为,使用自己的分配和释放函数,在使用第三方 C 库时非常有用。有一个 C 库,要求使用 my\_alloc() 进行分配和 my\_free() 进行释放:

\begin{cpp}
int* my_alloc(int value) { return new int { value }; }
void my_free(int* p) { delete p; }
\end{cpp}

为了在适当的时候调用 my\_free() 释放已分配的资源,可以使用带有自定义删除器的 unique\_ptr:

\begin{cpp}
unique_ptr<int, decltype(&my_free)> myIntSmartPtr { my_alloc(42), my_free };
\end{cpp}

这段代码使用 my\_alloc() 为整数分配内存,而 unique\_ptr 通过调用 my\_free() 函数来释放内存。unique\_ptr 的这个特性也可以用来管理其他资源,不仅仅是内存。可以用来在 unique\_ptr 超出作用域时自动关闭文件或网络套接字。

不幸的是,与 unique\_ptr 一起使用自定义删除器的语法有点笨拙。需要指定自定义删除器的类型作为模板类型参数,这应该是一个接受单个指针作为参数并返回 void 的函数指针类型。这个例子中,使用decltype(\&my\_free) ,返回 my\_free() 函数的指针类型。shared\_ptr 使用自定义删除器更容易,接下来的 shared\_ptr 部分将展示如何使用 shared\_ptr 在超出作用域时自动关闭文件。

\mySubsubsection{7.5.2.}{shared\_ptr}

有时,几个对象或代码片段需要相同的指针的副本。unique\_ptr 不能复制,因此不适用于这种情况。std::shared\_ptr 是一个支持共享所有权的智能指针,可以复制。但如果有多个 shared\_ptr 实例指向同一个资源,如何知道何时实际释放资源呢?这通过引用计数来解决,这是即将讨论的主题:“引用计数的必要性”。先来看看如何构造和使用 shared\_ptr。

\mySamllsection{创建和使用shared\_ptrs}

使用 shared\_ptr 的方式与 unique\_ptr 类似。创建一个 shared\_ptr,使用 make\_shared(),比直接创建一个 shared\_ptr 更有效率:

\begin{cpp}
auto mySimpleSmartPtr { make_shared<Simple>() };
\end{cpp}

\begin{myWarning}{WARNING}
始终使用 make\_shared() 来创建 shared\_ptr
\end{myWarning}

就像 unique\_ptr 一样,类模板参数推断在 shared\_ptr 上不起作用,必须指定模板类型。

make\_shared() 使用值初始化,类似于 make\_unique()。如果这不符合要求,可以使用 make\_shared\_for\_overwrite() 进行默认初始化,类似于 make\_unique\_for\_verwrite()。

shared\_ptr 可以用来存储指向动态分配的 C 风格数组的指针,就像使用 unique\_ptr 一样。可以使用 make\_shared() 进行此类操作,就像使用 make\_unique() 一样。尽管现在可以将 C 风格数组存储在 shared\_ptr 中,我仍然建议使用标准库容器,而非 C 风格数组。

shared\_ptr 也支持 get() 和 reset() 成员函数,就像 unique\_ptr 一样。唯一的区别是,调用 reset() 时,只有当最后一个 shared\_ptr 销毁或重置时,底层的资源才会释放。shared\_ptr 不支持 release(),可以使用 use\_count() 成员函数来获取共享同一资源 shared\_ptr 实例的数量。

与 unique\_ptr 一样,shared\_ptr 默认使用标准 new 和 delete 操作符来分配和释放内存,或者使用 new[] 和 delete[] 来存储 C 风格数组。可以按照以下方式更改这种行为:

\begin{cpp}
// Implementations of my_alloc() and my_free() as before.
shared_ptr<int> myIntSmartPtr { my_alloc(42), my_free };
\end{cpp}

如上,不必将自定义删除器的类型指定为模板类型参数,因此这比 unique\_ptr 中的自定义删除器更容易。

以下示例使用 shared\_ptr 来存储文件指针。这个例子中,当 shared\_ptr 在其超出作用域时销毁,文件指针会自动通过调用 close() 函数来关闭。C++ 有适当面向对象的类来处理文件(参见第 13 章),这些类已经自动关闭它们的文件。这个使用旧 C 风格 open() 和 fclose() 函数的示例只是为了展示 shared\_ptrs 除了纯粹的内存之外还能用于什么。必须使用一个 C 风格库,而没有 C++ 替代品,并且具有类似于 open 和 close 资源的类似函数,可以像这个示例中一样将其封装在 shared\_ptr 中。

\begin{cpp}
void close(FILE* filePtr)
{
    if (filePtr == nullptr) { return; }
    fclose(filePtr);
    println("File closed.");
}
int main()
{
    FILE* f { fopen("data.txt", "w") };
    shared_ptr<FILE> filePtr { f, close };
    if (filePtr == nullptr) {
        println(cerr, "Error opening file.");
    } else {
        println("File opened.");
        // Use filePtr
    }
}
\end{cpp}

\mySamllsection{引用计数的必要性}

当一个具有共享所有权的智能指针,如 shared\_ptr,超出作用域或重置时,应该只在最后一个引用的智能指针释放时,才释放所引用的资源。这是如何实现的?一种解决方案,如 shared\_ptr 标准库智能指针所使用的,就是引用计数。

作为一个通用概念,引用计数是一种跟踪类或特定对象实例使用次数的技术。引用计数的智能指针会跟踪有多少智能指针,指向单个实际指针或单个对象指针。每当这样的引用计数智能指针复制时,都会创建一个新实例,并指向同一资源,并且引用计数会增加。当这样的智能指针实例超出作用域或重置时,引用计数会减少。当引用计数降至零时,没有其他所有者拥有该资源,智能指针会释放资源。

引用计数的智能指针解决了许多内存管理问题,例如双重释放。例如,有两个指向相同内存的原始指针。imple 类在本书的这一章中首次介绍,当实例创建和销毁时,会打印出消息。

\begin{cpp}
Simple* mySimple1 { new Simple{} };
Simple* mySimple2 { mySimple1 }; // Make a copy of the pointer.
\end{cpp}

删除这两个原始指针会导致双重释放:

\begin{cpp}
delete mySimple2;
delete mySimple1;
\end{cpp}

(理想情况下)永远不会找到这样的代码,但在涉及多个函数调用层时,其中一个函数删除内存,而另一个函数已经做了,可能会发生这种情况。

通过使用引用计数的 shared\_ptr 智能指针,可以避免双重释放:

\begin{cpp}
auto smartPtr1 { make_shared<Simple>() };
auto smartPtr2 { smartPtr1 }; // Make a copy of the pointer.
\end{cpp}

两个智能指针超出作用域或重置时,Simple 实例只释放一次。

这一切只有在没有原始指针参与时才能正确工作!使用 new 定位一些内存,然后创建两个指向同一原始指针的 shared\_ptr 实例:

\begin{cpp}
Simple* mySimple { new Simple{} };
shared_ptr<Simple> smartPtr1 { mySimple };
shared_ptr<Simple> smartPtr2 { mySimple };
\end{cpp}

这两个智能指针在销毁时都会尝试删除同一个对象。根据编译器,这段代码可能会崩溃!如果得到输出,可能会是以下内容:

\begin{shell}
Simple constructor called!
Simple destructor called!
Simple destructor called!
\end{shell}

天哪!一个构造函数调用和两个析构函数调用?惊讶于即使具有引用计数的 shared\_ptr 类也会以这种方式行为,这是正确的行为。拥有多个 shared\_ptr 实例指向同一内存的唯一安全方法就是,简单地复制这些 shared\_ptr。

\mySamllsection{转换shared\_ptr}

正如某个类型的原始指针可以强制转换为另一个类型的指针一样,存储特定类型的 shared\_ptr 也可以强制转换为存储另一种类型的 shared\_ptr。可以转换的类型有限制,但并非所有转换都是有效的。用于转换 shared\_ptr 的函数包括 const\_pointer\_cast(), dynamic\_pointer\_cast(), static\_pointer\_cast()和 reinterpret\_pointer\_cast()。这些行为和工作方式与非智能指针转换函数 const\_cast(), dynamic\_cast(), static\_cast()和reinterpret\_cast() 类似,这些函数在第 10 章中有详细讨论和示例。

注意,这些转换只适用于 shared\_ptr,而不适用于 unique\_ptr。

\mySamllsection{别名}

shared\_ptr 支持别名,这允许 shared\_ptr 与另一个拥有相同资源的 shared\_ptr(拥有指针)共享所有权,但指向不同的对象(存储指针)。可用于 shared\_ptr 指向对象的成员,同时拥有对象本身。以下是示例:

\begin{cpp}
class Foo
{
    public:
    Foo(int value) : m_data { value } { }
    int m_data;
};

auto foo { make_shared<Foo>(42) };
auto aliasing { shared_ptr<int> { foo, &foo->m_data } };
\end{cpp}

只有当shared\_ptrs (Foo和别名)都销毁时,Foo对象才会销毁。

拥有的指针用于引用计数,存储的指针在引用指针或对其调用get()时返回。

\begin{myWarning}{WARNING}
现代C++编码中,只有在不涉及所有权的情况下才允许使用原始指针!如果涉及到所有权,通常应使用unique\_ptr,如果需要共享所有权,则使用shared\_ptr。此外,应使用make\_unique()和make\_shared()来创建这些智能指针。这样,不需要直接调用new操作符,也不需要delete。
\end{myWarning}

\mySubsubsection{7.5.3.}{weak\_ptr}

C++ 中还有一种与 shared\_ptr 相关的智能指针类,称为 weak\_ptr。weak\_ptr 可以包含一个由 shared\_ptr 管理的资源的引用。weak\_ptr 不拥有资源,所以 shared\_ptr 不会阻止释放资源。weak\_ptr 在 weak\_ptr 销毁(例如超出作用域)时,就不会销毁指向的资源,它可以用来确定关联的 shared\_ptr 是否已经释放了资源。weak\_ptr 的构造函数需要一个 shared\_ptr 或另一个 weak\_ptr 作为参数。要获取 weak\_ptr 中存储的指针,需要将其转换为 shared\_ptr。有两种方法可以做到这一点:

\begin{itemize}
\item
使用 weak\_ptr 实例上的 lock() 成员函数,该函数返回一个 shared\_ptr。如果在此期间关联的 shared\_ptr 已经释放,返回的 shared\_ptr 将为 nullptr。

\item
创建一个新的 shared\_ptr 实例,并将 weak\_ptr 作为 shared\_ptr 构造函数的参数传递。如果关联的 shared\_ptr 已经释放,这将抛出 std::bad\_weak\_ptr 异常。
\end{itemize}

以下示例展示了 weak\_ptr 的使用方式:

\begin{cpp}
void useResource(weak_ptr<Simple>& weakSimple)
{
    auto resource { weakSimple.lock() };
    if (resource) { println("Resource still alive."); }
    else { println("Resource has been freed!"); }
}

int main()
{
    auto sharedSimple { make_shared<Simple>() };
    weak_ptr<Simple> weakSimple { sharedSimple };

    // Try to use the weak_ptr.
    useResource(weakSimple);

    // Reset the shared_ptr.
    // Since there is only 1 shared_ptr to the Simple resource, this will
    // free the resource, even though there is still a weak_ptr alive.
    sharedSimple.reset();

    // Try to use the weak_ptr a second time.
    useResource(weakSimple);
}
\end{cpp}

这段代码的输出:

\begin{shell}
Simple constructor called!
Resource still alive.
Simple destructor called!
Resource has been freed!
\end{shell}

weak\_ptr 也支持 C 风格数组,就像 shared\_ptr 一样。

\mySubsubsection{7.5.4.}{传递给函数}

函数接受指针作为其参数时,当涉及所有权转移或共享时,函数才应接受智能指针。要共享shared\_ptr的所有权,只需按值接受一个shared\_ptr作为参数。要转移unique\_ptr的所有权,只需按值接受一个unique\_ptr作为参数。后者需要使用第9章详细讨论的移动语义。

如果不涉及所有权转移或共享,则函数应简单地具有对底层资源的非const引用或const引用参数,或者nullptr是参数的有效值,则为原始指针。具有诸如const shared\_ptr<T>\&或const unique\_ptr<T>\&的参数类型通常没有太大意义。

\mySubsubsection{7.5.5.}{从函数返回}

标准智能指针 shared\_ptr、unique\_ptr 和 weak\_ptr 可以通过值轻松且高效地从函数返回,这要归功于第 1 章中的强制性和非强制性复制消除,以及第 9 章中讨论的移动语义。重要的是,从函数返回智能指针非常高效。可以创建以下 create() 函数,并在 main() 中使用:

\begin{cpp}
unique_ptr<Simple> create()
{
    auto ptr { make_unique<Simple>() };
    // Do something with ptr...
    return ptr;
}
int main()
{
    unique_ptr<Simple> mySmartPtr1 { create() };
    auto mySmartPtr2 { create() };
}

\end{cpp}

\mySubsubsection{7.5.6.}{enable\_shared\_from\_this}

从 std::enable\_shared\_from\_this 派生类允许成员函数在对象上调用时,安全地返回指向自己的 shared\_ptr 或 weak\_ptr。如果没有这个基类,返回有效 shared\_ptr 或 weak\_ptr 到这个的方法是添加一个 weak\_ptr 作为类的成员,然后返回它的副本或返回由它构造的 shared\_ptr。enable\_shared\_from\_this 类为从它派生的类添加了以下两个成员函数:

\begin{itemize}
\item
shared\_from\_this(): 返回一个与对象共享所有权的 shared\_ptr

\item
weak\_from\_this(): 返回一个跟踪对象所有权的 weak\_ptr
\end{itemize}

这是一个高级特性,就不详细讨论了,但以下代码简要展示了如何使用。shared\_from\_this() 和 weak\_from\_this() 都是公共成员函数。可能在公共接口中对这个from\_this部分感到困惑,所以作为一个演示,下面的 Foo 类定义了自己的成员函数 getPointer():

\begin{cpp}
class Foo : public enable_shared_from_this<Foo>
{
    public:
        shared_ptr<Foo> getPointer() {
            return shared_from_this();
        }
};

int main()
{
    auto ptr1 { make_shared<Foo>() };
    auto ptr2 { ptr1->getPointer() };
}

\end{cpp}

只有在对象的指针已经存储在 shared\_ptr 中后,才能对对象调用 shared\_from\_this();否则,会抛出 bad\_weak\_ptr 异常。示例中,make\_shared() 在 main() 中用于创建一个包含 Foo 实例的 shared\_ptr,称为 ptr1。创建 shared\_ptr 之后,允许对 Foo 实例调用 shared\_from\_this()。另一方面,可以调用 weak\_from\_this(),但如果在尚未将指针存储在 shared\_ptr 中的对象上调用 weak\_from\_this(),可能会返回一个空的 weak\_ptr。

以下是对 getPointer() 成员函数的错误实现:

\begin{cpp}
class Foo
{
    public:
        shared_ptr<Foo> getPointer() {
            return shared_ptr<Foo>(this);
        }
};
\end{cpp}

如果使用前面展示的 main() 代码,这个 Foo 的实现会导致双重释放。两个完全独立的 shared\_ptr (ptr1 和 ptr2) 指向同一个对象,都会在超出作用域时尝试删除该对象。

\CXXTwentythreeLogo{-40}{-50}

\mySubsubsection{7.5.7.}{智能指针与 C 风格函数的互动性}

C风格函数使用返回类型来指示函数,是否正确执行或是否发生错误。由于返回类型已经用于报告错误,因此可使用输出参数来从函数返回其他数据:

\begin{cpp}
using errorcode = int;
errorcode my_alloc(int value, int** data) { *data = new int { value }; return 0; }
errorcode my_free(int* data) { delete data; return 0; }
\end{cpp}

使用这种 C 风格 API,my\_alloc() 函数返回一个 errorcode,并在名为 data 的输出参数中返回已分配的数据。在 C++23 之前,不能直接使用智能指针,如 unique\_ptr与 my\_alloc() 一起使用,可以这样做:

\begin{cpp}
unique_ptr<int, decltype(&my_free)> myIntSmartPtr(nullptr, my_free);
int* data { nullptr };
my_alloc(42, &data);
myIntSmartPtr.reset(data);
\end{cpp}

这相对简单,但相当繁琐。C++23 引入了 std::out\_ptr() 和 inout\_ptr() 函数来解决这个问题,这两个函数都定义在<memory>中。使用这些函数,代码段可以写的更优雅:

\begin{cpp}
unique_ptr<int, decltype(&my_free)> myIntSmartPtr(nullptr, my_free);
my_alloc(42, inout_ptr(myIntSmartPtr));
\end{cpp}

如果确定传递给 inout\_ptr() 的指针是 nullptr,则可以使用 out\_ptr 代替。

\mySubsubsection{7.5.8.}{已删除的auto\_ptr}

C++11之前的旧标准库中包含了一个基本的智能指针实现,称为 auto\_ptr。不幸的是,auto\_ptr 存在一些严重的问题。其中一个问题是,当在标准库容器(如 vectors)中使用时,不能正确工作。C++11 正式弃用了 auto\_ptr,并且从 C++17 开始,已经完全从标准库中移除,被 unique\_ptr 和 shared\_ptr 所取代。在这里提到 auto\_ptr仅是为了了解它,并确保永远不会使用。

\begin{myWarning}{WARNING}
永远不要使用旧的 auto\_ptr 智能指针!默认使用 unique\_ptr,如果需要共享所有权,则使用 shared\_ptr。
\end{myWarning}
