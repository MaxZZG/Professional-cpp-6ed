
内存是计算机的底层组件,即使在像 C++ 这样的高级编程语言中,也需要面对如此底层的东西。对 C++ 中动态内存真正的工作原理有一个坚实的理解,是成为一名专业 C++ 程序员的必要条件。

\mySubsubsection{7.1.1.}{如何理解内存}

如果有一个关于内存中对象看起来是什么样的概念,理解动态内存就会容易得多。本书中,内存单位表示为一个带有标签的盒子。标签表示与内存对应的变量名,而盒子内的数据显示了内存的当前值。

例如,图 7.1 显示了以下代码行执行后内存的状态。这行代码应该在函数中,所以 i 是一个局部变量:

\begin{cpp}
int i { 7 };
\end{cpp}

\myGraphic{0.5}{content/part3/chapter7/images/1.png}{图 7.1}

i 称为自动变量,在栈上分配。程序流程离开变量声明的作用域时,就会自动释放。

使用 new 关键字时,内存在自由存储上分配。如果没有显式初始化,new 调用的内存分配不进行初始化,会那个位置内存存储随机数据。这种未初始化的状态在本章的图表中用问号表示。以下代码在栈上创建了一个初始化为 nullptr 的变量 ptr,然后为 ptr 指向的自由存储上的内存分配内存:

\begin{cpp}
int* ptr { nullptr };
ptr = new int;
\end{cpp}

这也可以这样写:

\begin{cpp}
int* ptr { new int };
\end{cpp}

\myGraphic{0.5}{content/part3/chapter7/images/2.png}{图 7.2}

图 7.2 展示了执行此代码后内存的状态。请注意,尽管 ptr 指向自由存储上的内存,但它仍然位于栈上。指针只是变量,可以存在于栈上或堆上,但动态内存总是分配在堆上。

\begin{myWarning}{WARNING}
根据 C++ 核心指南(C++ 核心指南的 ES.20 指南:“切记初始化对象。”),每次声明指针变量时,应该立即用一个适当的指针或 nullptr 初始化。不要让它保持未初始化状态!
\end{myWarning}

接下来的例子显示,指针既可以存在于栈上,也可以存在于堆上:

\begin{cpp}
int** handle { nullptr };
handle = new int*;
*handle = new int;
\end{cpp}

这段代码首先声明了一个指向整型指针的指针作为变量 handle。然后,动态地分配足够的内存来存储一个指向整型的指针,并将指向新内存的指针存储在 handle 中。接下来,内存 (*handle) 分配了一个指向另一个足够大的动态内存区域的指针,该区域可以存储整数。图 7.3 显示了两个指针级别,其中一个指针位于栈上(handle),另一个位于堆上 (*handle)。

\myGraphic{0.5}{content/part3/chapter7/images/3.png}{图 7.3}

\mySubsubsection{7.1.2.}{分配和释放}

为变量创建空间,需要使用 new 操作符。要将该空间释放给程序的其他部分使用,需要使用 delete 操作符。

\mySamllsection{使用 new 和 delete}

想要分配一个内存块时,调用 new 并传入需要空间的变量类型。new 返回该内存的指针,尽管需要将这个指针存储在一个变量中。如果忽略 new 的返回值,或者指针变量超出作用域,内存就变得“无家可归”,无法再访问,这称为内存泄漏。

例如,以下代码遗弃了足够存储一个 int 的内存。图 7.4 显示了代码执行后内存的状态。当自由存储上有数据块,且无法从栈上直接或间接访问时,内存就会悬吊或泄漏。

\myGraphic{0.5}{content/part3/chapter7/images/4.png}{图 7.4}

\begin{cpp}
void leaky()
{
    new int; // BUG! Orphans/leaks memory!
    println("I just leaked an int!");
}
\end{cpp}

在找到一种方法来制造具有无限快速内存的计算机之前,将需要告诉编译器何时可以释放与对象关联的内存并用于其他目的。为了在自由存储上释放内存,使用 delete 关键字,并提供指向该内存的指针:

\begin{cpp}
int* ptr { new int };
delete ptr;
ptr = nullptr;
\end{cpp}

\begin{myWarning}{WARNING}
作为一条经验法则,每一行使用 new 分配内存的代码,以及使用原始指针而不是存储在智能指针中的指针,都应该对应于释放相同内存的另一行代码,即 delete。
\end{myWarning}

\begin{myNotic}{NOTE}
建议在释放内存后将其指针设置为 nullptr。这样,就不会意外地使用一个已经释放的内存的指针。值得注意的是,可以对 nullptr 指针调用 delete——什么也会不做。
\end{myNotic}

\mySamllsection{malloc 怎么样?}

如果是 C 语言开发者,可能会想知道 malloc() 函数有什么问题。在 C 中,malloc() 用于分配指定数量的字节的内存,使用 malloc() 简单直接。在 C++ 中,malloc() 函数仍然存在,但应该避免使用。new 相对于 malloc() 的主要优势在于,new 不仅仅分配内存,还会构造对象!

例如,考虑以下两行代码,使用一个假设的类 Foo:

\begin{cpp}
Foo* myFoo { (Foo*)malloc(sizeof(Foo)) };
Foo* myOtherFoo { new Foo{} };
\end{cpp}

执行这些行之后,myFoo 和 myOtherFoo 都指向堆中的内存区域,这些区域足够大以容纳一个 Foo 对象。可以使用这两个指针访问 Foo 的数据成员和成员函数。区别在于,myFoo 指向的 Foo 对象不是一个真正的对象,因为从未调用其构造函数。malloc() 函数只是为指定的大小分配了一块内存,不知道也不关心对象。相反,new 调用会分配适当的内存大小,并调用适当的构造函数来构造对象。

类似地,free() 函数和 delete 运算符之间也存在类似的区别。使用 free() 时,不会调用对象的析构函数。使用 delete 时,会调用析构函数,并且会正确清理对象。

\begin{myWarning}{WARNING}
C++ 中避免使用 malloc() 和 free()。
\end{myWarning}

\mySamllsection{内存分配失败时}

许多开发者编写代码时都假设 new 总是成功的。理由是如果 new 失败,则内存不足,这种情况非常糟糕。这通常是一个难以理解的状态,这种情况下,程序可能不知道能做什么。

默认情况下,当 new 失败时会抛出一个异常,例如如果请求的内存不足。这个异常没有捕捉,程序将终止。许多程序中,这种行为是可以接受的。第 1 章介绍了异常,第 14 章更详细地解释了异常,并提供了一些方法来优雅地从内存不足的情况中恢复。

还有一种 new 的替代版本,不会抛出异常。如果分配失败,它会返回 nullptr,类似于 C 中 malloc() 的行为。使用这种版本的语法如下:

\begin{cpp}
int* ptr { new(nothrow) int };
\end{cpp}

语法有点奇怪:真的会写“nothrow”作为 new 的一个参数(实际上就是)。

当然,仍然面临与抛出异常版本相同的問題——当结果为 nullptr 时该怎么办?编译器不检查该结果,所以nothrow 版本的 new 比抛出异常更容易导致其他错误。出于这个原因,建议你使用标准的 new。如果程序需要从内存不足的情况中恢复,第 14 章中的技术应足以满足需求。

\mySubsubsection{7.1.3.}{数组}

数组将多个相同类型的变量打包成一个单一的变量,并带有索引。对初学者来说,与数组一起工作很快就会变得自然,很容易想象值在编号的槽位中。数组在内存中的表示与这种方式差不多。

\mySamllsection{原始类型数组}

当程序为数组分配内存时,它是在分配连续的内存块,其中每个块都足够大以存储数组的一个元素。例如,一个本地包含五个 int 的数组可以在栈上声明如下:

\begin{cpp}
int myArray[5];
\end{cpp}

这种原始类型数组的单个元素是未初始化的,包含内存中的那个位置的任意内容。图 7.5 显示了数组创建后的内存状态。在栈上创建数组时,大小必须是编译时已知的常量值。

\myGraphic{0.5}{content/part3/chapter7/images/5.png}{图 7.5}

\begin{myNotic}{NOTE}
一些编译器允许在栈上创建变尺寸的数组。这不是 C++ 的标准特性,所以我建议谨慎使用。
\end{myNotic}

当在栈上创建数组时,可以使用初始化列表为提供初始元素:

\begin{cpp}
int myArray[5] { 1, 2, 3, 4, 5 };
\end{cpp}

如果初始化列表包含的元素少于数组的大小,数组中剩余的元素将被零初始化(见第 1 章),例如:

\begin{cpp}
int myArray[5] { 1, 2 }; // 1, 2, 0, 0, 0
\end{cpp}

要对所有元素进行零初始化,可以这样写:

\begin{cpp}
int myArray[5] { }; // 0, 0, 0, 0, 0
\end{cpp}

使用初始化列表时,编译器可以自动推断出元素的数量,从而无需显式声明数组的大小:

\begin{cpp}
int myArray[] { 1, 2, 3, 4, 5 };
\end{cpp}

在堆上声明数组没有什么不同,只是使用一个指针来引用数组的位置。下面的代码为5个未初始化的int型数组分配内存,并将指向该内存的指针存储在myArrayPtr变量:

\begin{cpp}
int* myArrayPtr { new int[5] };
\end{cpp}

如图7.6所示,基于存储的空数组类似于基于堆栈的数组,但位于不同的位置。myArrayPtr指向数组的第0个元素。

与new操作符一样,new[]接受一个nothrow参数来返回nullptr,而不是在分配失败时抛出异常:

\begin{cpp}
int* myArrayPtr { new(nothrow) int[5] };
\end{cpp}

\myGraphic{0.5}{content/part3/chapter7/images/6.png}{图 7.6}

在堆区动态创建的数组,也可以用初始化列表进行初始化:

\begin{cpp}
int* myArrayPtr { new int[] { 1, 2, 3, 4, 5 } };
\end{cpp}

每次对new[]的调用都应该与delete[]的调用配对以清理内存。注意delete[]后面的方括号[]!

\begin{cpp}
delete [] myArrayPtr;
myArrayPtr = nullptr;
\end{cpp}

将数组放在自由存储上的优势在于,可以根据运行时的情况定义其大小。例如,以下代码片段从假设的函数 askUserForNumberOfDocuments() 接收想要的文档数量,并使用该结果创建一个 Document 对象的数组。

\begin{cpp}
Document* createDocumentArray()
{
    size_t numberOfDocuments { askUserForNumberOfDocuments() };
    Document* documents { new Document[numberOfDocuments] };
    return documents;
}
\end{cpp}

记住,每次调用 new[] 都应该与一次调用 delete[] 配对。createDocumentArray() 的调用者使用 delete[] 来清理返回的内存是非常重要的。另一个问题是 C 风格的数组不知道它们的大小;因此,createDocumentArray() 的调用者不知道返回数组中有多少个元素!

前面的函数中,documents 是一个动态分配的数组。不要将这个与动态数组混淆。数组本身不是动态的,因为一旦分配,大小就不会改变。动态内存允许在运行时指定已分配块的大小,但不会自动调整其大小以适应数据。

\begin{myNotic}{NOTE}
有一些数据结构会动态调整大小,并且知道它们实际的大小,例如标准库容器。因为它们更安全,应该使用这样的容器而不是 C 风格数组。
\end{myNotic}

C++ 中有一个函数叫做 realloc(),这是 C 语言保留下来的。不要使用它!在 C 中,realloc() 用于有效地改变数组的大小,通过分配一个新块的内存,其大小为新大小,将所有旧数据复制到新位置,并删除原始块。这种方法在 C++ 中极其危险,因为用户定义的对象对位复制不会做出很好的响应。

\begin{myWarning}{WARNING}
在 C++ 中永远不要使用 realloc()!它不是你的朋友。
\end{myWarning}

\mySamllsection{对象数组}

对象数组与原始/基本类型数组没有不同,除了元素初始化。当使用 new[N] 分配一个 N 个对象的数组时,为 N 个连续的块分配了足够的空间,每个块都足够大以容纳一个单独的对象。对于对象数组,new[] 会自动为每个对象调用零参数(= default)构造函数,而原始类型数组默认具有未初始化的元素,所以使用 new[] 分配对象数组,会返回一个指向完全构造和初始化对象的数组的指针。

例如,考虑以下类:

\begin{cpp}
class Simple
{
    public:
        Simple() { println("Simple constructor called!"); }
        ~Simple() { println("Simple destructor called!"); }
};
\end{cpp}

如果分配一个包含四个Simple对象的数组,则调用Simple构造函数四次。

\begin{cpp}
Simple* mySimpleArray { new Simple[4] };
\end{cpp}

图7.7展示了这个数组的内存图,它与基本类型的数组没有什么不同。

\myGraphic{0.5}{content/part3/chapter7/images/7.png}{图 7.7}

\mySamllsection{删除数组}

使用 new[](即 new 的数组版本)分配内存时,必须使用 delete[](即 delete 的数组版本)来释放。这个版本会自动销毁数组中的对象,并释放相关内存。

\begin{cpp}
Simple* mySimpleArray { new Simple[4] };
// Use mySimpleArray...
delete [] mySimpleArray;

mySimpleArray = nullptr;
\end{cpp}

如果不使用数组版本的delete,程序可能会以奇怪的方式运行。对于某些编译器,只调用数组第一个元素的析构函数,编译器只知道正在删除指向对象的指针,而数组的所有其他元素将成为孤立对象。对于其他编译器,因为new和new[]可以使用完全不同的内存分配方案,很可能会发生内存损坏。

\begin{myWarning}{WARNING}
对分配new的对象使用delete,对分配new[]的对象使用delete[]。
\end{myWarning}

当然,只有当数组的元素是对象时才调用析构函数。如果有一个指针数组,仍然需要删除每个单独指向的对象,就像单独分配每个对象一样:

\begin{cpp}
const size_t size { 4 };
Simple** mySimplePtrArray { new Simple*[size] };

// Allocate an object for each pointer.
for (size_t i { 0 }; i < size; ++i) { mySimplePtrArray[i] = new Simple{}; }

// Use mySimplePtrArray...
// Delete each allocated object.
for (size_t i { 0 }; i < size; ++i) {
    delete mySimplePtrArray[i];
    mySimplePtrArray[i] = nullptr;
}

// Delete the array itself.
delete [] mySimplePtrArray;
mySimplePtrArray = nullptr;
\end{cpp}

\begin{myWarning}{WARNING}
现代C++中,当涉及所有权时,应避免使用原始C风格的指针。相比于在C风格的数组中存储原始指针,应该在现代标准库容器(如std::vector)中存储智能指针。智能指针会在适当的时候自动释放内存。
\end{myWarning}

\mySamllsection{多维数组}

多维数组将索引值的概念扩展到多个索引上。例如,井字游戏可能会使用二维数组来表示一个3x3的网格。下面的例子展示了如何在栈上声明这样一个数组,进行零初始化,并通过一些测试代码访问:

\begin{cpp}
char board[3][3] {};
// Test code
board[0][0] = 'X'; // X puts marker in position (0,0).
board[2][1] = 'O'; // O puts marker in position (2,1).
\end{cpp}

在二维数组中,第一个下标代表的是x坐标还是y坐标。实际上,这并不重要,只要保持一致即可。一个4x7的网格可以声明为char board[4][7]或char board[7][4]。对于大多数应用来说,将第一个下标视为x轴,第二个下标视为y轴最为直观。

\mySamllsection{基于栈的多维数组}

内存中,基于栈的3x3二维board数组看起来像图7.8所示。因为内存没有两个轴(地址仅仅是顺序的),计算机将二维数组就像一维数组一样表示,不同之处在于数组的大小和访问的方法。

\myGraphic{0.5}{content/part3/chapter7/images/8.png}{图 7.8}

多维数组的大小是其所有维度相乘,然后再乘以数组中单个元素的大小。图7.8中,3x3的board是3 × 3 × 1 = 9字节,假设字符占1字节。对于4x7的字符数组,数组将是4 × 7 × 1 = 28字节。

访问多维数组中的值时,计算机将每个下标视为如果正在访问多维数组内的另一个子数组。例如,在3x3的网格中,表达式board[0]实际上指的是图7.9中高亮显示的子数组。当添加第二个下标,比如board[0][2],就能够通过在子数组中查找第二个下标来访问正确的元素,如图7.10所示。

这些技术可扩展到N维数组,高于三维的维度往往难以理解,并且很少使用。

\myGraphic{0.5}{content/part3/chapter7/images/9.png}{图 7.9}

\myGraphic{0.5}{content/part3/chapter7/images/10.png}{图 7.10}

\mySamllsection{基于堆的多维数组}

如果需要在运行时确定多维数组的维度,可以使用基于堆的数组。如通过指针访问单一维度动态分配的数组一样,多维度动态分配的数组也可通过指针访问的。唯一的区别在于,在二维数组中,需要从指向指针的指针开始;而在N维数组中,需要N级指针。起初,似乎声明和分配动态多维数组的正确方式如下:

\begin{cpp}
char** board { new char[i][j] }; // BUG! Doesn't compile
\end{cpp}

这段代码无法编译,因为基于堆的多维数组工作方式与基于栈的数组不同,其内存布局不是连续的。相反,需要首先为基于自由存储器数组的第一个下标维度分配一个连续的数组。该数组的每个元素实际上是另一个指针,指向存储第二个下标维度元素的数组。图7.11展示了这种针对2x2动态分配board的布局。

\myGraphic{0.5}{content/part3/chapter7/images/11.png}{图 7.11}

不幸的是,编译器不会自动分配子数组的内存。可以像单一维度的自由存储器数组一样分配第一维数组,但各个子数组必须显式分配。以下函数正确地为二维数组分配了内存:

\begin{cpp}
char** allocateCharacterBoard(size_t xDimension, size_t yDimension)
{
    char** myArray { new char*[xDimension] }; // Allocate first dimension
    for (size_t i { 0 }; i < xDimension; ++i) {
        myArray[i] = new char[yDimension]; // Allocate ith subarray
    }
    return myArray;
}
\end{cpp}

同样地,想要释放与基于堆的多维数组相关的内存时,数组delete[]语法并不会清理子数组。释放数组的代码应该与分配数组的代码相呼应,如下所示:

\begin{cpp}
void releaseCharacterBoard(char**& myArray, size_t xDimension)
{
    for (size_t i { 0 }; i < xDimension; ++i) {
        delete [] myArray[i]; // Delete ith subarray
        myArray[i] = nullptr;
    }
    delete [] myArray; // Delete first dimension
    myArray = nullptr;
}
\end{cpp}

\begin{myNotic}{NOTE}
这个分配多维数组的例子并不是最有效的解决方案。它首先为第一个维度分配内存,然后为每个子数组分配内存。这导致内存中的内存块分散,这会影响在这样数据结构上工作的算法的性能。如果算法可以处理连续的内存,会运行得更快。一个更好的解决方案是分配一个足够大的单块内存来存储 xDimension * yDimension 个元素,并通过公式 x * yDimension + y 访问位置 (x,y) 的元素。
\end{myNotic}

现在了解了所有关于数组工作的细节,建议尽可能避免使用这些旧的 C 风格数组,它们没有提供任何内存安全性。新代码中,应该使用 C++ 标准库容器,如 std::array 和 vector。例如,使用 vector<T> 作为一维动态数组。对于二维动态数组,可以使用 vector<vector<T>{}>,对于更高维度也是如此。当然,直接使用像 vector<vector<T>{}> 这样的数据结构仍然很麻烦,尤其是在构建时,它们也受到与前述注释中讨论的相同内存碎片问题的影响。因此,如果应用程序确实需要 N 维动态数组,考虑编写辅助类,以提供更容易使用的接口。例如,为了与具有等长行的二维数据一起工作,应该考虑编写(或当然重新使用)一个 Matrix<T> 或 Table<T> 类模板,该模板隐藏了用户内存分配/释放和元素访问算法。有关编写类模板的详细信息,请参见第 12 章。

\begin{myWarning}{WARNING}
使用 C++ 标准库容器,如 std::array、vector 等,而不是 C 风格数组!
\end{myWarning}

\mySubsubsection{7.1.4.}{使用指针}

指针的坏名声来自于滥用。因为指针只是一个内存地址,理论上可以手动改变这个地址,甚至做一些像下面这行代码一样可怕的事情:

\begin{cpp}
char* scaryPointer { (char*)7 };
\end{cpp}

这行代码构建了一个指向内存地址 7 的指针,这很可能是随机的垃圾或应用程序中其他地方使用的内存。如果开始使用没有保留,例如通过 new 或在栈上,最终会破坏与对象相关的内存,或者涉及堆管理的内存,并且程序将出现故障。这种故障可能以几种方式表现出来。例如,因为数据破坏而导致的无效结果,或者由于访问不存在的内存或尝试写入受保护的内存而触发的硬件异常。如果很幸运,可能会得到通常导致操作系统或 C++ 运行时库终止程序的严重错误;如果不走运,只会得到错误的结果。

\mySamllsection{指针的概念}

有两种方式可以思考指针,更数学思维的读者可能会将指针视为地址。这种观点使得在本书稍后部分介绍的指针算术更容易理解。指针并不是内存中神秘的路径;它们是恰好对应于内存位置的数字。图 7.12 描述了基于地址的世界中的一个两乘二网格。(图 7.12 中的地址仅用于说明目的。真实系统中的地址高度依赖于硬件和操作系统。)

\myGraphic{0.5}{content/part3/chapter7/images/12.png}{图 7.12}

对于更擅长空间像想的读者来说,指针的“箭头”视图可能更有益。指针是一种间接性,告诉程序,“嘿!看那里。”在这个视图中,多个指针级别成为到达数据的路径上的单个步骤。图 7.11 展示了内存中的指针。

使用 * 运算符解引用一个指针时,是在告诉程序在内存中查看更深一层。基于地址的图中,将解引用视为一个跳转到指针指示的地址。在图中,每次解引用都对应于从其基础到其头部跟随箭头。

当获取一个位置的地址,使用 \& 运算符时,就是在内存中添加一个间接性引用。在基于地址的图中,程序是在记录位置的数字地址,可以存储为指针。在图中,\& 运算符创建了一个新箭头,其头部结束于表达式指定的位置,箭头的基部也可以存储为指针。

\mySamllsection{使用指针进行强制转换}

因为指针只是内存地址(或指向某处),所以它们在某种程度上是弱类型的。指向 XML 文档的指针与指向整数的指针大小相同。编译器允许使用 C 风格强制类型转换,地将指针类型转换为其他指针类型:

\begin{cpp}
Document* documentPtr { getDocument() };
char* myCharPtr { (char*)documentPtr };
\end{cpp}

当然,使用结果指针可能会导致灾难性的运行时错误。静态类型转换提供了一定程度的安全性。编译器拒绝在无关数据类型的指针之间执行静态类型转换:

\begin{cpp}
Document* documentPtr { getDocument() };
char* myCharPtr { static_cast<char*>(documentPtr) }; // BUG! Won't compile
\end{cpp}

第 10 章讨论了不同类型的强制类型转换。














