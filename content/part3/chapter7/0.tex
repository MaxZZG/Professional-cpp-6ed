\noindent
\textbf{内筒概要}

\begin{itemize}
\item
使用和管理内存

\item
数组和指针之间令人困惑的关系

\item
了解内存工作的底层实现

\item
常见的内存陷阱

\item
智能指针以及如何使用
\end{itemize}

本章的所有代码示例都可以在\url{https://github.com/Professional-CPP/edition-6/}获得。

C++ 是一个安全的语言,该语言提供了许多道路、行车道和交通灯,例如 C++ 核心指南(见附录 B),静态代码分析器来分析代码的正确性。

然而,C++ 确实允许离开行车道。一种离开行车道的方式是手动内存管理(分配和释放),这种方式是 C++ 编程中特别容易出错的一个领域。为了编写高质量的 C++ 程序,专业的 C++ 程序员需要理解内存工作背后的原理。第 3 部分的第一个章节探讨了内存管理的细节。将了解动态内存的陷阱,以及一些避免和消除它们的技术。

本章讨论了底层内存处理,因为专业的 C++ 开发则会遇到这样的代码。现代 C++ 代码中,应尽可能避免底层内存操作。例如,应使用标准库容器,如 vector,而不是动态分配的 C 风格数组,后者会自动为处理所有内存管理。应使用智能指针,如 unique\_ptr 和 shared\_ptr,这些将在本章后面讨论,它们会在不再需要时自动释放底层资源,如内存。基本目标是避免在代码中出现内存分配例程,如 new/new[] 和 delete/delete[]。当然,这可能并非总是可能的,而且在现有代码中很可能不是这样,因此作为专业的 C++ 开发者,仍然需要知道内存工作的原理。

\begin{myWarning}{WARNING}
现代 C++ 代码中,应尽可能避免底层内存操作,当涉及所有权时避免使用原始指针,并避免使用旧的 C 风格构造和函数。相反,使用安全的 C++ 替代品,如自动管理其内存的对象,如 C++ 字符串类、vector 容器、智能指针等!
\end{myWarning}




