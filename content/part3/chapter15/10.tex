
C++ 有一系列内置的字面量类型,可以在代码中使用它们。以下是一些示例:

\begin{itemize}
\item
'a': 字符

\item
"A string": 以零终止的字符序列,C 风格字符串

\item
3.14f: 单精度浮点值

\item
0xabc: 十六进制值
\end{itemize}

C++ 还允许定义自己的字面量,标准库就是这样做的;提供了构建标准库对象的字面量类型。我们先来看看这些,然后再看看如何自定义。

\mySubsubsection{15.10.1.}{标准库字面量}

C++ 标准库定义了以下标准字面量。注意,这些字面量不以下划线开头:

% Please add the following required packages to your document preamble:
% \usepackage{longtable}
% Note: It may be necessary to compile the document several times to get a multi-page table to line up properly
\begin{longtable}{|l|l|l|l|}
\hline
\textbf{字面量} &
\textbf{\begin{tabular}[c]{@{}l@{}}创建…的实例\end{tabular}} &
\textbf{示例} &
\textbf{需要的命名空间} \\ \hline
\endfirsthead
%
\endhead
%
s &
string &
\begin{tabular}[c]{@{}l@{}}auto myString \{\\ "Hello"s \};\end{tabular} &
string\_literals \\ \hline
sv &
string\_view &
\begin{tabular}[c]{@{}l@{}}auto myStringView \{\\ "Hello"sv \};\end{tabular} &
string\_view\_literals \\ \hline
\begin{tabular}[c]{@{}l@{}}h, min, s,\\ ms, us, ns\end{tabular} &
chrono::duration &
\begin{tabular}[c]{@{}l@{}}auto myDuration \{\\ 42min \};\end{tabular} &
chrono\_literals \\ \hline
y, d &
chrono::year 和 day &
\begin{tabular}[c]{@{}l@{}}auto thisYear \{\\ 2024y \};\end{tabular} &
chrono\_literals \\ \hline
i, il, if &
\begin{tabular}[c]{@{}l@{}}complex\textless{}T\textgreater with\\ T 可以是 double,\\ long double, float\end{tabular} &
\begin{tabular}[c]{@{}l@{}}auto\\ myComplexNumber \{\\ 1.3i \};\end{tabular} &
complex\_literals \\ \hline
\end{longtable}

所有这些都是在 std::literals 的子命名空间中定义的,例如 std::literals::string\_literals。然而,string\_literals 和 literals 都是内联命名空间,它们自动使它们的内容在其父命名空间中可用。因此,如果想使用字符串字面量,可以使用以下任何一个 using:

\begin{cpp}
using namespace std;
using namespace std::literals;
using namespace std::string_literals;
using namespace std::literals::string_literals;
\end{cpp}


\mySubsubsection{15.10.2.}{自定义的字面量}

自定义的字面量应以确切的单下划线开头,比如: \_i, \_s, \_km, \_miles, \_K 等。

自定义的字面量通过编写字面量运算符来实现,字面量运算符可以在原始模式或已处理模式下工作。在原始模式下,字面量运算符接收一个字符序列,而在已烹饪模式下,字面量运算符接收一个特定的解释类型。例如,考虑 C++ 字面量 123。原始模式的字面量运算符接收这个作为字符序列 '1', '2', '3',已处理模式的字面量运算符接收整数 123。字面量 0x23 原始运算符接收为字符 '0', 'x', '2', '3',而已处理运算符接收整数 35。像 3.14 这样的字面量原始运算符接收为 '3', '.', '1', '4',而已处理运算符接收浮点值 3.14。

\mySamllsection{已处理模式的字面量运算符}

已处理模式的字面量运算符应该有以下之一:

\begin{itemize}
\item
处理数值:一个类型为 unsigned long long, long double, char, wchar\_t, char8\_t, char16\_t 或 char32\_t 的参数

\item
处理字符串:两个参数,第一个是 C 风格字符串,第二个是字符串的长度,例如,(const char* str, std::size\_t len)
\end{itemize}

例如,以下代码定义了一个存储米长度的 Length 类。构造函数是private,因为用户应该只能使用提供的用户定义的字面量来构造 Length 实例。代码为用户定义的字面量 \_km 和 \_m 提供了已处理的字面量运算符。这两个都是 Length 的友元,以便可以访问private构造函数。这些运算符之间,不能有空格。

\begin{cpp}
// A class representing a length. The length is always stored in meters.
class Length
{
    public:
        long double getMeters() const { return m_length; }
        // The user-defined literals _km and _m are friends of Length so they
        // can use the private constructor.
        friend Length operator ""_km(long double d);
        friend Length operator ""_m(long double d);
    private:
        // Private constructor because users should only be able to construct a
        // Length using the provided user-defined literals.
        Length(long double length) : m_length { length } {}
        long double m_length;
};
Length operator ""_km(long double d) // Cooked _km literal operator
{
    return Length { d * 1000 }; // Convert to meters.
}
Length operator ""_m(long double d) // Cooked _m literal operator
{
    return Length { d };
}
\end{cpp}

字面量运算符的使用如下所示:

\begin{cpp}
Length d1 { 1.2_km };
auto d2 { 1.2_m };
println("d1 = {}m; d2 = {}m", d1.getMeters(), d2.getMeters());
\end{cpp}

输出如下:

\begin{shell}
d1 = 1200m; d2 = 1.2m
\end{shell}

为了演示接受 const char* 和 size\_t 的烹饪字面量运算符的变体,我们可以重新创建标准库提供的字符串字面量 s,以构建一个 std::string。我们称这个字面量为 \_s。

\begin{cpp}
string operator ""_s(const char* str, size_t len)
{
    return string { str, len };
}
\end{cpp}

这个字面量运算符的使用如下所示:

\begin{cpp}
string str1 { "Hello World"_s };
auto str2 { "Hello World"_s }; // str2 has as type string
\end{cpp}

如果没有 \_s 字面量运算符,auto 会将类型推导为 const char*:

\begin{cpp}
auto str3 { "Hello World" }; // str3 has as type const char*
\end{cpp}


\mySamllsection{原始模式字面量运算符}

一个原始模式字面量运算符需要一个参数,类型为 const char*,即一个零终止的 C 风格字符串。以下示例定义了之前提到的 \_m 字面量运算符作为一个原始模式字面量运算符:

\begin{cpp}
Length operator ""_m(const char* str)
{
    // Implementation omitted; it requires parsing the C-style string
    // converting it to a long double, and constructing a Length.
    ...
}
\end{cpp}

使用这个原始模式字面量运算符与使用已处理模式版本的方式相同。

\begin{myNotic}{NOTE}
原始模式字面量运算符只与非字符串字面量一起工作。例如,1.23\_m 可以实现为一个原始模式字面量运算符,但 "1.23"\_m 不能。后者需要一个已处理模式字面量,带有两个参数:零终止字符串和长度。
\end{myNotic}














