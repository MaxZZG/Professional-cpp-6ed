
回到 SpreadsheetCell 示例,看看这两行代码:

\begin{cpp}
SpreadsheetCell cell { 1.23 };
double d1 { cell }; // DOES NOT COMPILE!
\end{cpp}

SpreadsheetCell 包含一个双精度浮点数表示,所以可以将它赋值给一个双精度变量,但不能这么做。编译器会说它不知道如何将 SpreadsheetCell 转换为双精度。尝试强制编译器做转换,比如这样:

\begin{cpp}
double d1 { (double)cell }; // STILL DOES NOT COMPILE!
\end{cpp}

但上述代码仍然无法编译,因为编译器仍然不知道如何将 SpreadsheetCell 转换为双精度。它已经在第一行知道你想要它做什么。还有,在程序中随意添加强制类型转换通常不是一个好习惯。

如果想允许这种转换,必须告诉编译器如何执行。具体来说,可以写一个转换操作符将 SpreadsheetCell 转换为双精度,像这样:

\begin{cpp}
operator double() const;
\end{cpp}

函数名是 operator double,没有返回类型,因为返回类型由操作符的名称指定:double。操作符为const,因为它不会改变被调用的对象。实现如下:

\begin{cpp}
SpreadsheetCell::operator double() const
{
    return getValue();
}
\end{cpp}

这就是需要做的,将 SpreadsheetCell 转换为双精度的转换操作符。现在编译器接受以下行,并在运行时正确处理:

\begin{cpp}
SpreadsheetCell cell { 1.23 };
double d1 { cell }; // Works as expected
\end{cpp}

可以为任何类型使用相同的语法编写转换操作符。例如,这里为 SpreadsheetCell 编写一个 std::string 转换操作符:

\begin{cpp}
operator std::string() const;
\end{cpp}

实现如下:

\begin{cpp}
SpreadsheetCell::operator std::string() const
{
    return doubleToString(getValue());
}
\end{cpp}

现在可以将 SpreadsheetCell 转换为字符串,但由于 string 的构造函数,以下代码不起作用:

\begin{cpp}
string str { cell };
\end{cpp}

相反,可以使用正常的赋值语法,而非统一初始化,或者使用显式的 static\_cast():

\begin{cpp}
string str1 = cell;
string str2 { static_cast<string>(cell) };
\end{cpp}


\mySubsubsection{15.8.1.}{Operator auto}

可以通过指定 auto ,而不是显式指定转换运算符的返回类型,让编译器推断类型。例如,SpreadsheetCell 的 double 转换运算符可以这样写:

\begin{cpp}
operator auto() const { return getValue(); }
\end{cpp}

这里有一个警告,带有 auto 返回类型推断的成员函数的实现必须对类用户可见。因此,这个例子直接在类定义中实现了这个方法。

此外,在第1章中我们知道 auto 会去除引用和 const 限定符。所以,如果 operator auto 返回一个类型 T 的引用,那么推断出的类型将是 T 通过值返回,会创建一个副本。如果需要,可以显式添加引用和 const 限定符;这里是一个例子:

\begin{cpp}
operator const auto&() const { /* ... */ }
\end{cpp}

\mySubsubsection{15.8.2.}{使用显式转换运算符解决歧义问题}

为 SpreadsheetCell 对象编写 double 转换运算符会引入歧义问题。看看这一行:

\begin{cpp}
SpreadsheetCell cell { 6.6 };
double d1 { cell + 3.4 }; // DOES NOT COMPILE IF YOU DEFINE operator double()
\end{cpp}

这行现在无法编译。实现 operator double() 之前,这行可以编译通过,所以现在有什么问题呢?问题在于,编译器不知道应该使用 operator double() 将 cell 转换为 double 并进行 double 加法,还是使用 double 构造函数将 3.4 转换为 SpreadsheetCell 并进行 SpreadsheetCell 加法。在实现 operator double() 之前,编译器只有一个选择:使用 double 构造函数将 3.4 转换为 SpreadsheetCell 并进行 SpreadsheetCell 加法。

C++11 之前,解决这个问题的常见方法是将问题中的构造函数显式化,以防止使用该构造函数的自动转换(参见第8章)。然而,我们可能不希望显式化这个构造函数,因为通常希望自动将 double 转换为 SpreadsheetCell。从 C++11 开始,可以通过将 double 转换运算符显式化,而不是构造函数来解决这个问题:

\begin{cpp}
explicit operator double() const;
\end{cpp}

有了这个改变,以下行就可以编译通过了:

\begin{cpp}
double d1 { cell + 3.4 }; // 10
\end{cpp}

正如前一小节讨论的,operator auto 也可以标记为explicit。

\mySubsubsection{15.8.3.}{布尔表达式中的转换}

有时,可以在布尔表达式中使用对象。例如,在条件语句中使用指针:

\begin{cpp}
if (ptr != nullptr) { /* Perform some dereferencing action. */ }
\end{cpp}

有时会写这样简写条件:

\begin{cpp}
if (ptr) { /* Perform some dereferencing action. */ }
\end{cpp}

其他情况下,可能会看到这样的代码:

\begin{cpp}
if (!ptr) { /* Do something. */ }
\end{cpp}

目前,以上所有表达式都不能与前面定义的 Pointer 智能指针类模板一起编译。为了让它们工作,可以向类中添加一个转换运算符,将其转换为指针类型。然后,与 nullptr 的比较,以及在 if 语句中单独使用对象,都会触发转换为指针类型。转换运算符通常的指针类型是 void*,那是一个除了在布尔表达式中测试它之外,无法做太多操作的指针类型。以下是实现:

\begin{cpp}
operator void*() const { return m_ptr; }
\end{cpp}

现在以下代码可以编译并通过,并按预期工作:

\begin{cpp}
void process(const Pointer<SpreadsheetCell>& p)
{
    if (p != nullptr) { println("not nullptr"); }
    if (p != 0)       { println("not 0"); }
    if (p)            { println("not nullptr"); }
    if (!p)           { println("nullptr"); }
}

int main()
{
    Pointer<SpreadsheetCell> smartCell { nullptr };
    process(smartCell);
    println("");

    Pointer<SpreadsheetCell> anotherSmartCell { new SpreadsheetCell { 5.0 } };
    process(anotherSmartCell);
}
\end{cpp}

输出如下:

\begin{shell}
nullptr

not nullptr
not 0
not nullptr
\end{shell}

另一种选择是使用 operator bool(),而不是 operator void*() 来重载。毕竟,在布尔表达式中使用对象,为什么不像直接转换为 bool 一样转换它呢?

\begin{cpp}
operator bool() const { return m_ptr != nullptr; }
\end{cpp}

以下比较仍然可以工作:

\begin{cpp}
if (p != 0) { println("not 0"); }
if (p)      { println("not nullptr"); }
if (!p)     { println("nullptr"); }
\end{cpp}

然而,使用 operator bool(),以下与 nullptr 的比较会导致编译错误:

\begin{cpp}
if (p != nullptr) { println("not nullptr"); } // Error
\end{cpp}

这是因为 nullptr 有一个自己的类型 nullptr\_t,不能自动转换为整数 0(false)。编译器找不到一个接受 Pointer 对象和 nullptr\_t 对象的 operator !=。可以实现这样的 operator != ,作为 Pointer 类的友元:

\begin{cpp}
export template <typename T>
class Pointer
{
    public:
        // Omitted for brevity
        template <typename T>
        friend bool operator!=(const Pointer<T>& lhs, std::nullptr_t rhs);
        // Omitted for brevity
};

export template <typename T>
bool operator!=(const Pointer<T>& lhs, std::nullptr_t rhs)
{
    return lhs.m_ptr != rhs;
}
\end{cpp}

然而,实现了 operator != 之后,以下比较将不再工作,因为编译器不知道应该使用哪个 operator !=:

\begin{cpp}
if (p != 0) { println("not 0"); }
\end{cpp}

通过这个例子,可能会得出结论,operator bool() 技术只适用于那些不表示指针的对象,以及那些转换为指针类型没有意义的对象。不幸的是,即使在这种情况下,将 bool 添加为转换运算符也会带来一些未预料的后果。C++ 在有机会时会默默地使用提升规则将 bool 转换为 int,所以使用 operator bool(),以下代码可以编译并通过:

\begin{cpp}
Pointer<SpreadsheetCell> anotherSmartCell { new SpreadsheetCell { 5.0 } };
int i { anotherSmartCell }; // Converts Pointer to bool to int.
\end{cpp}

这通常不是期望或希望的行为。为了防止这样的赋值,可以显式删除转换运算符到 int、long、long long 等,但这样做会变得很混乱。因此,许多开发者更喜欢 operator void*(),而不是 operator bool()。

如上,重载运算符是需要设计的。关于要重载哪些运算符的决定,会直接影响客户使用类的方式。

