\noindent
\textbf{内筒概要}

\begin{itemize}
\item
什么是操作符重载

\item
重载操作符的基本原理

\item
操作符重载中的限制、注意事项和选择

\item
浅析可以、不能和不应该重载的操作符

\item
如何重载一元加号、一元减号、自增和自减

\item
如何重载I/O流操作符(operator<{}<和operator>{}>)

\item
如何重载下标(数组索引)操作符

\item
如何编写多维下标操作符

\item
如何重载函数调用操作符

\item
如何重载解引用操作符(*和->)

\item
如何编写转换操作符

\item
如何重载内存分配和释放操作符

\item
如何自定义字面量操作符

\item
可用的标准字面量操作符
\end{itemize}

本章的所有代码示例都可以在\url{https://github.com/Professional-CPP/edition-6/}获得。

C++允许重新定义+,-,和=等运算符,许多面向对象的语言不提供这种能力,所以可能会有冲动忽视其在C++中的作用。然而,这将使类表现得像内置类型(如int和double)一样。甚至可以编写看起来像数组、函数或指针的类。

第5章和第6章分别介绍了面向对象设计和运算符重载,第8章和第9章分别介绍了对象和基本运算符重载的语法细节。本章在第9章的基础上继续介绍运算符的重载。












