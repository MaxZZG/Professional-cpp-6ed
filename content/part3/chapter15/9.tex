
C++ 允许重新定义程序中内存分配和释放的方式,可以在全局级别和类级别提供这种定制。当担心内存碎片化时,这种能力最有用,这通常发生在大量分配和释放小对象时。例如,每次需要内存时,不必总是使用默认的 C++ 内存分配,可以编写一个内存池分配器,重用固定大小的内存块。这一节解释了内存分配和释放例程的微妙之处,并展示了如何进行定制。有了这些工具,应该能够编写自己的分配器。

\begin{myWarning}{WARNING}
除非对内存分配策略有深入的了解,否则尝试重载内存分配例程通常不值得。不要仅仅因为听起来不错,就去重载它们。只有在真正的需求和必要的知识时再这样做。
\end{myWarning}

\mySubsubsection{15.9.1.}{new 和 delete 的工作原理}

C++ 中最棘手的一个方面之一是 new 和 delete 的细节。看看以下代码行:

\begin{cpp}
SpreadsheetCell* cell { new SpreadsheetCell {} };
\end{cpp}

new 表达式的部分 new SpreadsheetCell\{\} 做了两件事。首先,通过调用 operator new 为 SpreadsheetCell 对象分配空间。其次,调用对象的构造函数。只有在构造函数完成后,才返回指针。

delete 的工作原理类似:

\begin{cpp}
delete cell;
\end{cpp}

这行代码称为 delete 表达式,首先调用 cell 的析构函数,然后调用 operator delete 来释放内存。

可以重载 operator new 和 operator delete 来控制内存分配和释放,但不能重载 new 表达式或 delete 表达式。因此,可以自定义实际的内存分配和释放,但不能自定义调用构造函数和析构函数。

\mySamllsection{new表达式和 operator new}

new 表达式有六种不同的形式,每种形式都有一个对应形式的 operator new。本书前面的章节已经展示了四种 new 表达式:new、new[]、new(nothrow) 和 new(nothrow)[]。以下列表显示了在 <new> 中定义的四个 operator new 重载:

\begin{cpp}
void* operator new(std::size_t size);
void* operator new[](std::size_t size);
void* operator new(std::size_t size, const std::nothrow_t&) noexcept;
void* operator new[](std::size_t size, const std::nothrow_t&) noexcept;
\end{cpp}

有两种特殊的 new 表达式,它们不会进行分配,但会在已经分配的内存上调用构造函数,这称为放置 new 运算符(包括单数和数组形式),允许在预先分配的内存中构造对象:

\begin{cpp}
void* ptr { allocateMemorySomehow() };
SpreadsheetCell* cell { new (ptr) SpreadsheetCell {} };
\end{cpp}

对应于这些的 operator new 重载如下所示,C++ 会禁止重载它们:

\begin{cpp}
void* operator new(std::size_t size, void* p) noexcept;
void* operator new[](std::size_t size, void* p) noexcept;
\end{cpp}

这个特性有点晦涩,但重要的是要知道它的存在。它可以在你想要实现内存池时派上用场,重用内存而非频繁分配和释放。这允许构造和析构对象的实例,而无需为每个新实例重新分配内存。第29章给出了一个内存池实现的示例。

\mySamllsection{delete 表达式和 operator delete}

可以调用的 delete 表达式只有两种形式:delete 和 delete[];没有 nothrow 或放置形式。然而,这里有六个 operator delete 重载。

为什么不对称呢?两个 nothrow 和两个放置重载只在构造函数抛出异常时使用。这种情况下,operator delete 调用与在构造函数调用之前用于分配内存的 operator new 匹配。然而,如果正常删除一个指针,delete 调用 operator delete,delete[] 调用 operator delete[](永远不会调用 nothrow 或放置形式)。实际上这并不重要,因为 C++ 标准指出,从 delete 抛出异常(例如,从由 delete 调用的析构函数)会导致未定义行为,这意味着 delete 无论如何都不应该抛出异常,所以 operator delete 的 nothrow 重载多余。此外,放置 delete 应该是一个空操作,因为放置 new 并没有分配内存,所以无需释放。

以下是六个 operator delete 重载的声明,对应于六个 operator new 重载:

\begin{cpp}
void operator delete(void* ptr) noexcept;
void operator delete[](void* ptr) noexcept;
void operator delete(void* ptr, const std::nothrow_t&) noexcept;
void operator delete[](void* ptr, const std::nothrow_t&) noexcept;
void operator delete(void* ptr, void*) noexcept;
void operator delete[](void* ptr, void*) noexcept;
\end{cpp}

\mySubsubsection{15.9.2.}{重载 operator new 和 operator delete}

可以替换全局的 operator new 和 operator delete ,这些函数在程序中的每个 new 表达式和 delete 表达式都会调用。然而,引用 Bjarne Stroustrup 的话:“... 替换全局的 operator new 和 operator delete 不要轻率为之”(《C++ 编程语言》第三版,Addison-Wesley,1997)。我也不推荐这么做!

\begin{myWarning}{WARNING}
如果不接受我的建议,决定替换全局的 operator new,请记住你无法在 operator 中放置任何调用 new 的代码,这会导致无限递归。例如,不能写一条消息到控制台。
\end{myWarning}

一个更有用的技术是为特定类重载 operator new 和 operator delete,这些重载的操作符只有在分配和释放该特定类的对象时才会调用。

以下是重载了四种非放置形式的 operator new 和 operator delete 的类的例子:

\begin{cpp}
export class MemoryDemo
{
    public:
        virtual ~MemoryDemo() = default;
        void* operator new(std::size_t size);
        void operator delete(void* ptr) noexcept;
        void* operator new[](std::size_t size);
        void operator delete[](void* ptr) noexcept;
        void* operator new(std::size_t size, const std::nothrow_t&) noexcept;
        void operator delete(void* ptr, const std::nothrow_t&) noexcept;
        void* operator new[](std::size_t size, const std::nothrow_t&) noexcept;
        void operator delete[](void* ptr, const std::nothrow_t&) noexcept;
};
\end{cpp}

以下是这些操作符的实现,只是向标准输出写入一条消息,并通过调用全局版本的操作符来传递参数。请注意,nothrow 实际上是一个 nothrow\_t 类型的变量。

\begin{cpp}
void* MemoryDemo::operator new(size_t size)
{
    println("operator new");
    return ::operator new(size);
}
void MemoryDemo::operator delete(void* ptr) noexcept
{
    println("operator delete");
    ::operator delete(ptr);
}
void* MemoryDemo::operator new[](size_t size)
{
    println("operator new[]");
    return ::operator new[](size);
}
void MemoryDemo::operator delete[](void* ptr) noexcept
{
    println("operator delete[]");
    ::operator delete[](ptr);
}
void* MemoryDemo::operator new(size_t size, const nothrow_t&) noexcept
{
    println("operator new nothrow");
    return ::operator new(size, nothrow);
}
void MemoryDemo::operator delete(void* ptr, const nothrow_t&) noexcept
{
    println("operator delete nothrow");
    ::operator delete(ptr, nothrow);
}
void* MemoryDemo::operator new[](size_t size, const nothrow_t&) noexcept
{
    println("operator new[] nothrow");
    return ::operator new[](size, nothrow);
}
void MemoryDemo::operator delete[](void* ptr, const nothrow_t&) noexcept
{
    println("operator delete[] nothrow");
    ::operator delete[](ptr, nothrow);
}
\end{cpp}

以下是分配和释放该类对象的代码示例,以几种不同的方式进行:

\begin{cpp}
MemoryDemo* mem { new MemoryDemo{} };
delete mem;
mem = new MemoryDemo[10];
delete [] mem;
mem = new (nothrow) MemoryDemo{};
delete mem;
mem = new (nothrow) MemoryDemo[10];
delete [] mem;
\end{cpp}

以下是运行该程序的输出:

\begin{shell}
operator new
operator delete
operator new[]
operator delete[]
operator new nothrow
operator delete
operator new[] nothrow
operator delete[]
\end{shell}

这些 operator new 和 operator delete 的实现很简单,并没有太大用处,只是为了展示语法示例。

\begin{myWarning}{WARNING}
无论何时重载 operator new,都应重载对应的 operator delete 形式。否则,内存将按照指定的方式分配,但释放时将遵循内置语义,这可能不兼容。
\end{myWarning}

为所有各种形式的 operator new 和 operator delete 重载可能看起来过于复杂,但这样做是为了避免内存分配中的不一致性。如果不想为某些重载提供实现,可以使用 =delete 显式删除这些,以避免任何人使用。

\begin{myWarning}{WARNING}
重载所有形式的 operator new 和 operator delete,或者显式删除你不希望使用的重载,避免内存分配中的不一致。
\end{myWarning}

\mySubsubsection{15.9.3.}{显式删除或默认 operator new 和 operator delete}

第8章展示了如何显式删除或默认一个构造函数或赋值运算符,显式删除或默认不仅仅限于构造函数和赋值运算符。例如,以下类删除了 operator new 和 new[],所以这个类的对象不能使用 new 或 new[] 动态分配:

\begin{cpp}
class MyClass
{
    public:
        void* operator new(std::size_t) = delete;
        void* operator new[](std::size_t) = delete;
        void* operator new(std::size_t, const std::nothrow_t&) noexcept = delete;
        void* operator new[](std::size_t, const std::nothrow_t&) noexcept = delete;
};
\end{cpp}

使用这个类会导致编译错误:

\begin{cpp}
MyClass* p1 { new MyClass };
MyClass* p2 { new MyClass[2] };
MyClass* p3 { new (std::nothrow) MyClass };
\end{cpp}

\mySubsubsection{15.9.4.}{带有额外参数的重载 operator new 和 operator delete}

除了重载 operator new 的标准形式,还可以编写带有参数的版本。这些额外参数可以用于向内存分配例程传递各种标志或计数器。例如,一些运行时库在调试模式下使用此方法,在对象分配时提供文件名和行号,以便在发生内存泄漏时可以确定进行分配的行。

作为示例,以下是对 MemoryDemo 类的额外 operator new 和 operator delete 的原型,带有一个整数参数:

\begin{cpp}
void* operator new(std::size_t size, int extra);
void operator delete(void* ptr, int extra) noexcept;
\end{cpp}

实现如下:

\begin{cpp}
void* MemoryDemo::operator new(std::size_t size, int extra)
{
    println("operator new with extra int: {}", extra);
    return ::operator new(size);
}
void MemoryDemo::operator delete(void* ptr, int extra) noexcept
{
    println("operator delete with extra int: {}", extra);
    return ::operator delete(ptr);
}
\end{cpp}

当实现带有参数的 overloaded operator new 时,编译器会自动允许相应的 new 表达式。new 表达式中外参数,通过函数调用语法传递(与 nothrow 重载一样)。因此,可以这样实现:

\begin{cpp}
MemoryDemo* memp { new(5) MemoryDemo{} };
delete memp;
\end{cpp}

输出如下所示:

\begin{shell}
operator new with extra int: 5
operator delete
\end{shell}

当定义带有参数的 operator new 时,也应该定义相应的带有相同参数的 operator delete。然而,不能自己调用这个带有参数的 operator delete;它只能在使用带有参数的 operator new,并且对象的构造函数抛出异常时才会调用。

\mySubsubsection{15.9.5.}{带有内存大小参数的 operator delete重载}

operator delete 的一个替代形式允许你获得应该释放的内存的大小以及指针,只需声明 operator delete 的原型,带有一个 size 参数。

\begin{myWarning}{WARNING}
如果一个类声明了两个 operator delete 重载,其中一个重载带有大小参数,而另一个不带,那么不带大小参数的重载将始终调用。如果想使用带有大小参数的重载,只需实现那个重载。
\end{myWarning}

可以用一个带有 size 参数的重载替换 operator delete 的任何版本。这里是对 MemoryDemo 类定义的修改,其中第一个 operator delete 修改为接受要删除的内存大小:

\begin{cpp}
export class MemoryDemo
{
    public:
        // Omitted for brevity
        void* operator new(std::size_t size);
        void operator delete(void* ptr, std::size_t size) noexcept;
        // Omitted for brevity
};
\end{cpp}

这个 operator delete 的实现,简单地调用全局的 operator delete:

\begin{cpp}
void MemoryDemo::operator delete(void* ptr, size_t size) noexcept
{
    println("operator delete with size {}", size);
    ::operator delete(ptr, size);
}
\end{cpp}

这种能力只有在编写复杂内存分配和释放方案时才有用。




