
假设你从未听说过标准库中的 vector 或数组类模板,因此决定自己编写一个动态分配数组的类。这个类将允许你在指定的索引处设置和检索元素,并会在后台处理所有的内存分配。动态分配数组的类定义看起来可能像这样:

\begin{cpp}
export template <typename T>
class Array
{
    public:
        // Creates an array with a default size that will grow as needed.
        Array();
        virtual ~Array();

        // Disallow copy constructor and copy assignment.
        Array& operator=(const Array& rhs) = delete;
        Array(const Array& src) = delete;

        // Move constructor and move assignment operator.
        Array(Array&& src) noexcept;
        Array& operator=(Array&& rhs) noexcept;

        // Returns the value at index x. Throws an exception of type
        // out_of_range if index x does not exist in the array.
        const T& getElementAt(std::size_t x) const;

        // Sets the value at index x. If index x is out of range,
        // allocates more space to make it in range.
        void setElementAt(std::size_t x, const T& value);

        // Returns the number of elements in the array.
        std::size_t getSize() const noexcept;
    private:
        static constexpr std::size_t AllocSize { 4 };
        void resize(std::size_t newSize);
        T* m_elements { nullptr };
        std::size_t m_size { 0 };
};
\end{cpp}

接口支持设置和访问元素,提供了随机访问保证:客户端可以创建一个默认的数组并设置元素 1, 100 和 1000,无需担心内存管理。

以下是成员函数的实现:

\begin{cpp}
template <typename T> Array<T>::Array()
{
    m_elements = new T[AllocSize] {}; // Elements are zero-initialized!
    m_size = AllocSize;
}

template <typename T> Array<T>::˜Array()
{
    delete[] m_elements;
    m_elements = nullptr;
    m_size = 0;
}

template <typename T> Array<T>::Array(Array&& src) noexcept
: m_elements { std::exchange(src.m_elements, nullptr) }
, m_size { std::exchange(src.m_size, 0) }
{}

template <typename T> Array<T>& Array<T>::operator=(Array<T>&& rhs) noexcept
{
    if (this == &rhs) { return *this; }
    delete[] m_elements;
    m_elements = std::exchange(rhs.m_elements, nullptr);
    m_size = std::exchange(rhs.m_size, 0);
    return *this;
}

template <typename T> void Array<T>::resize(std::size_t newSize)
{
    // Create new bigger array with zero-initialized elements.
    auto newArray { std::make_unique<T[]>(newSize) };

    // The new size is always bigger than the old size (m_size).
    for (std::size_t i { 0 }; i < m_size; ++i) {
        // Copy the elements from the old array to the new one.
        newArray[i] = m_elements[i];
    }
    // Delete the old array, and set the new array.
    delete[] m_elements;
    m_size = newSize;
    m_elements = newArray.release();
}

template <typename T> const T& Array<T>::getElementAt(std::size_t x) const
{
    if (x >= m_size) { throw std::out_of_range { "" }; }
    return m_elements[x];
}

template <typename T> void Array<T>::setElementAt(std::size_t x, const T& val)
{
    if (x >= m_size) {
        // Allocate AllocSize past the element the client wants.
        resize(x + AllocSize);
    }
    m_elements[x] = val;
}

template <typename T> std::size_t Array<T>::getSize() const noexcept
{
    return m_size;
}
\end{cpp}

注意 resize() 成员函数的异常安全实现。首先,使用 make\_unique() 创建一个适当大小的数组,并将其存储在一个 unique\_ptr 中,再将所有元素从旧数组复制到新数组。如果在复制值时发生问题,unique\_ptr 会自动清理新分配的内存。最后,当新数组的分配和所有元素的复制成功完成,即没有抛出异常时,我们才删除旧的 m\_elements 数组,并将新数组赋给它。最后一行必须使用 release() 来释放 unique\_ptr 对新数组的所有权;否则,当 unique\_ptr 的析构函数调用时,数组将销毁。

为了保证强异常安全性(参见第14章),resize() 复制了旧数组中的元素到新分配的数组中。第26章中实现了一个 move\_assign\_if\_noexcept() 辅助函数,这个辅助函数可以在 resize() 的实现中使用,以便在元素类型标记为 noexcept 的移动赋值运算符时,将元素从旧数组移动到新数组,如果不是这种情况,将复制元素。有了这个变化,无论元素是移动还是复制,强异常安全性都能得到保证。

这是一个如何使用这个类的示例:

\begin{cpp}
Array<int> myArray;
for (size_t i { 0 }; i < 20; i += 2) {
    myArray.setElementAt(i, 100);
}
for (size_t i { 0 }; i < 20; ++i) {
    print("{} ", myArray.getElementAt(i));
}
\end{cpp}

输出如下:

\begin{shell}
100 0 100 0 100 0 100 0 100 0 100 0 100 0 100 0 100 0 100 0
\end{shell}

不必告诉数组需要多少空间,根据给出的元素分配所需的空间。

\begin{myNotic}{NOTE}
这个实现不是最节省内存的。如果创建一个数组,只给索引为 4000 的元素赋值,那么它会分配内存用于 4004 个元素,所有元素都零初始化,除了索引为 4000 的元素。
\end{myNotic}

然而,需要使用 setElementAt() 和 getElementAt() 成员函数不是很方便。

这就是重载下标运算符的好时机。可以像这样向类中添加一个 operator[]:

\begin{cpp}
export template <typename T>
class Array
{
    public:
        T& operator[](std::size_t x);
        // Remainder omitted for brevity.
};
\end{cpp}

以下是实现:

\begin{cpp}
template <typename T> T& Array<T>::operator[](std::size_t x)
{
    if (x >= m_size) {
        // Allocate AllocSize past the element the client wants.
        resize(x + AllocSize);
    }
    return m_elements[x];
}
\end{cpp}

有了这个,就可以像这样使用传统的数组索引语法:

\begin{cpp}
Array<int> myArray;
for (size_t i { 0 }; i < 20; i += 2) {
    myArray[i] = 100;
}
for (size_t i { 0 }; i < 20; ++i) {
    print("{} ", myArray[i]);
}
\end{cpp}

operator[] 既可以用于设置,也可以用于获取元素,因为它返回了位置 x 处的元素的引用。这个引用可以用来赋值给那个元素。当 operator[] 用于赋值语句的左侧时,赋值实际上改变了 m\_elements 数组中位置 x 的值。

\mySubsubsection{15.5.1.}{ operator[] 提供只读访问}

有时 operator[] 返回一个可以作为左值使用的元素是很方便,但并不总是需要这样。能够提供数组元素的只读访问,通过返回一个引用为常量也很有用。为了实现这一点,需要两个 operator[] 重载:一个返回引用为非常量,另一个返回引用为常量:

\begin{cpp}
T& operator[](std::size_t x);
const T& operator[](std::size_t x) const;
\end{cpp}

不能仅基于返回类型来重载成员函数或运算符,所以第二个重载返回引用为常量并标记为 const。

这里是 const operator[] 的实现。如果索引超出范围,它将抛出异常而不是尝试分配新空间。仅尝试读取元素值时,分配新空间没有意义。

\begin{cpp}
template <typename T> const T& Array<T>::operator[](std::size_t x) const
{
    if (x >= m_size) { throw std::out_of_range { "" }; }
    return m_elements[x];
}
\end{cpp}

以下代码展示了这两种 operator[]:

\begin{cpp}
void printArray(const Array<int>& arr)
{
    for (size_t i { 0 }; i < arr.getSize(); ++i) {
        print("{} ", arr[i]); // Calls the const operator[] because arr is
                              // a const object.
    }
    println("");
}

int main()
{
    Array<int> myArray;
    for (size_t i { 0 }; i < 20; i += 2) {
        myArray[i] = 100; // Calls the non-const operator[] because
                          // myArray is a non-const object.
    }
    printArray(myArray);
}
\end{cpp}

注意,在 printArray() 中调用 const operator[] 仅仅是因为参数 arr 是 const 的。如果 arr 不是 const 的,将调用非 const operator[],尽管结果并未修改。

const operator[] 用于 const 对象,所以它不能增加数组的大小。当前实现当给定索引超出范围时会抛出异常。一个替代方案是在不抛出异常的情况下返回一个零初始化的元素,这可以这样做:

\begin{cpp}
template <typename T> const T& Array<T>::operator[](std::size_t x) const
{
    if (x >= m_size) {
        static T nullValue { T{} };
        return nullValue;
    }
    return m_elements[x];
}
\end{cpp}

nullValue 静态变量使用零初始化语法 T\{\} 初始化。这取决于用例,可以选择抛出异常或返回空值。

\begin{myNotic}{NOTE}
零初始化使用默认构造函数构造对象,并初始化原始整数类型(如 char、int 等)为零,原始浮点类型为 0.0,指针类型为 nullptr。
\end{myNotic}

\CXXTwentythreeLogo{-40}{-60}

\mySubsubsection{15.5.2.}{多维下标运算符}

从 C++23 开始,下标运算符可以支持多维索引。

为了演示,让我们回顾一下第12章中的 Grid 类模板。其接口包含一个 const 和非 const 的 at(x,y) 成员函数重载。这些 at() 成员函数可以用以下二维 const 和非 const 下标运算符替换:

\begin{cpp}
template <typename T>
class Grid
{
    public:
        std::optional<T>& operator[](std::size_t x, std::size_t y);
        const std::optional<T>& operator[](std::size_t x, std::size_t y) const;
        // Remainder omitted for brevity.
};
\end{cpp}

这个语法只是为这些二维下标运算符简单地指定了两个参数,x 和 y。实现与原始 at() 成员函数的实现相同:

\begin{cpp}
template <typename T>
const std::optional<T>& Grid<T>::operator[](std::size_t x, std::size_t y) const
{
    verifyCoordinate(x, y);
    return m_cells[x + y * m_width];
}
template <typename T>
std::optional<T>& Grid<T>::operator[](std::size_t x, std::size_t y)
{
    return const_cast<std::optional<T>&>(std::as_const(*this)[x, y]);
}
\end{cpp}

以下是在这些新运算符中使用的例子:

\begin{cpp}
Grid<int> myIntGrid { 4, 4 };
int counter { 0 };
for (size_t y { 0 }; y < myIntGrid.getHeight(); ++y) {
    for (size_t x { 0 }; x < myIntGrid.getWidth(); ++x) {
        myIntGrid[x, y] = ++counter;
    }
}
for (size_t y { 0 }; y < myIntGrid.getHeight(); ++y) {
    for (size_t x { 0 }; x < myIntGrid.getWidth(); ++x) {
        print("{:3} ", myIntGrid[x, y].value_or(0));
    }
    println("");
}
\end{cpp}

输出为:

\begin{shell}
 1  2  3  4
 5  6  7  8
 9 10 11 12
13 14 15 16
\end{shell}


\mySubsubsection{15.5.3.}{非整数数组索引}

将“索引”模式扩展到集合中提供某种键的自然延伸;向量(或一般情况下的线性数组)是特殊情况,其中“键”就是数组中的位置。operator[] 的参数作为两个域之间的映射:键的域和值的域。因此,可以写一个接受任何类型作为其索引的operator[],这个类型不需要是整数类型。这在标准库的关联容器中,如 std::map 中得到实现,这些将在第18章中介绍。

例如,可以创建一个使用字符串键,而不是整数索引的关联数组。这样的类的 operator[] 将会接受一个字符串,或者是字符串视图作为参数。实现这样的类,会作为是本章末尾的习题供读者练习。

\hspace*{\fill}

\mySubsubsection{15.5.4.}{静态下标计算操作符}

\CXXTwentythreeLogo{-40}{15}

从 C++23 开始,当操作符的实现不需要访问this时(不需要访问非静态数据成员和非静态成员函数),可将下标计算操作符可以标记为static。这允许编译器更好地优化代码,无需操心 this 指针。以下是将 operator[] 标记为静态、常量表达式(参见第9章)和无异常(第14章)的示例:

\begin{cpp}
enum class Figure { Diamond, Heart, Spade, Club };

class FigureEnumToString
{
    public:
        static constexpr string_view operator[](Figure figure) noexcept
        {
            switch (figure) {
                case Figure::Diamond: return "Diamond";
                case Figure::Heart: return "Heart";
                case Figure::Spade: return "Spade";
                case Figure::Club: return "Club";
            }
        }
};

int main()
{
    Figure f { Figure::Spade };
    FigureEnumToString converter;
    println("{}", converter[f]);
    println("{}", FigureEnumToString{}[f]);
}
\end{cpp}












