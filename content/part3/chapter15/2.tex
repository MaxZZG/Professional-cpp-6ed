
第9章展示了如何编写二元算术操作符和简写算术赋值操作符,但没有介绍如何重载其他算术操作符。

\mySubsubsection{15.2.1.}{重载一元加减}

C++ 有几个一元算术操作符。其中两个是一元减(-)和一元加(+)。以下是使用整数的一元操作符的示例:

\begin{cpp}
int i, j { 4 };
i = -j; // Unary minus
i = +i; // Unary plus
j = +(-i); // Apply unary plus to the result of applying unary minus to i.
j = -(-i); // Apply unary minus to the result of applying unary minus to i.
\end{cpp}

一元减取操作数的相反数,而一元加直接返回操作数,可以对一元加或一元减的结果应用一元加或一元减。这些操作符不改变调用的对象,因此应该使它们成为const成员函数。

以下是一元减操作符作为 SpreadsheetCell 类的成员函数的示例。一元加通常是一个恒等操作,所以这个类没有重载。

\begin{cpp}
SpreadsheetCell SpreadsheetCell::operator-() const
{
    return SpreadsheetCell { -getValue() };
}
\end{cpp}

operator- 不改变操作数,这个成员函数必须构造一个新的 SpreadsheetCell,并返回它的相反值,所以不能返回引用。可以像这样使用这个操作符:

\begin{cpp}
SpreadsheetCell c1 { 4 };
SpreadsheetCell c3 { -c1 };
\end{cpp}

\mySubsubsection{15.2.2.}{重载增减量操作符}

有多种方法可以将1加到变量中:

\begin{cpp}
i = i + 1;
i = 1 + i;
i += 1;
++i;
i++;
\end{cpp}

最后两种形式称为增量操作符。第一个是前置增量,将1加到变量,返回新增加的值,以供表达式的其余部分使用。第二个是后置增量,也将1加到变量,但返回旧的(未增加的)值,以供表达式的其余部分使用。减量操作符的工作方式类似。

想要重载 operator++ 和 operator-{}- 时,可能具有的前置和后置含义提出了一个问题。当编写一个重载的 operator++ 时,如何指定是在重载前置版本,还是后置版本?C++ 引入了一个技巧来做出这种区分:前置版本的 operator++ 和 operator-{}- 不接受参数,而后置版本接受一个未使用的类型 int 为参数。

SpreadsheetCell 类的重载操作符:

\begin{cpp}
SpreadsheetCell& operator++(); // Prefix
SpreadsheetCell operator++(int); // Postfix
SpreadsheetCell& operator--(); // Prefix
SpreadsheetCell operator--(int); // Postfix
\end{cpp}

前置形式的返回值与操作数的最终值相同,所以前置增量和减量可以返回对调用对象的引用。后置增量和减量的返回值与操作数的最终值不同,因此不能返回引用。

以下是 operator++ 的实现:

\begin{cpp}
SpreadsheetCell& SpreadsheetCell::operator++()
{
    set(getValue() + 1);
    return *this;
}

SpreadsheetCell SpreadsheetCell::operator++(int)
{
    auto oldCell { *this }; // Save current value
    ++(*this); // Increment using prefix ++
    return oldCell; // Return the old value
}
\end{cpp}

\begin{myNotic}{NOTE}
建议以后置操作符的形式实现前置操作符。
\end{myNotic}

operator-{}- 的实现几乎相同。现在可以随心所欲地增量和减量 SpreadsheetCell 对象了:

\begin{cpp}
SpreadsheetCell c1 { 4 };
SpreadsheetCell c2 { 4 };
c1++;
++c2;
\end{cpp}

增量和减量操作符也适用于指针。例如,使用智能指针类时,可以重载 operator++ 和 operator-{}- ,用以提供指针增减量功能。



