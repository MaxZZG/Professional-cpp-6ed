通过解决下面的练习,可以练习本章讨论的内容。所有练习的解决方案都可以在本书的网站\url{www.wiley.com/go/proc++6e}下载到源码。若在练习中卡住了,可以考虑先重读本章的部分内容,试着自己找到答案,再在从网站上寻找解决方案。

\begin{itemize}
\item
\textbf{练习 21-1}: 使用一个合适的特性来确定,用于格式化数字的十进制分隔符,以符合用户环境。查阅标准库手册,以了解所选特性的确切成员函数。

\item
\textbf{练习 21-2}: 编写一个应用程序,要求用户输入格式化的美国电话号码。例如:202-555-0108。使用正则表达式来验证电话号码的格式,即三个数字,后跟一个减号,再跟三个数字,再后跟一个减号,最后是四个数字。如果是有效的电话号码,将三个部分分别打印在单独的行上。例如,对于上述电话号码,结果必须如下所示:

\begin{shell}
202
555
0108
\end{shell}

\item
\textbf{练习 21-3}: 编写一个应用程序,要求用户输入一段可能跨越多行的源代码,并可能包含 // 风格的注释。为了指示输入的结束,使用一个哨兵字符,例如 @。可以使用 std::getline() 并以 '@' 作为分隔符,从标准输入控制台读取多行文本。最后,使用正则表达式将代码段的所有注释移除,确保代码正确地工作:

\begin{cpp}
string str; // A comment // Some more comments.
str = "Hello"; // Hello.
\end{cpp}

对于这个输入,结果必须如下所示:

\begin{shell}
string str;
str = "Hello";
\end{shell}

\item
\textbf{练习 21-4}: 本章前面“前瞻”部分提到了一个密码验证的正则表达式。编写一个程序来测试这个正则表达式,要求用户输入一个密码并验证它。当完成前面的验证后,再添加一个验证规则:密码必须至少包含两个数字。
\end{itemize}