过程式编程范式中,程序的基本单位是函数。函数之所以有用,因为可以编写与特定值无关的算法,因此可以用于不同的值。C++ 中的 sqrt() 函数计算调用者提供的值的平方根,一个只计算一个数字(如 4)平方根的函数并不特别有用!sqrt() 函数以参数的形式编写,参数代表调用者传递的任何值。用计算机科学家的话说,就是函数参数化。

面向对象编程范式增加了对象的概念,将相关数据和行为组合在一起,但并没有改变函数和成员函数参数化的方式。

模板将参数化的概念进一步推进,允许对值进行参数化,还可以对类型进行参数化。C++ 中的类型包括如 int 和 double 这样的原始类型,以及 SpreadsheetCell 和 CherryTree 这样的自定义类型。使用模板,可以编写与值和值类型无关的代码。为了避免给 int、Car 和 SpreadsheetCell 分别编写堆栈类,可以编写一个堆栈类模板定义,可以用于这些类型中的任何一个。

虽然模板是 C++ 中一个令人惊叹的语言特性,但 C++ 中的模板在语法上可能令人困惑,因此许多开发者会避免编写模板。每个专业的 C++ 开发者都需要知道如何编写模板,每个开发者至少需要知道如何使用模板,因为模板已广泛用于 C++ 标准库了。

本章将介绍 C++ 中模板的知识,重点是标准库中的那些。这个过程中,将了解一些可以应用于自己程序的巧妙技巧。
