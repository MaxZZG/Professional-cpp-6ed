This chapter started a discussion on using templates for generic programming. You saw the syntax on how to write templates and examples where templates are really useful. It explained how to write class templates, class member function templates, and how to use template parameters. It further discussed how to use class template specialization to write special implementations of a template where the template parameters are replaced with specific arguments.

You also learned about variable templates, function templates, and the elegant abbreviated function template syntax. The chapter finished with a discussion of concepts, allowing you to put constraints on template parameters.

Chapter 26 continues the discussion on templates with some more advanced features such as class template partial specializations, variadic templates, and metaprogramming.