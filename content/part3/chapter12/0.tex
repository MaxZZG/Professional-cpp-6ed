\noindent
\textbf{内筒概要}

\begin{itemize}
\item
定义类模板

\item
编译器处理模板

\item
组织模板源码

\item
使用非类型模板参数

\item
定义单个类成员函数的模板

\item
为特定类型定制类模板

\item
组合模板和继承

\item
定义别名模板

\item
定义函数模板

\item
让函数模板成为类模板的朋友(友元)

\item
使用缩写函数模板语法

\item
使用变量模板

\item
介绍概念

\item
使用概念类型来约束auto

\item
使用概念来指定模板参数的类型需求
\end{itemize}

本章的所有代码示例都可以在\url{https://github.com/Professional-CPP/edition-6}获得。

C++不仅支持面向对象编程,还支持泛型编程。如第6章所介绍,泛型编程的目标是编写可重用的代码,C++中用于泛型编程的基本工具是模板。虽然模板不是严格意义上的面向对象特性,但可以与面向对象编程结合使用,产生强大的效果。使用现有的模板,例如标准库提供的模板,如std::vector、unique\_ptr等。许多开发者认为编写自己的模板很困难,因此倾向于避免编写模板。作为一名专业的C++开发者,需要知道如何编写类模板和函数模板。

本章提供了实现第6章讨论的设计原则(泛型)的编码细节,而第26章则深入的介绍了一些更高级的模板特性。



















