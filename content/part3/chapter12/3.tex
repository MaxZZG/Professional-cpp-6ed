
也可以为独立的函数编写模板,模板的语法与类模板的语法相似。例如,可以编写以下泛型函数,在数组中查找值并返回其索引:

\begin{cpp}
template <typename T>
optional<size_t> Find(const T& value, const T* arr, size_t size)
{
    for (size_t i { 0 }; i < size; ++i) {
        if (arr[i] == value) {
            return i; // Found it; return the index.
        }
    }
    return {}; // Failed to find it; return empty optional.
}
\end{cpp}

Find() 函数模板可以用于任何类型的数组。例如,可以查找 int 在数组中的索引,或者在 SpreadsheetCell 数组中的索引。

可以以两种方式调用该函数:显式指定模板类型参数并使用尖括号,或者省略类型参数,让编译器从参数中推导出类型参数。

以下是一些示例:

\begin{cpp}
int myInt { 3 }, intArray[] {1, 2, 3, 4};
const size_t sizeIntArray { size(intArray) };

optional<size_t> res;
res = Find(myInt, intArray, sizeIntArray); // calls Find<int> by deduction.
res = Find<int>(myInt, intArray, sizeIntArray); // calls Find<int> explicitly.
if (res) { println("{}", *res); }
else { println("Not found"); }

double myDouble { 5.6 }, doubleArray[] {1.2, 3.4, 5.7, 7.5};
const size_t sizeDoubleArray { size(doubleArray) };

// calls Find<double> by deduction.
res = Find(myDouble, doubleArray, sizeDoubleArray);
// calls Find<double> explicitly.
res = Find<double>(myDouble, doubleArray, sizeDoubleArray);
if (res) { println("{}", *res); }
else { println("Not found"); }

//res = Find(myInt, doubleArray, sizeDoubleArray); // DOES NOT COMPILE!
                                                // Arguments are different types.

// calls Find<double> explicitly, even with myInt.
res = Find<double>(myInt, doubleArray, sizeDoubleArray);

SpreadsheetCell cell1 { 10 }
SpreadsheetCell cellArray[] { SpreadsheetCell { 4 }, SpreadsheetCell { 10 } };
const size_t sizeCellArray { size(cellArray) };

res = Find(cell1, cellArray, sizeCellArray);
res = Find<SpreadsheetCell>(cell1, cellArray, sizeCellArray);
\end{cpp}

Find()函数模板先前的实现需要将数组的大小作为参数之一。有时编译器知道数组的准确大小,例如对于基于栈的数组。如果能够在不传递数组大小的条件下调用 Find() 函数,那将非常方便。可以通过添加以下函数模板来实现这一点,该实现的目的是将调用转发到之前的 Find() 函数模板。函数模板还可以像类模板一样接受非类型参数。

\begin{cpp}
template <typename T, size_t N>
optional<size_t> Find(const T& value, const T(&arr)[N])
{
    return Find(value, arr, N);
}
\end{cpp}

这个 Find() 函数模板的重载看起来有点奇怪,但可以直接使用:

\begin{cpp}
int myInt { 3 }, intArray[] {1, 2, 3, 4};
optional<size_t> res { Find(myInt, intArray) };
\end{cpp}

与类模板成员函数定义一样,函数模板定义(不仅仅是原型)必须对所有使用它们的源文件可用。因此,应该将定义放在模块接口文件中,并在多个源文件使用时进行导出。

最后,函数模板的模板参数可以有默认值,就像类模板一样。

\begin{myNotic}{NOTE}
C++ 标准库提供了一个更强大的 std::find() 函数模板。有关详细信息,请参阅第 20 章。
\end{myNotic}

\mySubsubsection{12.3.1.}{函数重载与函数模板}

想要提供一个可以与不同数据类型一起工作的函数时,有两种选择:提供函数重载或提供函数模板。应该如何进行选择呢?

编写一个应该与不同数据类型一起工作的函数,且该函数体的所有数据类型都相同,那就提供一个函数模板。如果函数体对于每种数据类型都不同,就提供函数重载。

\mySubsubsection{12.3.2.}{函数模板的重载}

理论上,C++ 语言允许像编写类模板特化一样,编写函数模板的特化。然而,不太需要这样做,这样的函数模板的特化不参与重载解析,可能会有意外的行为。

相反,可以通过非模板函数或其他函数模板来重载函数模板。例如,想要为 const char* C 风格字符串编写一个 Find() 重载,使用 strcmp() ,而非 operator== 来比较(见第2章),因为 == 仅比较指针,而不是实际字符串。这是一个重载的例子:

\begin{cpp}
optional<size_t> Find(const char* value, const char** arr, size_t size)
{
    for (size_t i { 0 }; i < size; ++i) {
        if (strcmp(arr[i], value) == 0) {
            return i; // Found it; return the index.
        }
    }
    return {}; // Failed to find it; return empty optional.
}
\end{cpp}

这个函数重载可以这样使用:

\begin{cpp}
// Using an array for word to make sure no literal pooling happens, see Chapter 2.
const char word[] { "two" };
const char* words[] { "one", "two", "three", "four" };
const size_t sizeWords { size(words) };
optional<size_t> res { Find(word, words, sizeWords) }; // Calls non-template Find.
if (res) { println("{}", *res); }
else { println("Not found"); }
\end{cpp}

对 Find() 的调用正确地在索引 1 处找到了字符串“two”。

如果明确指定模板类型参数如下,那么函数模板将以 T=const char* 调用,而不是 const char* 的重载:

\begin{cpp}
res = Find<const char*>(word, words, sizeWords);
\end{cpp}

因为没有比较实际的字符串,而是指针,所以这个对 Find() 的调用没有找到任何匹配项。

当编译器的重载解析过程产生两个可能的候选项,一个是函数模板,另一个是非模板函数时,编译器总是倾向于使用非模板函数。

\mySubsubsection{12.3.3.}{函数模板作为类模板的友元}

想要在类模板中重载操作符时,函数模板非常有用。例如,想要为 Grid 类模板重载加法操作符(operator+),以便能够将两个网格相加。结果将是一个与两个操作数中最小的 Grid 大小相同的 Grid。只有当两个单元格都包含实际值时,相应的单元格才会相加。假设希望将 operator+ 作为一个独立的函数模板,定义应该放在 Grid.cppm 模块接口文件中,如下所示。实现使用 <algorithm> 中定义的 std::min(),返回两个参数的最小值:

\begin{cpp}
export template <typename T>
Grid<T> operator+(const Grid<T>& lhs, const Grid<T>& rhs)
{
    std::size_t minWidth { std::min(lhs.m_width, rhs.m_width) };
    std::size_t minHeight { std::min(lhs.m_height, rhs.m_height) };

    Grid<T> result { minWidth, minHeight };
    for (std::size_t y { 0 }; y < minHeight; ++y) {
        for (std::size_t x { 0 }; x < minWidth; ++x) {
            const auto& leftElement { lhs.at(x, y) };
            const auto& rightElement { rhs.at(x, y) };
            if (leftElement.has_value() && rightElement.has_value()) {
                result.at(x, y) = leftElement.value() + rightElement.value();
            }
        }
    }
    return result;
}
\end{cpp}

要查询一个可选值是否包含实际值,可以使用 has\_value() 成员函数,并用 value() 获取这个值。

这个函数模板适用于任何 Grid,只要Grid中存储的元素类型有一个加法操作符。这个实现唯一的问题在于,访问了 Grid 类的私有成员 m\_width 和 m\_height。显然的解决方案是使用public的 getWidth() 和 getHeight() 成员函数,来看看如何使一个函数模板成为类模板的朋友。这个例子中,可以使 operator+ 成为 Grid 类模板的友元。然而,Grid 和 operator+ 都是模板。真正想要的是,对于特定类型 T 的每个 operator+ 实例化,都成为同一类型的 Grid 模板实例化的友元。语法如下所示:

\begin{cpp}
export template <typename T>
class Grid
{
    public:
        friend Grid operator+<T>(const Grid& lhs, const Grid& rhs);
        // Omitted for brevity
};
\end{cpp}

这个友元声明很巧妙:对于类型 T 的类模板实例,T 实例化的 operator+ 是一个友元。换句话说,在类实例化和函数实例化之间有一对一的友元映射。特别注意 operator+ 上的显式模板指定 <T>,这个语法告诉编译器 operator+ 本身就是一个模板。

这个友元 operator+ 可以通过以下方式测试。以下代码首先定义了两个辅助函数模板:fillGrid(),用递增的数字填充 Grid,以及 printGrid(),其会将 Grid 中的值输出到控制台。


\begin{cpp}
template <typename T> void fillGrid(Grid<T>& grid)
{
    T index { 0 };
    for (size_t y { 0 }; y < grid.getHeight(); ++y) {
        for (size_t x { 0 }; x < grid.getWidth(); ++x) {
            grid.at(x, y) = ++index;
        }
    }
}

template <typename T> void printGrid(const Grid<T>& grid)
{
    for (size_t y { 0 }; y < grid.getHeight(); ++y) {
        for (size_t x { 0 }; x < grid.getWidth(); ++x) {
            const auto& element { grid.at(x, y) };
            if (element.has_value()) { print("{}\t", element.value()); }
            else { print("n/a\t"); }
        }
        println("");
    }
}

int main()
{
    Grid<int> grid1 { 2, 2 };
    Grid<int> grid2 { 3, 3 };
    fillGrid(grid1); println("grid1:"); printGrid(grid1);
    fillGrid(grid2); println("\ngrid2:"); printGrid(grid2);
    auto result { grid1 + grid2 };
    println("\ngrid1 + grid2:"); printGrid(result);
}
\end{cpp}

\mySubsubsection{12.3.4.}{关于模板类型参数推导的更多信息}

编译器根据传递给函数模板的参数,来推导函数模板参数的类型。无法推导的模板参数必须显式指定。

例如,以下 add() 函数模板需要三个模板参数:返回值的类型和两个操作数的类型:

\begin{cpp}
template <typename RetType, typename T1, typename T2>
RetType add(const T1& t1, const T2& t2) { return t1 + t2; }
\end{cpp}

可以像这样调用这个函数模板,指定所有参数:

\begin{cpp}
auto result { add<long long, int, int>(1, 2) };
\end{cpp}

然而,因为 T1 和 T2 是函数的参数,编译器可以推导出这两个参数,所以可以只指定返回值的类型来调用 add():

\begin{cpp}
auto result { add<long long>(1, 2) };
\end{cpp}

这只有在要推导的参数在参数列表的最后时才有效。假设函数模板定义如下:

\begin{cpp}
template <typename T1, typename RetType, typename T2>
RetType add(const T1& t1, const T2& t2) { return t1 + t2; }
\end{cpp}

必须指定 RetType,因为编译器无法推导这个类型。因为 RetType 是第二个参数,也必须显式指定 T1:

\begin{cpp}
auto result { add<int, long long>(1, 2) };
\end{cpp}

也可以为返回类型模板参数提供一个默认值,这样就可以在不指定类型的情况下调用 add()了:

\begin{cpp}
template <typename RetType = long long, typename T1, typename T2>
RetType add(const T1& t1, const T2& t2) { return t1 + t2; }
...
auto result { add(1, 2) };
\end{cpp}

\mySubsubsection{12.3.5.}{函数模板的返回类型}

继续使用 add() 函数模板的例子,是否可以让编译器推导返回值的类型呢?理想情况;返回类型取决于模板类型参数,如何做到这一点呢?例如,以下函数模板:

\begin{cpp}
template <typename T1, typename T2>
RetType add(const T1& t1, const T2& t2) { return t1 + t2; }
\end{cpp}

在这个例子中,RetType 应该是表达式 t1+t2 的类型,但不知道类型,因为不知道 T1 和 T2 是什么。

正如第 1 章所讨论的,从 C++14 开始,可以要求编译器自动推导函数的返回类型。因此,可以简单地将 add() 写成如下形式:

\begin{cpp}
template <typename T1, typename T2>
auto add(const T1& t1, const T2& t2) { return t1 + t2; }
\end{cpp}

然而,使用 auto 推导表达式的类型会剥离引用和 const 限定符,而 decltype 不会剥离。对于 add() 函数模板来说,因为 operator+ 通常返回一个新的对象,所以可以剥离,但对于某些其他函数模板来说,这可能不那么理想,所以让来看看如何避免这种情况。

继续函数模板示例之前,先看看 auto 和 decltype 在非模板示例中的区别。假设有以下函数:

\begin{cpp}
const std::string message { "Test" };

const std::string& getString() { return message; }
\end{cpp}

可以像这样调用 getString() ,并存储结果到一个带有指定类型的 auto 变量中:

\begin{cpp}
auto s1 { getString() };
\end{cpp}

因为 auto 剥离了引用和 const 限定符,s1 是类型 string 的,因此会进行复制。如果想要一个引用到 const,可以显式地将其作为一个引用并标记为 const:

\begin{cpp}
const auto& s2 { getString() };
\end{cpp}

另一种解决方案是使用 decltype,不会剥离任何限定符:

\begin{cpp}
decltype(getString()) s3 { getString() };
\end{cpp}

这种情况下,s3 是类型 const string\& ,但由于需要两次指定 getString(),这会导致代码重复,这在 getString() 是一个更复杂的表达式时会变得繁琐。这个问题可以通过使用 decltype(auto) 来解决:

\begin{cpp}
decltype(auto) s4 { getString() };
\end{cpp}

s4 的类型也是 const string\& 。

有了这些知识,可以使用 decltype(auto) 来编写 add() 函数模板,以避免剥离 const 和引用限定符:

\begin{cpp}
template <typename T1, typename T2>
decltype(auto) add(const T1& t1, const T2& t2) { return t1 + t2; }
\end{cpp}

C++14 之前,即在函数返回类型推导和 decltype(auto) 支持之前,可以使用 decltype(expression) 来解决,该表达式是在 C++11 中引入的。例如,有些读者可能觉得,可以写成以下形式:

\begin{cpp}
template <typename T1, typename T2>
decltype(t1+t2) add(const T1& t1, const T2& t2) { return t1 + t2; }
\end{cpp}

然而,这是错误的。原型的开头就使用了 t1 和 t2,但这些参数编译器在当时还了解。t1 和 t2 在语义分析器到达参数列表的末尾时,才会被编译器所知。

这个问题过去是通过替代函数语法来解决的。在这种语法中,auto 在原型行的开头使用,实际的返回类型在参数列表之后指定(后置返回类型);因此,参数的名称(以及类型,从而推导出 t1+t2 的类型)是已知的:

\begin{cpp}
template <typename T1, typename T2>
auto add(const T1& t1, const T2& t2) -> decltype(t1+t2)
{
    return t1 + t2;
}
\end{cpp}

另一个选项是使用 std::declval<>(),返回请求类型的 rvalue 引用。这并不是一个完全构造的对象,因为没有调用构造函数!不能在运行时使用这个对象,应该与 decltype() 结合使用。它对于泛型代码,以及当需要创建一个未知类型的对象时,非常有用。这种情况下,不能调用任何构造函数,因为不知道未知类型支持哪些构造函数。

来看一个例子。前面的 add() 代码片段在原型开头有一个明确的返回类型 decltype(t1+t2),但这在编译时不会通过,因为那时 t1 和 t2 的名称还不知道。为了修复这个问题,可以使用 declval<>():

\begin{cpp}
template <typename T1, typename T2>
decltype(std::declval<T1>() + std::declval<T2>()) add(const T1& t1, const T2& t2)
{
    return t1 + t2;
}
\end{cpp}

\begin{myNotic}{NOTE}
现在 C++ 支持 auto 返回类型推导和 decltype(auto),建议使用这些机制,而不是替代函数语法或 declval<>()。
\end{myNotic}

\mySubsubsection{12.3.6.}{简化函数模板语法}

简化函数模板语法使得编写函数模板更加容易,再次看看上一节中的 add() 函数模板:

\begin{cpp}
template <typename T1, typename T2>
decltype(auto) add(const T1& t1, const T2& t2) { return t1 + t2; }
\end{cpp}

从上面这个例子来看,指定一个简单函数模板的语法相当冗长。使用简化函数模板语法,可以更优雅地写成如下形式:

\begin{cpp}
decltype(auto) add(const auto& t1, const auto& t2) { return t1 + t2; }
\end{cpp}

使用这种语法,不再需要模板头 template<> 来指定模板参数。相反,之前实现中使用 T1 和 T2 作为函数参数的类型,现在指定为 auto。这种简化语法只是语法糖;编译器自动将这种简化的实现,翻译为更长的代码。基本上,每个指定为 auto 的函数参数都变成了模板类型参数。

需要记住两个注意事项。首先,每个指定为 auto 的参数都变成了不同的模板类型参数。假设有一个这样的函数模板:

\begin{cpp}
template <typename T>
decltype(auto) add(const T& t1, const T& t2) { return t1 + t2; }
\end{cpp}

这个版本只有一个模板类型参数,函数的两个参数 t1 和 t2 都是类型 const T\&。对于这样的函数模板,因为这会翻译为具有两个不同模板类型参数的函数模板,所以不能使用简化语法。

第二个问题是,因为自动推导出的类型没有名称,所以不能在函数模板的实现中显式使用推导出的类型。如果需要这样做,要么继续使用更长的函数模板语法,要么使用 decltype() 来确定类型。






















