By solving the following exercises, you can practice the material discussed in this chapter. Solutions to all exercises are available with the code download on the book’s website at \url{www.wiley.com/go/proc++6e}. However, if you are stuck on an exercise, first reread parts of this chapter to try to find an answer yourself before looking at the solution from the website.

\begin{itemize}
\item
Exercise 18-1: This exercise is to practice working with vectors. Create a program containing a vector of ints, called values, initialized with the numbers 2 and 5. Next, implement the following operations:
\begin{enumerate}
\item
Use insert() to insert the numbers 3 and 4 at the correct place in values.

\item
Create a second vector of ints initialized with 0 and 1, and then insert the contents of this new vector at the beginning of values.

\item
Create a third vector of ints. Loop over the elements of values in reverse, and insert them in this third vector.

\item
Print the contents of the third vector using println().

\item
Print the contents of the third vector using a range-based for loop.
\end{enumerate}

\item
Exercise 18-2: Take your implementation of the Person class from Exercise 15-4. Add a new module called phone\_book, defining a PhoneBook class that stores one or more phone numbers as strings for a person. Provide member functions to add and remove person’s phone numbers to/from a phonebook. Also provide a member function that returns a vector with all phone numbers for a given person. Test your implementation in your main() function. In your tests, use the user-defined person literal developed in Exercise 15-4.

\item
Exercise 18-3: In Exercise 15-1 you developed your own AssociativeArray. Modify the test code in main() from that exercise to use one of the Standard Library containers instead.

\item
Exercise 18-4: Write an average() function (not a function template) to calculate the average of a sequence of double values. Make sure it works with a sequence or subsequence from a vector or an array. Test your code with both a vector and an array in your main() function.

\item
Bonus exercise: Can you convert your average() function into a function template? The function template should only be instantiatable with integral or floating-point types. What effect does it have on your test code in main()?
\end{itemize}














