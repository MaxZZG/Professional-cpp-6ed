

C++标准库提供了几种不同类型的关联容器:

\begin{itemize}
\item
有序关联容器: map, multimap, set和 multiset.

\item
无序关联容器:unordered\_map, unordered\_multimap, unordered\_set, 和unordered\_multiset。 这些也称为哈希表。

\item
平面关联容器适配器: flat\_map, flat\_multimap, flat\_set和 flat\_multiset。这些适配顺序容器以表现出有序关联容器的特性。

\CXXTwentythreeLogo{-45}{20}
\end{itemize}

\mySubsubsection{18.5.1.}{有序关联容器}

与顺序容器不同,有序关联容器不将元素存储在线性配置中。相反,它们提供键到值的映射,插入、删除和查找的时间相等。

标准库提供了四种有序关联容器:map、multimap、set和multiset。这些容器将元素存储在有序、树状的数据结构中。

\mySamllsection{pair 工具类}

深入研究有序关联容器之前,简要回顾一下在第1章中介绍的pair类模板。其定义在<utility>中,并将两个可能不同类型的值组合在一起。这些值可以通过first和second公共数据成员访问,所有比较运算符都得到支持,并且可同时比较first和second值。以下是几个示例:

\begin{cpp}
// Two-argument constructor and default constructor
pair<string, int> myPair { "hello", 5 };
pair<string, int> myOtherPair;

// Can assign directly to first and second
myOtherPair.first = "hello";
myOtherPair.second = 6;

// Copy constructor
pair<string, int> myThirdPair { myOtherPair };

// operator<
if (myPair < myOtherPair) {
    println("myPair is less than myOtherPair");
} else {
    println("myPair is greater than or equal to myOtherPair");
}

// operator==
if (myOtherPair == myThirdPair) {
    println("myOtherPair is equal to myThirdPair");
} else {
    println("myOtherPair is not equal to myThirdPair");
}
\end{cpp}

输出为:

\begin{shell}
myPair is less than myOtherPair
myOtherPair is equal to myThirdPair
\end{shell}

使用类模板参数推导(CTAD),可以省略模板类型参数。这里有一个例子。注意对标准字符串字面量s的使用。

\begin{cpp}
pair myPair { "hello"s, 5 }; // Type is pair<string, int>.
\end{cpp}

C++17引入对CTAD的支持之前,可以使用std::make\_pair()工具函数模板从两个值构造一个pair。以下是构造一个int和double的pair的三种方式:

\begin{cpp}
pair<int, double> pair1 { make_pair(5, 10.10) };
auto pair2 { make_pair(5, 10.10) };
pair pair3 { 5, 10.10 }; // CTAD
\end{cpp}

\mySamllsection{map}

map,定义在< map >中,存储键/值对而不是单个值。插入、查找和删除都基于键;值只是“顺便携带”

map根据键保持元素排序,因此插入、删除和查找的时间复杂度都是对数。由于顺序,当枚举元素时,会按照类型operator<或按用户定义比较器的顺序出现。它通常实现为某种形式的平衡树,如红黑树,但树结构对客户端不可见。

当需要基于“键”存储和检索元素,并且希望按特定顺序排列时,应该使用map。

\mySamllsection{构造 map}

map类模板接受四种类型:键类型、值类型、比较器类型和分配器类型,本章忽略分配器。比较器与前面描述的priority\_queue的比较器类似,允许更改默认比较器。在本章中,只使用默认的less比较器。使用默认比较器时,请确保所有键都支持operator<。如果对更多细节感兴趣,第19章将解释如何自定义比较器。

如果忽略比较器和分配器参数,构造map就像构造vector或list一样,除了在模板实例化中分别指定键和值类型。例如,以下代码构造了一个使用int作为键和Data类对象作为值的map:

\begin{cpp}
class Data final
{
    public:
        explicit Data(int value = 0) : m_value { value } { }
        int getValue() const { return m_value; }
        void setValue(int value) { m_value = value; }
    private:
        int m_value;
};
...
map<int, Data> dataMap;
\end{cpp}

内部,dataMap为map中的每个元素存储一个pair<int, Data>。

map也支持统一初始化。以下map在内部存储pair<string, int>的实例:

\begin{cpp}
map<string, int> m {
    { "Marc G.", 12 }, { "Warren B.", 34 }, { "Peter V.W.", 56 }
};
\end{cpp}

类模板参数推导不按预期工作,以下代码无法编译:

\begin{cpp}
map m {
    { "Marc G."s, 12 }, { "Warren B."s, 34 }, { "Peter V.W."s, 56 }
};
\end{cpp}

这不起作用,因为编译器无法从\{"Marc G."s, 12\}推导出pair<string, int>。如果真的想这样做,可以编写以下内容(注意字符串字面量的s后缀!):

\begin{cpp}
map m {
    pair { "Marc G."s, 12 }, pair { "Warren B."s, 34 }, pair { "Peter V.W."s, 56 }
};
\end{cpp}

\CXXTwentythreeLogo{-40}{-50}

\mySamllsection{格式化和打印Map}

与vector一样,std::format()和print()函数可以用来格式化和打印整个map,只需一条语句。对于vector,输出用方括号括起来,每个元素之间用逗号分隔。对于map,输出略有不同:输出用花括号括起来,每个键/值对之间用逗号分隔,键和值之间用冒号分隔。例如,打印上一节中的map m,输出为:

\begin{cpp}
{"Marc G.": 12, "Peter V.W.": 56, "Warren B.": 34}
\end{cpp}

\mySamllsection{插入元素}

将元素插入顺序容器,如vector和list,需要指定元素要添加的位置。与map以及其他有序关联容器不同,其内部实现决定了新元素应存储的位置,只需提供键和值。

\begin{myNotic}{NOTE}
map和其他有序关联容器确实提供了一个带有迭代器位置的insert()版本。那个位置只是对容器的一个“提示”,即正确位置。容器不必将元素插入那个位置。
\end{myNotic}

在插入元素时,重要的是要注意map需要唯一的键:map中的每个元素都必须有不同的键。如果想支持具有相同键的多个元素,有两个选择:要么使用map,并将另一个容器(如vector)作为键的值存储,或者使用稍后介绍的multimap。

\mySamllsection{insert()成员函数}

insert()成员函数可用于向map添加元素,其优点是可以检测键是否已经存在。必须将键/值对作为pair对象或作为初始化列表指定,insert()基本形式的返回类型是一个包含迭代器和布尔值的pair。返回类型复杂的原因是,insert()不会覆盖已经存在指定键的值。返回的pair中的布尔值元素指明insert()是否实际插入了新的键/值对,迭代器指向具有指定键的map中的元素(根据插入是否成功或失败,是新的还是旧的值)。map迭代器的详细介绍将在下一节中进行。继续上一节的map示例,可以这样使用insert():

\begin{cpp}
map<int, Data> dataMap;

auto ret { dataMap.insert({ 1, Data { 4 } }) }; // Using an initializer_list
if (ret.second) { println("Insert succeeded!"); }
else { println("Insert failed!"); }

ret = dataMap.insert(make_pair(1, Data { 6 })); // Using a pair object
if (ret.second) { println("Insert succeeded!"); }
else { println("Insert failed!"); }
\end{cpp}

ret变量的类型为:

\begin{cpp}
pair<map<int, Data>::iterator, bool> ret;
\end{cpp}

pair的第一个元素是具有int类型键和Data类型值的map的迭代器,pair的第二个元素是一个布尔值。

程序的输出如下:

\begin{shell}
Insert succeeded!
Insert failed!
\end{shell}

使用if语句初始化器,可以在单个语句中将数据插入map并检查结果:

\begin{cpp}
if (auto result { dataMap.insert({ 1, Data { 4 } }) }; result.second) {
    println("Insert succeeded!");
} else {
    println("Insert failed!");
}
\end{cpp}

这还可以与结构化绑定一起使用:

\begin{cpp}
if (auto [iter, success] { dataMap.insert({ 1, Data { 4 } }) }; success) {
    println("Insert succeeded!");
} else {
    println("Insert failed!");
}
\end{cpp}

\mySamllsection{The insert\_or\_assign()成员函数}

insert\_or\_assign()的返回类型与insert()类似,但如果已经存在具有给定键的元素,insert\_or\_assign()将用新值覆盖旧值,而insert()在这种情况下不会覆盖旧值。insert\_or\_assign()与insert()的另一个区别在于,它有两个独立的参数:键和值。以下是使用insert\_or\_assign()的示例:

\begin{cpp}
auto ret { dataMap.insert_or_assign(1, Data { 7 }) };
if (ret.second) { println("Inserted."); }
else { println("Overwritten."); }
\end{cpp}

\CXXTwentythreeLogo{-40}{-50}

\mySamllsection{insert\_range()成员函数}

C++23为map添加了insert\_range(),将给定范围内的所有元素插入到map中,并返回添加的第一个元素的迭代器。以下是使用insert\_range()的示例:

\begin{cpp}
vector<pair<int, Data>> moreData { {2, Data{22}}, {3, Data{33}}, {4, Data{44}} };
dataMap.insert_range(moreData);
\end{cpp}

\mySamllsection{operator[]}

向map中插入元素的另一种成员函数是通过重载的operator[]。主要区别在于语法:分别指定键和值。此外,operator[]总是成功。如果没有与给定键对应的值,将创建一个具有该键和值的新元素。如果已经有具有该键的元素,operator[]将用新指定的值替换该值。以下是使用operator[]代替insert()的示例的一部分:

\begin{cpp}
map<int, Data> dataMap;
dataMap[1] = Data { 4 };
dataMap[1] = Data { 6 }; // Replaces the element with key 1
\end{cpp}

然而,operator[]有一个主要缺点:总是构造一个新的值对象,即使不需要使用。因此,需要元素的默认构造函数,并且可能不如insert()高效。

由于operator[]在请求的元素不存在时在map中创建新元素,所以这个运算符没有标记为const。例如,有以下函数:

\begin{cpp}
void func(const map<int, int>& m)
{
    println("{}", m[1]); // Error
}
\end{cpp}

这无法编译,即使只是想读取m[1]的值。这是因为参数m是对map的const引用,而operator[]没有标记为const。这种情况下,应该使用find()或at()成员函数。

\mySamllsection{Emplace成员函数}

map支持emplace()和emplace\_hint(),用于在原地构造元素,类似于vector的emplace成员函数。还有一个try\_emplace()成员函数,如果给定的键不存在,则在原地插入元素,如果键已经存在于map中,则不做任何事情。

\mySamllsection{map迭代器}

map迭代器的工作方式与顺序容器上的迭代器类似。主要区别在于,迭代器指向键/值对,而不仅仅是值。要访问值,必须检索pair对象的第二字段。map迭代器是双向的,所以可以双向遍历。以下是遍历map的方式:

\begin{cpp}
for (auto iter { cbegin(dataMap) }; iter != cend(dataMap); ++iter) {
    println("{}", iter->second.getValue());
}
\end{cpp}

再看看用于访问值的表达式:

\begin{cpp}
iter->second.getValue()
\end{cpp}

iter指向一个键/值对,因此可以使用->运算符访问该对的第二字段,这是一个Data对象。然后,可以调用该Data对象的getValue()成员函数。

请注意,以下代码是功能等效的:

\begin{cpp}
(*iter).second.getValue()
\end{cpp}

使用基于范围的for循环,循环可以写得更具有可读性并减少错误:

\begin{cpp}
for (const auto& p : dataMap) {
    println("{}", p.second.getValue());
}
\end{cpp}

可以使用基于范围的for循环和结构化绑定更优雅地实现元素修改:

\begin{cpp}
for (const auto& [key, data] : dataMap) {
    println("{}", data.getValue());
}
\end{cpp}

\begin{myWarning}{WARNING}
可以通过非常量迭代器修改元素值,但如果尝试通过非常量迭代器修改元素的键,编译器将生成错误,这将破坏map中元素的排序顺序。
\end{myWarning}

\mySamllsection{查找元素}

map提供基于键的对数查找元素。如果已经知道给定键的元素在map中,最简单的方法是通过operator[]进行查找,只要在非常量map或对非常量map的引用上调用即可。operator[]的好处是返回一个引用,可以直接使用和修改,而不必担心从pair对象中提取值。以下是对前一个示例的扩展,以调用具有键1的Data对象值的setValue()成员函数:

\begin{cpp}
map<int, Data> dataMap;
dataMap[1] = Data { 4 };
dataMap[1] = Data { 6 };
dataMap[1].setValue(100);
\end{cpp}

作为替代,map提供了find()成员函数,其返回一个迭代器,指向具有请求键的键/值对,如果存在,或者如果键在map中找不到,则返回end()迭代器。这对于以下情况很有用:

\begin{itemize}
\item
如果不知道元素是否存在,可能不想使用operator[],因为如果它找不到已存在的键,会插入一个新的元素。

\item
如果有一个常量或对常量map的引用,这种情况下不能使用operator[]。
\end{itemize}

以下是一个使用find()执行对键1的Data对象修改的示例:

\begin{cpp}
auto it { dataMap.find(1) };
if (it != end(dataMap)) {
    it->second.setValue(100);
}
\end{cpp}

使用find()稍微有些笨拙,但在某些情况下是必要的。

或者,可以使用at()成员函数,其与operator[]一样。如果存在,则返回对map中具有请求键的值的引用。如果请求的键在map中找不到,会抛出out\_of\_range异常。at()成员函数在常量或对常量map上正常工作。例如:

\begin{cpp}
dataMap.at(1).setValue(200);
\end{cpp}

如果只想知道具有特定键的元素是否在map中,可以使用count()成员函数,返回与给定键在map中的元素数量。对于map,结果将始终为0或1,因为不能有具有重复键的元素。

此外,所有关联容器(有序、无序和扁平)都有一个名为contains()的成员函数。如果给定的键存在于容器中,返回true,否则返回false。有了这个,不再需要使用count()来确定关联容器中是否包含某个键。以下是包含键1的示例:

\begin{cpp}
auto isKeyInMap { dataMap.contains(1) };
\end{cpp}

\mySamllsection{删除元素}

map可以在特定迭代器位置删除一个元素,或在给定迭代器范围内删除所有元素,分别以常数时间和对数时间完成。从客户端的角度来看,这两个erase()成员函数与顺序容器中的erase()成员函数等效。然而,map的一个伟大特性是,还提供了一个erase(),用于删除匹配特定键的元素。以下是使用erase()的示例:

\begin{cpp}
map<int, Data> dataMap;
dataMap[1] = Data { 4 };
println("There are {} elements with key 1.", dataMap.count(1));
dataMap.erase(1);
println("There are {} elements with key 1.", dataMap.count(1));
\end{cpp}

输出为:

\begin{shell}
There are 1 elements with key 1.
There are 0 elements with key 1.
\end{shell}

\mySamllsection{节点}

所有有序和无序关联容器都是基于节点的数据结构。标准库提供了对节点的直接访问,形式为节点句柄。确切类型未指定,但每个容器都有一个类型别名node\_type,指定了该容器的节点句柄类型。节点句柄只能移动,并且是存储在节点中的元素的拥有者,其提供了对键和值的读写访问。

可以使用extract()成员函数基于给定的迭代器位置,或给定的键从关联容器中提取节点,作为节点句柄。从容器中提取节点会将其从容器中删除,因为返回的节点句柄是提取元素的唯一拥有者。

提供了新的insert()重载,允许节点句柄插入到容器中。

通过使用extract()提取节点句柄和使用insert()插入节点句柄,可以有效地将数据从一个关联容器转移到另一个关联容器,而无需复制或移动。甚至可以将节点从map转移到multimap,并将节点从set转移到multiset。继续使用前面部分的示例,以下代码片段将dataMap中键为1的节点,转移到一个名为dataMap2的第二个map中:

\begin{cpp}
map<int, Data> dataMap2;
auto extractedNode { dataMap.extract(1) };
dataMap2.insert(move(extractedNode));
\end{cpp}

最后两行可以合并为一行:

\begin{cpp}
dataMap2.insert(dataMap.extract(1));
\end{cpp}

对移动所有节点从一个关联容器到另一个容器的一个操作是merge()。如果移动节点会导致目标容器中出现重复(目标容器不允许重复),则源容器中的节点将保持不变。以下是使用merge()的示例:

\begin{cpp}
map<int, int> src { {1, 11}, {2, 22} };
map<int, int> dst { {2, 22}, {3, 33}, {4, 44}, {5, 55} };
dst.merge(src);
println("src = {}", src); // src = {2: 22}
println("dst = {}", dst); // dst = {1: 11, 2: 22, 3: 33, 4: 44, 5: 55}
\end{cpp}

merge操作后,src仍然包含一个元素\{2: 22\},因为目标容器已经包含这样的元素,所以无法移动。

\mySamllsection{map 示例: 银行账户}

可以通过使用map来实现一个简单的银行账户数据库。一个常见的模式是将键设置为一个类或结构体的一个字段,键是账户号码。下面是简单的BankAccount和BankDB类:

\begin{cpp}
export class BankAccount final
{
    public:
        explicit BankAccount(int accountNumber, std::string name)
            : m_accountNumber { accountNumber }, m_clientName { std::move(name) }{}
        void setAccountNumber(int accountNumber) {
            m_accountNumber = accountNumber; }
        int getAccountNumber() const { return m_accountNumber; }

        void setClientName(std::string name) { m_clientName = std::move(name); }
        const std::string& getClientName() const { return m_clientName; }
    private:
        int m_accountNumber;
        std::string m_clientName;
};
export class BankDB final
{
    public:
        // Adds account to the bank database. If an account exists already
        // with that account number, the new account is not added. Returns true
        // if the account is added, false if it's not.
        bool addAccount(const BankAccount& account);

        // Removes the account with accountNumber from the database.
        void deleteAccount(int accountNumber);

        // Returns a reference to the account represented
        // by its account number or the client name.
        // Throws out_of_range if the account is not found.
        BankAccount& findAccount(int accountNumber);
        BankAccount& findAccount(std::string_view name);

        // Adds all the accounts from db to this database.
        // Deletes all the accounts from db.
        void mergeDatabase(BankDB& db);
    private:
        std::map<int, BankAccount> m_accounts;
};
\end{cpp}

以下是BankDB成员函数的实现:

\begin{cpp}
bool BankDB::addAccount(const BankAccount& account)
{
    // Do the actual insert, using the account number as the key.
    auto res { m_accounts.emplace(account.getAccountNumber(), account) };
    // or: auto res { m_accounts.insert(
        // pair { account.getAccountNumber(), account }) };
    // Return the bool field of the pair specifying success or failure.
    return res.second;
}

void BankDB::deleteAccount(int accountNumber)
{
    m_accounts.erase(accountNumber);
}

BankAccount& BankDB::findAccount(int accountNumber)
{
    // Finding an element via its key can be done with find().
    auto it { m_accounts.find(accountNumber) };
    if (it == end(m_accounts)) {
        throw out_of_range { format("No account with number {}.", accountNumber) };
    }
    // Remember that iterators into maps refer to pairs of key/value.
    return it->second;
}

BankAccount& BankDB::findAccount(string_view name)
{
    // Finding an element by a non-key attribute requires a linear
    // search through the elements. The following uses structured bindings.
    for (auto& [accountNumber, account] : m_accounts) {
        if (account.getClientName() == name) {
            return account; // found it!
        }
    }
    throw out_of_range { format("No account with name '{}'.", name) };
}

void BankDB::mergeDatabase(BankDB& db)
{
    // Use merge().
    m_accounts.merge(db.m_accounts);
    // Or: m_accounts.insert(begin(db.m_accounts), end(db.m_accounts));

    // Now clear the source database.
    db.m_accounts.clear();
}
\end{cpp}

可以用下面的代码测试BankDB类:

\begin{cpp}
BankDB db;
db.addAccount(BankAccount { 100, "Nicholas Solter" });
db.addAccount(BankAccount { 200, "Scott Kleper" });

try {
    auto& account { db.findAccount(100) };
    println("Found account 100");
    account.setClientName("Nicholas A Solter");

    auto& account2 { db.findAccount("Scott Kleper") };
    println("Found account of Scott Kleper");

    auto& account3 { db.findAccount(1000) };
} catch (const out_of_range& caughtException) {
    println("Unable to find account: {}", caughtException.what());
}
\end{cpp}

输出为:

\begin{shell}
Found account 100
Found account of Scott Kleper
Unable to find account: No account with number 1000.
\end{shell}

\mySamllsection{multimap}

multimap中同一键可对应多个元素。与map一样,multimap支持统一初始化。其接口与map接口几乎相同,但有以下区别:

\begin{itemize}
\item
multimap不提供 operator[] 和 at()。如果可以有多个元素对应单个键,则这些操作的语义没有意义。

\item
multimap的插入操作总是成功的,添加单个元素的 multimap::insert() 成员函数只返回一个迭代器,而不是一个 pair。

\item
支持的 insert\_or\_assign() 和 try\_emplace() 成员函数在multimap中不支持。
\end{itemize}

\begin{myNotic}{NOTE}
multimap允许插入相同的键/值对。如果想要避免这种冗余,必须在插入新元素之前显式检查。
\end{myNotic}

multimap最棘手的部分是查找元素,但不能使用 operator[]。find() 并不是很有用,因为只返回一个迭代器,指向具有给定键的任一元素(不一定是该键的第一个元素)。

然而,multimap将所有具有相同键的元素存储在一起,并提供成员函数来获取容器中具有相同键的元素子范围的迭代器。lower\_bound() 和 upper\_bound() 成员函数各自返回一个迭代器,指向与给定键匹配的第一个元素和最后一个元素之后的位置。如果没有与该键匹配的元素,lower\_bound() 和 upper\_bound() 返回的迭代器将彼此相等。

如果需要获取界定具有给定键的元素的两个迭代器,使用 equal\_range() 比先调用 lower\_bound() 再调用 upper\_bound() 更高效。equal\_range() 成员函数返回 lower\_bound() 和 upper\_bound(),将返回的那两个迭代器组成的一对。

\begin{myNotic}{NOTE}
lower\_bound()、upper\_bound() 和 equal\_range() 成员函数在map中同样存在,但它们的实用性有限,因为map不能有多个元素对应同一键。
\end{myNotic}

\mySamllsection{multimap 示例: 好友列表}

大多数在线聊天程序允许用户拥有一个“好友列表”或朋友列表,聊天程序赋予好友列表中的用户特殊权限,例如:允许他们向用户发送未经请求的消息。

实现好友列表的一种方法是将信息存储在multimap中,一个multimap可以存储每个用户的好友列表。容器中的每个条目存储一个用户的好友。键是用户,值是好友。例如,如果哈利·波特和罗恩·韦斯莱彼此都在各自的好友列表中,就会有两条形式为“哈利·波特”映射到“罗恩·韦斯莱”和“罗恩·韦斯莱”映射到“哈利·波特”的条目。多映射允许同一键有多个值,因此同一用户可以有多个好友。以下是BuddyList类的定义:


\begin{cpp}
export class BuddyList final
{
    public:
        // Adds buddy as a friend of name.
        void addBuddy(const std::string& name, const std::string& buddy);
        // Removes buddy as a friend of name.
        void removeBuddy(const std::string& name, const std::string& buddy);
        // Returns true if buddy is a friend of name, false otherwise.
        bool isBuddy(const std::string& name, const std::string& buddy) const;
        // Retrieves a list of all the friends of name.
        std::vector<std::string> getBuddies(const std::string& name) const;
    private:
        std::multimap<std::string, std::string> m_buddies;
};
\end{cpp}

这里提供了代码实现,展示了 lower\_bound()、upper\_bound() 和 equal\_range() 的使用方法。

\begin{cpp}
void BuddyList::addBuddy(const string& name, const string& buddy)
{
    // Make sure this buddy isn't already there. We don't want
    // to insert an identical copy of the key/value pair.
    if (!isBuddy(name, buddy)) {
        m_buddies.insert({ name, buddy }); // Using initializer_list
    }
}

void BuddyList::removeBuddy(const string& name, const string& buddy)
{
    // Obtain the beginning and end of the range of elements with
    // key 'name'. Use both lower_bound() and upper_bound() to demonstrate
    // their use. Otherwise, it's more efficient to call equal_range().
    auto begin { m_buddies.lower_bound(name) }; // Start of the range
    auto end { m_buddies.upper_bound(name) }; // End of the range

    // Iterate through the elements with key 'name' looking
    // for a value 'buddy'. If there are no elements with key 'name',
    // begin equals end, so the loop body doesn't execute.
    for (auto iter { begin }; iter != end; ++iter) {
        if (iter->second == buddy) {
            // We found a match! Remove it from the map.
            m_buddies.erase(iter);
            break;
        }
    }
}

bool BuddyList::isBuddy(const string& name, const string& buddy) const
{
    // Obtain the beginning and end of the range of elements with
    // key 'name' using equal_range(), and structured bindings.
    auto [begin, end] { m_buddies.equal_range(name) };

    // Iterate through the elements with key 'name' looking
    // for a value 'buddy'.
    for (auto iter { begin }; iter != end; ++iter) {
        if (iter->second == buddy) {
            // We found a match!
            return true;
        }
    }
    // No matches
    return false;
}

vector<string> BuddyList::getBuddies(const string& name) const
{
    // Obtain the beginning and end of the range of elements with
    // key 'name' using equal_range(), and structured bindings.
    auto [begin, end] { m_buddies.equal_range(name) };

    // Create a vector with all names in the range (all buddies of name).
    vector<string> buddies;
    for (auto iter { begin }; iter != end; ++iter) {
        buddies.push_back(iter->second);
    }
    return buddies;
}
\end{cpp}

请注意,removeBuddy() 不能简单地使用erase()删除所有具有给定键的元素,因为它应该只删除该键的一个元素,而不是所有元素。还要注意,getBuddies() 不能在vector上使用insert()来插入equal\_range()返回范围中的元素,因为由multimap迭代器引用的元素是键/值对,而不是字符串。getBuddies() 成员函数必须显式地遍历该范围,从每个键/值对中提取字符串并将其推送到要返回的新vector中。

或者,使用第17章讨论的C++23范围功能,getBuddies() 可以进行如下实现,而无需显式循环:

\begin{cpp}
vector<string> BuddyList::getBuddies(const string& name) const
{
    auto [begin, end] { m_buddies.equal_range(name) };
    return ranges::subrange { begin, end } | views::values | ranges::to<vector>();
}
\end{cpp}

下面是对BuddyList的测试:

\begin{cpp}
BuddyList buddies;
buddies.addBuddy("Harry Potter", "Ron Weasley");
buddies.addBuddy("Harry Potter", "Hermione Granger");
buddies.addBuddy("Harry Potter", "Hagrid");
buddies.addBuddy("Harry Potter", "Draco Malfoy");
// That's not right! Remove Draco.
buddies.removeBuddy("Harry Potter", "Draco Malfoy");
buddies.addBuddy("Hagrid", "Harry Potter");
buddies.addBuddy("Hagrid", "Ron Weasley");
buddies.addBuddy("Hagrid", "Hermione Granger");

auto harrysFriends { buddies.getBuddies("Harry Potter") };

println("Harry's friends: ");
for (const auto& name : harrysFriends) {
    println("\t{}", name);
}
\end{cpp}

输出为:

\begin{shell}
Harry's friends:
        Ron Weasley
        Hermione Granger
        Hagrid
\end{shell}

\mySamllsection{set}

set(集合),定义在<set>中,与map类似。不同之处在于,set不存储键/值对,而是值就是键。set适用于存储没有显式键但希望有序且不重复的信息,同时支持快速插入、查找和删除。

set提供的接口与map几乎相同。主要区别在于set不提供operator[]、insert\_or\_assign()和try\_emplace()。

不能更改set中元素的值,因为修改集合中的元素会破坏其顺序。

\mySamllsection{set 示例: 访问控制列表}

计算机系统上实现基本安全性的一种方法是,访问控制列表。系统上的每个实体(如文件或设备)都有一个用户列表,这些用户有权访问该实体。通常,只有具有特殊权限的用户才能将其他用户添加到或从权限列表中移除。在内部,集合提供了一种很好的方式来表示访问控制列表。可以为每个实体使用一个set,包含所有允许访问该实体的用户名。这是一个简单访问控制列表的类定义:

\begin{cpp}
export class AccessList final
{
    public:
        // Default constructor
        AccessList() = default;
        // Constructor to support uniform initialization.
        AccessList(std::initializer_list<std::string_view> users)
        {
            m_allowed.insert(begin(users), end(users));
        }
        // Adds the user to the permissions list.
        void addUser(std::string user)
        {
            m_allowed.emplace(std::move(user));
        }
        // Removes the user from the permissions list.
        void removeUser(const std::string& user)
        {
            m_allowed.erase(user);
        }
        // Returns true if the user is in the permissions list.
        bool isAllowed(const std::string& user) const
        {
            return m_allowed.contains(user);
        }
        // Returns all the users who have permissions.
        const std::set<std::string>& getAllUsers() const
        {
            return m_allowed;
        }
        // Returns a vector of all the users who have permissions.
        std::vector<std::string> getAllUsersAsVector() const
        {
            return { begin(m_allowed), end(m_allowed) };
        }
    private:
        std::set<std::string> m_allowed;
};
\end{cpp}

getAllUsersAsVector()中有一行实现很有趣,将m\_allowed的begin和end迭代器传递给vector构造函数,来构造一个vector<string>。如果愿意,可以将其拆分为两行:

\begin{cpp}
std::vector<std::string> users { begin(m_allowed), end(m_allowed) };
return users;
\end{cpp}

最后,这里有一个简单的测试程序:

\begin{cpp}
AccessList fileX { "mgregoire", "baduser" };
fileX.addUser("pvw");
fileX.removeUser("baduser");

if (fileX.isAllowed("mgregoire")) { println("mgregoire has permissions"); }
if (fileX.isAllowed("baduser")) { println("baduser has permissions"); }

// C++23 supports formatting/printing of ranges, see Chapter 2.
println("Users with access: {:n:}", fileX.getAllUsers());

// Iterating over the elements of a set.
print("Users with access: ");
for (const auto& user : fileX.getAllUsers()) { print("{} ", user); }
println("");

// Iterating over the elements of a vector.
print("Users with access: ");
for (const auto& user : fileX.getAllUsersAsVector()) { print("{} ", user); }
println("");
\end{cpp}

AccessList的一个构造函数使用initializer\_list作为参数,这样就可以使用统一初始化语法,如在测试程序中对fileX进行的初始化。

此程序的输出如下:

\begin{shell}
mgregoire has permissions
Users with access: mgregoire, pvw
Users with access: mgregoire pvw
Users with access: mgregoire pvw
\end{shell}

注意,m\_allowed数据成员需要是一个std::string的集合,而不是string\_view的集合。将其更改为string\_view的set,会引入悬空指针的问题。例如,有以下代码:

\begin{cpp}
AccessList fileX;
{
    string user { "someuser" };
    fileX.addUser(user);
}
\end{cpp}

这段代码创建了一个名为user的string,然后将其添加到fileX访问控制列表中。然而,string和addUser()的调用位于一对花括号内;所以,string的生命周期比fileX短。关闭花括号时,string超出作用域并销毁。这将导致fileX访问控制列表中的string\_view指向一个销毁的string,即悬空指针!通过使用string的set可以避免这个问题。

\mySamllsection{multiset}

multiset对于set来说,就像multimap对于map一样。multiset支持set的所有操作,也可同时存储多个相等的元素。这里没有显示multiset的示例,因为它与set和multimap非常相似。

\mySubsubsection{18.5.2.}{无序关联容器或哈希表}

标准库支持无序关联容器或哈希表,有四个:unordered\_map、unordered\_multimap、unordered\_set和unordered\_multiset。前面讨论的map、multimap、set和multiset容器会对其元素进行排序,而这些无序版本不会对元素进行排序。

\mySamllsection{哈希函数}

无序关联容器是哈希表,因为实现使用了哈希函数。实现通常由某种数组组成,数组中的每个元素都称为一个桶(bucket)。每个桶都有一个特定的数字索引,如0、1、2,直到最后一个桶。哈希函数将键转换为一个哈希值,然后该值转换为桶索引。与该键关联的值会存储在该桶中。

哈希函数的结果并不唯一,当两个或多个键哈希到同一个桶索引时,这种情况称为碰撞。当不同的键产生相同的哈希值,或者不同的哈希值转换到相同的桶索引时,可能会发生碰撞。处理碰撞有许多方法,包括二次哈希和线性链表等。如果感兴趣,可以参考附录B中的参考文献。标准库没有指定需要哪种碰撞处理算法,但大多数当前实现选择通过线性链表解决碰撞。线性链表中,桶不直接包含与键关联的数据值,而是包含指向链表的指针,这个链表包含该特定桶的所有数据值。图18.1显示了这是如何工作的。

\myGraphic{0.9}{content/part3/chapter18/images/1.png}{图 18.1}

图18.1中,有两个碰撞。第一个碰撞是因为对键“Marc G.”和“John D.”应用哈希函数得到相同的哈希值,该值映射到桶索引128。然后这个桶指向一个链表,包含键“Marc G.”和“John D.”以及它们关联的数据值。第二个碰撞是由键“Scott K.”和“Johan G.”的哈希值映射到相同的桶索引129引起的。

从图18.1中也可以清楚地看出基于键的查找是如何工作的,以及其复杂度如何。

查找涉及一次哈希函数调用以计算哈希值,哈希值转换为一个桶索引。当知道了桶索引,就需要进行一个或多个相等性操作以在链表中找到正确的键。这查找可比map的查找快得多,但也取决于碰撞的数量。

哈希函数的选择很重要,不产生碰撞的哈希函数称为完美哈希。完美哈希的查找时间恒定;普通哈希的查找时间平均接近1,与元素数量无关。随着碰撞数量的增加,查找时间增加,性能降低。通过增加基本哈希表的大小可以减少碰撞,但需要考虑缓存大小。

C++标准为指针和所有基本数据类型(如bool、char、int、float、double等)提供了哈希函数。还为几个标准库类提供了哈希函数,如optional、bitset、unique\_ptr、shared\_ptr、string、string\_view、vector<bool>等。如果想要使用的键类型没有标准的哈希函数,那么必须实现自己的哈希函数。即使键集固定且已知,创建完美哈希也不简单,需要深度的数学分析。即使创建一个不完美,但足够好且性能不错的哈希函数仍然具有挑战性。解释哈希函数背后的数学知识超出了本书的范围,这里只给出了一个简单哈希函数的示例。

以下代码演示了如何编写自定义哈希函数。代码定义了一个只包装单个整数的类IntWrapper。提供了一个operator==,因为这是无序关联容器中使用键的要求。

\begin{cpp}
class IntWrapper
{
    public:
        explicit IntWrapper(int i) : m_wrappedInt { i } {}
        int getValue() const { return m_wrappedInt; }
        bool operator==(const IntWrapper&) const = default;// = default since C++20
    private:
        int m_wrappedInt;
};
\end{cpp}

要为IntWrapper编写实际的哈希函数,需要为IntWrapper编写std::hash类模板特化。std::hash类模板定义在<functional>中,这个特化需要一个实现函数调用运算符,该运算符计算,并返回给定IntWrapper实例的哈希值。对于这个例子,请求只是简单地转发给标准整数哈希函数:

\begin{cpp}
namespace std
{
    template<> struct hash<IntWrapper>
    {
        size_t operator()(const IntWrapper& x) const {
            return std::hash<int>{}(x.getValue());
        }
    };
}
\end{cpp}

注意,通常不允许在std命名空间中放置内容;然而,std类模板特化是这一规则的例外。函数调用运算符的实现只有一行。创建了一个标准整数哈希函数的实例——std::hash<int>\{\}——然后在它上面调用函数调用运算符,参数是x.getValue()。注意,这种转发在这个例子中有效,因为IntWrapper只包含一个数据成员,一个整数。如果类包含多个数据成员,则需要计算一个考虑所有这些数据成员的哈希值,但这些细枝末节都超出了本书介绍的范围。

\mySamllsection{unordered\_map}

unordered\_map定义在<unordered\_map>中,作为一个类模板:

\begin{cpp}
template <typename Key,
          typename T,
          typename Hash = hash<Key>,
          typename Pred = std::equal_to<Key>,
          typename Alloc = std::allocator<std::pair<const Key, T>>>
    class unordered_map;
\end{cpp}

这里有五个模板类型参数:键类型、值类型、哈希类型、相等比较器类型和分配器类型。最后三个参数有默认值。最重要的参数是前两个。与map一样,可以使用统一初始化来初始化一个unordered\_map。遍历元素也与映射类似,如下所示:

\begin{cpp}
unordered_map<int, string> m {
    {1, "Item 1"}, {2, "Item 2"}, {3, "Item 3"}, {4, "Item 4"}
};
// Using C++23 support for formatting/printing ranges.
println("{}", m);
// Using structured bindings.
for (const auto& [key, value] : m) { print("{} = {}, ", key, value); }
println("");
// Without structured bindings.
for (const auto& p : m) { print("{} = {}, ", p.first, p.second); }
\end{cpp}

输出如下所示:

\begin{shell}
{4: "Item 4", 3: "Item 3", 2: "Item 2", 1: "Item 1"}
4 = Item 4, 3 = Item 3, 2 = Item 2, 1 = Item 1,
4 = Item 4, 3 = Item 3, 2 = Item 2, 1 = Item 1,
\end{shell}

下表总结了map和unordered\_map之间的区别。一个填充的方框(■)表示容器支持该操作,而一个空方框(□)表示容器不支持该操作。

% Please add the following required packages to your document preamble:
% \usepackage{longtable}
% Note: It may be necessary to compile the document several times to get a multi-page table to line up properly
\begin{longtable}{|l|l|l|}
\hline
\textbf{操作}                           & \textbf{map} & \textbf{unordered\_map} \\ \hline
\endfirsthead
%
\endhead
%
at()                                         & ■            & ■                       \\ \hline
begin()                                      & ■            & ■                       \\ \hline
begin(n)                                     & □            & ■                       \\ \hline
bucket()                                     & □            & ■                       \\ \hline
bucket\_count()                              & □            & ■                       \\ \hline
bucket\_size()                               & □            & ■                       \\ \hline
cbegin()                                     & ■            & ■                       \\ \hline
cbegin(n)                                    & □            & ■                       \\ \hline
cend()                                       & ■            & ■                       \\ \hline
cend(n)                                      & □            & ■                       \\ \hline
clear()                                      & ■            & ■                       \\ \hline
contains()                                   & ■            & ■                       \\ \hline
count()                                      & ■            & ■                       \\ \hline
crbegin()                                    & ■            & □                       \\ \hline
crend()                                      & ■            & □                       \\ \hline
emplace()                                    & ■            & ■                       \\ \hline
emplace\_hint()                              & ■            & ■                       \\ \hline
empty()                                      & ■            & ■                       \\ \hline
end()                                        & ■            & ■                       \\ \hline
end(n)                                       & □            & ■                       \\ \hline
equal\_range()                               & ■            & ■                       \\ \hline
erase()                                      & ■            & ■                       \\ \hline
extract()                                    & ■            & ■                       \\ \hline
find()                                       & ■            & ■                       \\ \hline
insert()                                     & ■            & ■                       \\ \hline
insert\_or\_assign()                         & ■            & ■                       \\ \hline
insert\_range() (C++23)                      & ■            & ■                       \\ \hline
iterator / const\_iterator                   & ■            & ■                       \\ \hline
load\_factor()                               & □            & ■                       \\ \hline
local\_iterator / const\_local\_iterator     & □            & ■                       \\ \hline
lower\_bound()                               & ■            & □                       \\ \hline
max\_bucket\_count()                         & □            & ■                       \\ \hline
max\_load\_factor()                          & □            & ■                       \\ \hline
max\_size()                                  & ■            & ■                       \\ \hline
merge()                                      & ■            & ■                       \\ \hline
operator{[}{]}                               & ■            & ■                       \\ \hline
rbegin()                                     & ■            & □                       \\ \hline
rehash()                                     & □            & ■                       \\ \hline
rend()                                       & ■            & □                       \\ \hline
reserve()                                    & □            & ■                       \\ \hline
reverse\_iterator / const\_reverse\_iterator & ■            & □                       \\ \hline
size()                                       & ■            & ■                       \\ \hline
swap()                                       & ■            & ■                       \\ \hline
try\_emplace()                               & ■            & ■                       \\ \hline
upper\_bound()                               & ■            & □                       \\ \hline
\end{longtable}

与map一样,unordered\_map中的所有键必须唯一。前一个表包括了一些特定于哈希的成员函数。例如,load\_factor()返回每个桶中的平均元素数量,以给定一个碰撞数量的指数。bucket\_count()成员函数返回容器中的桶数量;还提供了local\_iterator和const\_local\_iterator,允许遍历单个桶中的元素;然而,这些不能用于跨桶遍历。bucket(key)成员函数返回包含给定键的桶的索引;begin(n)返回一个指向索引为n的桶中第一个元素的local\_iterator,end(n)返回一个指向索引为n的桶中最后一个元素之后的local\_iterator。下一节的示例演示了如何使用其中一些成员函数。

\mySamllsection{unordered\_map示例:电话簿}

以下示例使用unordered\_map来表示一个电话簿。人的名字是键,而电话号码是与该键关联的值。

\begin{cpp}
void printMap(const auto& m) // Abbreviated function template
{
    for (auto& [key, value] : m) {
        println("{} (Phone: {})", key, value);
    }
    println("-------");
}

int main()
{
    // Create a hash table.
    unordered_map<string, string> phoneBook {
        { "Marc G.", "123-456789" },
        { "Scott K.", "654-987321" } };
    printMap(phoneBook);

    // Add/remove some phone numbers.
    phoneBook.insert(make_pair("John D.", "321-987654"));
    phoneBook["Johan G."] = "963-258147";
    phoneBook["Freddy K."] = "999-256256";
    phoneBook.erase("Freddy K.");
    printMap(phoneBook);

    // Find the bucket index for a specific key.
    const size_t bucket { phoneBook.bucket("Marc G.") };
    println("Marc G. is in bucket {} containing the following {} names:",
        bucket, phoneBook.bucket_size(bucket));
    // Get begin and end iterators for the elements in this bucket.
    // 'auto' is used here. The compiler deduces the type of
    // both as unordered_map<string, string>::const_local_iterator
    auto localBegin { phoneBook.cbegin(bucket) };
    auto localEnd { phoneBook.cend(bucket) };
    for (auto iter { localBegin }; iter != localEnd; ++iter) {
        println("\t{} (Phone: {})", iter->first, iter->second);
    }
    println("-------");

    // Print some statistics about the hash table
    println("There are {} buckets.", phoneBook.bucket_count());
    println("Average number of elements in a bucket is {}.",
    phoneBook.load_factor());
}
\end{cpp}

一种可能的输出如下。注意,输出在不同的系统上可能会有所不同,它取决于所使用的哈希函数和unordered\_map的实现。

\begin{shell}
Scott K. (Phone: 654-987321)
Marc G. (Phone: 123-456789)
-------
Scott K. (Phone: 654-987321)
Marc G. (Phone: 123-456789)
Johan G. (Phone: 963-258147)
John D. (Phone: 321-987654)
-------
Marc G. is in bucket 1 containing the following 2 names:
        Scott K. (Phone: 654-987321)
        Marc G. (Phone: 123-456789)
-------
There are 8 buckets.
Average number of elements in a bucket is 0.5
\end{shell}

\mySamllsection{unordered\_multimap}

unordered\_multimap是一个允许具有相同键的多个元素的无序map。它们的接口几乎相同,但有以下区别:

\begin{itemize}
\item
unordered\_multimap不提供operator[]和at()。如果有多个具有单个键的元素,这些语义就没有意义。

\item
unordered\_multimap上的插入总会成功。因此,添加单个元素的unordered\_multimap::insert()成员函数只返回一个迭代器,而不是一个pair。

\item
unordered\_map支持的insert\_or\_assign()和try\_emplace()成员函数,但在unordered\_multimap中没有这几个成员函数。
\end{itemize}

\begin{myNotic}{NOTE}
unordered\_multimap可以插入相同的键/值对。如果想要避免这种重复,必须在插入新元素之前进行显式检查。
\end{myNotic}

正如之前与multimap的讨论,unordered\_multimap中的元素查找不能使用operator[]。可以使用find(),但它返回一个指向具有给定键的任意一个元素的迭代器(不一定是该键的第一个元素)。最好使用equal\_range()成员函数,返回一对迭代器:一个指向与给定键匹配的第一个元素,另一个指向与该键匹配的最后一个元素之后的位置。equal\_range()的使用与之前讨论的multimap相同,所以以查看为multimap提供的示例来了解它。

\mySamllsection{unordered\_set/unordered\_multiset}

<unordered\_set>定义了unordered\_set和unordered\_multiset,它们类似于set和multiset,不同之处在于它们不排序其键,而是使用哈希函数。unordered\_set和unordered\_map之间的区别与本章前面讨论的set和map之间的区别类似,因此在这里不详细讨论。要了解unordered\_set和unordered\_multiset操作的完整总结,请参阅标准库手册。

\mySubsubsection{18.5.3.}{扁平set和扁平map关联容器适配器}

\CXXTwentythreeLogo{-40}{15}

C++23引入了以下容器适配器:

\begin{itemize}
\item
std::flat\_set 和 flat\_multiset,定义在<flat\_set>中

\item
std::flat\_map 和 flat\_multimap,定义在<flat\_map>中
\end{itemize}

这些适配器在顺序容器之上提供了一个关联容器接口。flat\_set和flat\_map要求唯一键,就像set和map一样,而flat\_multiset和flat\_multimap支持重复键,就像multiset和multimap一样。它们都使用std::less作为默认比较器,按键排序存储其数据。flat\_set和flat\_multiset提供快速检索键的功能,而flat\_map和flat\_multimap提供基于键快速检索值的功能。flat\_set和flat\_multiset需要一个底层顺序容器来存储它们的键。flat\_map和flat\_multimap需要两个底层容器,一个用于存储键,另一个用于存储值。底层容器必须支持随机访问迭代器,如vector和deque。默认情况下,使用vector。

所有扁平关联容器适配器的接口与有序对应容器相似,只是扁平容器适配器不是基于节点的数据结构,因此没有节点处理的概念。另一个区别是,扁平版本提供了随机访问迭代器,而有序对应物只提供双向迭代器。

随着这些扁平容器适配器的加入,标准库现在为每种关联容器类型提供了三种版本;例如,现在有三种映射容器:map、unordered\_map和flat\_map。所有三种基本上都以类似的方式工作,但它们以截然不同的数据结构存储数据,因此具有不同的时间和空间效率。由于扁平关联容器适配器按顺序容器中的键排序存储数据,所有添加和删除元素的时间复杂度都为线性,这可能比添加和删除有序和无序容器中的元素要慢。查找具有对数复杂度,就像有序关联容器一样。但对于扁平版本,查找和特别是遍历元素比有序变体更有效,因为前者在顺序容器中存储数据,因此具有更高效和缓存友好的内存布局,每个元素所需的内存也少于有序或无序版本。对于每种类型的三种版本中选择哪一种,取决于特定用例的确切要求。如果性能很重要,我建议为特定用例对所有三种进行基准测试,以找出最适合的那一种。基准测试将在第29章中介绍。

扁平关联容器适配器通常只是相应有序容器的替换。例如,前面提到的访问控制列表示例有一个名为m\_allowed的set<string>类型的数据成员,这是一个有序关联容器,可以很轻松更改为使用flat\_set:

\begin{cpp}
std::flat_set<std::string> m_allowed;
\end{cpp}

其次,getAllUsers()的返回类型可更改为flat\_set:

\begin{cpp}
const std::flat_set<std::string>& getAllUsers() const { return m_allowed; }
\end{cpp}

其他一切保持不变。

\mySubsubsection{18.5.4.}{关联容器的性能}

从这一节可以看出,C++标准库包含几种不同的关联容器。如何知道为特定任务使用哪一种?如果用例中遍历关联容器的内部内容很重要,那么扁平关联容器适配器具有最佳性能,因为它们以连续内存的方式存储数据。如果更关心其他操作,那么无序关联容器通常比有序容器更快。如果性能非常重要,那么决定正确的容器唯一的方法是通过为特定用例对所有容器进行基准测试。通常,可以选择更容易使用的那个。要使用有序版本,必须为自定义类类型实现比较操作,而对于无序版本,需要编写一个哈希函数。通常,后者更难实现。









