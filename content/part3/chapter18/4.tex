
除了标准顺序容器之外,标准库还提供了三个容器适配器:queue、priority\_queue和stack。这些适配器都是顺序容器的包装器,允许交换底层容器,而无需更改其余代码。适配器的目的是简化接口,并提供仅适用于stack、queue或priority\_queue抽象的适当功能。这些适配器不提供对底层容器的访问,因此是第4章中解释的数据隐藏原则的完美示例。例如,适配器不支持同时擦除多个元素的功能,也不提供迭代器。后者意味着不能使用它们进行基于范围的for循环,也不能使用第20章中讨论的标准迭代器算法。但从C++23开始,std::format()和print()函数支持格式化和打印这些容器适配器的内容。

\mySubsubsection{18.4.1.}{queue}

queue容器适配器,定义在< queue >中,提供标准的先进先出语义。通常,它是一个类模板:

\begin{cpp}
template <typename T, typename Container = deque<T>> class queue;
\end{cpp}

T模板参数指定存储在队列中的类型,第二个模板参数指定队列适配的底层容器。然而,队列要求顺序容器支持push\_back()和pop\_front(),因此只有两个内置选择:deque和list。对于大多数情况,可以使用默认的deque。

\mySamllsection{queue的操作}

queue接口非常简单:只有九个成员函数、一组构造函数和比较运算符。C++23中新增了一个接受迭代器对的构造函数[begin, end),构造一个包含给定迭代器范围内元素的队列。push()和emplace()成员函数将新元素添加到队列的尾部,而pop()删除队列头部的元素。C++23添加了push\_range()以将一系列元素添加到队列中,可以使用front()和back()分别检索对队列头尾元素的引用(不移除)。通常,当在const对象上调用时,front()和back()返回对const的引用;当在非const对象上调用时,返回对非const的引用(读/写)。

\begin{myWarning}{WARNING}
pop()不返回弹出的元素。如果想保留一个副本,必须先用front()检索。
\end{myWarning}

queue还支持size()、empty()和swap()。

\mySamllsection{queue示例:网络数据包缓冲区}

当两台计算机通过网络通信时,它们会将信息划分为数据包的离散块并发送给对方。计算机的操作系统网络层,必须接收这些数据包,并在包到达时存储起来。然而,计算机可能没有足够的带宽一次性处理所有数据包,所以网络层通常会缓冲或存储数据包,直到有机会处理它们。数据包应该按照它们到达的顺序进行处理,所以这个问题非常适合使用队列结构。以下是一个小的 PacketBuffer 类,代码中有注释解释,它将传入的数据包存储在队列中,直到它们被处理。它是一个类模板,因此网络堆栈的不同层次可以使用它来处理不同类型的数据包,例如: IP 数据包或 TCP 数据包,允许客户端指定最大大小,因为操作系统通常限制可以存储的数据包数量,以避免使用过多内存。当缓冲区已满时,将忽略随后到达的数据包。

\begin{cpp}
export template <typename T>
class PacketBuffer final
{
    public:
        // If maxSize is 0, the size is unlimited, because creating
        // a buffer of size 0 makes little sense. Otherwise only
        // maxSize packets are allowed in the buffer at any one time.
        explicit PacketBuffer(std::size_t maxSize = 0);

        // Stores a packet in the buffer.
        // Returns false if the packet has been discarded because
        // there is no more space in the buffer, true otherwise.
        bool bufferPacket(const T& packet);

        // Returns the next packet. Throws out_of_range
        // if the buffer is empty.
        [[nodiscard]] T getNextPacket();
    private:
        std::queue<T> m_packets;
        std::size_t m_maxSize;
};

template <typename T> PacketBuffer<T>::PacketBuffer(std::size_t maxSize/*= 0*/)
    : m_maxSize { maxSize }
{}

template <typename T> bool PacketBuffer<T>::bufferPacket(const T& packet)
{
    if (m_maxSize > 0 && m_packets.size() == m_maxSize) {
        // No more space. Drop the packet.
        return false;
    }
    m_packets.push(packet);
    return true;
}

template <typename T> T PacketBuffer<T>::getNextPacket()
{
    if (m_packets.empty()) {
        throw std::out_of_range { "Buffer is empty" };
    }
    // Retrieve the head element
    T temp { m_packets.front() };
    // Pop the head element
    m_packets.pop();
    // Return the head element
    return temp;
}
\end{cpp}

这个类的一个实际应用场景需要使用多个线程,但如果没有显式的同步机制,当至少一个线程修改该对象时,就不能安全地从多个线程中使用标准库对象。C++提供了同步类,以允许对共享对象进行线程安全的访问。这将在第27章中进行介绍。这个例子的重点是在queue类上,所以这里是使用了PacketBuffer的单线程示例:

\begin{cpp}
class IPPacket final
{
    public:
        explicit IPPacket(int id) : m_id { id } {}
        int getID() const { return m_id; }
    private:
        int m_id;
};

int main()
{
    PacketBuffer<IPPacket> ipPackets { 3 };
    // Add 4 packets
    for (int i { 1 }; i <= 4; ++i) {
        if (!ipPackets.bufferPacket(IPPacket { i })) {
            println("Packet {} dropped (queue is full).", i);
        }
    }
    while (true) {
        try {
            IPPacket packet { ipPackets.getNextPacket() };
            println("Processing packet {}", packet.getID());
        } catch (const out_of_range&) {
            println("Queue is empty.");
            break;
        }
    }
}
\end{cpp}

代码的输出如下所示:

\begin{shell}
Packet 4 dropped (queue is full).
Processing packet 1
Processing packet 2
Processing packet 3
Queue is empty.
\end{shell}

\mySubsubsection{18.4.2.}{priority\_queue}

优先队列是一种保持其元素排序的队列。与严格的先进先出(FIFO)排序不同,优先队列在任何给定时间队列头部的元素是优先级最高的元素。这个元素可能是队列中最老的,也可能是最近的。如果两个元素具有相同的优先级,它们在队列中的相对顺序未定义。

priority\_queue 容器适配器也在 <queue> 中定义,其模板定义看起来像这样(略微简化):

\begin{cpp}
template <typename T, typename Container = vector<T>,
          typename Compare = less<T>>;
\end{cpp}

看起来并没有那么复杂,之前已经见过前两个参数:T 是存储在 priority\_queue 中的元素类型,Container 是优先队列适配的容器。priority\_queue 默认使用 vector,也适用于 deque,但不适用于list,因为 priority\_queue 需要对其元素进行随机访问。第三个参数Compare,比较复杂,将在第 19 章中了解更多关于 less 的信息。它是一个类模板,支持使用 operator< 比较两个类型为 T 的对象。所以优先队列中元素的优先级是根据 operator< 确定,可以自定义使用的比较,但这是第 19 章的内容。现在,只需确保存储在 priority\_queue 中的类型支持 operator<。当然,从 C++20 起,提供 operator<=> 就足够了(会自动提供 operator<)。

\begin{myNotic}{NOTE}
priority\_queue 的头部元素是具有“最高”优先级的元素;默认情况下,这是根据 operator< 确定的,使得“小于”其他元素的元素具有较低的优先级。
\end{myNotic}

\mySamllsection{priority\_queue的操作}

priority\_queue提供的操作比queue还要少。push()、emplace()和push\_range()(C++23)成员函数可以插入元素,pop()移除元素,而top()返回对头部元素的const引用。

\begin{myWarning}{WARNING}
即使在对非const对象调用时,top()也返回const引用,因为修改元素可能会改变其顺序,这是不允许的。priority\_queue不提供获取尾部元素的方法。
\end{myWarning}

\begin{myWarning}{WARNING}
pop()不返回弹出的元素。如果想保留一个副本,必须先用top()检索。
\end{myWarning}

与queue一样,priority\_queue支持size()、empty()和swap(),但不提供比较运算符。

\mySamllsection{priority\_queue 示例: 错误相关器}

系统上的单个故障通常会导致来自不同组件的多个错误,一个好的错误处理系统使用错误相关性来首先处理最重要的错误。可以使用优先级队列来编写一个简单的错误相关器。假设所有错误事件都编码了优先级,错误相关器可以简单地根据错误的优先级对错误事件进行排序,以便总是首先处理优先级最高的错误。以下是类定义:

\begin{cpp}
// Sample Error class with just a priority and a string error description.
export class Error final
{
    public:
        explicit Error(int priority, std::string errorString)
            : m_priority { priority }, m_errorString { std::move(errorString) } { }
        int getPriority() const { return m_priority; }
        const std::string& getErrorString() const { return m_errorString; }
        // Compare Errors according to their priority.
        auto operator<=>(const Error& rhs) const {
            return getPriority() <=> rhs.getPriority(); }
    private:
        int m_priority;
        std::string m_errorString;
};

// Stream insertion overload for Errors.
export std::ostream& operator<<(std::ostream& os, const Error& err)
{
    std::print(os, "{} (priority {})", err.getErrorString(), err.getPriority());
    return os;
}

// Simple ErrorCorrelator class that returns highest priority errors first.
export class ErrorCorrelator final
{
    public:
        // Add an error to be correlated.
        void addError(const Error& error) { m_errors.push(error); }
        // Retrieve the next error to be processed.
        [[nodiscard]] Error getError()
        {
            // If there are no more errors, throw an exception.
            if (m_errors.empty()) {
                throw std::out_of_range { "No more errors." };
            }
            // Save the top element.
            Error top { m_errors.top() };
            // Remove the top element.
            m_errors.pop();
            // Return the saved element.
            return top;
        }
    private:
        std::priority_queue<Error> m_errors;
};
\end{cpp}

以下是一个简单的测试,展示了如何使用ErrorCorrelator。实际应用场景需要使用多个线程,一个线程添加错误,而另一个线程处理它们。正如前面在队列示例中提到的,这需要显式同步,这将在第27章中介绍。

\begin{cpp}
ErrorCorrelator ec;
ec.addError(Error { 3, "Unable to read file" });
ec.addError(Error { 1, "Incorrect entry from user" });
ec.addError(Error { 10, "Unable to allocate memory!" });

while (true) {
    try {
        Error e { ec.getError() };
        cout << e << endl;
    } catch (const out_of_range&) {
        println("Finished processing errors");
        break;
    }
}
\end{cpp}

输出为:

\begin{shell}
Unable to allocate memory! (priority 10)
Unable to read file (priority 3)
Incorrect entry from user (priority 1)
Finished processing errors
\end{shell}

\mySubsubsection{18.4.3.}{stack}

stack与queue几乎相同,不同之处在于它提供的是先进后出(FILO)语义,也称为后进先出(LIFO),而不是FIFO。定义在<stack>中,模板定义如下所示:

\begin{cpp}
template <typename T, typename Container = deque<T>> class stack;
\end{cpp}

可以使用vector、list或deque作为stack的基础容器。

\mySamllsection{stack的操作}

与deque一样,C++23为stack添加了一个接受迭代器对的构造函数[begin, end),使用给定迭代器范围内的元素构建一个stack。也与deque类似,stack提供了push()、emplace()、pop()和push\_range()(C++23)。不同之处在于,push()和push\_range()将新元素添加到栈顶,将之前插入的所有元素“推下”,而pop()移除栈顶的元素,即最近插入的元素。top()成员函数在调用const对象时返回对栈顶元素的常量引用,在调用非常量对象时返回非常量引用。

\begin{myWarning}{WARNING}
pop()不会返回弹出的元素。如果想保留一个副本,必须首先使用top()检索。
\end{myWarning}

stack支持empty()、size()、swap()以及标准比较运算符。

\mySamllsection{stack示例:修改后的错误相关器}

可以将之前的ErrorCorrelator类重写,使其输出最近的错误,而不是优先级最高的错误。所需的唯一更改是将m\_errors从priority\_queue更改为stack。这样,错误以LIFO顺序,而不是以优先级顺序分发。因为push()、pop()、top()和empty()成员函数在priority\_queue和stack上都存在,所以成员函数的定义不需要更改。











