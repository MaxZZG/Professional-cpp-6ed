
还有一些C++语言和标准库的其他特性与容器有关,包括标准C风格的数组、字符串、流和bitset。

\mySubsubsection{18.6.1.}{标准C风格数组}

原始指针是真正的迭代器,因为它们支持所需的操作,所以可以通过使用指向其元素的指针,将标准C风格数组视为标准库容器。当然,标准C风格数组不提供size()、empty()、insert()和erase()等成员函数,所以它们不是真正的标准库容器,但确实通过指针变相的支持了迭代器。

例如,可以使用vector的insert()成员函数将标准C风格数组的所有元素复制到一个vector中,该函数接受容器的迭代器范围。这个insert()成员函数的原型为:

\begin{cpp}
template <typename InputIterator> iterator insert(const_iterator position,
    InputIterator first, InputIterator last);
\end{cpp}

如果想使用标准C风格int数组作为源,那么模板类型参数InputIterator就变成了int*。这是一个完整的例子:

\begin{cpp}
const size_t count { 10 };
int values[count]; // standard C-style array
// Initialize each element of the array to the value of its index.
for (int i { 0 }; i < count; ++i) { values[i] = i; }

// Insert the contents of the array at the end of a vector.
vector<int> vec;
vec.insert(end(vec), values, values + count);

// Print the contents of the vector.
println("{:n} ", vec);
\end{cpp}

注意,指向数组第一个元素的迭代器是第一个元素的地址,即values。数组名称本身解释为第一个元素的地址。指向最后一个元素的迭代器必须是最后一个元素之后的一个,所以它是第一个元素的地址加上count,或者values+count。

使用std::begin()或cbegin()获取静态分配的C风格数组第一个元素的迭代器,而不通过指针访问,以及使用std::end()或cend()获取此类数组最后一个元素之后的迭代器更容易。例如,前面例子中insert()的调用可以这样写:

\begin{cpp}
vec.insert(end(vec), cbegin(values), cend(values));
\end{cpp}

从C++23开始,可以用更优雅的方式——append\_range()——来写:

\begin{cpp}
vec.append_range(values);
\end{cpp}

\begin{myWarning}{WARNING}
std::begin()和end()等函数只对未通过指针访问的静态分配的C风格数组有效,并不适用于涉及指针或动态分配的C风格数组。
\end{myWarning}

\mySubsubsection{18.6.2.}{字符串}

可以将字符串视为一个字符序列容器,将C++字符串作为一个完整的顺序容器,它有begin()和end()成员函数,返回字符串中的迭代器;以及insert()、push\_back()、erase()、size()和empty()等成员函数,以及所有顺序容器的其他基本功能。它与vector非常相似,甚至提供了reserve()和capacity()等成员函数。

可以像使用vector一样使用字符串作为标准库容器。这是一个例子:

\begin{cpp}
string myString;
myString.insert(cend(myString), 'h');
myString.insert(cend(myString), 'e');
myString.push_back('l');
myString.push_back('l');
myString.push_back('o');

for (const auto& letter : myString) {
    print("{}", letter);
}
println("");

for (auto it { cbegin(myString) }; it != cend(myString); ++it) {
    print("{}", *it);
}
println("");
\end{cpp}

除了标准库顺序容器的成员函数外,字符串还提供了一组有用的成员函数和友元函数。字符串接口实际上是设计陷阱的好例子,这些设计陷阱已经在第6章中介绍过。字符串类在第2章中有详细的介绍。

\mySubsubsection{18.6.3.}{流}

输入和输出流在传统意义上不是容器,它们不存储元素,但它们可以视为元素序列,因此与标准库容器有一些相似之处。C++流不直接提供与标准库相关的成员函数,但标准库提供了特殊的迭代器istream\_iterator和ostream\_iterator,允许“遍历”输入和输出流。第17章解释过如何使用它们。

\mySubsubsection{18.6.4.}{bitset}

bitset是一个固定长度的位序列抽象。位只能表示两个值,1和0,可以称为开/关,真/假等。bitset也使用set和unset这样的术语,可以切换或翻转位从一个值到另一个值。

bitset不是真正的标准库容器:长度固定,不根据元素类型参数化,也不支持迭代。然而,它是一个有用的工具类,通常与容器一起提及,所以在这里简要介绍。要了解bitset操作的完整总结,请参阅标准库手册。

\CXXTwentythreeLogo{-40}{-50}

\begin{myNotic}{NOTE}
从C++23开始,bitset是一个constexpr类,可以在编译时使用。
\end{myNotic}

\mySamllsection{bitset基础知识}

bitset,定义在<bitset>中,根据存储的位数进行参数化。默认构造函数初始化所有bitset字段为0。一个替代构造函数,从一个0和1字符组成的字符串创建一个bitset。

可以使用set()、reset()和flip()成员函数调整单个位的值,并且可以使用重载的operator[]访问和设置单个字段。注意,operator[]在非const对象上返回一个代理对象,可以向其分配布尔值,调用flip(),或使用operator~进行补码操作。还可以使用test()成员函数访问单个字段。位使用零基索引进行访问,可以使用to\_string()将bitset转换为0和1字符组成的字符串。

这是一个小例子:

\begin{cpp}
bitset<10> myBitset;

myBitset.set(3);
myBitset.set(6);
myBitset[8] = true;
myBitset[9] = myBitset[3];

if (myBitset.test(3)) { println("Bit 3 is set!"); }
println("{}", myBitset.to_string());
\end{cpp}

输出为:

\begin{shell}
Bit 3 is set!
1101001000
\end{shell}

注意,输出字符串中最左边的字符是最高编号位。这与我们对二进制数表示法的直觉相符,其中最低位的位表示$2^0 = 1$,打印表示中是右边的位。

\mySamllsection{位运算符}

除了基本的位操作例程外,bitset还提供了所有位运算符的实现 \&, |, \^{}, ~, <{}<, >{}>, \&=, |=, \^{}=, <{}<=, and >{}>=。其行为就像在“真实”的位序列上一样:

\begin{cpp}
auto str1 { "0011001100" };
auto str2 { "0000111100" };
bitset<10> bitsOne { str1 };
bitset<10> bitsTwo { str2 };

auto bitsThree { bitsOne & bitsTwo };
println("{}", bitsThree.to_string());
bitsThree <<= 4;
println("{}", bitsThree.to_string());
\end{cpp}

程序的输出如下所示:

\begin{shell}
0000001100
0011000000
\end{shell}

\mySamllsection{bitset示例:有线电视频道}

bitset的一个用途是跟踪有线电视订阅者的频道。每个订阅者都可以有一个与他们的订阅关联的bitset,其中设置的位表示他们实际订阅的频道。这个系统还可以支持“套餐”频道,也用bitset表示,代表常见的订阅频道组合。

以下CableCompany类是这种模型的一个简单示例。使用两个map,都以字符串为键,映射到bitset。一个存储有线电视套餐,另一个存储订阅者信息。

\begin{cpp}
export class CableCompany final
{
    public:
    // Number of supported channels.
    static constexpr std::size_t NumChannels { 10 };
    // Adds package with the channels specified as a bitset to the database.
    void addPackage(const std::string& packageName,
        const std::bitset<NumChannels>& channels);
    // Adds package with the channels specified as a string to the database.
    void addPackage(const std::string& packageName,
        const std::string& channels);
    // Removes the specified package from the database.
    void removePackage(const std::string& packageName);
    // Retrieves the channels of a given package.
    // Throws out_of_range if the package name is invalid.
    const std::bitset<NumChannels>& getPackage(
        const std::string& packageName) const;
    // Adds customer to database with initial channels found in package.
    // Throws out_of_range if the package name is invalid.
    // Throws invalid_argument if the customer is already known.
    void newCustomer(const std::string& name, const std::string& package);
    // Adds customer to database with given initial channels.
    // Throws invalid_argument if the customer is already known.
    void newCustomer(const std::string& name,
        const std::bitset<NumChannels>& channels);
    // Adds the channel to the customer's profile.
    // Throws invalid_argument if the customer is unknown.
    void addChannel(const std::string& name, int channel);
    // Removes the channel from the customer's profile.
    // Throws invalid_argument if the customer is unknown.
    void removeChannel(const std::string& name, int channel);
    // Adds the specified package to the customer's profile.
    // Throws out_of_range if the package name is invalid.
    // Throws invalid_argument if the customer is unknown.
    void addPackageToCustomer(const std::string& name,
    const std::string& package);
    // Removes the specified customer from the database.
    void deleteCustomer(const std::string& name);
    // Retrieves the channels to which a customer subscribes.
    // Throws invalid_argument if the customer is unknown.
    const std::bitset<NumChannels>& getCustomerChannels(
        const std::string& name) const;
    private:
        // Retrieves the channels for a customer. (non-const)
        // Throws invalid_argument if the customer is unknown.
        std::bitset<NumChannels>& getCustomerChannelsHelper(
            const std::string& name);

        using MapType = std::map<std::string, std::bitset<NumChannels>>;
        MapType m_packages, m_customers;
};
\end{cpp}

以下是所有成员函数的实现:

\begin{cpp}
void CableCompany::addPackage(const string& packageName,
    const bitset<NumChannels>& channels)
{
    m_packages.emplace(packageName, channels);
}

void CableCompany::addPackage(const string& packageName, const string& channels)
{
    addPackage(packageName, bitset<NumChannels> { channels });
}

void CableCompany::removePackage(const string& packageName)
{
    m_packages.erase(packageName);
}

const bitset<CableCompany::NumChannels>& CableCompany::getPackage(
    const string& packageName) const
{
    // Get an iterator to the specified package.
    if (auto it { m_packages.find(packageName) }; it != end(m_packages)) {
        // Found package. Note that 'it' is an iterator to a name/bitset pair.
        // The bitset is the second field.
        return it->second;
    }
    throw out_of_range { format("Invalid package '{}'.", packageName) };
}

void CableCompany::newCustomer(const string& name, const string& package)
{
    // Get the channels for the given package.
    auto& packageChannels { getPackage(package) };
    // Create the account with the bitset representing that package.
    newCustomer(name, packageChannels);
}

void CableCompany::newCustomer(const string& name,
    const bitset<NumChannels>& channels)
{
    // Add customer to the customers map.
    if (auto [iter, success] { m_customers.emplace(name, channels) }; !success) {
        // Customer was already in the database. Nothing changed.
        throw invalid_argument { format("Duplicate customer '{}'.", name) };
    }
}

void CableCompany::addChannel(const string& name, int channel)
{
    // Get the current channels for the customer.
    auto& customerChannels { getCustomerChannelsHelper(name) };
    // We found the customer; set the channel.
    customerChannels.set(channel);
}

void CableCompany::removeChannel(const string& name, int channel)
{
    // Get the current channels for the customer.
    auto& customerChannels { getCustomerChannelsHelper(name) };
    // We found this customer; remove the channel.
    customerChannels.reset(channel);
}

void CableCompany::addPackageToCustomer(const string& name, const string& package)
{
    // Get the channels for the given package.
    auto& packageChannels { getPackage(package) };
    // Get the current channels for the customer.
    auto& customerChannels { getCustomerChannelsHelper(name) };
    // Or-in the package to the customer's existing channels.
    customerChannels |= packageChannels;
}

void CableCompany::deleteCustomer(const string& name)
{
    m_customers.erase(name);
}

const bitset<CableCompany::NumChannels>& CableCompany::getCustomerChannels(
    const string& name) const
{
    // Find an iterator to the customer.
    if (auto it { m_customers.find(name) }; it != end(m_customers)) {
        // Found customer. Note that 'it' is an iterator to a name/bitset pair.
        // The bitset is the second field.
        return it->second;
    }
    throw invalid_argument { format("Unknown customer '{}'.", name) };
}

bitset<CableCompany::NumChannels>& CableCompany::getCustomerChannelsHelper(
    const string& name)
{
    // Forward to const getCustomerChannels() to avoid code duplication.
    return const_cast<bitset<NumChannels>&>(getCustomerChannels(name));
}
\end{cpp}

最后,这里有一个简单的程序来演示如何使用CableCompany类:

\begin{cpp}
CableCompany myCC;
myCC.addPackage("basic", "1111000000");

myCC.addPackage("premium", "1111111111");
myCC.addPackage("sports", "0000100111");

myCC.newCustomer("Marc G.", "basic");
myCC.addPackageToCustomer("Marc G.", "sports");
println("{}", myCC.getCustomerChannels("Marc G.").to_string());

try { println("{}", myCC.getCustomerChannels("John").to_string()); }
catch (const exception& e) { println("Error: {}", e.what()); }
\end{cpp}

输出如下:

\begin{shell}
1111100111
Error: Unknown customer 'John'.
\end{shell}






