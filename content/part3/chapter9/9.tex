
已经了解了在 C++ 中编写类的所有语法,回顾第 5 章和第 6 章中的设计原则会有更多帮助。类是C++中的主要抽象单位,应该将抽象原则应用于类,尽可能地将接口与实现分离。具体来说,应该将所有数据成员设置为private,并提供 getter 和 setter 成员函数(可选)。SpreadsheetCell 类就是这样实现的:m\_value 是私有的,而 public set() 成员函数设置值,public getValue() 和 getString() 获取值。

\mySubsubsection{9.9.1.}{使用接口和实现类}

即使采取了上述措施并遵循最佳设计原则,C++ 语言在本质上不利于抽象原则。语法要求将public接口和private(或protected)的数据成员和成员函数组合在一个类定义中,从而向类客户端暴露了一些类的内部实现细节。这种做法的缺点是,如果向类添加新的非公共成员函数或数据成员,所有客户端都必须重新编译。在大的项目中,这可能会成为一个负担。

好消息是,可以使接口更加清晰,并隐藏所有实现细节,从而产生稳定的接口。坏消息是这需要一些编码工作。基本原则是为想要编写的每个类定义两个类:接口类和实现类。实现类与如果不采用这种方法编写的类完全相同。接口类呈现与实现类相同的公共成员函数,但它只有一个数据成员:指向实现类对象的指针。这称为 pimpl 模式、private实现模式或桥接模式。接口类的成员函数实现,只是调用实现类对象上的等效成员函数。结果是,无论实现如何变化,都不会影响公共接口类。这减少了重新编译的需要。如果实现(仅实现)发生变化,所有使用接口类的客户端都不需要重新编译。这种模式只有在单数据成员是指向实现类的指针时才有效。如果它是值数据成员,则在实现类定义发生变化时,客户端必须重新编译。

要使用这种方法,定义以下公共接口类,称为 Spreadsheet。

\begin{cpp}
export module spreadsheet;

export import spreadsheet_cell;
import std;

export class Spreadsheet
{
    public:
        explicit Spreadsheet(
            std::size_t width = MaxWidth, std::size_t height = MaxHeight);
        Spreadsheet(const Spreadsheet& src);
        Spreadsheet(Spreadsheet&&) noexcept;
        ~Spreadsheet();

        Spreadsheet& operator=(const Spreadsheet& rhs);
        Spreadsheet& operator=(Spreadsheet&&) noexcept;

        void setCellAt(std::size_t x, std::size_t y, const SpreadsheetCell& cell);
        SpreadsheetCell& getCellAt(std::size_t x, std::size_t y);
        const SpreadsheetCell& getCellAt(std::size_t x, std::size_t y) const;

        std::size_t getId() const;

        static constexpr std::size_t MaxHeight { 100 };
        static constexpr std::size_t MaxWidth { 100 };

        void swap(Spreadsheet& other) noexcept;

    private:
        class Impl;
        std::unique_ptr<Impl> m_impl;
};
export void swap(Spreadsheet& first, Spreadsheet& second) noexcept;
\end{cpp}

这个实现类 Impl 是一个私有嵌套类,因为除了 Spreadsheet 类之外,没有人需要知道这个实现类的信息。Spreadsheet 类现在只包含一个数据成员:一个指向 Impl 实例的指针。公共成员函数与旧的 Spreadsheet 类完全相同。

嵌套的 Spreadsheet::Impl 类在 spreadsheet 模块的实现文件中定义,应该对客户端隐藏,因此 Impl 类没有导出。Spreadsheet.cpp 模块的实现如下所示:

\begin{cpp}
module spreadsheet;
import std;
using namespace std;

// Spreadsheet::Impl class definition.
class Spreadsheet::Impl
{
    public:
        Impl(size_t width, size_t height);
        // Remainder omitted for brevity.
};
\end{cpp}

Impl 类几乎与原始的 Spreadsheet 类有相同的接口,可以在下载的源代码存档中找到完整的定义。对于成员函数的实现,需要记住 Impl 是一个嵌套类,需要指定作用域为 Spreadsheet::Impl。对于构造函数,就变成了 Spreadsheet::Impl::Impl(…):

\begin{cpp}
// Spreadsheet::Impl member function definitions.
Spreadsheet::Impl::Impl(size_t width, size_t height)
    : m_id { ms_counter++ }
    , m_width { min(width, Spreadsheet::MaxWidth) }
    , m_height { min(height, Spreadsheet::MaxHeight) }
{
    m_cells = new SpreadsheetCell*[m_width];
    for (size_t i { 0 }; i < m_width; ++i) {
        m_cells[i] = new SpreadsheetCell[m_height];
    }
}
// Other member function definitions omitted for brevity.
\end{cpp}

Spreadsheet 类有一个unique\_ptr指向到 Impl 实例,Spreadsheet 类需要有一个用户声明的析构函数。由于这个析构函数中无事可做,所以可以在实现文件中显示默认:

\begin{cpp}
Spreadsheet::~Spreadsheet() = default;
\end{cpp}

实际上,它必须在实现文件中默认,而不是直接在类定义中。原因是因为 Impl 类只在 Spreadsheet 类定义中进行了声明,因为编译器知道会有一个 Spreadsheet::Impl 类,但还不知道其定义。因为编译器会尝试使用还尚未定义的 Impl 类的析构函数,所以不能在类定义中默认析构函数。对于其他成员函数也是如此,例如:移动构造函数和移动赋值操作符。

Spreadsheet 成员函数的实现,如 setCellAt() 和 getCellAt(),只是将请求传递给底层 Impl 对象:

\begin{cpp}
void Spreadsheet::setCellAt(size_t x, size_t y, const SpreadsheetCell& cell)
{
    m_impl->setCellAt(x, y, cell);
}

const SpreadsheetCell& Spreadsheet::getCellAt(size_t x, size_t y) const
{
    return m_impl->getCellAt(x, y);
}

SpreadsheetCell& Spreadsheet::getCellAt(size_t x, size_t y)
{
    return m_impl->getCellAt(x, y);
}
\end{cpp}

Spreadsheet 的构造函数必须构造新的 Impl 来进行工作:

\begin{cpp}
Spreadsheet::Spreadsheet(size_t width, size_t height)
{
    : m_impl { make_unique<Impl>(width, height) }
}
}

Spreadsheet::Spreadsheet(const Spreadsheet& src)
: m_impl { make_unique<Impl>(*src.m_impl) }
{}
\end{cpp}

复制构造函数看起来有点奇怪,需要复制源 Spreadsheet 底层的 Impl。复制构造函数需要引用 Impl,而不是指针,所以必须解引用 m\_impl 指针才能得到对象本身。

Spreadsheet 赋值操作符也必须将赋值传递给底层的 Impl:

\begin{cpp}
Spreadsheet& Spreadsheet::operator=(const Spreadsheet& rhs)
{
    *m_impl = *rhs.m_impl;
    return *this;
}
\end{cpp}

赋值操作符的第一行看起来有点奇怪。Spreadsheet 赋值操作符需要将调用传递给 Impl 赋值操作符,这个操作符只在复制对象时运行。通过解引用 m\_impl 指针,强制执行直接对象赋值,这会调用 Impl 的赋值操作符。

swap() 成员函数只是交换单个数据成员:

\begin{cpp}
void Spreadsheet::swap(Spreadsheet& other) noexcept
{
    std::swap(m_impl, other.m_impl);
}
\end{cpp}

这种将接口与实现分离的技术很强大。虽然一开始有点笨拙,但习惯了后,会发现它会自然地工作。大多数工作场所环境中,这不是常见的做法,所以可能会发现同事不愿意尝试它。支持它最有力的论据,不是将接口分离的美学原因,而是如果类实现发生变化时,可以加快构建时间。当一个类没有使用 pimpl 模式时,对其实现细节的更改可能会触发长时间的重构建。向类定义中添加新的数据成员。使用 pimpl 模式,可以随意修改实现类的定义,只要公共接口类保持不变,就不会触发长时间的重构建。

\begin{myNotic}{NOTE}
具有稳定接口的类可以减少构建时间。
\end{myNotic}

将实现与接口分离的另一种替代方法是使用抽象接口——一个只有纯虚成员函数的接口——然后有一个实现类来实现该接口。不过,这是下一章的内容了。












