本章与第 8 章一起提供了编写稳健、设计良好的类,并介绍了有效使用对象所需的全部工具。

您发现对象中的动态内存分配带来了新的挑战:需要实现一个析构函数、复制构造函数、复制赋值运算符、移动构造函数和移动赋值运算符,它们正确地复制、移动和释放内存。学会了如何通过显式删除复制构造函数和赋值运算符,来避免按值传递和赋值。发现了使用复制和交换模式来实现复制赋值运算符,以及使用移动和交换模式来实现移动赋值运算符,同时还学习了零规则。

阅读了更多有关不同类型的数据成员,包括静态、常量、常量引用和可变成员。还学习了静态、内联和常量成员函数、成员函数重载和默认参数。本章还描述了嵌套类定义,以及友元类、函数和成员函数。

还遇到了运算符重载,并学会了如何重载算术和比较运算符,既作为全局函数也作为类成员函数。还发现了如何使用三向比较运算符,可使类添加比较支持变得如此简单。

最后,学习了如何通过提供单独的接口和实现类来实现极端的抽象。

现在,已经掌握了面向对象编程的语言,是时候解决继承问题了,这将在下一章中进行介绍。