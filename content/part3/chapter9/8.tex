
若希望对对象执行操作,例如添加、比较或将它们流式传输到文件或从文件中读取。可以对电子表格执行算术操作(例如对整行单元格求和)时,电子表格才是有用的。所有这些都可以通过重载操作符来实现。

许多人发现操作符重载的语法复杂且令人困惑,但它本来是为了降低代码的复杂度。这不是为编写类的开发者简化,而是为使用类的使用者简化。重点是将新类应尽可能与内置类型(如 int 和 double)相似:使用 + 添加对象比应该调用的成员函数 add() 还是 sum() 要简单。

\begin{myNotic}{NOTE}
提供操作符重载可视为对客户端的服务。
\end{myNotic}

此时,可能想知道究竟可以重载哪些操作符——(几乎)所有操作符——甚至有一些从未听说过的。本章只是触及了表面:本章前面介绍了赋值操作符,本节将介绍基本的算术操作符、简化算术操作符和比较操作符,重载流插入和提取操作符也很实用。可以使用操作符重载做一些复杂的操作,标准库广泛使用了操作符重载。第 15 章解释了如何以及何时重载其余的操作符,第 16 到 24 章将介绍标准库。

\mySubsubsection{9.8.1.}{示例:为 SpreadsheetCell 实现 加法操作符}

SpreadsheetCell 对象应该能够以真正的面向对象方式,将自己的内容与其他 SpreadsheetCell 对象中的内容相加。两个单元格的内容进行加法时,其结果会生成到第三个(新的)单元格中。对于 SpreadsheetCell 来说,加法的含义是,单元格内的数值相加。

\mySamllsection{第一次尝试:add 成员函数}

可以为 SpreadsheetCell 类声明和定义一个 add() 成员函数:

\begin{cpp}
export class SpreadsheetCell
{
    public:
        SpreadsheetCell add(const SpreadsheetCell& cell) const;
        // Omitted for brevity
};
\end{cpp}

这个成员函数将两个单元格的内容相加,返回一个新的第三个单元格,其值是前两个单元格的和。该函数声明为 const,并接受一个对 const SpreadsheetCell 的引用,所以 add() 不会改变源单元格。下面是实现:

\begin{cpp}
SpreadsheetCell SpreadsheetCell::add(const SpreadsheetCell& cell) const
{
    return SpreadsheetCell { getValue() + cell.getValue() };
}
\end{cpp}

可以这样使用 add() 成员函数:

\begin{cpp}
SpreadsheetCell myCell { 4 }, anotherCell { 5 };
SpreadsheetCell aThirdCell { myCell.add(anotherCell) };
auto aFourthCell { aThirdCell.add(anotherCell) };
\end{cpp}

这样可行,但略显笨拙。

\mySamllsection{第二次尝试:将 operator+ 作为成员函数重载}

能够使用加号将两个单元格相加,就像将两个 int 或两个 double 相加一样:

\begin{cpp}
SpreadsheetCell myCell { 4 }, anotherCell { 5 };
SpreadsheetCell aThirdCell { myCell + anotherCell };
auto aFourthCell { aThirdCell + anotherCell };
\end{cpp}

C++ 允许为类编写自己的加法操作符,使其正确地为类工作。为此,需要一个名为 operator+ 的成员函数:

\begin{cpp}
export class SpreadsheetCell
{
    public:
        SpreadsheetCell operator+(const SpreadsheetCell& cell) const;
        // Omitted for brevity
};
\end{cpp}

\begin{myNotic}{NOTE}
可以在 operator 和加号之间插入空格。可以写 operator +,而非 operator+。本书采用了不带空格的方式。
\end{myNotic}

重载的 operator+ 成员函数定义与 add() 成员函数的实现相同:

\begin{cpp}
SpreadsheetCell SpreadsheetCell::operator+(const SpreadsheetCell& cell) const
{
    return SpreadsheetCell { getValue() + cell.getValue() };
}
\end{cpp}

现在,可以使用加号操作符将两个单元格相加,如前所示。

这种语法需要一些时间来适应。不要过于担心奇怪的成员函数名 operator+ ——只是一个名称,就像 foo 或 add 一样。要理解其余的语法,有助于了解真正发生的事情。当 C++ 编译器解析程序并遇到操作符时,例如 +、-、= 或 <{}<,尝试找到具有相应参数的函数或成员函数,分别名为 operator+、operator-、operator= 或 operator<{}<。当编译器看到以下代码时,会尝试在 SpreadsheetCell 类中找到名为 operator+ 的成员函数,该函数接受另一个 SpreadsheetCell 作为参数(或者,如本章后面所讨论的,接受两个 SpreadsheetCell 参数的全局函数):

\begin{cpp}
SpreadsheetCell aThirdCell { myCell + anotherCell };
\end{cpp}

如果 SpreadsheetCell 类包含这样的 operator+ 成员函数,前面的行将转换为:

\begin{cpp}
SpreadsheetCell aThirdCell { myCell.operator+(anotherCell) };
\end{cpp}

operator+ 接受的参数类型不必与编写它的类相同,可以为 SpreadsheetCells 编写一个接受 Spreadsheet 的 operator+ 以加和 SpreadsheetCell的值。这对开发者来说可能没有意义,但编译器会允许。下一节将给出一个接受 double 值的 operator+ 示例。

可以为 operator+ 分配想要的返回类型,但应该遵循最小惊讶原则;也就是说,operator+ 的返回类型通常应该为用户所期望。

\mySamllsection{隐式转换}

前面展示的 operator+,不仅可以添加两个单元格,还可以将单元格添加到 string\_view、double 或 int!这里有一些例子:

\begin{cpp}
SpreadsheetCell myCell { 4 }, aThirdCell;
string str { "hello" };
aThirdCell = myCell + string_view{ str };
aThirdCell = myCell + 5.6;
aThirdCell = myCell + 4;
\end{cpp}

这段代码能够工作是因为,编译器为了找到一个合适的 operator+,不仅仅查找具有指定确切类型的,还会尝试找到合适的类型转换。SpreadsheetCell 类有转换构造函数(在第 8 章中介绍),可以将 double 或 string\_view 转换为 SpreadsheetCell。前面的示例中,编译器看到 SpreadsheetCell 尝试与 double 类型变量求和时,会找到接受 double 的 SpreadsheetCell 构造函数,并构造一个临时 SpreadsheetCell 对象传递给 operator+。当编译器看到尝试将 SpreadsheetCell 添加到 string\_view 的行时,也会接受 string\_view 的 SpreadsheetCell 构造函数,创建临时 SpreadsheetCell传递给 operator+。

因为必须创建临时对象,所以使用隐式转换的构造函数可能效率不高。为了避免在与 double 进行加和时进行隐式构造,可以创建第二个 operator+:

\begin{cpp}
SpreadsheetCell SpreadsheetCell::operator+(double rhs) const
{
    return SpreadsheetCell { getValue() + rhs };
}
\end{cpp}

\mySamllsection{第三次尝试:全局 operator+}

隐式转换允许使用 operator+ 成员函数将 SpreadsheetCell 对象的值与 int 和 double 相加。如以下代码所示,操作符不可交换:

\begin{cpp}
aThirdCell = myCell + 5.6; // Works fine.
aThirdCell = myCell + 4; // Works fine.
aThirdCell = 5.6 + myCell; // FAILS TO COMPILE!
aThirdCell = 4 + myCell; // FAILS TO COMPILE!
\end{cpp}

当 SpreadsheetCell 对象在操作符的左侧时,隐式转换工作正常,但当在右侧时则不工作。加法应该是交换的呀!这里的问题是 operator+ 成员函数必须在 SpreadsheetCell 对象上使用,并且该对象必须在操作符的左侧,这就是 C++ 语言的定义方式,这里无法使用operator+成员函数。

然而,如果用一个不绑定到特定对象的全局 operator+ 函数,替换类内的 operator+ 成员函数即可。该函数如下所示:

\begin{cpp}
SpreadsheetCell operator+(const SpreadsheetCell& lhs,
    const SpreadsheetCell& rhs)
{
    return SpreadsheetCell { lhs.getValue() + rhs.getValue() };
}
\end{cpp}

需要在模块接口文件中声明此操作符并将其导出:

\begin{cpp}
export class SpreadsheetCell { /* Omitted for brevity */ };

export SpreadsheetCell operator+(const SpreadsheetCell& lhs,
    const SpreadsheetCell& rhs);
\end{cpp}

现在,四个加法操作都能按预期工作。

\begin{cpp}
aThirdCell = myCell + 5.6; // Works fine.
aThirdCell = myCell + 4; // Works fine.
aThirdCell = 5.6 + myCell; // Works fine.
aThirdCell = 4 + myCell; // Works fine.
\end{cpp}

想知道如果编写以下代码会发生什么?

\begin{cpp}
aThirdCell = 4.5 + 5.5;
\end{cpp}

它会编译并运行,但它没有调用 operator+,而进行了正常的 double 加法,4.5 和 5.5 相加:

\begin{cpp}
aThirdCell = 10;
\end{cpp}

为了使这个赋值方式工作,右侧应该有一个 SpreadsheetCell 对象。编译器将找到一个非显式的用户定义构造函数,该构造函数接受 double,使用此构造函数隐式地将 double 值转换为临时 SpreadsheetCell 对象,然后调用赋值操作符。

\mySubsubsection{9.8.2.}{重载算术操作符}

现在已经了解了如何编写 operator+,其余的基本算术操作符就简单了。这里是 +、-、* 和 / 的声明,要用 +、-、* 和 / 替换 ,得到四个函数,也可以重载 \%,但对于存储在 SpreadsheetCells 中的 double 值来说,没有意义。

\begin{cpp}
export class SpreadsheetCell { /* Omitted for brevity */ };

export SpreadsheetCell operator<op>(const SpreadsheetCell& lhs,
    const SpreadsheetCell& rhs);
\end{cpp}

operator- 和 operator* 的实现与 operator+ 的实现类似,这里先忽略。对于 operator/,唯一棘手的是除零。这个实现在检测到除零时抛出异常:

\begin{cpp}
SpreadsheetCell operator/(const SpreadsheetCell& lhs,
    const SpreadsheetCell& rhs)
{
    if (rhs.getValue() == 0) {
        throw invalid_argument { "Divide by zero." };
    }
    return SpreadsheetCell { lhs.getValue() / rhs.getValue() };
}
\end{cpp}

C++ 不要求在 operator* 中实现乘法,在 operator/ 中实现除法。可以在 operator/ 中实现乘法,在 operator+ 中实现除法。然而,这样做会非常令人困惑,而且没有理由这样做。

\begin{myNotic}{NOTE}
不能改变C++中操作符的优先级。例如,* 和 / 总是在 + 和 - 之前计算。用户定义的操作符能做的是,在确定了运算的优先级之后指定实现。C++ 也不允许发明新的操作符符号或改变操作符的参数数量。操作符重载在第 15 章中详细介绍。
\end{myNotic}

\mySamllsection{重载算术简写操作符}

除了基本算术操作符之外,C++ 还提供了简写操作符,如 += 和 -=。有些读者可能认为,已经为类编写 operator+ 了,就会自动生成 operator+=,但事实并非如此。必须显式地重载算术简写操作符,这些操作符与基本算术操作符的区别在于,改变操作符左侧的对象,而不是创建一个新对象。另一个不同之处在于,像赋值操作符一样,会产生一个引用,为修改对象后的结果。

算术简写操作符始终需要在类对象的左侧,所以应该可写为成员函数,而不是全局函数。这里是 SpreadsheetCell 类的声明:

\begin{cpp}
export class SpreadsheetCell
{
    public:
        SpreadsheetCell& operator+=(const SpreadsheetCell& rhs);
        SpreadsheetCell& operator-=(const SpreadsheetCell& rhs);
        SpreadsheetCell& operator*=(const SpreadsheetCell& rhs);
        SpreadsheetCell& operator/=(const SpreadsheetCell& rhs);
        // Omitted for brevity
};
\end{cpp}

这是 operator+= 的实现,其他实现类似。

\begin{cpp}
SpreadsheetCell& SpreadsheetCell::operator+=(const SpreadsheetCell& rhs)
{
    set(getValue() + rhs.getValue());
    return *this;
}
\end{cpp}

算术简写操作符是基本算术操作符和赋值操作符的组合。有了前面的定义,现在可以有如下代码:

\begin{cpp}
SpreadsheetCell myCell { 4 }, aThirdCell { 2 };
aThirdCell -= myCell;
aThirdCell += 5.4;
\end{cpp}

但不能这样写:

\begin{cpp}
5.4 += aThirdCell;
\end{cpp}

\begin{myNotic}{NOTE}
当有某个操作符的正常版本和简写版本时,建议将正常版本实现为简写版本的子函数,以避免代码重复。
\end{myNotic}

一个栗子:

\begin{cpp}
SpreadsheetCell operator+(const SpreadsheetCell& lhs, const SpreadsheetCell& rhs)
{
    auto result { lhs }; // Local copy
    result += rhs; // Forward to +=()
    return result;
}
\end{cpp}

\mySubsubsection{9.8.3.}{重载比较操作符}

比较操作符 >、<、<=、>=、== 和 != 是为类定义的另一组有用的操作符。C++20 标准为这些操作符带来了很多变化,并添加了三向比较操作符,也称为空格船操作符,<=>。为了更了解自从 C++20 以来发生了什么变化,让我们首先看看在 C++20 之前需要做什么,若编译器还不支持三向比较操作符,开发者需要做什么。

\mySamllsection{C++20 前的重载比较操作符}

像基本算术操作符一样,六个预 C++20 比较操作符应该是全局函数,这样就可以在操作符的左、右两侧参数上使用隐式转换。比较操作符都返回一个 bool。当然,可以改变返回类型,但不推荐这样做。

以下是声明,需要用 ==、<、>、!=、<= 和 >= 替换 ,得到六个函数:

\begin{cpp}
class SpreadsheetCell { /* Omitted for brevity */ };

bool operator<op>(const SpreadsheetCell& lhs, const SpreadsheetCell& rhs);
\end{cpp}

这是 operator== 的定义,其他定义类似。

\begin{cpp}
bool operator==(const SpreadsheetCell& lhs, const SpreadsheetCell& rhs)
{
    return (lhs.getValue() == rhs.getValue());
}
\end{cpp}

\begin{myNotic}{NOTE}
这些重载的比较操作符是在比较 double 值。大多数时候,对浮点值进行相等性或不等性测试不是一个好主意。应该使用一个接近零的测试,但这超出了本书的范围。
\end{myNotic}

具有更多数据成员的类中,比较每个数据成员可能会很痛苦。当实现了 == 和 <,就可以在那些两个的基础上实现其余的比较操作符。下面是一个使用 operator< 定义 operator>= 的示例:

\begin{cpp}
bool operator>=(const SpreadsheetCell& lhs, const SpreadsheetCell& rhs)
{
    return !(lhs < rhs);
}
\end{cpp}

可以使用这些操作符来比较 SpreadsheetCell 与其他 SpreadsheetCell,以及与 double 和 int:

\begin{cpp}
if (myCell > aThirdCell || myCell < 10) {
    cout << myCell.getValue() << endl;
}
\end{cpp}

需要编写六个单独的函数来支持六个比较操作符,而且这只是用来比较两个 SpreadsheetCells 之间。当前实现的六个比较函数可以用来比较 SpreadsheetCell 与 double,因为 double 参数会隐式转换为 SpreadsheetCell。就像之前对 operator+ 一样,可以通过实现显式函数来比较 double 来避免。对于每个操作符<op>,需要以下三个重载:

\begin{cpp}
bool operator<op>(const SpreadsheetCell& lhs, const SpreadsheetCell& rhs);
bool operator<op>(double lhs, const SpreadsheetCell& rhs);
bool operator<op>(const SpreadsheetCell& lhs, double rhs);
\end{cpp}

如果想支持所有的比较操作符,需要写很多重复代码!

\mySamllsection{ C++20 起的重载比较操作符}

现在转换一下话题,看看 C++20 及以后版本带来了什么。从 C++20 开始,为类添加比较操作符的支持变得更加简单。首先,推荐将 operator== 作为类的成员函数而不是全局函数来实现。同时,添加 [[nodiscard]] 属性也是一个好主意,这样操作符的结果就不能忽略:

\begin{cpp}
[[nodiscard]] bool operator==(const SpreadsheetCell& rhs) const;
\end{cpp}

从 C++20 开始,这个 operator== 重载能让以下比较工作:

\begin{cpp}
if (myCell == 10) { println("myCell == 10"); }
if (10 == myCell) { println("10 == myCell"); }
\end{cpp}

现在, 10 == myCell 这样的表达式在编译器中会重写为 myCell == 10,其中可以调用 operator== 成员函数。通过实现 operator==,编译器自动添加了对 != 的支持;使用 != 的表达式会重写为使用 == 的表达式。

接下来,为了实现对完整比较操作符集的支持,只需要实现一个额外的重载操作符,即 operator<=>。当类有了 operator== 和 <=> 的重载,编译器自动提供六个比较操作符!对于 SpreadsheetCell 类,operator<=> 为:

\begin{cpp}
[[nodiscard]] std::partial_ordering operator<=>(const SpreadsheetCell& rhs) const;
\end{cpp}

\begin{myNotic}{NOTE}
编译器不会将 == 或 != 比较重写为 <=>。这是为了避免性能问题,operator== 的显式实现通常比使用 <=> 更有效率。例如, std::string的 operator== 和 != 实现,首先比较两个字符串的长度。如果长度不同,operator== 和 != 可以立即返回 false 和 true, 而不必检查每个字符。但operator<=> 总是必须比较每个字符,直到找到不匹配的两个字符。
\end{myNotic}

SpreadsheetCell 类中存储的值是一个 double。回想一下第 1 章,浮点类型只有偏序,所以返回的类型是 std::partial\_ordering。实现很简单:

\begin{cpp}
partial_ordering SpreadsheetCell::operator<=>(const SpreadsheetCell& rhs) const
{
    return getValue() <=> rhs.getValue();
}
\end{cpp}

通过实现 operator<=>,编译器自动为提供 >、<、<= 和 >= ,通过重写使用这些操作符的表达式来使用 <=> 。例如,myCell < aThirdCell 会自动重写为 std::is\_lt(myCell <=> aThirdCell) ,其中 is\_lt() 是一个比较函数。

因此,只实现 operator== 和 operator<=>,SpreadsheetCell 类就支持完整的比较操作符集:

\begin{cpp}
if (myCell < aThirdCell) { println("myCell < aThirdCell"); }
if (aThirdCell < myCell) { println("aThirdCell < myCell"); }

if (myCell <= aThirdCell) { println("myCell <= aThirdCell"); }
if (aThirdCell <= myCell) { println("aThirdCell <= myCell"); }

if (myCell > aThirdCell) { println("myCell > aThirdCell"); }
if (aThirdCell > myCell) { println("aThirdCell > myCell"); }

if (myCell >= aThirdCell) { println("myCell >= aThirdCell"); }
if (aThirdCell >= myCell) { println("aThirdCell >= myCell"); }

if (myCell == aThirdCell) { println("myCell == aThirdCell"); }
if (aThirdCell == myCell) { println("aThirdCell == myCell"); }

if (myCell != aThirdCell) { println("myCell != aThirdCell"); }
if (aThirdCell != myCell) { println("aThirdCell != myCell"); }
\end{cpp}

由于 SpreadsheetCell 类支持从 double 到 SpreadsheetCell 的隐式转换,因此以下比较也支持:

\begin{cpp}
if (myCell < 10) { println("myCell < 10"); }
if (10 < myCell) { println("10 < myCell"); }
if (10 != myCell) { println("10 != myCell"); }
\end{cpp}

与比较两个 SpreadsheetCell 对象一样,编译器会重写这样的表达式以使用 operator== 和 <=>,并选择性的交换参数顺序。例如,10 < myCell 首先重写为 is\_lt(10 <=> myCell),这个表达式不会工作,因为只有一种 <=> 作为成员的重载,所以左操作数必须是 SpreadsheetCell。注意这一点后,编译器尝试将表达式重写为 is\_gt(myCell <=> 10),这个表达式就可以工作了。

与之前一样,如果想要避免隐式转换带来的性能影响,可以为 double 提供特定的重载。从 C++20 开始,这就很简单了。只需要提供以下两个重载操作符作为成员函数即可:

\begin{cpp}
[[nodiscard]] bool operator==(double rhs) const;
[[nodiscard]] std::partial_ordering operator<=>(double rhs) const;
\end{cpp}

实现如下:

\begin{cpp}
bool SpreadsheetCell::operator==(double rhs) const
{
    return getValue() == rhs;
}
partial_ordering SpreadsheetCell::operator<=>(double rhs) const
{
    return getValue() <=> rhs;
}
\end{cpp}

\mySamllsection{编译器生成的比较操作符}

观察 SpreadsheetCell 的 operator== 和 <=> 实现,只是比较了所有数据成员,C++20(以及后续版本)可以进一步减少所需的代码量。如可以显式默认复制构造函数一样,operator== 和 <=> 也可以显式默认,则编译器会生成它们,并通过比较类定义中声明的每个数据成员的顺序来实现,这称为成员的字典序比较。

此外,如果只是显式默认 operator<=>,编译器还会自动包含一个默认的 operator==。对于没有显式为 double 实现 operator== 和 <=> 的 SpreadsheetCell (我将在本节的稍后部分回到这一点),可以简单地用下面的代码来为 SpreadsheetCell 添加对六个比较操作符的支持:

\begin{cpp}
[[nodiscard]] std::partial_ordering operator<=>(
    const SpreadsheetCell&) const = default;
\end{cpp}

可以使用 auto 作为 operator<=> 的返回类型,编译器会根据数据成员的 <=> 操作符的返回类型推断返回类型:

\begin{cpp}
[[nodiscard]] auto operator<=>(const SpreadsheetCell&) const = default;
\end{cpp}

如果类中不是所有数据成员都支持 operator<=>,默认的 operator<=> 会退回为使用 < 和 == 来比较数据成员。返回类型推断不会起作用,需要显式指定返回类型为 strong\_ordering、partial\_ordering 或 weak\_ordering。如果数据成员没有可用的 < 和 ==,默认的 operator<=> 会隐式删除。

为了使编译器能够生成默认的 <=> 操作符,类中所有数据成员要么支持 operator<=>,这种情况下返回类型可以是 auto;要么支持 < 和 ==,但在这种情况下返回类型不能是 auto。由于 SpreadsheetCell 只有一个 double 数据成员,编译器推断返回类型为 partial\_ordering。

本节的开始,这个显式默认的 operator<=> 对于没有显式为 double 实现 operator== 和 <=> 的 SpreadsheetCell 版本有效。如果确实添加了那些显式的 double 版本,将添加一个声明 operator==(double)。编译器将不会自动生成 operator==(const SpreadsheetCell\&),所以需要显式默认:

\begin{cpp}
export class SpreadsheetCell
{
    public:
        // Omitted for brevity
        [[nodiscard]] auto operator<=>(const SpreadsheetCell&) const = default;
        [[nodiscard]] bool operator==(const SpreadsheetCell&) const = default;

        [[nodiscard]] bool operator==(double rhs) const;
        [[nodiscard]] std::partial_ordering operator<=>(double rhs) const;
        // Omitted for brevity
};
\end{cpp}

我建议显式地为类默认生成<=>,而不是自己实现。让编译器生成,将保持与新添加或修改的数据成员同步。如果自己实现 <=>,那每当添加数据成员或更改现有数据成员时,都需要更新<=> 的实现。对于 == 也是如此。

只有在 == 和 <=> 作为参数是定义该操作符的类类型的引用为常量时,才能显式地默认生成,以下方法将不起作用:

\begin{cpp}
[[nodiscard]] auto operator<=>(double) const = default; // Does not work!
\end{cpp}

\begin{myNotic}{NOTE}
要在 C++20 或更新版本中为类添加对所有六个比较操作符的支持:

\begin{itemize}
\item
如果默认的<=> 适用于类,只需要通过成员函数显式地默认生成<=> 即可。某些情况下,可能还需要显式地默认生成 ==。

\item
否则,只需重载并实现 == 和 <=> 成员函数即可。
\end{itemize}

从而,不需要手动实现其他比较操作符。
\end{myNotic}
