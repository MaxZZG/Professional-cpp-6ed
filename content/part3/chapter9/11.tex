
通过解决下面的练习,可以练习本章讨论的内容。所有练习的解决方案都可以在本书的网站\url{www.wiley.com/go/proc++6e}下载到源码。若在练习中卡住了,可以考虑先重读本章的部分内容,试着自己找到答案,再在从网站上寻找解决方案。

\begin{itemize}
\item
\textbf{练习 9-1}: 请使用在练习 8-3 中实现的 Person 类,并尽可能以最佳方式传递字符串。同时,为该类添加移动构造函数和移动赋值运算符。两个成员函数中,编写一条消息到控制台,以便了解它们何时调用。实现其他的成员函数,实现移动成员函数并改进练习 8-3 中其他成员函数的实现,以避免代码重复。修改 main() 函数并测试新的成员函数。

\item
\textbf{练习 9-2}: 使用练习 8-4 中的 Person 类。与练习 9-1 类似,以最佳方式传递字符串,并为比较两个 Person 对象添加完整的六个比较运算符支持。尽量以最少的代码实现此支持。在 main() 中执行各种比较测试,以验证实现。Person 是如何排序的?排序是基于姓氏、姓氏还是基于姓氏和名字的组合?

\item
\textbf{练习 9-3}:  C++20 之前,为六个比较运算符添加支持需要编写更多的代码。从练习 9-2 中的 Person 类开始,移除 operator<=>,并添加必要的代码,以不使用 <=> 添加所有比较运算符,以便比较两个 Person 对象。执行与练习 9-2 相同的测试集。

\item
\textbf{练习 9-4}: 练习编写稳定的接口。使用练习 8-4 中的 Person 类,将其拆分为一个稳定的公共接口类和一个单独的实现类。

\item
\textbf{练习 9-5}: 从练习 9-2 的解决方案开始,在参数为右值时,优化 getFirstName(), getLastName()和 getInitials() 成员函数。
\end{itemize}






