
C++ 中,有时在编写程序之前不知道需要多少内存。正如在第 7 章所言,解决方案是在程序执行期间动态分配所需的空间,类也不例外。编写类时,不知道一个对象需要多少内存。这时,对象应该动态分配内存。对象中的动态分配内存带来了几个挑战,包括释放内存、处理对象复制和处理对象赋值。

\mySubsubsection{9.2.1.}{Spreadsheet类}

第 8 章介绍了 SpreadsheetCell 类,本章将定义 Spreadsheet 类。与 SpreadsheetCell 类一样,Spreadsheet 类在本章中不断发展,各种尝试并不是展示编写类的最佳方式。

首先,Spreadsheet 是一个二维数组,其中包含 SpreadsheetCells,具有设置和检索 Spreadsheet 中特定位置的单元格的成员函数。虽然大多数电子表格应用程序使用一个方向上的字母,和另一个方向上的数字来引用单元格,但 Spreadsheet 使用两个方向上的数字。

Spreadsheet.cppm 模块接口文件第一行定义了模块的名称:

\begin{cpp}
export module spreadsheet;
\end{cpp}

Spreadsheet 类需要访问 SpreadsheetCell 类,需要导入 spreadsheet\_cell 模块。为了让 SpreadsheetCell 类对 spreadsheet 模块的用户可见,需要导入并导出 spreadsheet\_cell 模块:

\begin{cpp}
export import spreadsheet_cell;
\end{cpp}

Spreadsheet 类使用 std::size\_t 类型,该类型定义在名为<cstddef>的 C 头文件中。可以通过以下导入来获取:

\begin{cpp}
import std;
\end{cpp}

这是 Spreadsheet 类定义的第一次尝试:

\begin{cpp}
export class Spreadsheet
{
    public:
        Spreadsheet(std::size_t width, std::size_t height);
        void setCellAt(std::size_t x, std::size_t y, const SpreadsheetCell& cell);
        SpreadsheetCell& getCellAt(std::size_t x, std::size_t y);
    private:
        bool inRange(std::size_t value, std::size_t upper) const;
        std::size_t m_width { 0 };
        std::size_t m_height { 0 };
        SpreadsheetCell** m_cells { nullptr };
};
\end{cpp}

\begin{myNotic}{NOTE}
Spreadsheet 类使用普通指针作为 m\_cells 数组。这是本章中为了展示结果,并解释如何在类中正确处理资源(如动态内存)而采用的做法。生产代码中,应该使用标准 C++ 容器,如 std::vector,这极大地简化了 Spreadsheet 的实现,但这样就无法了解如何正确使用原始指针处理动态内存。现代 C++ 中,永远不应该使用具有所有权语义的原始指针,但可能会在已有代码中遇到它们。在这种情况下,需要知道如何与它们一起工作即可。
\end{myNotic}

Spreadsheet 类不包含标准的二维 SpreadsheetCell 数组,它包含一个 SpreadsheetCell** 数据成员,这是一个指向指针的指针,表示一个数组的数组。这是因为每个 Spreadsheet 对象可能有不同的维度,所以类的构造函数必须根据客户端指定的高度和宽度动态分配二维数组。

为了动态分配二维数组,需要编写以下代码。 C++ 不像在 Java,不能简单地写成 new SpreadsheetCell[m\_width][m\_height]。

\begin{cpp}
Spreadsheet::Spreadsheet(size_t width, size_t height)
: m_width { width }, m_height { height }
{
    m_cells = new SpreadsheetCell*[m_width];
    for (size_t i { 0 }; i < m_width; ++i) {
        m_cells[i] = new SpreadsheetCell[m_height];
    }
}
\end{cpp}

图 9.1 显示了名为 s1 的 Spreadsheet 在栈上具有宽度 4 和高度 3 的内存布局。

\myGraphic{0.7}{content/part3/chapter9/images/1.png}{图 9.1}

inRange() 和 set 和检索成员函数的实现很简单:

\begin{cpp}
bool Spreadsheet::inRange(size_t value, size_t upper) const
{
    return value < upper;
}

void Spreadsheet::setCellAt(size_t x, size_t y, const SpreadsheetCell& cell)
{
    if (!inRange(x, m_width)) {
        throw out_of_range {
            format("x ({}) must be less than width ({}).", x, m_width) };
    }
    if (!inRange(y, m_height)) {
        throw out_of_range {
            format("y ({}) must be less than height ({}).", y, m_height) };
    }
    m_cells[x][y] = cell;
}

SpreadsheetCell& Spreadsheet::getCellAt(size_t x, size_t y)
{
    if (!inRange(x, m_width)) {
        throw out_of_range {
            format("x ({}) must be less than width ({}).", x, m_width) };
    }
    if (!inRange(y, m_height)) {
        throw out_of_range {
            format("y ({}) must be less than height ({}).", y, m_height) };
    }
    return m_cells[x][y];
}
\end{cpp}

setCellAt() 和 getCellAt() 都使用一个名为 inRange() 的辅助函数来检查 x 和 y 是否代表 Spreadsheet 中的有效坐标,尝试访问数组中一个越界的索引会导致程序异常。

setCellAt() 和 getCellAt() 的实现中,会看到有一些明显的代码重复。第6章解释说,应该不惜一切代价避免代码重复。因此,不再使用名为 inRange() 的辅助函数,而是定义 verifyCoordinate() 成员函数:

\begin{cpp}
void verifyCoordinate(std::size_t x, std::size_t y) const;
\end{cpp}

实现检查给定的坐标,并在坐标无效时抛出异常:

\begin{cpp}
void Spreadsheet::verifyCoordinate(size_t x, size_t y) const
{
    if (x >= m_width) {
        throw out_of_range {
            format("x ({}) must be less than width ({}).", x, m_width) };
    }
    if (y >= m_height) {
        throw out_of_range {
            format("y ({}) must be less than height ({}).", y, m_height) };
    }
}
\end{cpp}

现在,setCellAt() 和 getCellAt() 的实现可以简化为:

\begin{cpp}
void Spreadsheet::setCellAt(size_t x, size_t y, const SpreadsheetCell& cell)
{
    verifyCoordinate(x, y);
    m_cells[x][y] = cell;
}

SpreadsheetCell& Spreadsheet::getCellAt(size_t x, size_t y)
{
    verifyCoordinate(x, y);
    return m_cells[x][y];
}
\end{cpp}

\mySubsubsection{9.2.2.}{使用析构函数释放内存}

完成对动态分配内存的使用时,都应该对其进行释放。如果在对象中动态分配内存,释放该内存的位置应该在析构函数中。编译器保证当对象销毁时,会调用析构函数。以下是 Spreadsheet 类定义中的析构函数:

\begin{cpp}
export class Spreadsheet
{
    public:
        Spreadsheet(std::size_t width, std::size_t height);
        ~Spreadsheet();
        // Code omitted for brevity
};
\end{cpp}

析构函数具有与类名(以及构造函数)相同的名称,前面有一个波浪号 (\textasciitilde)。析构函数不接受参数,并且只能有一个。析构函数绝不应该抛出异常,原因在第14章中详细介绍。

以下是 Spreadsheet 类析构函数的实现:

\begin{cpp}
Spreadsheet::~Spreadsheet()
{
    for (size_t i { 0 }; i < m_width; ++i) {
        delete[] m_cells[i];
    }
    delete[] m_cells;
    m_cells = nullptr;
}
\end{cpp}

这个析构函数释放了在构造函数中分配的内存,但没有规则要求必须在析构函数中释放内存。可以在析构函数中编写任何代码,但最好只用它来释放内存或处理其他资源。

\mySubsubsection{9.2.3.}{复制和赋值}

从第 8 章回顾一下,如果没有编写自己的复制构造函数和复制赋值操作符,C++ 会生成它们。这些编译器生成的成员函数递归地调用对象数据成员的复制构造函数或复制赋值操作符。对于基本类型(如 int、double 和指针),提供浅层或位操作复制或赋值:会直接从源对象复制或赋值数据成员到目标对象,这在对象动态分配内存时会带来问题。以下代码复制了 spreadsheet s1 来初始化 s,当 s1 传递给 printSpreadsheet() 函数时:

\begin{cpp}
import spreadsheet;

void printSpreadsheet(Spreadsheet s) { /* Code omitted for brevity. */ }

int main()
{
    Spreadsheet s1 { 4, 3 };
    printSpreadsheet(s1);
}
\end{cpp}

Spreadsheet 包含一个指针变量:m\_cells。Spreadsheet 的浅层复制只给了目标对象 m\_cells 指针的副本,而不是底层数据的副本。因此,s 和 s1 都指向相同的数据,如图 9.2 所示。

\myGraphic{0.7}{content/part3/chapter9/images/2.png}{图 9.2}

如果 s 更改了 m\_cells 指向的内容,也会在 s1 中显示出来。更糟糕的是,当 printSpreadsheet() 函数退出时,调用s 的析构函数,释放了 m\_cells 指向的内存。这使得 m\_cells 在 s1 中不再指向有效的内存,如图 9.3 所示,这称为悬挂指针。

\myGraphic{0.7}{content/part3/chapter9/images/3.png}{图 9.3}

使用赋值时问题甚至更糟:

\begin{cpp}
Spreadsheet s1 { 2, 2 }, s2 { 4, 3 };
s1 = s2;
\end{cpp}

第一条语句之后,当构造 s1 和 s2 Spreadsheet 对象时,得到了如图 9.4 的内存布局。

\myGraphic{0.7}{content/part3/chapter9/images/4.png}{图 9.4}

赋值语句之后,得到了如图 9.5 的布局。

\myGraphic{0.7}{content/part3/chapter9/images/5.png}{图 9.5}

s1 和 s2 中的 m\_cells 指针不仅指向相同的内存,而且遗弃了 s1 之前指向的内存,这称为内存泄漏。

复制构造函数和复制赋值操作符必须执行深度复制,不仅应该复制指针数据成员,还应该复制这些指针指向的实际数据。

所以,依赖于 C++ 的默认复制构造函数和默认复制赋值操作符并不总是一个好主意。

\begin{myWarning}{WARNING}
每当在类中动态分配资源时,应该编写自己的复制构造函数和复制赋值操作符,以提供内存的深度复制。
\end{myWarning}

\mySamllsection{Spreadsheet的复制构造函数}

这是 Spreadsheet 类中复制构造函数的声明:

\begin{cpp}
export class Spreadsheet
{
    public:
        Spreadsheet(const Spreadsheet& src);
        // Code omitted for brevity
};
\end{cpp}

定义如下:

\begin{cpp}
Spreadsheet::Spreadsheet(const Spreadsheet& src)
    : Spreadsheet { src.m_width, src.m_height }
{
    for (size_t i { 0 }; i < m_width; ++i) {
        for (size_t j { 0 }; j < m_height; ++j) {
            m_cells[i][j] = src.m_cells[i][j];
        }
    }
}
\end{cpp}

这里使用了委托构造函数。这个复制构造函数的构造函数初始化列表,首先委托给非复制构造函数来分配适当的内存量。然后,复制构造函数复制实际的值,这两个过程共同实现了对 m\_cells 动态分配的二维数组的深度复制。

不需要删除现有的 m\_cells,因为这是复制构造函数,因此在这个对象中还没有m\_cells。

\mySamllsection{Spreadsheet的赋值操作符}

下面展示了带有赋值操作符的 Spreadsheet 类定义:

\begin{cpp}
export class Spreadsheet
{
    public:
        Spreadsheet& operator=(const Spreadsheet& rhs);
        // Code omitted for brevity
};
\end{cpp}

简单的实现如下所示:

\begin{cpp}
Spreadsheet& Spreadsheet::operator=(const Spreadsheet& rhs)
{
    // Check for self-assignment
    if (this == &rhs) {
        return *this;
    }

    // Free the old memory
    for (size_t i { 0 }; i < m_width; ++i) {
        delete[] m_cells[i];
    }
    delete[] m_cells;
    m_cells = nullptr;

    // Allocate new memory
    m_width = rhs.m_width;
    m_height = rhs.m_height;

    m_cells = new SpreadsheetCell*[m_width];
    for (size_t i { 0 }; i < m_width; ++i) {
        m_cells[i] = new SpreadsheetCell[m_height];
    }

    // Copy the data
    for (size_t i { 0 }; i < m_width; ++i) {
        for (size_t j { 0 }; j < m_height; ++j) {
            m_cells[i][j] = rhs.m_cells[i][j];
        }
    }

    return *this;
}
\end{cpp}

代码首先检查自我赋值,然后释放 this 对象的当前内存,分配新内存,最后复制各个元素。函数中发生了很多事情,可能会出很多问题!this 对象可能会处于无效状态。

内存成功释放,m\_width 和 m\_height 正确设置,但在分配内存的循环中抛出了异常,执行会跳过函数的其余部分退出。现在 Spreadsheet 实例已经损坏,m\_width 和 m\_height 数据成员表示一定的大小,但 m\_cells 数据成员不再指向正确的内存,代码异常不安全!

我们需要一个全有或全无的机制;要么一切顺利,要么 this 对象保持不变。为了实现这样的异常安全的赋值操作符,使用了复制和交换模式,在 Spreadsheet 类中添加了 swap() 成员函数。建议提供非成员 swap() 函数,以便各种标准库算法使用。以下是带有赋值操作符、swap() 成员函数和非成员函数的 Spreadsheet 类定义:

\begin{cpp}
export class Spreadsheet
{
    public:
        Spreadsheet& operator=(const Spreadsheet& rhs);
        void swap(Spreadsheet& other) noexcept;
        // Code omitted for brevity
};
export void swap(Spreadsheet& first, Spreadsheet& second) noexcept;
\end{cpp}

实现交换操作符的要求是 swap() 不会抛出异常,所以标记为 noexcept。

\begin{myNotic}{NOTE}
函数可以使用 noexcept 关键字来指定它不会抛出异常。noexcept 修饰符必须出现在 const 关键字之后:

\begin{cpp}
void myNonThrowingConstFunction() const noexcept { /* ... */ }
\end{cpp}

一个标记为 noexcept 的函数确实抛出了异常,程序将终止。关于 noexcept 的更多细节在第 14 章中介绍。
\end{myNotic}

成员函数 swap() 使用标准库中 <utility> 提供的 std::swap(),该函数可高效地交换两个值:

\begin{cpp}
void Spreadsheet::swap(Spreadsheet& other) noexcept
{
    std::swap(m_width, other.m_width);
    std::swap(m_height, other.m_height);
    std::swap(m_cells, other.m_cells);
}
\end{cpp}

非成员 swap() 函数简单地将交换操作转发给成员函数:

\begin{cpp}
void swap(Spreadsheet& first, Spreadsheet& second) noexcept
{
    first.swap(second);
}
\end{cpp}

现在我们有了一个异常安全的 swap(),可以用来实现赋值操作符:

\begin{cpp}
Spreadsheet& Spreadsheet::operator=(const Spreadsheet& rhs)
{
    Spreadsheet temp { rhs }; // Do all the work in a temporary instance
    swap(temp); // Commit the work with only non-throwing operations
    return *this;
}
\end{cpp}

该实现使用了复制和交换模式。首先,将右侧值复制到一个临时实例中,称为 temp。然后,当前对象与这个副本进行交换。这种模式是实现赋值操作符的推荐方式,保证了强异常安全性,即发生异常,当前 Spreadsheet 对象的状态将保持不变。这个模式分为三个阶段:

\begin{itemize}
\item
第一个阶段创建一个临时副本。这个阶段不会修改当前 Spreadsheet 对象的状态,如果在这个阶段抛出异常,没有问题。

\item
第二个阶段使用 swap() 函数与当前对象交换创建的临时副本。swap() 函数绝不抛出异常。

\item
第三个阶段是临时对象的销毁,现在包含原始对象(因为进行了交换)。
\end{itemize}

不使用复制和交换模式来实现赋值操作符时,为了效率和有时为了正确性,赋值操作符的第一行代码通常会检查自我赋值:

\begin{cpp}
Spreadsheet& Spreadsheet::operator=(const Spreadsheet& rhs)
{
    // Check for self-assignment
    if (this == &rhs) { return *this; }
    // ...
    return *this;
}
\end{cpp}

使用复制和交换模式,不需要自我赋值测试。

\begin{myWarning}{WARNING}
实现赋值操作符时,使用复制和交换模式来避免代码重复,必须保证异常安全性。
\end{myWarning}

\begin{myNotic}{NOTE}
复制和交换模式不仅可以用于赋值操作符,可用于其他步骤。并且若希望将其转换为全有或全无操作的操作:首先,创建一个副本;然后,在副本上进行所有修改;最后,如果没出错,执行一个非抛出异常的交换操作。
\end{myNotic}

\mySamllsection{禁止赋值和按值传递}

当在类中动态分配内存时,最简单的方法就是避免复制或赋值对象。可以通过显式删除 operator= 和复制构造函数来实现这一点。这样,如果尝试按值传递对象、从函数返回对象或赋值对象,编译器会发出警告。以下是一个防止赋值和按值传递的 Spreadsheet 类定义:

\begin{cpp}
export class Spreadsheet
{
    public:
        Spreadsheet(std::size_t width, std::size_t height);
        Spreadsheet(const Spreadsheet& src) = delete;
        ~Spreadsheet();
        Spreadsheet& operator=(const Spreadsheet& rhs) = delete;
        // Code omitted for brevity
};
\end{cpp}

不需要为删除的成员函数提供实现。编译器不会允许代码调用它们,所以链接器永远不会寻找它们。当编写复制或赋值 Spreadsheet 对象的代码时,编译器会报错:

\begin{shell}
'Spreadsheet &Spreadsheet::operator =(const Spreadsheet &)': attempting to
reference a deleted function
\end{shell}

\mySubsubsection{9.2.4.}{使用移动语义}

类的移动语义需要移动构造函数和移动赋值操作符,编译器可将其用于对象的临时对象,该对象在操作完成后将销毁,或显式使用 std::move()移动。移动将内存和其他资源的所有权从一个对象转移到另一个对象,是数据成员的浅层复制与已分配内存和其他资源所有权切换的组合,以防止悬挂指针或资源,并防止内存泄漏。

移动构造函数和移动赋值操作符将源对象的数据成员移动到新对象中,使源对象处于某种不确定的状态。源对象的数据成员重置为“null”值,但这不是严格的要求。我确实建议,确保源对象在移动操作后处于明确定义的空状态。安全起见,因为可能会触发未定义的行为,所以永远不要使用已移动的对象。标准库中的一些例外是 std::unique\_ptr 和 shared\_ptr。标准库明确声明,这些智能指针在移动后必须将其内部指针重置为 nullptr,这使得在移动操作后安全地重用此类智能指针成为可能。

实现移动语义之前,先来了解一下右值和右值引用。

\mySamllsection{右值引用}

左值是可以取地址的某种东西,一个有名字的变量。这个名字来源于左值可以出现在赋值操作符的左侧。右值是指不是左值的对象,比如一个字面量,或者一个临时对象或值(从技术上讲,C++ 标准定义了三个更多类别(xvalues、prvalues 和 glvalues),但那些细节对于当前讨论并不重要)。通常,右值出现在赋值操作符的右侧。考虑以下语句:

\begin{cpp}
int a { 4 * 2 };
\end{cpp}

a 是一个左值,有名字,并且可以用 \&a 来取它的地址。另一方面,表达式 4*2 的结果是一个右值,是一个临时值,语句执行完毕时销毁。这个例子中,这个临时值的副本存储在名为 a 的变量中。 如果一个函数通过值返回某个结果,调用该函数的结果是一个临时右值。

如果函数返回一个对非常量的引用,调用该函数的结果是一个左值,可以使用这个结果作为赋值操作符左侧的表达式。

右值是一个临时对象或使用 std::move() 显式移动的对象。右值引用的目的是使得当涉及到右值时,可以有选择地选择特定的函数重载。这使得某些通常涉及复制大值的操作可以改为复制这些值的指针。

函数可以通过在其参数说明中使用 \&\& 来指定一个右值引用参数,例如:type\&\& name。临时对象会视为 const type\&,但是当有一个使用右值引用的函数重载时,临时对象可以解析为那个重载。以下示例展示了这一点。代码首先定义了两个 handleMessage() 函数,一个接受一个左值引用,另一个接受一个右值引用:

\begin{cpp}
void handleMessage(string& message) // lvalue reference parameter
{
    println("handleMessage with lvalue reference: {}", message);
}

void handleMessage(string&& message) // rvalue reference parameter
{
    println("handleMessage with rvalue reference: {}", message);
}
\end{cpp}

可以用命名变量作为参数来调用 handleMessage():

\begin{cpp}
string a { "Hello " };
handleMessage(a); // Calls handleMessage(string& value)
\end{cpp}

a 是一个命名变量,所以调用了接受左值引用的 handleMessage() 函数。通过其引用参数所做的更改都会改变 a 的值。

也可以用一个表达式作为参数来调用 handleMessage():

\begin{cpp}
string b { "World" };
handleMessage(a + b); // Calls handleMessage(string&& value)
\end{cpp}

a + b 表达式产生一个临时对象,而临时对象不是一个左值,所以不能使用接受左值引用的 handleMessage() 函数,调用了接受右值引用的重载。参数是一个临时对象,handleMessage() 通过其引用参数所做的更改在函数返回后都会丢失。

字面量也可以作为 handleMessage() 的参数使用。这也会触发对接受右值引用的重载的调用,字面量不能是一个左值(字面量可以作为对常量引用的参数传递):

\begin{cpp}
handleMessage("Hello World"); // Calls handleMessage(string&& value)
\end{cpp}

如果删除了接受左值引用的 handleMessage() 函数,调用 handleMessage() 并传递一个命名的变量,如 handleMessage(b),将会导致编译错误,因为一个右值引用参数(string\&\&)永远不会绑定到一个左值(b)。可以通过使用 std::move() 来强制编译器调用 handleMessage() 的右值引用重载。move() 所做的唯一事情就是将一个左值转换为一个右值引用,不执行任何实际的移动操作。通过返回右值引用,编译器找到一个接受右值引用的 handleMessage() 的重载,该重载可以执行移动操作。以下是使用 move() 的示例:

\begin{cpp}
handleMessage(std::move(b)); // Calls handleMessage(string&& value)
\end{cpp}

命名的变量是左值,所以在 handleMessage(string\&\& message) 函数内部,message 右值引用参数本身是左值,因为它有名字!如果想将这个右值引用参数作为右值传递给另一个函数,则需要使用 std::move() 将左值转换为右值引用。例如,添加了以下具有右值引用参数的函数:

\begin{cpp}
void helper(string&& message) { }
\end{cpp}

下面这样使用,无法通过编译:

\begin{cpp}
void handleMessage(string&& message) { helper(message); }
\end{cpp}

helper()函数需要一个右值引用,而handleMessage()传递的message因为有命名,所以是一个左值,这会导致编译错误。下面是std::move()的正确使用方法:

\begin{cpp}
void handleMessage(string&& message) { helper(std::move(message)); }
\end{cpp}

\begin{myWarning}{WARNING}
命名的右值引用,比如右值引用形参,本身就是左值!
\end{myWarning}

右值引用不仅限于函数的参数,也可以声明一个右值引用类型的变量,并给它赋值,但这种用法不常见。以下代码在 C++ 中非法:

\begin{cpp}
int& i { 2 }; // Invalid: reference to a constant
int a { 2 }, b { 3 };
int& j { a + b }; // Invalid: reference to a temporary
\end{cpp}

使用右值引用,下面的代码完全合法:

\begin{cpp}
int&& i { 2 };
int a { 2 }, b { 3 };
int&& j { a + b };
\end{cpp}

这种独立的右值引用很少这样使用。

\begin{myNotic}{NOTE}
如果将临时变量赋值给右值引用,则只要右值引用在作用域中,临时变量的生命周期就会延长。
\end{myNotic}

\CXXTwentythreeLogo{-40}{-50}

\mySamllsection{副本退化}

如果一个对象 x,写成 “auto y\{x\}” 会创建 x 的一个副本并给命名为 y,它是一个左值。

C++23 引入了 auto(x) 或 auto\{x\} 语法,以创建一个作为右值的 x 的副本,而不是左值。

假设只有一个前面部分中的右值引用 handleMessage(string\&\&) 函数,没有左值引用重载。这种情况下,下面的代码将不起作用:

\begin{cpp}
string value { "Hello " };
handleMessage(value); // Error
\end{cpp}

可以使用std::move():

\begin{cpp}
handleMessage(std::move(value));
\end{cpp}

在这个操作之后,因为已经移动了,所以不应该再使用这个值。

使用 C++23 的副本退化语法:

\begin{cpp}
handleMessage(auto { value });
\end{cpp}

这使得将对象 value 作为一个右值进行临时复制,并将这个右值传递给 handleMessage()。如果 handleMessage() 从这个副本中移动,原始对象 value 将保留,不受任何影响。

\mySamllsection{实现移动语义}

移动语义通过使用右值引用实现。为了向类中添加移动语义,需要实现一个移动构造函数和一个移动赋值操作符。移动构造函数和移动赋值操作符应该标记为 noexcept,说明其不会抛出异常。这对于与标准库的兼容性尤为重要,因为标准库容器的完全符合规定的实现将只在实现移动语义的同时保证不会抛出异常,以提供强异常安全性。

以下是带有移动构造函数和移动赋值操作符的 Spreadsheet 类定义。此外,还引入了两个辅助成员函数:cleanup(),用于析构函数和移动赋值操作符中;moveFrom(),将源对象的数据成员移动到目标对象,然后重置源对象。

\begin{cpp}
export class Spreadsheet
{
    public:
        Spreadsheet(Spreadsheet&& src) noexcept; // Move constructor
        Spreadsheet& operator=(Spreadsheet&& rhs) noexcept; // Move assignment
        // Remaining code omitted for brevity
    private:
        void cleanup() noexcept;
        void moveFrom(Spreadsheet& src) noexcept;
        // Remaining code omitted for brevity
};
\end{cpp}

实现如下所示:

\begin{cpp}
void Spreadsheet::cleanup() noexcept
{
    for (size_t i { 0 }; i < m_width; ++i) {
        delete[] m_cells[i];
    }
    delete[] m_cells;
    m_cells = nullptr;
    m_width = m_height = 0;
}

void Spreadsheet::moveFrom(Spreadsheet& src) noexcept
{
    // Shallow copy of data
    m_width = src.m_width;
    m_height = src.m_height;
    m_cells = src.m_cells;

    // Reset the source object, because ownership has been moved!
    src.m_width = 0;
    src.m_height = 0;
    src.m_cells = nullptr;
}

// Move constructor
Spreadsheet::Spreadsheet(Spreadsheet&& src) noexcept
{
    moveFrom(src);
}

// Move assignment operator
Spreadsheet& Spreadsheet::operator=(Spreadsheet&& rhs) noexcept
{
    // Check for self-assignment
    if (this == &rhs) {
        return *this;
    }

    // Free the old memory and move ownership
    cleanup();
    moveFrom(rhs);
    return *this;
}
\end{cpp}

移动构造函数和移动赋值操作符是将 m\_cells 内存的所有权,从一个源对象转移到一个新的对象。将源对象的 m\_cells 指针置为空指针,并将源对象的 m\_width 和 m\_height 设置为零,以防止源对象的析构函数释放内存。

只有在知道源对象不再需要时,移动语义才有用。

这个实现包括在移动赋值操作符中的自我赋值测试。根据前面的实例,这个自我赋值测试可能是不必要的。然而,应该始终包括它,就像 C++ 核心指南所建议的那样(C++ 核心指南 C.65 条目(见附录 B)指出,“使移动赋值对自我赋值安全。”),以确保以下代码不会在运行时崩溃:

\begin{cpp}
sheet1 = std::move(sheet1);
\end{cpp}

移动构造函数和移动赋值操作符可以显式删除或默认,就像复制构造函数和复制赋值操作符一样。

如果类没有用户声明的复制构造函数、复制赋值操作符、移动构造函数或析构函数,编译器会自动为该类生成默认的移动构造函数。如果类没有用户声明的复制构造函数、移动构造函数、复制赋值操作符或析构函数,编译器也会为该类生成默认的移动赋值操作符。

\begin{myWarning}{WARNING}
声明一个或多个特殊成员函数(析构函数、复制构造函数、移动构造函数、复制赋值操作符和移动赋值操作符)时,建议声明所有这些函数,这称为五法则。要么为它们提供显式实现,要么显式默认(=default)或删除(=delete)。
\end{myWarning}

\mySamllsection{std::exchange}

可以使用定义在  <utility> 中的 std::exchange() 来用新值替换一个值并返回旧值:

\begin{cpp}
int a { 11 };
int b { 22 };
println("Before exchange(): a = {}, b = {}", a, b);
int returnedValue { exchange(a, b) };
println("After exchange(): a = {}, b = {}", a, b);
println("exchange() returned: {}", returnedValue);
\end{cpp}

输出为:

\begin{shell}
Before exchange(): a = 11, b = 22
After exchange(): a = 22, b = 22
exchange() returned: 11
\end{shell}

exchange() 函数在实现移动赋值操作符时非常有用。移动赋值操作符需要将数据从源对象移动到目标对象,之后源对象中的数据会清除。上一节中,这是这样做的:

\begin{cpp}
void Spreadsheet::moveFrom(Spreadsheet& src) noexcept
{
    // Shallow copy of data
    m_width = src.m_width;
    m_height = src.m_height;
    m_cells = src.m_cells;

    // Reset the source object, because ownership has been moved!
    src.m_width = 0;
    src.m_height = 0;
    src.m_cells = nullptr;
}
\end{cpp}

这个成员函数复制了源对象的数据成员 m\_width、m\_height 和 m\_cells,然后将它们设置为 0 或 nullptr。使用 exchange(),可以写的更紧凑:

\begin{cpp}
void Spreadsheet::moveFrom(Spreadsheet& src) noexcept
{
    m_width = exchange(src.m_width, 0);
    m_height = exchange(src.m_height, 0);
    m_cells = exchange(src.m_cells, nullptr);
}
\end{cpp}

\mySamllsection{移动数据成员}

因为它们是基本类型,所以moveFrom() 成员函数使用直接赋值来处理三个数据成员。如果对象包含其他对象作为数据成员,应该使用 std::move() 来移动这些对象。假设 Spreadsheet 类有一个名为 m\_name 的 std::string 数据成员。moveFrom() 成员函数应该这样实现:

\begin{cpp}
void Spreadsheet::moveFrom(Spreadsheet& src) noexcept
{
    // Move object data members
    m_name = std::move(src.m_name);

    // Move primitives:
    m_width = exchange(src.m_width, 0);
    m_height = exchange(src.m_height, 0);
    m_cells = exchange(src.m_cells, nullptr);
}
\end{cpp}

\mySamllsection{移动构造函数和移动赋值操作符的使用}

之前的移动构造函数和移动赋值操作符都使用了 moveFrom() ,该函数通过执行浅复制来移动所有数据成员。如果向 Spreadsheet 类添加一个新的数据成员,就必须修改 swap() 函数和 moveFrom() 函数。如果忘记更新其中一个,就会引入错误。为了避免这样的错误,可以用移动构造函数和移动赋值操作符替代 swap() 函数。

首先,可以删除 cleanup() 和 moveFrom() 辅助函数。将 cleanup() 函数中的代码移到析构函数中,移动构造函数和移动赋值操作符可以按照以下方式实现:

\begin{cpp}
Spreadsheet::Spreadsheet(Spreadsheet&& src) noexcept
{
    swap(src);
}

Spreadsheet& Spreadsheet::operator=(Spreadsheet&& rhs) noexcept
{
    auto moved { std::move(rhs) }; // Move rhs into moved (noexcept)
    swap(moved); // Commit the work with only non-throwing operations
    return *this;
}
\end{cpp}

移动构造函数简单地将默认构造的 *this 与给定的源对象进行交换。移动赋值操作符使用移动-交换模式,这与之前讨论的复制-交换模式类似。

\begin{myNotic}{NOTE}
使用 swap() 实现移动构造函数和移动赋值操作符可以减少代码量。当添加数据成员时,引入错误的概率也较低,只需要更新 swap() 实现来包含这些新数据成员。
\end{myNotic}

Spreadsheet 的移动赋值操作符也可以进行如下实现:

\begin{cpp}
Spreadsheet& Spreadsheet::operator=(Spreadsheet&& rhs) noexcept
{
    swap(rhs);
    return *this;
}
\end{cpp}

这样做并不能保证立即清理 this 的内容,而this 的内容会通过 rhs 转移到移动赋值操作符的调用者处,可能比预期存活得更久。

\mySamllsection{测试 Spreadsheet 的移动操作}

Spreadsheet 的移动构造函数和移动赋值操作符可以通过以下代码进行测试:

\begin{cpp}
Spreadsheet createObject()
{
    return Spreadsheet { 3, 2 };
}

int main()
{
    vector<Spreadsheet> vec;
    for (size_t i { 0 }; i < 2; ++i) {
        println("Iteration {}", i);
        vec.push_back(Spreadsheet { 100, 100 });
        println("");
    }

    Spreadsheet s { 2, 3 };
    s = createObject();

    println("");

    Spreadsheet s2 { 5, 6 };
    s2 = s;
}
\end{cpp}

第 1 章介绍了 vector。一个 vector 动态地增长其大小以适应新的对象。这是通过分配更大的内存块,然后将对象从旧 vector 复制或移动到新的更大的 vector 实现的。如果编译器发现一个 noexcept 移动构造函数,则使用移动而不是复制。因为移动,所以不需要深度复制,会更有效率。

Spreadsheet 类的所有构造函数和赋值操作符中添加打印语句后,前面测试程序的输出可能如下所示。每行右侧的数字不是实际输出的一部分,而是添加到本文字中,以便更容易地引用以下讨论中的特定行。此输出和以下讨论基于使用移动和交换模式实现其移动操作的 Spreadsheet 类版本,以及使用 Microsoft Visual C++ 2022 编译器的代码的发布构建。C++ 标准没有指定 vector 的初始容量及其增长策略,不同编译器的输出可能会有所不同。

\begin{shell}
Iteration 0
Normal constructor          (1)
Move constructor            (2)

Iteration 1
Normal constructor          (3)
Move constructor            (4)
Move constructor            (5)

Normal constructor          (6)
Normal constructor          (7)
Move assignment operator    (8)
Move constructor            (9)

Normal constructor         (10)
Copy assignment operator   (11)
Normal constructor         (12)
Copy constructor           (13)
\end{shell}

循环的第一次迭代中,vector 仍然为空。考虑循环中的以下代码行:

\begin{cpp}
vec.push_back(Spreadsheet { 100, 100 });
\end{cpp}

执行此行代码时,创建一个新的 Spreadsheet 对象,调用正常构造函数(1)。vector 调整大小以为新对象腾出空间。然后,创建的 Spreadsheet 对象移动到 vector 中,调用移动构造函数(2)。

循环的第二次迭代中,使用正常构造函数创建了第二个 Spreadsheet 对象(3)。此时,vector 可以容纳一个元素,再次调整大小以为新对象腾出空间。因为 vector 调整大小,之前添加的元素需要从旧 vector 移动到新的 vector 中。这触发了对之前每个添加元素的移动构造函数的调用。vector 中有一个元素,所以调用移动构造(4)。最后,新的 Spreadsheet 对象移动到 vector 中,调用移动构造函数(5)。

接下来,使用正常构造函数创建了一个 Spreadsheet 对象 s(6)。createObject() 函数创建了一个临时 Spreadsheet 对象,调用正常构造函数(7),该临时对象从函数返回并赋值给变量 s。因为 createObject() 返回的临时对象在赋值后不再存在,编译器调用移动赋值操作符(8)而不是复制赋值操作符。移动赋值操作符使用移动和交换模式,因此委托给移动构造函数(9)。

另一个 Spreadsheet 对象 s2 使用正常构造函数创建(10)。赋值 s2 = s 调用复制赋值操作符(11),右侧对象不是临时对象,而是一个命名对象。这个复制赋值操作符使用复制和交换模式,创建了一个临时副本,触发了对复制构造函数的调用,该复制构造函数首先委托给正常构造函数(12 和 13)。

如果 Spreadsheet 类没有实现移动语义,所有对移动构造函数和移动赋值操作符的调用,都会替换为对复制构造函数和复制赋值操作符的调用。之前的示例中,循环中的 Spreadsheet 对象有 10,000(100 × 100)个元素。Spreadsheet 移动构造函数和移动赋值操作符的实现不需要内存分配,而复制构造函数和复制赋值操作符每次都需要 101 次分配。某些情况下,使用移动语义可以显著提高性能。

\mySamllsection{使用移动语义实现交换函数}

作为另一个示例 swap() 函数,该函数交换两个 Object。以下 swapCopy() 实现不使用移动语义:

\begin{cpp}
void swapCopy(Object& a, Object& b)
{
    Object temp { a };
    a = b;
    b = temp;
}
\end{cpp}

将 a 复制到 temp,然后将 b 复制到 a,最后将 temp 复制到 b。如果 Object 复制成本较高,这种实现会严重影响性能。使用移动语义,可以避免所有复制:

\begin{cpp}
void swapMove(Object& a, Object& b)
{
    Object temp { std::move(a) };
    a = std::move(b);
    b = std::move(temp);
}
\end{cpp}

这就是标准库中的 std::swap() 的实现方式。

\mySamllsection{返回语句中使用 std::move()}

从 C++17 开始,编译器不允许对 return object; 的语句中的匿名临时对象 object 执行复制或移动操作。这称为强制省略复制/移动操作,所以按值返回 object 没有性能开销。如果 object 是一个不是函数参数的局部变量,则允许非强制性省略复制/移动操作,也称为命名返回值优化(NRVO)。这种优化不是标准保证的。一些编译器仅在发布构建中执行此优化,而不在调试构建中执行。通过强制性和非强制性省略,编译器可以避免从函数返回对象的复制,这就是零拷贝按值传递语义。

\begin{myWarning}{WARNING}
对于 NRVO,即使不再调用复制/移动构造函数,仍然需要可访问;否则,根据标准,程序不正确。
\end{myWarning}

现在,当在返回语句中使用 std::move() 时会发生什么?例如以下代码:

\begin{cpp}
return std::move(object);
\end{cpp}

使用此代码,编译器无法再应用强制性省略(NRVO)或非强制性省略复制/移动操作,这些仅适用于形式为 return object; 的语句。由于无法再应用复制/移动省略,如果对象支持移动语义,则使用移动语义;如果不支持,则退回到复制语义。

与 NRVO 相比,退回到移动语义会有轻微的性能影响,但退回到复制语义可能会产生巨大的性能影响!因此:

\begin{myWarning}{WARNING}
从一个函数返回局部变量或匿名临时对象时,只需写 return object; 而不使用 std::move()。
\end{myWarning}

如果想要从一个类的成员函数返回其数据成员,想要移动它(而不是返回其副本),则需要使用 std::move()。

此外,对于以下表达式要小心:

\begin{cpp}
return condition ? obj1 : obj2;
\end{cpp}

这不符合形式 return object;,所以编译器无法应用复制/移动省略。更糟糕的是,条件 ? obj1 : obj2 形式的表达式是一个左值,所以编译器使用复制构造函数来返回其中一个对象。为了至少触发移动语义,可以将返回语句重写为:

\begin{cpp}
return condition ? std::move(obj1) : std::move(obj2);
\end{cpp}

或

\begin{cpp}
return std::move(condition ? obj1 : obj2);
\end{cpp}

更清晰的做法是将返回语句重写为以下形式,编译器可以自动使用移动语义而无需显式使用 std::move():

\begin{cpp}
if (condition) {
    return obj1;
} else {
    return obj2;
}
\end{cpp}

\mySamllsection{向函数传递参数的最优方式}

建议使用常量引用参数,来避免非基本函数参数的冗余昂贵的复制。当右值介入时,情况略有变化。想象一个函数无论如何都会复制作为其参数传递的对象。这种情况会出现在类成员函数中:

\begin{cpp}
class DataHolder
{
    public:
        void setData(const vector<int>& data) { m_data = data; }
    private:
        vector<int> m_data;
};
\end{cpp}

setData() 复制传入的数据。现在熟悉了右值和右值引用,可能想为 setData() 添加一个重载,以优化避免在右值的情况下进行复制:

\begin{cpp}
class DataHolder
{
    public:
        void setData(const vector<int>& data) { m_data = data; }
        void setData(vector<int>&& data) { m_data = move(data); }
    private:
        vector<int> m_data;
};
\end{cpp}

传递临时对象为setData()的参数时,不进行复制——数据移动了。

以下代码片段触发了对 setData() 的引用为常量重载的调用,数据进行了复制:

\begin{cpp}
DataHolder wrapper;
vector myData { 11, 22, 33 };
wrapper.setData(myData);
\end{cpp}

另外,以下代码片段以临时对象调用 setData(),这触发了对 setData() 的右值引用重载的调用。随后数据是移动,而不是复制:

\begin{cpp}
wrapper.setData({ 22, 33, 44 });
\end{cpp}

为了优化 setData() 以适应左值和右值,需要实现两个重载。幸运的是,有一个更好的方法,涉及使用按值传递的单个成员函数——按值传递!到目前为止,建议总是使用常量引用的参数来传递对象,以避免不必要的复制,但现在建议使用按值传递。对于那些不会复制的参数,使用常量引用的参数仍然是正确的做法。按值传递的建议只适用于函数会无论如何复制的参数。这种情况下,使用按值传递语义,代码对于左值和右值都最优。如果传递了一个左值,将复制一次,就像使用引用为常量参数一样。如果传递了一个右值,则不会进行复制,就像使用右值引用参数一样。来看一些代码:

\begin{cpp}
class DataHolder
{
    public:
        void setData(vector<int> data) { m_data = move(data); }
    private:
        vector<int> m_data;
};
\end{cpp}

如果传递了一个左值给 setData(),将复制到 data 参数中,然后移动到 m\_data。如果传递了一个右值给 setData(),将移动到 data 参数中,然后再移动到 m\_data。

\begin{myNotic}{NOTE}
对于函数内在会复制的参数,建议使用按值传递,但仅当参数类型支持移动语义,并且当不需要参数上的多态行为时。否则,请使用引用为常量的参数。按值传递多态类型可能导致切片,这将在第10章进行介绍。
\end{myNotic}

\mySubsubsection{9.2.5.}{零法则}

本章早些时候,介绍了五法则。复习一下,声明了五个特殊成员函数中的一个(析构函数、复制构造函数、移动构造函数、复制赋值操作符和移动赋值操作符),那么应该声明所有这些函数,要么实现它们,要么显式默认(=default)或删除(=delete)。原因是编译器遵循复杂的规则来决定,是否自动生成这些特殊成员函数。通过自己声明所有这些函数,不会让编译器决定任何事情,并且将使类的意图更加明确。

目前为止的所有讨论,都解释了如何编写这五个特殊成员函数。

现代 C++ 中,应该采用零法则。零法则指出,应该去设计类,不需要这五个特殊成员函数。怎么做到这点呢?可以为非多态类型做到这一点,避免使用旧式的动态分配内存或其他资源,而使用现代构造,如标准库容器和智能指针。可以使用 vector<vector<SpreadsheetCell>{}> ,而不是 Spreadsheet 类中的 SpreadsheetCell** 数据成员,或者使用一个 vector<SpreadsheetCell> 来存储电子表格的线性化表示。vector 会自动处理内存,所以不需要这五个特殊成员函数。

\begin{myWarning}{WARNING}
现代 C++ 中,采用零法则!
\end{myWarning}

五法则应该限制在自定义资源获取即初始化(RAII)类中。RAII 类拥有资源,并在正确的时间处理其释放。这是一种设计技术,例如 vector 和 unique\_ptr 使用了这种技术,会在第 32 章中进行了进一步介绍。此外,第 10 章会说明多态类型也需要遵循五法则。









