
std::pair 类定义在 <utility> 头文件中,并在第 1 章中引入,可以存储两个值,每个值都有特定的类型。每个值的类型必须在编译时已知:

\begin{cpp}
pair<int, string> p1 { 16, "Hello World" };
pair p2 { true, 0.123f }; // Using CTAD.
println("p1 = ({}, {})", p1.first, p1.second);
println("p2 = ({}, {})", p2.first, p2.second);
\end{cpp}

输出如下所示:

\begin{shell}
p1 = (16, Hello World)
p2 = (true, 0.123)
\end{shell}

从 C++23 开始,std::format() 和打印函数完全支持 std::pair。例如,前面代码片段中的两个 println() 语句可以写成以下形式:

\begin{cpp}
println("p1 = {}", p1);
println("p2 = {}", p2);
\end{cpp}

输出如下,双引号包围字符串:

\begin{shell}
p1 = (16, "Hello World")
p2 = (true, 0.123)
\end{shell}

std::tuple,定义在 <tuple> 头文件中,是对 std::pair 的扩展。允许存储任意数量的值,每个值都有其自己的特定类型。就像 std::pair 一样,std::tuple 有固定的大小和固定的值类型,在编译时确定。

可以使用 tuple 构造函数创建一个 tuple,指定模板类型和实际值。例如,下面的代码创建了一个 tuple,其中第一个元素是整数,第二个元素是字符串,最后一个元素是布尔值:

\begin{cpp}
using MyTuple = tuple<int, string, bool>;
MyTuple t1 { 16, "Test", true };
\end{cpp}

就像使用 std::pair 一样,从 C++23 开始,std::format() 和打印函数完全支持 std::tuple:

\begin{cpp}
println("t1 = {}", t1);
// Outputs: t1 = (16, "Test", true)
\end{cpp}

std::get<i>() 用于从 tuple 中获取第 i 个元素,其中 i 是一个零基索引;也就是说,<0> 是 tuple 的第一个元素,<1> 是 tuple 的第二个元素,依此类推。返回的值具有 tuple 中该索引的正确类型:

\begin{cpp}
println("t1 = ({}, {}, {})", get<0>(t1), get<1>(t1), get<2>(t1));
// Outputs: t1 = (16, Test, true)
\end{cpp}

可以通过使用 typeid(),从 <typeinfo>,来检查 get<i>() 返回的正确类型。以下代码确认了 get<1>(t1) 返回的值确实是一个 std::string(正如之前所述,typeid().name() 返回的确切字符串取决于编译器):

\begin{cpp}
println("Type of get<1>(t1) = {}", typeid(get<1>(t1)).name());
// Outputs: Type of get<1>(t1) = class std::basic_string<char,
// struct std::char_traits<char>,class std::allocator<char> >
\end{cpp}

可以使用 std::tuple\_element 类模板在编译时根据元素的索引,获取元素的类型。tuple\_element 需要指定 tuple 的类型(这个例子中是 MyTuple),而不是实际的 tuple 实例,如 t1。以下是一个例子:

\begin{cpp}
println("Type of element with index 2 = {}",
    typeid(tuple_element<2, MyTuple>::type).name());
// Outputs: Type of element with index 2 = bool
\end{cpp}

可以根据元素的类型从 tuple 中检索元素,使用 std::get<T>(),其中 T 是要检索的元素的类型,而不是索引。如果 tuple 有多个具有请求类型的元素,编译器将生成一个错误。例如,可以像以下这样从 t1 中检索字符串元素:

\begin{cpp}
println("String = {}", get<string>(t1));
// Outputs: String = Test
\end{cpp}

不幸的是,遍历 tuple 的值并不是那么直接。不能编写一个简单的循环,并调用类似 get<i>(mytuple) 的东西,因为 i 的值必须在编译时已知。一个可能的解决方案是使用模板元编程,这在第 26 章中有详细介绍,其中还包括一个如何打印 tuple 值的示例。

tuple 的大小可以通过 std::tuple\_size 类模板查询。与 tuple\_element 一样,tuple\_size 需要指定 tuple 的类型:

\begin{cpp}
println("Tuple Size = {}", tuple_size<MyTuple>::value);
// Outputs: Tuple Size = 3
\end{cpp}

如果不知道 tuple 的确切类型,可以使用 decltype() 来查询类型:

\begin{cpp}
println("Tuple Size = {}", tuple_size<decltype(t1)>::value);
// Outputs: Tuple Size = 3
\end{cpp}

使用类模板参数推断(CTAD),当构造一个 tuple 时,可以省略模板类型参数,让编译器自动根据传递给构造函数的参数类型推断它们。例如,下面的代码定义了与 t1 相同的 tuple,包含一个整数、一个字符串和一个布尔值。请注意,现在必须使用 s 字符串字面量来指定 "Test",以确保它是一个 std::string:

\begin{cpp}
tuple t1 { 16, "Test"s, true };
\end{cpp}

使用 CTAD时,不需要显式指定存储在 tuple 中的类型,因此不能使用 \& 来指定引用。如果想要使用 CTAD 来生成一个包含非常量引用或常量引用的 tuple,那么需要使用 <functional> 中定义的 ref() 或 cref(),这些函数会创建 reference\_wrapper<T> 或 reference\_wrapper<const T> 的实例。例如,以下语句会创建一个类型为 tuple<int, double\&, const double\&, string\&> 的 tuple:

\begin{cpp}
double d { 3.14 };
string str1 { "Test" };
tuple t2 { 16, ref(d), cref(d), ref(str1) };
\end{cpp}

为了测试存储在 t2 中的双精度浮点数引用,以下代码首先将双精度浮点数变量的值写入控制台。对 get<1>(t2) 的调用返回对 d 的引用,因为 ref(d) 用于第二个(索引 1)的 tuple 元素。第二个语句改变了引用的变量的值,而最后一个语句显示,通过 tuple 中存储的引用确实改变了 d 的值。注意,第三个语句因为使用了 cref(d) 作为第三个 tuple 元素而无法编译;所以,是对 d 的常量引用:

\begin{cpp}
println("d = {}", d);
get<1>(t2) *= 2;
//get<2>(t2) *= 2; // ERROR because of cref().
println("d = {}", d);
// Outputs: d = 3.14
// d = 6.28
\end{cpp}

如果没有类模板参数推断,可以使用 std::make\_tuple() 函数模板来创建一个 tuple。由于它是一个函数模板,支持函数模板参数推断,因此也允许通过指定实际值来创建一个 tuple。类型在编译时自动推断。这是一个例子:

\begin{cpp}
auto t2 { make_tuple(16, ref(d), cref(d), ref(str1)) };
\end{cpp}

\mySubsubsection{24.3.1.}{分解Tuple}

可以通过两种方式将tuple分解为其单独的元素:结构化绑定和 std::tie()。

\mySamllsection{结构化绑定}

自从C++17以来,结构化绑定使分解元组到单独变量变得容易。例如,以下代码定义了一个由整数、字符串和布尔值组成的tuple,然后使用结构化绑定将其分解为三个不同的变量:

\begin{cpp}
tuple t1 { 16, "Test"s, true };
auto [i, str, b] { t1 };
println("Decomposed: i = {}, str = \"{}\", b = {}", i, str, b);
\end{cpp}

还可以将tuple分解为引用,通过这些引用修改元组的内容。这里是一个例子:

\begin{cpp}
auto& [i2, str2, b2] { t1 };
i2 *= 2;
str2 = "Hello World";
b2 = !b2;
\end{cpp}

使用结构化绑定时,不能在分解元组时忽略特定的元素。如果元组有三个元素,则结构化绑定时也需要三个变量。

\mySamllsection{tie}

如果想在没有结构化绑定的情况下分解元组,可以使用 std::tie() 辅助函数,它生成一个引用的tuple。以下示例首先创建了一个由整数、字符串和布尔值组成的元组。然后它创建了三个变量——一个整数、一个字符串和一个布尔值——并将这些变量的值写入控制台。tie(i, str, b) 调用创建了一个包含对 i、对 str 和对 b 的引用的tuple。赋值运算符用于将元组 t1 分配给 tie() 的结果。因为 tie() 的结果是一个引用的元组,所以赋值实际上改变了三个单独变量中的值,如下面的输出所显示:

\begin{cpp}
tuple t1 { 16, "Test"s, true };
int i { 0 };
string str;
bool b { false };
println("Before: i = {}, str = \"{}\", b = {}", i, str, b);
tie(i, str, b) = t1;
println("After: i = {}, str = \"{}\", b = {}", i, str, b);
\end{cpp}

结果为:

\begin{shell}
Before: i = 0, str = "", b = false
After: i = 16, str = "Test", b = true
\end{shell}

使用 tie() 时,可以忽略您不想分解的某些元素。对于要分解的元素,使用特殊的 std::ignore 值即可。例如,可以通过将上一个示例中的 tie() 语句替换为以下内容,来忽略 t1 中的字符串元素:

\begin{cpp}
tie(i, ignore, b) = t1;
\end{cpp}

\mySubsubsection{24.3.2.}{连接}

可以使用 std::tuple\_cat() 将两个元组连接成一个。在以下示例中,t3 的类型是 tuple<int, string, bool, double, string>:

\begin{cpp}
tuple t1 { 16, "Test"s, true };
tuple t2 { 3.14, "string 2"s };
auto t3 { tuple_cat(t1, t2) };
println("t3 = {}", t3);
\end{cpp}

输出为:

\begin{shell}
t3 = (16, "Test", true, 3.14, "string 2")
\end{shell}

\mySubsubsection{24.3.3.}{比较}

tuple支持所有比较运算符。为了使比较运算符工作,tuple中存储的元素类型也应支持比较运算符。以下是示例:

\begin{cpp}
tuple t1 { 123, "def"s };
tuple t2 { 123, "abc"s };
if (t1 < t2) { println("t1 < t2"); }
else { println("t1 >= t2"); }
\end{cpp}

输出为:

\begin{shell}
t1 >= t2
\end{shell}

tuple比较可以用来轻松实现自定义类型的字典顺序比较运算符,这些自定义类型具有多个数据成员。例如,有如下类及其三个数据成员:

\begin{cpp}
class Foo
{
    public:
        explicit Foo(int i, string s, bool b)
            : m_int { i }, m_str { move(s) }, m_bool { b } { }
    private:
        int m_int;
        string m_str;
        bool m_bool;
};
\end{cpp}

正确实现一个完整的比较运算符集合,以便比较 Foo 的所有数据成员,可以通过显式默认 operator<=> 来实现:

\begin{cpp}
auto operator<=>(const Foo& rhs) const = default;
\end{cpp}

这会自动比较所有数据成员,但如果类的语义是这样的,即两个实例之间的比较应该只考虑数据成员的一个子集,那么正确实现这样一个类的完整的比较运算符集合就很困难了!但有了 std::tie() 和三向比较运算符(operator<=>),这就会变得很容易,只需要一行代码。以下是 Foo 类中仅比较 m\_int 和 m\_str 数据成员,并忽略 m\_bool 的 operator<=> 实现:

\begin{cpp}
auto operator<=>(const Foo& rhs) const
{
    return tie(m_int, m_str) <=> tie(rhs.m_int, rhs.m_str);
}
\end{cpp}

以下是使用示例:

\begin{cpp}
Foo f1 { 42, "Hello", false };
Foo f2 { 42, "World", false };
println("{}", (f1 < f2)); // Outputs true
println("{}", (f2 > f1)); // Outputs true
\end{cpp}

\mySubsubsection{24.3.4.}{make\_from\_tuple}

std::make\_from\_tuple<T>() 构造一个给定类型 T 的对象,并将给定tuple中的元素作为 T 构造函数的参数传递。例如,有如下类:

\begin{cpp}
class Foo
{
    public:
        explicit Foo(string str, int i) : m_str { move(str) }, m_int { i } { }
    private:
        string m_str;
        int m_int;
};
\end{cpp}

可以使用 make\_from\_tuple():

\begin{cpp}
tuple myTuple { "Hello world.", 42 };
auto foo { make_from_tuple<Foo>(myTuple) };
\end{cpp}

传递给 make\_from\_tuple() 的参数不一定是tuple,但必须支持 std::get<>() 和 tuple\_size,std::array 和 pair 也满足这些要求。

这个函数在日常使用中并不实用,但在编写使用模板和模板元编程的代码时会派上用场。

\mySubsubsection{24.3.5.}{apply}

std::apply() 调用一个给定的可调用对象,并将给定tuple中的元素作为参数传递。以下是示例:

\begin{cpp}
int add(int a, int b) { return a + b; }
...
println("{}", apply(add, tuple { 39, 3 }));
\end{cpp}

与 make\_from\_tuple() 一样,这个函数在编写使用模板和模板元编程的通用代码时,会更有用。







