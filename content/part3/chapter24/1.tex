std::variant定义在 <variant> 头文件中,可以存储一组给定类型中的一种值。当定义一个 variant 时,必须指定它可能包含的类型。例如,下面的代码定义了一个 variant,可以包含整数、字符串或浮点值,但一次只能存储一个:

\begin{cpp}
variant<int, string, float> v;
\end{cpp}

variant 的模板类型参数必须是唯一的;例如,variant<int,int> 是无效的。一个默认构造的 variant 包含其第一个类型(这个例子中是 int)的默认构造值。如果想能够默认构造一个 variant,必须确保 variant 的第一个类型是可以默认构造的。例如,下面的代码因为 Foo 不是默认构造的,所以无法编译:

\begin{cpp}
class Foo { public: Foo() = delete; Foo(int) {} };
class Bar { public: Bar() = delete; Bar(int) {} };
...
variant<Foo, Bar> v;
\end{cpp}

实际上,Foo 和 Bar 都不是默认构造的。如果仍然希望能够默认构造这样的 variant,那么可以将 std::monostate(一个行为良好的空值替代品),作为 variant 的第一个类型:

\begin{cpp}
variant<monostate, Foo, Bar> v;
\end{cpp}

可以使用赋值运算符将值存储在 variant 中:

\begin{cpp}
variant<int, string, float> v;
v = 12;
v = 12.5f;
v = "An std::string"s;
\end{cpp}

variant 在任何给定时间只能包含一个值。所以,通过这三个赋值语句,首先将整数 12 存储在 variant 中,然后 variant 修改为一个浮点值,最后 variant 再次被修改为包含一个字符串值。

可以使用 index() 成员函数来获取,当前存储在 variant 中的值的零基索引,可以使用 std::holds\_alternative() 函数模板,来确定 variant 是否当前包含特定类型的值:

\begin{cpp}
println("Type index: {}", v.index());
println("Contains an int: {}", holds_alternative<int>(v));
\end{cpp}

输出为:

\begin{shell}
Type index: 1
Contains an int: false
\end{shell}

使用 std::get<index>() 或 get<T>() 从 variant 中检索值,其中 index 是想要检索的类型的零基索引,T 是想要检索的类型。如果使用类型索引或类型,函数将抛出一个 bad\_variant\_access 异常:

\begin{cpp}
println("{}", get<string>(v));
try {
    println("{}", get<0>(v));
} catch (const bad_variant_access& ex) {
    println("Exception: {}", ex.what());
}
\end{cpp}

输出为:

\begin{shell}
An std::string
Exception: bad variant access
\end{shell}

为了避免异常,可以使用 std::get\_if<index>() 或 get\_if<T>() 辅助函数。这些函数接受一个指向 variant 的指针,并返回一个指向请求值的指针,或者在错误时返回 nullptr:

\begin{cpp}
string* theString { get_if<string>(&v) };
int* theInt { get_if<int>(&v) };
println("Retrieved string: {}", (theString ? *theString : "n/a"));
println("Retrieved int: {}", (theInt ? to_string(*theInt) : "n/a"));
\end{cpp}

输出为:

\begin{shell}
Retrieved string: An std::string
Retrieved int: n/a
\end{shell}

std::visit() 是一个辅助函数,可以使用它在variant上使用用访问者模式 。访问者必须是一个可调用的对象,例如一个函数、一个 Lambda 表达式或一个函数对象,它可以接受 variant 中可能存储的类型。一个简单的例子只是使用一个通用的 Lambda 作为第一个参数传递给 visit() 的可调用的对象:

\begin{cpp}
visit([](auto&& value) { println("Value = {}", value); }, v);
\end{cpp}

输出为:

\begin{shell}
Value = An std::string
\end{shell}

如果想以不同的方式处理 variant 中存储的每个类型,可以编写自己的访问者类。假设有以下访问者类,定义了多个重载的函数调用运算符,一个用于 variant 中可能存储的每个可能类型。这个实现将所有它的函数调用运算符标记为静态的(自 C++23 开始),因为不需要访问非静态成员函数或数据成员:

\begin{cpp}
class MyVisitor
{
    public:
        static void operator()(int i) { println("int: {}", i); }
        static void operator()(const string& s) { println("string: {}", s); }
        static void operator()(float f) { println("float: {}", f); }
};
\end{cpp}

可以这样使用 std::visit():

\begin{cpp}
visit(MyVisitor{}, v);
\end{cpp}

结果是根据 variant 中当前存储的值,调用适当的函数调用运算符。这个例子中的输出如下所示:

\begin{shell}
string: An std::string
\end{shell}

variant 不能存储数组,并且与第 1 章中介绍的 optional 一样,也不能存储引用。可以存储指针或 reference\_wrapper<T> 或 reference\_wrapper<const T> 的实例(请参阅第 18 章)。

\begin{myNotic}{NOTE}
C++23 开始,variant 是一个 constexpr 类,因此可以在编译时使用。有关 constexpr 类的更多信息,参见第 9 章。
\end{myNotic}

\CXXTwentythreeLogo{-40}{35}




