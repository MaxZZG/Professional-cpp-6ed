\CXXTwentythreeLogo{-40}{15}

第14章中解释了,C++中的一个函数只能返回一个类型,如果函数可能会失败,应该通知调用者关于失败的信息。过去,有几个选择来做到这一点,可以抛出一个带有错误详细信息的异常,或者可以尝试提出一个特殊的返回类型值来表示错误。

例如,一个函数返回一个指针,该函数可以在发生错误时返回nullptr。如果一个函数只为其正常操作返回正整数,可以使用负值来表示不同的错误等。但提出这样一个特殊值并不总是可行。如果一个函数的返回类型是int,有效的返回值范围是整数的整个范围,就没有剩余的整数可以用作特殊错误值了。这时,可以使用std::optional类型,这是一个可以包含某种类型的值或为空的类型,函数可以返回一个空的optional来表示错误。

但是当函数的调用者收到一个空的optional时,就无法知道到底出了什么问题;即,函数不能返回错误的真正原因。这些问题可以通过std::expected得到了解决,它在<introduced>中定义,并随着C++23引入,是一个接受两个模板类型参数的类模板:

\begin{itemize}
\item
T: 期望值的类型

\item
E: 错误值的类型,也称为意外值
\end{itemize}

expected永远不会为空,总是包含类型T的值或类型E的值。这是与optional最大的不同,它可以为空,让使用者不知道为什么它为空。因此,返回expected的函数应该返回期望类型的值,或者返回错误类型的值来表示失败的准确原因。错误类型可以是想要的类型,可以是简单的整数或复杂的类。通常,最好将错误编码放在能够表示尽可能多错误信息的类中,例如:解析某些数据文件失败的文件名、行号和列号。

expected<T,E>的一个实例可以从类型T的值隐式创建,就像optional<T>一样。要创建一个包含错误类型E值的expected<T,E>实例,必须使用std::unexpected<E>,默认构造的expected<T,E>包含默认构造的期望类型T的值。这与optional不同,默认构造的optional是空的!换句话说,默认构造的expected表示成功,而默认构造的optional表示错误。

下面是一个接收字符串,并尝试将字符串解析为整数的函数。stoi()函数在字符串不能表示整数时抛出invalid\_argument,解析的整数大于可以表示为int的范围时抛出out\_of\_range。假设不希望parseInteger()抛出这样的异常,而是返回一个expected。函数捕获这两个异常,并将它们转换为字符串,这是返回expected的错误类型。

\begin{cpp}
expected<int, string> parseInteger(const string& str)
{
    try { return stoi(str); }
    catch (const invalid_argument& e) { return unexpected { e.what() }; }
    catch (const out_of_range& e) { return unexpected { e.what() }; }
}
\end{cpp}

expected具有以下成员函数。除了error()之外,都与optional的类似命名的成员函数相似。

\begin{itemize}
\item
has\_value() 和 operator bool: 如果expected有类型T的值,则返回true,否则返回false。

\item
value(): 返回类型T的值。如果在包含类型E值的expected上调用,则抛出std::bad\_expected\_access。

\item
operator* 和 ->: 访问类型T的值。如果expected不包含类型T的值,则行为未定义。

\item
error(): 返回类型E的错误。如果expected不包含类型E的值,则行为未定义。

\item
value\_or(): 返回类型T的值,或者如果expected不包含这样的值,则返回另一个给定的值。
\end{itemize}

下面的例子展示了这些成员函数的大多数:

\begin{cpp}
auto result1 { parseInteger("123456789") };
if (result1.has_value()) { println("result1 = {}", result1.value()); }
if (result1) { println("result1 = {}", *result1); }

println("result1 = {}", result1.value_or(0));

auto result2 { parseInteger("123456789123456") };
if (!result2) { println("result2 contains an error: {}", result2.error()); }

auto result3 { parseInteger("abc") };
if (!result3) { println("result3 contains an error: {}", result3.error()); }
\end{cpp}

输出如下:

\begin{shell}
result1 = 123456789
result1 = 123456789
result1 = 123456789
result2 contains an error: stoi argument out of range
result3 contains an error: invalid stoi argument
\end{shell}

expected还支持一些函数式编程中的操作,比如and\_then()、transform()、or\_else()和transform\_error()。前三个操作与optional支持的类似。

\begin{itemize}
\item
transform(F): 如果*this包含期望值,则返回调用F的结果,并将期望值作为参数;否则,直接返回原expected对象。

\item
and\_then(F): 如果*this包含期望值,则返回调用F的结果(结果必须是一个expected),并将期望值作为参数;否则,直接返回原expected对象。

\item
or\_else(F): 如果*this包含期望值,则返回*this;否则,返回调用F的结果(结果必须是一个expected),并将错误值作为参数。

\item
transform\_error(F): 如果*this包含期望值,则返回*this;否则,返回一个包含转换后的错误值的expected对象,转换是通过调用F并将错误值作为参数完成。
\end{itemize}

下面是使用and\_then()的一个例子。就像在optional上使用一元操作一样,在调用parseInteger()之前,不需要显式检查结果是否包含期望值。错误处理会自动完成。

\begin{cpp}
auto transformedResult { parseInteger("123456789")
    .and_then([](int value) -> expected<int, string> { return value * 2; }) };
\end{cpp}

expected的错误类型可以是任何类型,使用本章前面讨论的variant类型,也可以返回多种错误类型。 例如,parseInteger()函数可以针对两种错误情况返回两种不同的错误类型。以下函数返回两种自定义类型OutOfRange和InvalidArgument的错误:


\begin{cpp}
expected<int, variant<OutOfRange, InvalidArgument>>
    parseInteger(const string& str) { ... }
\end{cpp}

总之,optional和expected有些相似。特定情况下,请根据以下规则决定使用哪一个:

\begin{myNotic}{NOTE}
当错误是未预期的时候,使用expected。此时expected应该准确表示错误的原因,以便调用者可以适当地处理错误。

如果可以接受没有值,并且不需要为缺失的值提供原因,则使用optional。例如,对于接受可选输入参数的函数或返回类型为可能没有值的函数(如查找相关函数)。
\end{myNotic}

\mySubsubsection{24.5.1.}{异常、错误返回码和expected}

处理函数中的错误有三种主要方式。函数可以抛出异常,这在第14章中详细讨论过;可以返回错误码;或者返回expected。下表根据std::expected的官方提案P0323R12总结了它们的特点:

% Please add the following required packages to your document preamble:
% \usepackage{longtable}
% Note: It may be necessary to compile the document several times to get a multi-page table to line up properly
\begin{longtable}{|l|l|l|l|}
\hline
&
\textbf{异常} &
\textbf{错误返回码} &
\textbf{expected} \\ \hline
\endfirsthead
%
\endhead
%
\textbf{可见性} &
\begin{tabular}[c]{@{}l@{}}除非阅读函数文档或\\分析代码,否则不可\\见。\end{tabular} &
\begin{tabular}[c]{@{}l@{}}从函数原型中立即\\可见。但容易忽略\\返回值。\end{tabular} &
\begin{tabular}[c]{@{}l@{}}从函数原型中立即可见。\\不能被忽略,因为它包\\含函数的结果。\end{tabular} \\ \hline
\textbf{详情} &
\begin{tabular}[c]{@{}l@{}}包含尽可能多的错误\\详情。\end{tabular} &
\begin{tabular}[c]{@{}l@{}}通常只是一个简单\\的整数。\end{tabular} &
\begin{tabular}[c]{@{}l@{}}包含尽可能多的错误详\\情。\end{tabular} \\ \hline
\textbf{代码噪音} &
\begin{tabular}[c]{@{}l@{}}允许编写干净的代码\\,并将错误处理分开。\end{tabular} &
\begin{tabular}[c]{@{}l@{}}错误处理与正常流\\程交织,使代码更\\难读和维护。\end{tabular} &
\begin{tabular}[c]{@{}l@{}}允许编写干净的代码。\\由于有一元操作,错误\\处理不会与正常流程交\\织。\end{tabular} \\ \hline
\end{longtable}

