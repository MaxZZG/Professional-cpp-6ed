通过解决下面的练习,可以练习本章讨论的内容。所有练习的解决方案都可以在本书的网站\url{www.wiley.com/go/proc++6e}下载到源码。然而,若在练习中卡住了,在从网站上寻找解决方案之前,可以考虑先重读本章的部分内容,试着自己找到答案。

\begin{itemize}
\item
\textbf{练习24-1}:第14章解释了C++中的错误处理,并解释说基本上有两种主要选项:要么使用错误代码,要么使用异常。我推荐使用异常进行错误处理,但在这个练习中,将使用错误代码。编写一个简单的Error类,只存储单个消息,有一个构造函数来设置消息,并且有一个getter来检索消息。接下来,编写一个名为getData()的函数,有一个名为fail的布尔参数。如果fail为false,函数返回一个包含一些数据的向量;否则,返回Error的一个实例。不允许使用引用-非常量输出参数,尝试想出一个解决方案,并且不使用C++23的std::expected类模板。在main()函数中测试实现。

\item
\textbf{练习24-2}:修改练习24-1中的解决方案,使用C++23的std::expected类模板,并了解它如何使解决方案更容易阅读和理解。

\item
\textbf{练习24-3}:大多数命令行应用程序接受命令行参数。本书中的大多数,main函数仅仅是main()。然而,main()也可以接受参数:main(int argc, char** argv),其中argc是命令行参数的数量,argv是一个字符串数组,每个参数一个字符串。假设在这个练习中,命令行参数的格式是name=value。编写一个函数,可以解析单个参数,并返回一个包含参数名称和包含值作为布尔值的变体(如果值可以解析为布尔值(true或false)、整数如果值可以解析为整数,否则为字符串)的pair。为了分割name=value字符串,可以使用正则表达式(见第21章)。要解析整数,可以使用第2章中介绍的函数。在main()函数中,遍历所有命令行参数,解析它们,并使用holds\_alternative()将解析结果输出到标准输出。

\item
\textbf{练习24-4}:修改练习24-3中的解决方案。不是使用holds\_alternative(),使用访问者将解析结果输出到标准输出。

\item
\textbf{练习24-5}:修改练习24-4中的解决方案,使用tuple完成。
\end{itemize}


