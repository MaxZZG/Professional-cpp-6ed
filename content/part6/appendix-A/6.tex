
Interviewers rarely ask questions about designing reusable code. This omission is unfortunate because having programmers on staff who can write only single-purpose code can be detrimental to a programming organization. Occasionally, you’ll find a company that is savvy on code reuse and asks about it in their interviews. Such a question is an indication that it might be a good company to work for.

\mySubsubsection{A.6.1.}{Things to Remember}

\begin{itemize}
\item
The principle of abstraction

\item
The creation of subsystems and class hierarchies

\item
The general rules for good interface design, which are interfaces with no implementation details and no public data members

\item
When to use templates for polymorphism and when to use inheritance
\end{itemize}

\mySubsubsection{A.6.2.}{Types of Questions}

The interviewer might ask you to explain the principle of abstraction and its benefits and to give some concrete examples.

Questions about reuse will almost certainly be about previous projects on which you have worked.
For example, if you worked at a company that produced both consumer and professional videoediting applications, the interviewer may ask how code was shared between the two applications. Even if you aren’t explicitly asked about code reuse, you might be able to sneak it in. When you’re describing some of your past work, tell the interviewer if the modules you wrote were used in other projects. Even when answering apparently straight coding questions, make sure to consider and mention the interfaces involved. As always, be careful not to expose intellectual property from your previous jobs, though.
















