

字符串在几乎每种类型的应用程序中都至关重要,因此面试官很可能至少会问一个与C++字符串处理相关的问题。

\mySubsubsection{A.2.1.}{需要记住的事项}

\begin{itemize}
\item
std::string和string\_view类

\item
优先使用const string\&或string作为函数返回类型,而不是string\_view

\item
C++的std::string类与C风格(char*)字符串的区别,包括为什么应该避免使用C风格字符串

\item
将字符串转换为整数和浮点数等数字类型,反之亦然

\item
使用std::format()进行字符串格式化

\item
使用std::print()和println()进行字符串打印 (C++23)

\item
如何一次性格式化和打印整个范围 (C++23)

\item
原始字符串字面量

\item
本地化的重要性

\item
Unicode背后的理念

\item
locale和facet的高层次概念

\item
正则表达式的概念
\end{itemize}

\mySubsubsection{A.2.2.}{问题的类型}

面试官可能会问如何将两个字符串拼接在一起。对于这个问题,面试官想要了解你是以C++程序员,还是以C程序员的方式思考。如果遇到这样的问题,你应该解释std::string类,并展示如何使用它来拼接两个字符串。同时,也应该提到字符串类会自动处理所有内存管理,并与C风格字符串进行对比。

大多数面试官不会询问关于本地化的具体细节。如果问及本地化经验,确保提到从项目开始时就考虑全球使用的重要性。

你可能会被问到locale和facet的基本概念。尽管可能不需要解释确切的语法,但应该解释它们可以根据特定语言或国家的规则来格式化文本和数字。

你可能会问到关于Unicode的问题,但几乎可以肯定的是,这个问题是要求你解释Unicode背后的理念和基本概念,而不是实现细节。确保理解Unicode的高层次概念,并能解释它们在本地化上下文中的使用。还应该知道不同的Unicode字符编码选项,如UTF-8和UTF-16,而不需要具体细节。

正如第21章所讨论的,正则表达式的语法可能令人望而生畏。面试官不太可能问你正则表达式的细节,但应该能够解释正则表达式的概念,以及可以使用它们进行哪些字符串操作。





