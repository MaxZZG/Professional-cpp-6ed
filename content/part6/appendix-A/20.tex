
You should be suspicious if you go through the complete interview process with a company and the interviewers do not ask any process questions—it may mean that they don’t have any process in place or that they don’t care about it. Alternatively, they might not want to scare you away with their process behemoth.

Having a well defined process in place is important. Similarly, version control should be mandatory for any project of any size.

Most of the time, you’ll get a chance to ask questions regarding the company. I suggest you consider asking about the company’s engineering processes and version control solution as one of your standard questions.

\mySubsubsection{A.20.1.}{Things to Remember}

\begin{itemize}
\item
Traditional life-cycle models

\item
The trade-offs of different models

\item
The main principles behind Extreme Programming

\item
Scrum as an example of an agile process

\item
Other processes you have used in the past

\item
What version control is
\end{itemize}

\mySubsubsection{A.20.2.}{Types of Questions}

The most common question you’ll be asked is to describe the process that your previous employer used. Be careful, though, not to disclose any confidential information. When answering, you should mention what worked well and what failed, but try not to denounce any particular methodology. The methodology you criticize could be the one that your interviewer uses.

Almost every candidate is listing Scrum/Agile as a skill these days. If the interviewer asks you about Scrum, she probably doesn’t want you to simply recite the textbook definition—the interviewer knows that you can read the table of contents of a Scrum book. Instead, pick a few ideas from Scrum that you find appealing. Explain each one to the interviewer along with your thoughts on it. Try to engage the interviewer in a conversation, proceeding in a direction in which she is interested based on the cues that she gives.

If you get a question regarding version control, it will most likely be a high-level question. You should explain why it should be used and what its benefits are. You could also explain the difference between local, client/server, and distributed solutions, and possibly explain how version control was implemented by your previous employer.

