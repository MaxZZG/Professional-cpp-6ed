
Object-oriented design questions are used to weed out C programmers who merely know what a class is, from C++ programmers who actually use the object-oriented features of the language. Interviewers don’t take anything for granted; even if you’ve been using object-oriented languages for years, they may still want to see evidence that you understand the methodology.

\mySubsubsection{A.5.1.}{Things to Remember}

\begin{itemize}
\item
The differences between the procedural and object-oriented paradigms

\item
The difference between a class and an object

\item
Expressing classes in terms of components, properties, and behaviors

\item
Is-a and has-a relationships

\item
The tradeoffs involved in multiple inheritance
\end{itemize}

\mySubsubsection{A.5.2.}{Types of Questions}

There are typically two ways to ask object-oriented design questions: you can be asked to define an object-oriented concept, or you can be asked to sketch out an object-oriented hierarchy. The former is pretty straightforward. Remember that examples might earn you extra credit.

If you’re asked to sketch out an object-oriented hierarchy, the interviewer will usually provide a simple application, such as a card game, for which you should design a class hierarchy. Interviewers often ask design questions about games because those are applications with which most people are already familiar. They also help lighten the mood a bit when compared to questions about things like database implementations. The hierarchy you generate will, of course, vary based on the game or application they are asking you to design. Here are some points to consider:

\begin{itemize}
\item
The interviewer wants to see your thought process. It is very important to think aloud, brainstorm, and engage the interviewer in a discussion. Don’t be afraid to erase and go in a different direction!

\item
The interviewer may assume that you are familiar with the application. If you’ve never heard of blackjack and you get a question about it, ask the interviewer to clarify or change the question.

\item
Unless the interviewer gives you a specific format to use when describing the hierarchy, it’s recommended that your class diagrams take the form of inheritance trees with rough lists of member functions and data members for each class.

\item
You may have to defend your design or revise it to take added requirements into consideration. Try to gauge whether the interviewer sees actual flaws in your design, or whether she just wants to put you on the defensive to see your skills of persuasion.
\end{itemize}















