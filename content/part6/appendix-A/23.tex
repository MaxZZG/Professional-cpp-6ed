Engineering organizations look for candidates who are able to debug their own code as well as code that they’ve never seen before. Technical interviews often attempt to size up your debugging muscles.

\mySubsubsection{A.23.1.}{Things to Remember}

\begin{itemize}
\item
Debugging doesn’t start when bugs appear; you should instrument your code ahead of time, so you’re prepared for bugs when they arrive.

\item
Logs and debuggers are your best tools.

\item
You should know how to use assertions.

\item
The symptoms that a bug exhibits may appear to be unrelated to the actual cause.

\item
Object diagrams can be helpful in debugging, especially during an interview.
\end{itemize}

\mySubsubsection{A.23.2.}{Types of Questions}

During an interview, you might be challenged with an obscure debugging problem. Remember that the process is the most important thing, and the interviewer probably knows that. Even if you don’t find the bug during the interview, make sure that the interviewer knows what steps you would go through to track it down. If the interviewer hands you a function and tells you that it crashes during execution, she should award you just as many points or even more if you properly discuss the sequence of steps to find the bug, as if you find the bug right away.
