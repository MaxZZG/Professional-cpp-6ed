
几乎所有的系统,从服务器到笔记本电脑甚至手机,现在都有多个核心的处理器。多线程编程对于利用所有这些核心的功率至关重要。面试官可能会问你几个多线程问题。C++包括一个标准的线程支持库,所以了解它是很好的主意。

\mySubsubsection{A.19.1.}{需要注意的事项}

\begin{itemize}
\item
什么是竞争条件和死锁,以及如何预防它们

\item
std::jthread来创建线程,以及为什么比std::thread更好

\item
原子类型和原子操作

\item
互斥的概念,包括使用不同的互斥量和锁类,为线程提供同步

\item
条件变量及其如何用于向其他线程发送信号

\item
信号量、锁存器和栅栏的概念

\item
Future 和 Promise

\item
跨线程的异常复制和重新抛出

\item
什么是协程,包括一个高层次的概述,以及它们是如何工作的

\item
可等待的标准std::generator(C++23)
\end{itemize}

\mySubsubsection{A.19.2.}{问题的类型}

多线程编程是一个复杂的话题,所以除非正在面试一个特定的多线程编程职位,否则你不需要期望详细的问题。

相反,面试官可能会问你解释多线程代码中可能遇到的不同问题:例如竞争条件、死锁和撕裂。她可能会问你解释为什么需要原子类型和原子操作,可能会要求解释多线程编程背后的基本概念。这是一个很宽泛的问题,但它允许面试官了解你的多线程知识。解释互斥量、信号量、栅栏和锁存器的概念将为你赢得额外分数。还可以提到,标准库中的许多算法都有选项可以并行运行,以提高它们的性能。

编写自己的协程很复杂,但从C++23开始,标准库包含了一个标准的std::generator类型。如果能用一个小例子解释生成器如何工作,将获得额外分数。

