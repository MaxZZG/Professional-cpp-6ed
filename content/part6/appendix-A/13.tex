
管理层有时会避开雇佣最近毕业的或初学者程序员来从事关键(且薪酬较高)的工作,因为他们编写的代码不具备生产质量。你可以通过在面试中展示你的错误处理技能,来证明你的代码不会随机崩溃。

\mySubsubsection{A.13.1.}{需要注意的事项}

\begin{itemize}
\item
异常的语法

\item
将异常作为引用常量来捕获

\item
为什么异常层次优于几个通用的异常

\item
异常被抛出时,调用堆展开的基本工作原理

\item
如何在构造函数和析构函数中处理错误

\item
智能指针如何帮助避免在抛出异常时发生内存泄漏

\item
std::source\_location类作为某些C风格预处理器宏的替代品

\item
std::stacktrace类以在程序执行的时候获取堆栈跟踪,并检查个别堆栈帧 (C++23)
\end{itemize}

\mySubsubsection{A.13.2.}{问题的类型}

准备好讨论错误处理,如果面试官提到它。但不要试图将错误处理硬塞进讨论中,并迫使面试官与你讨论它,如果真的想专注于其他事情,比如数据结构或算法。

面试官可能会问你不同的错误处理策略,可能会被要求给出一个高层次的概述,说明当异常被抛出时,堆展开的工作原理,而不涉及实现细节。

当然,并非所有程序员都欣赏异常,有些人甚至因为性能原因而反对它们。如果面试官要求你做一些不使用异常的事情,你必须回归到传统的nullptr检查和错误代码。

了解source\_location和stacktrace类的知识及其最重要的用例,将为你赢得额外的分数。

面试官还可以问你是否会避免使用异常,因为它们的性能影响。你应该解释说,使用现代编译器,抛出异常可能会有性能开销,但只是有能够处理潜在异常的代码几乎不会有性能开销。

