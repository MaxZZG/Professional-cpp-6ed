
效率问题在面试中非常常见,因为许多组织面临着其代码的可扩展性问题,并需要了解性能的开发者。

\mySubsubsection{A.21.1.}{需要注意的事项}

\begin{itemize}
\item
语言层面的效率很重要,只能走这么远;设计层面的选择最终更为重要。

\item
具有不良复杂度的算法,如二次算法,应避免使用。

\item
引用参数更有效率,因为它们避免了复制。

\item
对象池可以帮助避免创建和销毁对象的开销。

\item
性能分析至关重要,以确定哪些操作实际上消耗了最多的时间,这样就不会浪费精力试图优化不是性能瓶颈的代码。
\end{itemize}

\mySubsubsection{A.21.2.}{问题的类型}

通常,面试官会用自己的产品作为例子来提出效率问题。有时,面试官会描述一个较旧的设计以及她经历的一些与性能相关的症状。候选人应提出一个新的设计来缓解这个问题,但这种问题有一个主要问题:你有多大几率会提出与公司解决实际问题时所采用的相同解决方案?因为几率很低,需要格外小心地证明你的设计。可能不会提出实际的解决方案,但答案仍然可能是正确的,甚至比公司的最新设计还要好。

其他类型的效率问题可能会要求调整一些C++代码以提高性能或迭代算法。例如,面试官可能会向你展示包含多余的复制或低效循环的代码。

面试官还可能会问关于性能分析工具(如gprof或Visual C++)的高层次描述、好处,以及为什么要使用它们。






