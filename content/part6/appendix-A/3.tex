
Anybody who’s coded in the professional world has had a co-worker who writes messy code. That is something companies don’t want, so interviewers sometimes attempt to determine a candidate’s style skills.

\mySubsubsection{A.3.1.}{Things to Remember}

\begin{itemize}
\item
Style matters, even during interview questions that aren’t explicitly style related.

\item
Well-written code doesn’t need extensive comments.

\item
Comments can be used to convey meta information.

\item
Decomposition is the practice of breaking up code into smaller pieces.

\item
Refactoring is the act of restructuring your code, for example to clean up previously written code.

\item
Naming techniques are important, so pay attention to how you name your variables, classes, and so on.
\end{itemize}

\mySubsubsection{A.3.2.}{Types of Questions}

Style questions can come in a few different forms. A friend of mine was once asked to write the code for a relatively complex algorithm on a whiteboard. As soon as he wrote the first variable name, the interviewer stopped him and told him he passed. The question wasn’t about the algorithm; it was just a red herring to see how well he named his variables. More commonly, you may be asked to submit code that you’ve written or to give your opinions on style.

You need to be careful when a potential employer asks you to submit a code sample. You probably cannot legally submit code that you wrote for a previous employer. You also have to find a piece of code that shows off your skills without requiring too much background knowledge. For example, you wouldn’t want to submit your master’s thesis on high-speed image rendering to a company that is interviewing you for a database administration position. If the company gives you a specific program to write, that’s a perfect opportunity to show off what you’ve learned in this book. If the potential employer doesn’t specify the program, you could consider writing a small program specifically to submit to the company. Instead of selecting some code you’ve already written, start from scratch to produce code that is relevant to the job and highlights good style.

If you have documentation that you have written and that can be released, meaning it is not confidential, use it to show your skills to communicate; it will give you extra points. Websites you have built or maintained, and articles you have submitted to places like Stack Overflow (stackoverflow.com), CodeGuru (codeguru.com), CodeProject (codeproject.com), and so on, are useful. This tells the interviewer that you can not only write code, but also communicate to others how to use that code effectively.

If you are contributing to active open-source projects, for example on GitHub (github.com), you can score extra points. Even better would be if you have your own open-source project that you actively maintain. That’s the perfect opportunity to show off your coding style and your communication skills.
Profile pages on websites such as GitHub are taken as part of your résumé by certain employers.













