
实际工作中编写代码的人都有过,与编写混乱代码的同事合作的经历。公司不希望这样的事情发生,所以面试官有时会试图确定候选人的风格技能。

\mySubsubsection{A.3.1.}{需要注意的事项}

\begin{itemize}
\item
风格很重要,即使是在与风格无关的面试问题中。

\item
编写良好的代码,不需要大量的注释。

\item
注释可以用来传达元信息。

\item
分解是将代码拆分为更小的做法。

\item
重构是重构代码的行为,例如清理以前编写的代码。

\item
命名技术很重要,所以要注意你如何命名变量、类等。
\end{itemize}

\mySubsubsection{A.3.2.}{问题的类型}

风格问题可以以几种不同的形式出现。我的一位朋友曾被要求在白板上编写一个相对复杂的算法。他一写出第一个变量名,面试官就告诉他通过了。这个问题不是关于算法的;这只是个诱饵,看看他如何命名变量。更常见的是,可能会被要求提交你写的代码样本,或者表达你对风格的看法。

当潜在雇主要求你提交代码样本时,需要小心。你可能无法合法地提交为前雇主编写的代码。你还需要找到一个展示你技能的代码样本,而不需要太多的背景知识。例如,不应该向一个正在面试你担任数据库管理员职位的公司,提交自己的硕士论文,关于高速图像渲染。如果公司给你一个特定的程序来编写,那是一个展示在这本书中学到的东西的完美机会。如果潜在雇主没有指定程序,可以考虑写一个小的程序专门提交给公司。不要选择你以前编写的某些代码,从零开始产生与工作相关的代码,并突出良好的风格。

如果有可以发布的文档,也就是说它不是保密的,可以用它来展示你的技能,有效地传达信息;这将为你赢得额外的好感。你建立或维护的网站,以及在Stack Overflow(stackoverflow.com)、CodeGuru(codeguru.com)、CodeProject(codeproject.com)等地方提交的文章都是有用的。这告诉面试官你不仅能编写代码,还能有效地向他人传达如何有效使用这些代码。

如果你正在为活跃的开源项目做贡献,例如在GitHub(github.com),可以得到额外的分数。更好的是,如果有自己的开源项目,并且你积极维护它。这是展示你的编码风格和沟通技巧的完美机会。

某些雇主将网站如GitHub上的个人资料页面,也视为简历的一部分。













