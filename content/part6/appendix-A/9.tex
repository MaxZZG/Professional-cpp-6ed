
关于继承的问题通常与类的问题形式相同。面试官还可能会让你实现一个类层次结构,以展示已经足够熟悉C++,能够编写派生类而不用查书。

\mySubsubsection{A.9.1.}{需要注意的事项}

\begin{itemize}
\item
继承的语法

\item
从派生类的角度来看,private和protected的区别

\item
成员函数重写和虚拟

\item
重载、重写和隐藏之间的区别

\item
为什么基类析构函数应该声明为虚

\item
链式构造函数

\item
向上转型和向下转型的细节

\item
C++中的不同类型的强制转换

\item
多态的原则

\item
纯虚成员函数和抽象基类

\item
多重继承

\item
运行时类型信息(RTTI)

\item
继承的构造函数

\item
类的final关键字

\item
成员函数上的override和final关键字
\end{itemize}

\mySubsubsection{A.9.2.}{问题的类型}

继承问题中的许多陷阱都与细节的正确性有关。当编写一个基类时,不要忘记使成员函数为虚。如果将所有成员函数都标记为虚,请准备好解释这个决定。能够解释什么是虚函数,以及它是如何工作的。此外,不要忘记在派生类定义中在基类名称前加上public关键字(例如,class Derived : public Base)。在面试中,不太可能被要求执行非公有继承。

更具挑战性的继承问题与基类和派生类之间的关系有关。务必了解不同访问级别的运作方式,尤其是private和protected之间的区别。提醒自己了解所谓的切片现象,当某些类型的强制转换导致类丢失其派生类信息。










