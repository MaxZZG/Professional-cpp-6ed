The following header files define the available Standard Library algorithms, iterators, and allocators, and the ranges library:

% Please add the following required packages to your document preamble:
% \usepackage{longtable}
% Note: It may be necessary to compile the document several times to get a multi-page table to line up properly
\begin{longtable}{|l|l|}
\hline
\textbf{HEADER} &
\textbf{CONTENTS} \\ \hline
\endfirsthead
%
\endhead
%
\textless{}algorithm\textgreater{} &
\begin{tabular}[c]{@{}l@{}}Prototypes for most of the algorithms in the Standard Library, and\\ min(), max(), minmax(), and clamp(). See Chapter 20, “Mastering\\ Standard Library Algorithms.”\end{tabular} \\ \hline
\textless{}bit\textgreater{} &
\begin{tabular}[c]{@{}l@{}}Defines the endian class enumeration, see Chapter 34, “Developing\\ Cross-Platform and Cross-Language Applications,” and provides\\ function prototypes to perform low-level operations on bit\\ sequences, such as bit\_ceil(), rotl(), countl\_zero(), and\\ more, see Chapter 16, “Overview of the C++ Standard Library.”\end{tabular} \\ \hline
\textless{}execution\textgreater{} &
\begin{tabular}[c]{@{}l@{}}Defines the execution policy types for use with the Standard Library\\ algorithms. See Chapter 20.\end{tabular} \\ \hline
\textless{}functional\textgreater{} &
\begin{tabular}[c]{@{}l@{}}Defines the built-in function objects, negators, binders, and adaptors.\\ See Chapter 19, “Function Pointers, Function Objects, and Lambda\\ Expressions.”\end{tabular} \\ \hline
\textless{}iterator\textgreater{} &
\begin{tabular}[c]{@{}l@{}}Definitions of iterator\_traits, iterator tags, iterator, reverse\_\\ iterator, insert iterators (such as back\_insert\_iterator), and\\ stream iterators. See Chapter 17, “Understanding Iterators and the\\ Ranges Library.”\end{tabular} \\ \hline
\textless{}memory\textgreater{} &
\begin{tabular}[c]{@{}l@{}}Defines the default allocator and function prototypes for dealing with\\ uninitialized memory inside containers. Also provides unique\_ptr,\\ shared\_ptr, weak\_ptr, make\_unique(), and make\_shared(),\\ introduced in Chapter 7, “Memory Management.”\end{tabular} \\ \hline
\textless{}memory\_resource\textgreater{} &
\begin{tabular}[c]{@{}l@{}}Defines polymorphic allocators and memory resources. See\\ Chapter 25, “Customizing and Extending the Standard Library.”\end{tabular} \\ \hline
\textless{}numeric\textgreater{} &
\begin{tabular}[c]{@{}l@{}}Prototypes for some numerical algorithms: accumulate(), inner\_\\ product(), partial\_sum(), adjacent\_difference(), gcd(),\\ lcm(), and a few others. See Chapter 20.\end{tabular} \\ \hline
\textless{}ranges\textgreater{} &
Provides all functionality for the Ranges library. See Chapter 17. \\ \hline
\textless{}scoped\_allocator\textgreater{} &
\begin{tabular}[c]{@{}l@{}}An allocator that can be used with nested containers such as a\\ vector of strings, or a vector of maps.\end{tabular} \\ \hline
\end{longtable}























