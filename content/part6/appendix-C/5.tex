C++为数值处理提供了一些设施,尽管本书没有详细描述这些功能,但可以参考附录B中列出的标准库参考资料以获取更多信息:

% Please add the following required packages to your document preamble:
% \usepackage{longtable}
% Note: It may be necessary to compile the document several times to get a multi-page table to line up properly
\begin{longtable}{|l|l|}
\hline
\textbf{头文件}                  & \textbf{内容}                                                          \\ \hline
\endfirsthead
%
\endhead
%
\textless{}complex\textgreater{} & 定义了用于处理复数的complex类模板。       \\ \hline
\textless{}numbers\textgreater{} & 提供了多个数学常数,如pi,phi,log2e等。 \\ \hline
\textless{}stdfloat\textgreater (C++23) &
\begin{tabular}[c]{@{}l@{}}提供了float16\_t,float32\_t,float64\_t,float128\_t,和bfloat16\_t固定宽度浮点\\数类型。参见第1章。\end{tabular} \\ \hline
\textless{}valarray\textgreater{} &
\begin{tabular}[c]{@{}l@{}}定义了valarray及相关类和类模板,用于处理数学向量和矩阵。\end{tabular} \\ \hline
\end{longtable}











