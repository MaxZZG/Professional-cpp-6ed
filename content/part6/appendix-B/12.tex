
\begin{itemize}
\item
埃里希·伽玛(Erich Gamma)、理查德·海尔姆(Richard Helm)、拉尔夫·约翰逊(Ralph Johnson)和约翰·维尔利德斯(John Vlissides)。《设计模式:可重用面向对象软件的元素》(Design Patterns: Elements of Reusable Object-Oriented Software)。Addison-Wesley Professional,1994年。ISBN:0-201-63361-2。

\hspace*{\fill}

称为“四人组”(GoF)书籍(因为有四位作者),这是关于设计模式的开创性作品。

\hspace*{\fill}

\item
安德烈·亚历山德雷斯库(Andrei Alexandrescu)。《现代C++设计:泛型编程与设计模式应用》(Modern C++ Design: Generic Programming and Design Patterns Applied)。Addison-Wesley Professional,2001年。ISBN:0-201-70431-5。

\hspace*{\fill}

提出了一种采用高度可重用代码和模式的C++编程方法。

\hspace*{\fill}

\item
约翰·维尔利德斯(John Vlissides)。《模式孵化:应用设计模式》(Pattern Hatching: Design Patterns Applied)。Addison-Wesley Professional,1998年。ISBN:0-201-43293-5。

\hspace*{\fill}

作为GoF书籍的补充,解释了如何实际应用模式。

\hspace*{\fill}

\item
埃里克·弗里曼(Eric Freeman)、伯特·贝茨(Bert Bates)、凯西·西埃拉(Kathy Sierra)和伊丽莎白·罗布森(Elisabeth Robson)。《头脑第一设计模式》(Head First Design Patterns)。O’Reilly Media,2004年。ISBN:0-596-00712-4。

\hspace*{\fill}

这本书不仅仅列出设计模式。作者展示了使用模式的好坏示例,并给出了每个模式背后的坚实理由。

\hspace*{\fill}

\item
维基百科贡献者。软件设计模式。维基百科,自由的百科全书,\url{en.wikipedia.org/wiki/Design_pattern_(computer_science)}(访问日期为2023年8月15日)。

\hspace*{\fill}

包含大量用于计算机编程的设计模式的描述。
\end{itemize}


