
C++始于1982年,作为C语言的继承者,由丹麦计算机科学家Bjarne Stroustrup创建。1985年,《C++编程语言》第一版发布。C++的第一个标准于1998年发布,称为C++98。2003年,C++03发布,进行了一些小的更新。之后,C++的发展一度沉寂,但慢慢地开始积聚动力,最终在2011年有了重大更新,C++11。此后,C++标准委员会开始每三年发布一次更新版本,于是有了C++14、C++17、C++20,以及现在的C++23。随着C++23在2023年的发布,C++虽已年过四旬,但依旧强大。2023年的编程语言排名中,C++位于前四名,可应用在很多硬件上,从小型嵌入式微处理器的设备到多机架超级计算机。除了可观的硬件支持,C++还可以处理各种编程任务,无论是移动平台上的游戏、性能敏感的人工智能(AI)和机器学习(ML)软件、自动驾驶组件、实时3D图形引擎、底层硬件驱动、操作系统、网络设备的软件、网络浏览器等。C++程序在性能方面具有其他编程语言不可比拟的优势,因此成为了编写高效、强大和企业级程序的默认语言。像微软、Facebook、亚马逊、谷歌等大型科技公司都在使用用C++编写服务来运行他们的基础设施。尽管C++已经如此流行,但完全掌握这门语言还是很有难度。专业C++开发者使用的一些技巧,可能在传统教材中根本找不到,所以即便是资深C++开发者,也会有知识盲区。


很多时候,编程书籍更多地关注语法,而非实际应用。传统的C++教科书会在每一章中介绍语言的某个主要部分,解释语法并提供一个例子。但《Professional C++》不遵循这种模式,不会只介绍语言的基本知识和少量实践,而是会展示C++的实战用例。本书将展示一些鲜为人知的功能,可使读者们的编程更加轻松,同时还会向菜鸟与专业程序员展示不同的编程技巧。

\mySubsectionNoFile{}{适读人群}

即使已经使用C++语言多年,也可能不熟悉C++的一些更高级特性,或者没有充分利用该语言的所有功能,甚至还不知道最新版本C++23引入的新特性。或许已经能写出合格的C++代码,但想要了解更多关于设计和良好编程风格的内容。或者您是C++的新手,但希望一开始就了解“正确”的编程方式。这本书将满足这些需求,并将您的C++技能提升到专业水平。

本书会侧重于从C++的基本或中级知识提升到成为资深C++开发者,所以需要阅读本书的读者们已经具备一些基础C++编程知识和经验。

第1章,“C++和标准库速成课程”,涵盖了C++的基础知识作为复习,但并不能替代实际的编程训练。若是刚开始学习C++,但有C、Java、C\#等其他编程语言的经验,应该能够从第1章中获得学习后面章节所需的大部分知识。

无论如何,读者们都需要扎实的编程基础知识,应该了解循环、函数和变量。知道如何构建程序,并且应该熟悉递归等基本算法,对常见的数据结构(如队列)和一些有用的算法(如排序和搜索)也要有所了解。不过,不需要非常了解面向对象编程——这将在第5章“使用类进行设计”中详细介绍。

并且还需要熟悉将用于编译代码的编译器。后面将介绍两个编译器,Microsoft Visual C++和GCC。对于其他编译器,请参阅编译器提供的文档。

\mySubsectionNoFile{}{内容简介}

《Professional C++》采用的方法既可以提高代码质量,又可以提高编程效率。在第六版中,关于新C++23特性的讨论将贯穿全书。这些特性并不局限于几章或几节;并且,示例会在适当的时候使用新特性重新实现。

《Professional C++》不仅教授C++的语法和语言特性,还强调编程方法学、可重用的设计模式和良好的编程风格。《Professional C++》的方法学涵盖了整个软件开发过程,从设计和编写代码到调试和团队协作。这种方法将使您能够掌握C++语言及其特性,并利用其强大的功能进行大规模软件开发。

试想,学习了C++的所有语法,但从未见过一个实际用例,那这些语法只是反而会让学习者们陷入危险!没有例子,他们可能会认为所有的代码都应该写在主函数main()中,或者所有的变量都应该是全局变量——但我们知道,这并不是一个好的编程习惯。

资深C++开发者除了了解语法外,还需要了解如何正确使用C++语言。会重视良好的设计,理解面向对象编程,并清楚使用现有库的最佳方法。并且,还会研制一系列辅助代码和可重用的设计。

通过阅读和理解本书,你将成为一名专业的C++开发者。本书将扩展你对C++的了解,涵盖一些鲜为人知,且常有误解的语言特性。你将更加欣赏面向对象的设计,并获得一流的调试能力。也许最重要的是,你将使用丰富的、可实用的、可重用的思想来完成这本书。

努力成为一名专业的C++开发者吧,了解语言的真正工作原理将提高代码的质量,学习不同的编程方法和流程将有助于团队合作,发现可重用的库和设计模式将提高工作效率,并避免重复造轮子。所有这些经验将使你成为一个更好的开发者和更有价值的员工。虽然这本书不能保证你升职加薪,但它肯定会给你带来成就感。

\mySubsectionNoFile{}{本书结构}

这本书由五个部分组成。

第一部分,以C++基础知识速成课程开始,确保读者具备基础知识。之后,将更深入地探讨字符串的使用,因为字符串在本书的大多数示例中都得到了广泛的应用。第一部分的最后一章将探讨如何编写具有可读性的C++代码。

第二部分,讨论了C++设计方法。将了解到设计的重要性、面向对象的方法论,以及代码复用的重要性。

第三部分,从专业角度提供了C++的技术性概述。将了解C++中最佳管理内存的方法、如何创建可重用类,以及如何利用继承等语言特性。还将了解输入和输出、错误处理、字符串本地化,如何使用正则表达式,如何将代码结构化为模块的可重用组件,如何实现运算符重载,如何编写模板,如何使用概念对模板参数施加限制,以及如何解锁Lambda表达式和函数对象的力量。这部分还解释了C++标准库,包括容器、迭代器、范围和算法,还将了解到标准库中一些其他标准库,例如可用于处理时间、日期、时区、随机数和文件系统。

第四部分,展示了如何充分利用C++。这一部分展示了C++的神秘之处,并描述了如何使用一些更高级的特性。将了解如何根据需要自定义和扩展C++标准库,模板编程的高级细节,包括模板元编程,以及如何使用多线程来利用多处理器和多核系统。

第五部分,专注于编写企业级质量的软件。将了解当今编程组织正在使用的工程实践;如何编写高效的C++代码;软件测试概念,如单元测试和回归测试;调试C++程序的技术;如何将设计技术、框架和概念性的面向对象设计模式融入自己的代码;以及跨语言和跨平台代码的解决方案。

这本书以一个实用的章节指南作为结尾,可能会帮助你在C++技术面试中取得成功,该指南按章节进行了注释,还包括了标准中可用的C++头文件摘要,以及统一建模语言(UML)的简单介绍。

本书并不是C++的参考手册。《C++17标准库快速参考》由Peter Van Weert和Marc Gregoire编写(Apress, 2019. ISBN: 978-1-4842-4923-9)是一本浓缩的参考书,涵盖了直到C++17标准的C++标准库提供的所有基本数据结构、算法和函数(撰写本文时,正在制作《C++23标准库快速参考》,这是一本类似的浓缩参考书,包括了所有C++20和C++23特性)。附录B“注释参考书目”列出了一些其他的参考资料。这里有两个优秀的在线参考:

\begin{itemize}
\item
\url{cppreference.com}: 您可以使用这个在线参考,或者下载一个离线版本在本地使用。

\item
\url{cplusplus.com/reference}
\end{itemize}

本书中提到“标准库参考”时,指的是这些详细的C++参考之一。

以下是另外的优秀在线资源:

\begin{itemize}
\item
\url{github.com/isocpp/CppCoreGuidelines}: C++核心指南是由C++语言缔造者Bjarne Stroustrup领导的一项合作项目。这些指南的目的是帮助人们有效地使用现代C++,其关注的是相对较高层次的问题,如接口、资源管理、内存管理和并发性。

\item
\url{github.com/Microsoft/GSL}: 这是微软实现的一个纯头文件的指南库(GSL),其中包含C++核心指南建议使用的函数和类型。

\item
\url{isocpp.org/faq}: 这里是一个关于C++常见问题的汇集地。

\item
\url{stackoverflow.com}: 用于搜索常见编程问题的答案——或者提出自己的问题。
\end{itemize}

\mySubsectionNoFile{}{惯例}

\CXXTwentythreeLogo{-40}{-40}

讨论C++23标准的段落或部分,会在左侧有一个的C++23小图标,就像这段话一样。而C++11、C++14、C++17和C++20的特性则没有标记。

\mySubsectionNoFile{}{编译环境}

需要一台装有C++编译器的计算机。本书仅关注已标准化的C++,而不关注厂商的编译器扩展。

\mySubsubsection{}{C++编译器}

可以使用任何C++编译器。若还没有C++编译器,可以免费下载一个。有很多选择。例如,对于Windows,可以下载Microsoft Visual Studio社区版,免费的,并且包括Visual C++。对于Linux,可以使用GCC或Clang,也是免费的。

以下两部分简要说明了如何使用Visual C++和GCC。有关更多详细信息,请参阅编译器的帮助文档。

\begin{myTip}{编译器:支持C++23特性的}
本书讨论了C++23标准引入的新特性。撰写本文时,还没有任何编译器完全符合C++23标准。某些编译器仅支持了部分新特性,还有一些特性没有任何编译器支持。编译器供应商正在努力赶上新特性的步伐,我相信不久之后就会有多款完全符合C++23标准的编译器。可以在以下网站查看哪个编译器支持哪些特性:\url{en.cppreference.com/w/cpp/compiler_support}。
\end{myTip}

\begin{myTip}{编译器:支持C++模块}
撰写本书时,并非所有的编译器都支持模块;尽管主流编译器有部分支持。本书在所有地方都使用了模块,若目前使用的编译器尚不支持模块,可以将模块化代码转换为非模块化代码。在第11章“模块、头文件和其他主题”中,介绍了转换的方法。
\end{myTip}

\mySubsubsection{}{Microsoft Visual C++ 2022}

首先,需要创建一个项目。启动Visual C++ 2022,然后在欢迎屏幕上点击“创建新项目”按钮。如果欢迎屏幕没有显示,请选择“文件”->“新建”->“项目”。在“创建新项目”对话框中,搜索带有C++、Windows和Console标签的“控制台应用程序”项目模板,然后点击“下一步”。为项目指定一个名称和一个保存位置,然后点击“创建”。

当新项目加载完成后,可以在解决方案资源管理器中看到项目文件的列表。若这个停靠窗口不可见,请选择“视图”->“解决方案资源管理器”。一个新创建的项目将包含一个名为<projectname>.cpp的文件,位于解决方案资源管理器的“源文件”部分。可以在那个.cpp文件中开始编写你的C++代码,或者若想要编译本书可下载源代码存档中的源代码文件,请在解决方案资源管理器中选择<projectname>.cpp文件并将其删除。通过在解决方案资源管理器中右键点击项目名称,然后选择“添加”->“新建项”或“添加”->“现有项”,来向项目添加新文件或现有文件。

撰写本文时,Visual C++ 2022尚未自动启用C++23特性。要启用C++23特性,请在解决方案资源管理器窗口中,右键点击您的项目并选择“属性”。在属性窗口中,转到“配置属性”->“常规”,将“C++语言标准”选项设置为“ISO C++23标准”或“预览 - 最新C++工作草案中的特性”,根据Visual C++版本选择可用选项,然后点击“确定”。

最后,选择“生成”->“生成解决方案”来编译代码。当无误地编译完成后,可以使用“调试”->“开始调试”来运行。

\begin{myNotic}{NOTE}
Microsoft Visual C++完全支持模块,包括C++23标准的命名模块std。
\end{myNotic}

\mySubsubsection{}{GCC}

可以使用任何文本编辑器创建源代码文件,并将它们保存到目录中。要编译的代码,打开一个终端并运行以下命令,指定要编译的所有.cpp文件:

\begin{shell}
g++ -std=c++2b -o <executable_name> <source1.cpp> [source2.cpp ...]
\end{shell}

需要使用-std=c++2b选项来告诉GCC启用C++23特性。当GCC完全支持C++23标准,这个选项将可变为-std=c++23。

\mySamllsection{模块支持}

在GCC中,可使用-fmodules-ts选项来启用对模块的支持。

撰写本文时,GCC尚不支持C++23标准中引入的命名模块std。为了让这样的代码编译通过,必须将import std;声明替换为单个标准库头的import声明。标准库头的import声明,如下所示,并需要预编译:

\begin{cpp}
import <iostream>;
\end{cpp}

以下是一个预编译<iostream>的示例:

\begin{shell}
g++ -std=c++2b -fmodules-ts -xc++-system-header iostream
\end{shell}

作为一个例子,第一章中的AirlineTicket代码使用了模块。为了用GCC编译它,首先将std::println()的使用替换为std::cout,因为截至撰写本文时,GCC尚不支持<print>功能。之后,将import std;声明替换为适当的import声明,对于这个例子是<string>和<iostream>。可以在下载的源代码存档中的Examples\verb|\|Ch00\verb|\|AirlineTicket目录找到改编后的代码。

然后,编译两个标准头文件<iostream>和<string>:

\begin{shell}
g++ -std=c++2b -fmodules-ts -xc++-system-header iostream
g++ -std=c++2b -fmodules-ts -xc++-system-header string
\end{shell}

编译模块接口文件:

\begin{shell}
g++ -std=c++2b -fmodules-ts -c -x c++ AirlineTicket.cppm
\end{shell}

最后,编译应用程序本身:

\begin{shell}
g++ -std=c++2b -fmodules-ts -o AirlineTicket AirlineTicket.cpp AirlineTicketTest.cpp AirlineTicket.o
\end{shell}

当没有编译错误时,可以这样运行:

\begin{shell}
./AirlineTicket
\end{shell}

\begin{myNotic}{Note}
使用GCC编译使用C++模块的C++代码的过程可能会在将来发生变化,将添加对C++23标准命名模块std的支持。所以请查阅GCC文档,以获取更新后的编译此类代码的程序。
\end{myNotic}

\mySamllsection{C++23对输出范围的支持}

第2章“使用字符串和字符串视图”展示了可以轻松地将标准库容器(如std::vector)的全部内容输出到屏幕上,这是C++23的一个新特性。撰写本文时,并非所有编译器都支持此特性。

作为一个例子,第2章解释了std::vector的内容可以这样写。若还没有理解所用语法也不用担心,看到第2章的末尾就好了。

\begin{cpp}
std::vector values { 11, 22, 33 };
std::print("{:n}", values);
\end{cpp}

输出为:

\begin{shell}
11, 22, 33
\end{shell}

若编译器还不支持使用std::print()输出容器内容的C++23特性,则可以将第二行代码转换为以下内容:

\begin{cpp}
for (const auto& value : values) { std::cout << value << ", "; }
\end{cpp}

输出为:

\begin{shell}
11, 22, 33
\end{shell}

还不理解语法也不用担心,第2章结束时,一切都会清楚的。

\mySubsectionNoFile{}{本书的读者支持}

以下各节描述获得本书支持的不同方式。

\mySubsubsection{}{配套文件下载}

在阅读本书中的示例时,可以选择手动输入所有代码,也可以选择使用本书附带的源代码文件。但是,我建议您手动去敲所有代码,因为这对学习过程和记忆知识点非常有益。本书中使用的所有源代码都可以从\url{www.wiley.com/go/proc++6e}或从GitHub \url{github.com/Professional-CPP/edition-6}下载到。

\begin{myNotic}{Note}
因为许多书都有相似的标题,会发现按ISBN搜索最容易;这本书的ISBN是978-1-394-19317-2。
\end{myNotic}

下载代码后,只需使用解压缩工具对其进行解压缩即可。

\mySubsubsection{}{如何联系出版商}

若在这本书中发现了错误,请告诉我们。在John Wiley \& Sons,我们明白为客户提供准确的内容是多么重要,但即使我们尽了最大的努力,也可能会出现错误。

要提交可能的勘误表,请通过电子邮件发送到我们的客户服务团队wileysupport@wiley.com,主题为“可能的图书勘误表提交”(Possible Book Errata Submission)。

\mySubsubsection{}{如何联系作者}

若在阅读这本书的时候有任何问题,可以通过\url{marc.gregoire@nuonsoft.com}联系到作者,他会及时的进行回复。