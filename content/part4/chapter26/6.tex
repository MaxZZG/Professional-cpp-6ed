
这一节涉及到模板元编程。这是一个复杂的主题,有关它的书籍已经详细解释了所有细节。本书没有足够的空间来涵盖所有这些细节,这一节介绍了其中最重要的概念,并通过几个示例进行说明。

模板元编程的目的是在编译时,而不是在运行时进行一些计算。基本上是在另一种编程语言之上的编程语言。接下来的部分将以一个简单的示例开始,该示例在编译时计算一个数的阶乘,并将结果作为简单的常量在运行时提供。

\mySubsubsection{26.6.1.}{编译时的阶乘}

模板元编程允许在编译时,而不是运行时进行计算。以下代码是一个示例,计算一个数的阶乘。代码使用了之前在本章中解释的模板递归,这需要一个递归模板和一个基础模板来停止递归。根据数学定义,0 的阶乘是 1,因此将其用作基础情况。

\begin{cpp}
template <int f>
class Factorial
{
    public:
    static constexpr unsigned long long value { f * Factorial<f - 1>::value };
};

template <>
class Factorial<0>
{
    public:
    static constexpr unsigned long long value { 1 };
};

int main()
{
    println("{}", Factorial<6>::value);
}
\end{cpp}

这计算了 6 的阶乘,数学上写作 6!,即 1×2×3×4×5×6 或 720。

\begin{myNotic}{NOTE}
阶乘计算是在编译时进行的。运行时,只需通过 value 数据成员访问编译时计算的值,这只是一个静态常量值。
\end{myNotic}

对于这个特定的示例,计算一个数的阶乘在编译时,不需要使用模板元编程。可以将其实现为一个 consteval 立即函数,如下所示。不需要任何模板,尽管模板实现仍然作为一个很好的例子,展示了如何实现递归模板。

\begin{cpp}
consteval unsigned long long factorial(int f)
{
    if (f == 0) { return 1; }
    else { return f * factorial(f - 1); }
}
\end{cpp}

可以像调用其他函数一样调用 factorial(),唯一的区别是 consteval 函数保证在编译时执行。例如:

\begin{cpp}
println("{}", factorial(6));
\end{cpp}

\mySubsubsection{26.6.2.}{循环展开}

模板元编程的第二个示例,在编译时展开循环,而不是在运行时执行循环。注意,循环展开应该只在真正需要情况下进行,例如:性能关键的代码中。通常,编译器足够智能,可以展开相应的循环。

这个示例再次使用模板递归,需要在编译时循环中执行某些操作。每次递归中,Loop类模板都会实例化自己为i-1。当达到0时,递归停止。

\begin{cpp}
template <int i>
class Loop
{
    public:
        template <typename FuncType>
        static void run(FuncType func) {
            Loop<i - 1>::run(func);
            func(i);
        }
};

template <>
class Loop<0>
{
    public:
        template <typename FuncType>
        static void run(FuncType /* func */) { }
};
\end{cpp}

Loop模板可以这样使用:

\begin{cpp}
void doWork(int i) { println("doWork({})", i); }

int main()
{
    Loop<3>::run(doWork);
}
\end{cpp}

这段代码会让编译器展开循环,并连续调用函数doWork()三次。程序的输出如下:

\begin{shell}
doWork(1)
doWork(2)
doWork(3)
\end{shell}

\mySubsubsection{26.6.3.}{打印元组}

这个示例使用模板元编程来打印std::tuple中的各个元素,存储任意数量的值,每个值都有其自己的特定类型。元组有固定的大小和固定的值类型,在编译时确定。然而,元组没有内置的机制来遍历它们的元素。以下示例展示了如何使用模板元编程在编译时遍历元组的元素。

与模板元编程中的许多示例一样,这个示例再次使用模板递归。TuplePrint类模板有两个模板参数:元组类型和整数,初始化为元组的大小。在构造函数中递归实例化自己,并在每次调用时递减整数,TuplePrint的偏特化在整数达到0时停止递归。main()函数展示了如何使用这个TuplePrint类模板。

\begin{cpp}
template <typename TupleType, int N>
class TuplePrint
{
    public:
        explicit TuplePrint(const TupleType& t) {
            TuplePrint<TupleType, N - 1> tp { t };
            println("{}", get<N - 1>(t));
        }
};

template <typename TupleType>
class TuplePrint<TupleType, 0>
{
    public:
        explicit TuplePrint(const TupleType&) { }
};

int main()
{
    using MyTuple = tuple<int, string, bool>;
    MyTuple t1 { 16, "Test", true };
    TuplePrint<MyTuple, tuple_size<MyTuple>::value> tp { t1 };
}
\end{cpp}

TuplePrint语句在main()中看起来有点复杂,需要作为模板参数提供确切的tuple类型和大小。这可以通过引入辅助函数模板来简化,该模板可以自动推断模板参数。简化后的实现如下:

\begin{cpp}
template <typename TupleType, int N>
class TuplePrintHelper
{
    public:
        explicit TuplePrintHelper(const TupleType& t) {
            TuplePrintHelper<TupleType, N - 1> tp { t };
            println("{}", get<N - 1>(t));
        }
};

template <typename TupleType>
class TuplePrintHelper<TupleType, 0>
{
    public:
        explicit TuplePrintHelper(const TupleType&) { }
};

template <typename T>
void tuplePrint(const T& t)
{
    TuplePrintHelper<T, tuple_size<T>::value> tph { t };
}

int main()
{
    tuple t1 { 16, "Test"s, true };
    tuplePrint(t1);
}
\end{cpp}

这里所做的第一个改变是,将原始的TuplePrint类模板重命名为TuplePrintHelper。然后,代码实现了一个小的函数模板tuplePrint(),接受元组的类型作为模板类型参数,并接受一个指向元组的引用作为函数参数。该函数体的内容实例化了TuplePrintHelper类模板,main()函数展示了如何使用这个简化版本。不需要指定函数模板参数,因为编译器可以从提供的参数自动推断出这些参数。

\mySamllsection{constexpr if}

constexpr if,在本章早些时候引入,可以用来简化许多模板元编程技术。例如,可以使用constexpr if简化打印元组元素的先前代码,如下所示。因为递归在constexpr if语句处停止,所以不再需要模板递归。

\begin{cpp}
template <typename TupleType, int N>
class TuplePrintHelper
{
    public:
        explicit TuplePrintHelper(const TupleType& t) {
            if constexpr (N > 1) {
                TuplePrintHelper<TupleType, N - 1> tp { t };
            }
            println("{}", get<N - 1>(t));
        }
};

template <typename T>
void tuplePrint(const T& t)
{
    TuplePrintHelper<T, tuple_size<T>::value> tph { t };
}
\end{cpp}

现在甚至可以去掉类模板本身,并用一个简单的函数模板tuplePrintHelper()替换:

\begin{cpp}
template <typename TupleType, int N>
void tuplePrintHelper(const TupleType& t)
{
    if constexpr (N > 1) {
        tuplePrintHelper<TupleType, N - 1>(t);
    }
    println("{}", get<N - 1>(t));
}

template <typename T>
void tuplePrint(const T& t)
{
    tuplePrintHelper<T, tuple_size<T>::value>(t);
}
\end{cpp}

这还可以进一步简化,两个函数模板可以合并成一个:

\begin{cpp}
template <typename TupleType, int N = tuple_size<TupleType>::value>
void tuplePrint(const TupleType& t)
{
    if constexpr (N > 1) {
        tuplePrint<TupleType, N - 1>(t);
    }
    println("{}", get<N - 1>(t));
}
\end{cpp}

仍然可以像以前一样使用:

\begin{cpp}
tuple t1 { 16, "Test"s, true };
tuplePrint(t1);
\end{cpp}

\mySamllsection{使用编译时整数序列进行折叠}

C++ 支持使用 std::integer\_sequence 的编译时整数序列,该序列定义在 <utility> 中。模板元编程的一个常见用例是生成编译时索引序列,即类型为 size\_t 的整数序列,有一个辅助的 std::index\_sequence 可用。可以使用 std::make\_index\_sequence 来生成一个索引序列,其长度与给定参数包的长度相同。

可以使用变长模板、编译时索引序列和折叠表达式实现元组打印器:

\begin{cpp}
template <typename Tuple, size_t... Indices>
void tuplePrintHelper(const Tuple& t, index_sequence<Indices...>)
{
    (println("{}", get<Indices>(t)) , ...);
}

template <typename... Args>
void tuplePrint(const tuple<Args...>& t)
{
    tuplePrintHelper(t, make_index_sequence<sizeof...(Args)>{});
}
\end{cpp}

可以用以前的方式:

\begin{cpp}
tuple t1 { 16, "Test"s, true };
tuplePrint(t1);
\end{cpp}

调用中,tuplePrintHelper() 函数模板中的单右折叠表达式展开为以下内容:

\begin{cpp}
((println("{}", get<0>(t)) ,
 (println("{}", get<1>(t)) ,
  println("{}", get<2>(t)))));
\end{cpp}


\mySubsubsection{26.6.4.}{类型特性}

类型特性允许你在编译时根据类型做出决策,可以验证一个类型是否派生自另一个类型,是否可以转换为另一个类型,是否为整数类型等。C++ 标准库提供了一系列的类型特性,所有与类型特性相关的功能都定义在 <type\_traits> 中。类型特性分为几个不同的类别,以下列表给出了每个类别中可用的一些类型特性示例。请查阅附录 B 的以获取完整的列表。

\begin{itemize}
\item
主要类型类别
\begin{itemize}
\item
is\_void

\item
is\_integral

\item
is\_floating\_point

\item
is\_pointer

\item
is\_function

\item
...
\end{itemize}

\item
类型属性
\begin{itemize}
\item
is\_const

\item
is\_polymorphic

\item
is\_unsigned

\item
is\_constructible

\item
is\_copy\_constructible

\item
is\_move\_constructible

\item
is\_assignable

\item
is\_trivially\_copyable

\item
is\_swappable

\item
is\_nothrow\_swappable

\item
has\_virtual\_destructor

\item
has\_unique\_object\_representations

\item
is\_scoped\_enum*

\item
is\_implicit\_lifetime*

\item
...
\end{itemize}

\item
属性查询
\begin{itemize}
\item
alignment\_of

\item
rank

\item
extent
\end{itemize}

\item
复合类型类别
\begin{itemize}
\item
is\_arithmetic

\item
is\_reference

\item
is\_object

\item
is\_scalar

\item
...
\end{itemize}

\item
类型关系

\begin{itemize}
\item
is\_same

\item
is\_base\_of

\item
is\_convertible

\item
is\_invocable

\item
is\_nothrow\_invocable

\item
...
\end{itemize}

\item
修改const-volatile

\begin{itemize}
\item
remove\_const

\item
add\_const

\item
...
\end{itemize}

\item
修改

\begin{itemize}
\item
make\_signed

\item
make\_unsigned
\end{itemize}

\item
修改数组

\begin{itemize}
\item
remove\_extent

\item
remove\_all\_extents
\end{itemize}

\item
逻辑运算符特性

\begin{itemize}
\item
conjunction

\item
disjunction

\item
negation
\end{itemize}

\item
修改引用

\begin{itemize}
\item
remove\_reference

\item
add\_lvalue\_reference

\item
add\_rvalue\_reference
\end{itemize}

\item
修改指针

\begin{itemize}
\item
remove\_pointer

\item
add\_pointer
\end{itemize}

\item
上下文的常量计算

\begin{itemize}
\item
is\_constant\_evaluated
\end{itemize}

\item
其他转换

\begin{itemize}
\item
enable\_if

\item
conditional

\item
invoke\_result

\item
type\_identity

\item
remove\_cvref

\item
common\_reference

\item
decay

\item
...
\end{itemize}
\end{itemize}

标记有星号 (*) 的类型特性仅从 C++23 开始可用。

类型特性是 C++ 中相当高级的特性。前面的列表,只是 C++ 标准中列表的缩减版本,也可以清楚地看出,本书无法解释所有关于所有类型特性的详细信息。本节只解释了几个用例,以展示如何使用类型特性。

\mySamllsection{使用类型分类}

给出使用类型特性的模板示例之前,首先需要了解更多诸如is\_integral这样的类是如何工作的。C++标准定义了一个integral\_constant类,如下所示:

\begin{cpp}
template <class T, T v>
struct integral_constant {
    static constexpr T value = v;
    using value_type = T;
    using type = integral_constant<T, v>;
    constexpr operator value_type() const noexcept { return value; }
    constexpr value_type operator()() const noexcept { return value; }
};
\end{cpp}

还定义了bool\_constant、true\_type和false\_type类型别名:

\begin{cpp}
template <bool B>
using bool_constant = integral_constant<bool, B>;

using true_type = bool_constant<true>;
using false_type = bool_constant<false>;
\end{cpp}

当访问true\_type::value时,会得到true,而访问false\_type::value时,会得到false。还可以访问true\_type::type,会得到true\_type类型的值,同样适用于false\_type。像is\_integral这样的类,会检查一个类型是否为整数类型,和is\_class,检查一个类型是否为类,都继承自true\_type或false\_type。例如,标准库对bool类型特化了is\_integral:

\begin{cpp}
template <> struct is_integral<bool> : public true_type { };
\end{cpp}

可以访问is\_integral<bool>::value,其结果为true。注意,不需要自己写这些特化;它们是标准库的一部分。

以下代码展示了使用类型分类的最简单示例:

\begin{cpp}
if (is_integral<int>::value) { println("int is integral"); }
else { println("int is not integral"); }

if (is_class<string>::value) { println("string is a class"); }
else { println("string is not a class"); }
\end{cpp}

输出如下:

\begin{shell}
int is integral
string is a class
\end{shell}

对于每个具有值成员的特性,标准库都添加了一个变量模板,该模板的名字与特性名相同,后面跟着\_v。而不是some\_trait<T>::value,可以使用some\_trait\_v<T>——例如,is\_integral\_v<T>,is\_const\_v<T>等。以下是如何在标准库中定义is\_integral\_v<T>变量模板的示例:

\begin{cpp}
template <class T>
inline constexpr bool is_integral_v = is_integral<T>::value;
\end{cpp}

使用这些变量模板,前面的示例可以这样:

\begin{cpp}
if (is_integral_v<int>) { println("int is integral"); }
else { println("int is not integral"); }

if (is_class_v<string>) { println("string is a class"); }
else { println("string is not a class"); }
\end{cpp}

实际上,因为is\_integral\_v<T>的值是一个编译时常量,可以用constexpr if代替普通的if。

当然,很可能永远不会以这种方式使用类型特性。它们在与模板结合使用时更有用,以根据类型的某些属性生成代码,以下的函数模板演示了这一点。代码定义了两个重载的processHelper()函数模板,接受类型作为模板参数。这些函数的第一个参数是值,第二个参数是true\_type或false\_type的一个实例。process()函数模板接受一个参数并调用processHelper()。

\begin{cpp}
template <typename T>
void processHelper(const T& t, true_type)
{
    println("{} is an integral type.", t);
}

template <typename T>
void processHelper(const T& t, false_type)
{
    println("{} is a non-integral type.", t);
}

template <typename T>
void process(const T& t)
{
    processHelper(t, is_integral<T>{});
}
\end{cpp}

在调用processHelper()时,第二个参数是is\_integral<T>\{\}。这个参数使用is\_integral<T>来确定T是否为整数类型,is\_integral<T>要么从true\_type继承,要么从false\_type继承。processHelper()函数需要一个true\_type或false\_type的实例作为第二个参数,这就是为什么有空的括号\{\}。这两个重载的processHelper()函数,没有费心为true\_type和false\_type类型的参数命名,它们是无名的,因为函数体内部不使用这些参数。这些参数仅用于函数重载解析。

代码可以按照以下方式进行测试:

\begin{cpp}
process(123);
process(2.2);
process("Test"s);
\end{cpp}

输出如下:

\begin{shell}
123 is an integral type.
2.2 is a non-integral type.
Test is a non-integral type.
\end{shell}

前面的示例可以写成一个单一的函数模板,如下所示。然而,这并不是如何使用类型特性,根据类型选择不同的重载。

\begin{cpp}
template <typename T>
void process(const T& t)
{
    if constexpr (is_integral_v<T>) {
        println("{} is an integral type.", t);
    } else {
        println("{} is a non-integral type.", t);
    }
}
\end{cpp}


\mySamllsection{使用类型关系}

类型关系的示例包括 is\_same、is\_base\_of 和 is\_convertible。本节给出了一个如何使用 is\_same 的示例;其他类型关系的工作方式类似。

以下 same() 函数模板使用 is\_same 类型特质来判断,两个给定的参数是否具有相同的类型,并输出适当的消息:

\begin{cpp}
template <typename T1, typename T2>
void same(const T1& t1, const T2& t2)
{
    bool areTypesTheSame { is_same_v<T1, T2> };
    println("'{}' and '{}' are {} types.", t1, t2,
    (areTypesTheSame ? "the same" : "different"));
}

int main()
{
    same(1, 32);
    same(1, 3.01);
    same(3.01, "Test"s);
}
\end{cpp}

输出如下所示:

\begin{shell}
'1' and '32' are the same types.
'1' and '3.01' are different types
'3.01' and 'Test' are different types
\end{shell}

或者,可以不使用任何类型特性,而是使用两个函数模板的重载集来实现这个示例:

\begin{cpp}
template <typename T1, typename T2>
void same(const T1& t1, const T2& t2)
{
    println("'{}' and '{}' are different types.", t1, t2);
}
template <typename T>
void same(const T& t1, const T& t2)
{
    println("'{}' and '{}' are the same type.", t1, t2);
}
\end{cpp}

第二个函数模板比第一个更具体,所以当它可行时,即当两个参数都是同一类型 T 时,将首选重载版本。

\mySamllsection{使用条件类型特征}

第 18 章解释了标准库辅助函数模板 std::move\_if\_noexcept(),可以根据前者是否标记为 noexcept ,有条件地调用移动构造函数或复制构造函数。标准库没有提供类似的辅助函数模板,根据前者是否 noexcept 来调用移动赋值运算符或复制赋值运算符。现在了解了模板元编程和类型特征,来看看如何自己实现 move\_assign\_if\_noexcept()。

move\_if\_noexcept() 中,如果移动构造函数标记为 noexcept,只是将给定的引用转换为 右值 引用,否则转换为引用为常量。

move\_assign\_if\_noexcept() 需要做类似的事情,如果移动赋值运算符标记为 noexcept,则将给定的引用转换为 右值 引用,否则转换为引用为常量。

std::conditional 类型特征可以用来实现条件。这个类型特征有三个模板参数:一个布尔值、布尔值为true的类型,以及布尔值为false的类型。conditional 类型特征的实现如下:

\begin{cpp}
template <bool B, class T, class F>
struct conditional { using type = T; };

template <class T, class F>
struct conditional<false, T, F> { using type = F; };
\end{cpp}

is\_nothrow\_move\_assignable 类型特征可以用来判断某个类型,是否可以不抛出异常地进行移动赋值。对于类类型,检查类型是否有一个标记为 noexcept 的移动赋值运算符。以下是 move\_assign\_if\_noexcept() 函数模板的完整实现:

\begin{cpp}
template <typename T>
constexpr conditional<is_nothrow_move_assignable_v<T>, T&&, const T&>::type
    move_assign_if_noexcept(T& t) noexcept
{
    return move(t);
}
\end{cpp}

标准库为具有类型成员的特性定义了别名模板,例如 conditional。这些具有与特性相同的名字,以 \_t 结尾。例如,conditional\_t<B,T,F> 别名模板用于 conditional<B,T,F>::type,由标准库定义如下:

\begin{cpp}
template <bool B, class T, class F>
using conditional_t = typename conditional<B,T,F>::type;
\end{cpp}

所以,无需这样写:

\begin{cpp}
conditional<is_nothrow_move_assignable_v<T>, T&&, const T&>::type
\end{cpp}

只要这样写即可:

\begin{cpp}
conditional_t<is_nothrow_move_assignable_v<T>, T&&, const T&>
\end{cpp}

move\_assign\_if\_noexcept() 函数模板可以像这样测试:

\begin{cpp}
class MoveAssignable
{
    public:
        MoveAssignable& operator=(const MoveAssignable&) {
            println("copy assign"); return *this; }
        MoveAssignable& operator=(MoveAssignable&&) {
            println("move assign"); return *this; }
};

class MoveAssignableNoexcept
{
    public:
        MoveAssignableNoexcept& operator=(const MoveAssignableNoexcept&) {
            println("copy assign"); return *this; }
        MoveAssignableNoexcept& operator=(MoveAssignableNoexcept&&) noexcept {
            println("move assign"); return *this; }
};

int main()
{
    MoveAssignable a, b;
    a = move_assign_if_noexcept(b);
    MoveAssignableNoexcept c, d;
    c = move_assign_if_noexcept(d);
}
\end{cpp}

输出如下所示:

\begin{shell}
copy assign
move assign
\end{shell}

\mySamllsection{使用类型修改类型特征}

有许多类型特征可以修改给定的类型。例如,add\_const 类型特征给一个类型添加 const,remove\_pointer 类型特征从类型中移除指针等。这里有一个例子:

\begin{cpp}
println("{}", is_same_v<string, remove_pointer_t<string*>>);
\end{cpp}

输出是 true。

自己实现这样的类型修改特质并不困难,这里是一个简化的 my\_remove\_pointer 类型特征实现:

\begin{cpp}
// my_remove_pointer class template.
template <typename T> struct my_remove_pointer { using type = T; };
// Partial specialization for pointer types.
template <typename T> struct my_remove_pointer<T*> { using type = T; };
// Partial specialization for const pointer types.
template <typename T> struct my_remove_pointer<T* const> { using type = T; };
// Alias template for ease of use.
template <typename T>
using my_remove_pointer_t = typename my_remove_pointer<T>::type;

int main()
{
    println("{}", is_same_v<string, my_remove_pointer_t<string*>>);
}
\end{cpp}

\mySamllsection{使用 enable\_if}

enable\_if 的使用基于一个称为替换失败不是错误(SFINAE)的原则,这是 C++ 的高级特性。这个原则指出,对于给定的模板参数集,函数模板未特化不应该视为编译错误,这样的特化应该从函数重载集中移除。本节只解释 SFINAE 的基本概念。

如果有一组重载的函数,可以使用 enable\_if 根据某些类型特征有选择地禁用某些重载。enable\_if 特质通常用于重载组的返回类型,或者用于未命名的非类型模板参数。enable\_if 接受两个模板参数。第一个是一个布尔值,第二个是一个类型。如果布尔值为true,则 enable\_if 类模板有一个类型别名,可以通过 ::type 访问,这个类型别名的类型是作为第二个模板参数给出的类型。如果布尔值为false,则没有这样的类型别名。以下是这个类型特征的实现:

\begin{cpp}
template <bool B, class T = void>
struct enable_if {};

template <class T>
struct enable_if<true, T> { typedef T type; };
\end{cpp}

same() 函数模板可以通过使用 enable\_if 重新编写成重载的 checkType() 函数模板,如下所示。这个实现中,checkType() 函数根据给定值的类型是否相同返回 true 或 false。如果不想从 checkType() 返回任何内容,可以删除返回语句并从 enable\_if 删除第二个模板参数。

\begin{cpp}
template <typename T1, typename T2>
enable_if_t<is_same_v<T1, T2>, bool>
checkType(const T1& t1, const T2& t2)
{
    println("'{}' and '{}' are the same types.", t1, t2);
    return true;
}

template <typename T1, typename T2>
enable_if_t<!is_same_v<T1, T2>, bool>
checkType(const T1& t1, const T2& t2)
{
    println("'{}' and '{}' are different types.", t1, t2);
    return false;
}

int main()
{
    checkType(1, 32);
    checkType(1, 3.01);
    checkType(3.01, "Test"s);
}
\end{cpp}

输出与之前相同:

\begin{shell}
'1' and '32' are the same types.
'1' and '3.01' are different types.
'3.01' and 'Test' are different types.
\end{shell}

代码定义了两个 checkType() 重载,使用 is\_same\_v 来检查两个类型是否相同。结果传递给 enable\_if\_t,当 enable\_if\_t 的第一个参数为true时,enable\_if\_t 有一个类型别名,可以通过 ::type 访问。这个类型别名的类型是作为第二个模板参数给出的类型。如果布尔值为false,则没有这样的类型别名,这就是 SFINAE 发挥作用的地方。

当编译器开始编译 main() 函数中的第一条语句时,尝试找到一个接受两个整数值的 checkType() 函数。找到了第一个 checkType() 函数模板重载,并推断出它可以使用这个函数模板的一个实例,通过将 T1 和 T2 都设置为整数,尝试确定返回类型。因为两个参数都是整数,所以其类型相同,所以 is\_same\_v<T1, T2> 为true,从而 enable\_if\_t<true, bool> 为类型 bool。有了这个实例化,编译器将这个重载添加到候选重载集中。当它看到 checkType() 的第二个重载时,再次推断出它可以使用这个函数模板的一个实例,通过将 T1 和 T2 都设置为整数。然而,当尝试确定返回类型时,发现 !is\_same\_v<T1, T2> 为false。由于这个原因,enable\_if\_t<false, bool> 不表示一个类型,使得这个重载的 checkType() 函数没有返回类型。编译器注意到这个错误,因为 SFINAE,还没有生成真正的编译错误,只是没有将这个重载添加到候选重载集中。因此,对于 main() 函数中的第一条语句,重载集包含一个候选 checkType() 函数,所以编译器将使用哪个函数就很明显了。

当编译器尝试编译 main() 函数中的第二条语句时,再次尝试找到一个合适的 checkType() 函数。从第一个 checkType() 开始,决定使用这个重载,通过将 T1 设置为 int 类型,T2 设置为 double 类型,再尝试确定返回类型。这次,T1 和 T2 是不同类型,所以 is\_same\_v<T1, T2> 为false。因此,enable\_if\_t<false, bool> 不表示一个类型,使得这个重载的 checkType() 函数没有返回类型。编译器注意到这个错误,因为 SFINAE还没有生成真正的编译错误,编译器只是没有将这个重载添加到候选重载集中。当编译器看到第二个 checkType() 函数时,推断出可以使用这个函数模板的一个实例,因为 T1 和 T2 是不同类型,所以 !is\_same\_v<T1, T2> 为true,因此 enable\_if\_t<true, bool> 为类型 bool。最后,对于 main() 函数中的第二条语句,重载集再次只包含一个重载。

如果不想让 enable\_if 污染返回类型,那么另一种选择是使用 enable\_if 和非类型模板参数,这会让代码更容易阅读。例如:

\begin{cpp}
template <typename T1, typename T2, enable_if_t<is_same_v<T1, T2>>* = nullptr>
bool checkType(const T1& t1, const T2& t2)
{
    println("'{}' and '{}' are the same types.", t1, t2);
    return true;
}

template <typename T1, typename T2, enable_if_t<!is_same_v<T1, T2>>* = nullptr>
bool checkType(const T1& t1, const T2& t2)
{
    println("'{}' and '{}' are different types.", t1, t2);
    return false;
}
\end{cpp}

如果想在构造函数集中使用 enable\_if,不能使用它与返回类型,因为构造函数没有返回类型。这种情况下,必须使用它与非类型模板参数。

本节中解释的 enable\_if 语法在 C++20 之前是业界标准。自 C++20 以来,应该更倾向于使用概念,这在第 12 章中有讨论。注意较早的 enable\_if 代码与使用概念的以下示例之间的语法相似性,但使用概念显然更易于阅读。

\begin{cpp}
template <typename T1, typename T2> requires is_same_v<T1, T2>
bool checkType(const T1& t1, const T2& t2)
{
    println("'{}' and '{}' are the same types.", t1, t2);
    return true;
}

template <typename T1, typename T2> requires !is_same_v<T1, T2>
bool checkType(const T1& t1, const T2& t2)
{
    println("'{}' and '{}' are different types.", t1, t2);
    return false;
}
\end{cpp}

推荐谨慎使用 SFINAE,只有在其他技术(如特殊化、概念等)无法解决重载歧义时才使用。例如,只是想在使用错误的类型时让编译失败,请使用概念或使用本章后面介绍的 static\_assert(),而不是 SFINAE。当然,SFINAE 有适合的使用场景,但请记住以下几点。

\begin{myNotic}{NOTE}
依赖 SFINAE 是复杂且棘手的。如果 SFINAE 和 enable\_if 选择性禁用了重载集中错误的重载,将得到难以追踪的神秘编译错误。
\end{myNotic}

\mySamllsection{使用 constexpr if 简化 enable\_if 的构造}

正如从之前的示例中可以看到的,使用 enable\_if 可能会变得相当复杂。constexpr if 特性可以帮助显著简化 enable\_if 的某些使用场景。

例如,有两个类:

\begin{cpp}
class IsDoable
{
    public:
        virtual void doit() const { println("IsDoable::doit()"); }
};

class Derived : public IsDoable { };
\end{cpp}

可以编写一个函数模板 callDoit(),如果成员函数可用,则调用 doit() 成员函数;否则,打印一个错误消息。可以使用 enable\_if 通过检查给定类型是否派生自 IsDoable 来做到:

\begin{cpp}
template <typename T>
enable_if_t<is_base_of_v<IsDoable, T>, void> callDoit(const T& t)
{
    t.doit();
}

template <typename T>
enable_if_t<!is_base_of_v<IsDoable, T>, void> callDoit(const T&)
{
    println("Cannot call doit()!");
}
\end{cpp}

以下代码测试了这个实现:

\begin{cpp}
Derived d;
callDoit(d);
callDoit(123);
\end{cpp}

输出为:

\begin{shell}
IsDoable::doit()
Cannot call doit()!
\end{shell}

可以通过使用 constexpr if 极大地简化 enable\_if 实现:

\begin{cpp}
template <typename T>
void callDoit(const T& t)
{
    if constexpr (is_base_of_v<IsDoable, T>) {
        t.doit();
    } else {
        println("Cannot call doit()!");
    }
}
\end{cpp}

不能使用普通的 if 语句来完成这个任务。使用普通的 if 语句,两个分支都需要编译,如果提供了一个不是从 IsDoable 派生的类型 T,这将会失败。这种情况下,语句 t.doit() 将无法编译,而使用 constexpr if 语句,如果提供的类型不是从 IsDoable 派生,那么语句 t.doit() 甚至不会编译。

除了使用 is\_base\_of 类型特征,也可以使用 requires 表达式;请参见第 12 章。以下是使用 requires 表达式检查是否可以在对象 t 上调用 doit() 成员函数的 callDoit() 实现的示例。

\begin{cpp}
template <typename T>
void callDoit(const T& t)
{
    if constexpr (requires { t.doit(); }) {
        t.doit();
    } else {
        println("Cannot call doit()!");
    }
}
\end{cpp}

\mySamllsection{逻辑运算符特征}

有三个逻辑运算符特征:conjunction、disjunction 和 negation。此外,还提供了以 \_v 结尾的变量模板。这些特征接受可变数量的模板类型参数,并可用于在类型特征上执行逻辑操作,例如:

\begin{cpp}
print("{} ", conjunction_v<is_integral<int>, is_integral<short>>);
print("{} ", conjunction_v<is_integral<int>, is_integral<double>>);

print("{} ", disjunction_v<is_integral<int>, is_integral<double>,
                        is_integral<short>>);

print("{} ", negation_v<is_integral<int>>);
\end{cpp}

输出为:

\begin{shell}
true false true false
\end{shell}

\mySamllsection{静态断言}

static\_assert() 允许在编译时断言某些条件。断言是必须为true,如果断言为假,编译器将发出错误。调用 static\_assert() 接受两个参数:一个在编译时评估的表达式和(可选)一个字符串。当表达式计算为false时,编译器将发出包含给定字符串的错误。例如,检查是否使用 64 位编译器:

\begin{cpp}
static_assert(sizeof(void*) == 8, "Requires 64-bit compilation.");
\end{cpp}

如果用一个 32 位编译器编译这个,其中指针占用四个字节,编译器将发出错误,可能看起来像这样:

\begin{shell}
test.cpp(3): error C2338: Requires 64-bit compilation.
\end{shell}

字符串参数是可选的,例如:

\begin{cpp}
static_assert(sizeof(void*) == 8);
\end{cpp}

这种情况下,如果表达式计算为false,将得到一个与编译器相关的错误消息。例如,Microsoft Visual C++ 给出以下错误:

\begin{shell}
test.cpp(3): error C2607: static assertion failed
\end{shell}

static\_assert() 可以与类型特征结合使用。例如:

\begin{cpp}
template <typename T>
void foo(const T& t)
{
    static_assert(is_integral_v<T>, "T must be an integral type.");
}
\end{cpp}

\mySubsubsection{26.6.5.}{对元编程的总结}

正如在本节中看到的,模板元编程可以是一个强大的工具,但它也可能变得相当复杂。模板元编程的一个问题,之前没有提到,是所有事情都在编译时发生,所以不能使用调试器来定位问题。如果决定在代码中使用模板元编程,确保编写了良好的注释来解释确切发生了什么,以及为什么以某种方式完成。如果没有正确地记录模板元编程代码,其他人可能难以理解这段代码,甚至在未来的自己也很难理解这段代码。






















