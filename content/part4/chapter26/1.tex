
有三种模板参数:模板类型参数、非类型模板参数和双重模板参数。目前为止,已经了解了类型和非类型参数的例子(在第12章中),但没有看到双重模板参数。类型和非类型参数都有一些棘手的问题,这些在第12章中并没有涵盖。本节将深入探讨所有三种类型的模板参数。

\mySubsubsection{26.1.1.}{模板类型参数}

模板类型参数是模板的主要目的,可以声明任意数量的类型参数。例如,可以从第12章的grid模板中添加第二个类型参数,以指定构建网格的容器。标准库定义了几个参数化的容器类,包括vector和deque。原始的Grid类使用vector来存储网格的元素,Grid类的用户可能希望使用deque。使用另一个模板类型参数,可以允许用户指定底层容器是vector还是deque。Grid实现要求底层容器支持随机访问,还使用容器的resize()成员函数和容器的value\_type类型别名。使用概念(参见第12章)来强制提供的类型支持这些操作。以下是带有概念和模板类型参数的类模板定义:

\begin{cpp}
template <typename Container>
concept GridContainerType =
    std::ranges::random_access_range<Container> &&
    requires(Container c) {
        typename Container::value_type;
        c.resize(1);
    };

export template <typename T, GridContainerType Container>
class Grid
{
    public:
        explicit Grid(std::size_t width = DefaultWidth,
            std::size_t height = DefaultHeight);
        virtual ˜Grid() = default;

        // Explicitly default a copy constructor and assignment operator.
        Grid(const Grid& src) = default;
        Grid& operator=(const Grid& rhs) = default;

        // Explicitly default a move constructor and assignment operator.
        Grid(Grid&& src) = default;
        Grid& operator=(Grid&& rhs) = default;

        typename Container::value_type& at(std::size_t x, std::size_t y);
        const typename Container::value_type& at(
            std::size_t x, std::size_t y) const;

        std::size_t getHeight() const { return m_height; }
        std::size_t getWidth() const { return m_width; }

        static constexpr std::size_t DefaultWidth { 10 };
        static constexpr std::size_t DefaultHeight { 10 };

    private:
        void verifyCoordinate(std::size_t x, std::size_t y) const;

        Container m_cells;
        std::size_t m_width { 0 }, m_height { 0 };
};
\end{cpp}

这个模板现在有两个参数:T 和 Container。无论之前在哪里引用了 Grid<T>,现在必须引用 Grid<T, Container>。

\begin{myNotic}{NOTE}
这个 Grid 类模板的实现中,可以移除 T 模板类型参数;它只使用 Container 参数。但请耐心点,因为下一节将在这个例子基础上进一步构建,使其对用户更友好。
\end{myNotic}

m\_cells 数据成员现在是 Container 类型,而不是 vector<optional<T>{}>。每个 Container 类型都有一个叫做 value\_type 的类型别名。这一点可以通过 GridContainerType 概念得到了验证。在 Grid 类模板定义,及其成员函数定义内部,可以使用作用域解析运算符访问 value\_type 类型名:Container::value\_type。由于 Container 是一个模板类型参数,Container::value\_type 是一个依赖类型名。通常,编译器不会将依赖类型名视为类型名,这可能会导致一些神秘的编译器错误信息。为了确保编译器将其解释为类型名,需要使用 typename 关键字作为前缀,如 typename Container::value\_type。这就是 at() 成员函数的返回类型所做的,返回类型是存储在给定容器类型内部的元素类型,即 typename Container::value\_type。

下面是构造函数的定义:

\begin{cpp}
template <typename T, GridContainerType Container>
Grid<T, Container>::Grid(std::size_t width, std::size_t height)
    : m_width { width }, m_height { height }
{
    m_cells.resize(m_width * m_height);
}
\end{cpp}

下面是其余成员函数的实现:

\begin{cpp}
template <typename T, GridContainerType Container>
void Grid<T, Container>::verifyCoordinate(std::size_t x, std::size_t y) const
{
    if (x >= m_width) {
        throw std::out_of_range {
            std::format("x ({}) must be less than width ({}).", x, m_width) };
    }
    if (y >= m_height) {
        throw std::out_of_range {
            std::format("y ({}) must be less than height ({}).", y, m_height) };
    }
}

template <typename T, GridContainerType Container>
const typename Container::value_type&
    Grid<T, Container>::at(std::size_t x, std::size_t y) const
{
    verifyCoordinate(x, y);
    return m_cells[x + y * m_width];
}

template <typename T, GridContainerType Container>
typename Container::value_type&
    Grid<T, Container>::at(std::size_t x, std::size_t y)
{
    return const_cast<typename Container::value_type&>(
        std::as_const(*this).at(x, y));
}
\end{cpp}

现在可以这样实例化和使用Grid对象:

\begin{cpp}
Grid<int, vector<optional<int>>> myIntVectorGrid;
Grid<int, deque<optional<int>>> myIntDequeGrid;

myIntVectorGrid.at(3, 4) = 5;
println("{}", myIntVectorGrid.at(3, 4).value_or(0));

myIntDequeGrid.at(1, 2) = 3;
println("{}", myIntDequeGrid.at(1, 2).value_or(0));

Grid<int, vector<optional<int>>> grid2 { myIntVectorGrid };
grid2 = myIntVectorGrid;
\end{cpp}

可以尝试用容器模板类型参数double实例化Grid类模板:

\begin{cpp}
Grid<int, double> test; // WILL NOT COMPILE.
\end{cpp}

这一行代码无法编译,编译器会抱怨double类型不满足与Container模板类型参数关联概念的限制。

就像函数参数一样,可以为模板参数提供默认值。例如,Grid的默认容器是vector。然后类模板定义是这样:

\begin{cpp}
export template <typename T,
    GridContainerType Container = std::vector<std::optional<T>>>
class Grid
{
    // Everything else is the same as before.
};
\end{cpp}

可以使用第一个模板类型参数中的类型T,作为第二个模板类型参数默认值中optional模板的参数。C++语法要求,不能在成员函数定义的模板参数处重复默认值。有了这个默认参数,客户端现在可以实例化一个Grid,并选择性地指定一个底层容器:

\begin{cpp}
Grid<int, deque<optional<int>>> myDequeGrid;
Grid<int, vector<optional<int>>> myVectorGrid;
Grid<int> myVectorGrid2 { myVectorGrid };
\end{cpp}

标准库使用了这种方法。stack、queue、priority\_queue、flat\_(multi)set和flat\_(multi)map类模板都接受一个Container模板类型参数,并指定了底层容器的默认值。

\mySubsubsection{26.1.2.}{引入模板模板参数}

上一节中的Container参数存在一个问题,实例化类模板时:

\begin{cpp}
Grid<int, vector<optional<int>>> myIntGrid;
\end{cpp}

注意int类型的重复,必须同时指定Grid和vector中的optional的元素类型。如果这样呢:

\begin{cpp}
Grid<int, vector<optional<SpreadsheetCell>>> myIntGrid;
\end{cpp}

那就不会很好地工作。能够写出以下内容会很不错,这样就不会犯错误了:

\begin{cpp}
Grid<int, vector> myIntGrid;
\end{cpp}

Grid类模板应该能够确定它想要一个vector,包含int的optional。

然而,编译器不允许将那个参数传递给普通类型参数,因为vector本身不是一个类型而是一个模板。

如果想将一个模板作为模板类型参数,必须使用一种特殊类型的参数,称为双重模板参数。指定双重模板参数有点像在普通函数中指定函数指针参数,函数指针类型包括函数的返回类型和参数类型。类似地,当你指定双重模板参数时,双重模板参数的完整规范包括该模板的参数。

例如,vector和deque这样的容器有一个看起来像这样的模板参数列表。E参数是元素类型,Allocator参数在第25章中介绍过。

\begin{cpp}
template <typename E, typename Allocator = std::allocator<E>>
class vector { /* Vector definition */ };
\end{cpp}

要将这样的容器作为双重模板参数传递,所要做的就是复制粘贴类模板的声明(本例中,template <typename E, typename Allocator = std::allocator<E>{}> class vector)并用参数名(Container)替换类名(vector)。鉴于前面的模板规范,这里是将容器模板作为第二个模板参数的Grid类模板定义:

\begin{cpp}
export template <typename T,
    template <typename E, typename Allocator = std::allocator<E>> class Container
        = std::vector>
class Grid
{
    public:
        // Omitted code that is the same as before.
        std::optional<T>& at(std::size_t x, std::size_t y);
        const std::optional<T>& at(std::size_t x, std::size_t y) const;
        // Omitted code that is the same as before.
    private:
        void verifyCoordinate(std::size_t x, std::size_t y) const;
        Container<std::optional<T>> m_cells;
        std::size_t m_width { 0 }, m_height { 0 };
};
\end{cpp}

这里发生了什么?第一个模板参数与之前相同:元素类型 T。第二个模板参数现在是一个模板本身,用于诸如 vector 或 deque 这样的容器。这个“模板类型”必须接受两个参数:一个元素类型 E 和一个分配器类型。在 Grid 模板中的这个参数的名字是 Container(与之前一样)。现在的默认值是 vector,而不是 vector<T>,因为 Container 参数现在是一个模板,而不是一个实际类型。

双重模板参数的语法规则是这样的:

\begin{cpp}
template <..., template <TemplateTypeParams> class ParameterName, ...>
\end{cpp}

\begin{myNotic}{NOTE}
也可以使用 typename 关键字代替 class,如下面的例子:

\begin{cpp}
template <..., template <Params> typename ParameterName, ...>
\end{cpp}
\end{myNotic}

代码中,不能只使用 Container,必须指定 Container<std::optional<T>{}> 作为容器类型。例如,m\_cells 的声明如下所示:

\begin{cpp}
Container<std::optional<T>> m_cells;
\end{cpp}

成员函数定义不需要改变,除了必须改变模板声明,例如:

\begin{cpp}
template <typename T,
    template <typename E, typename Allocator = std::allocator<E>> class Container>
void Grid<T, Container>::verifyCoordinate(std::size_t x, std::size_t y) const
{
    // Same implementation as before...
}
\end{cpp}

这个 Grid 类模板可以按以下方式使用:

\begin{cpp}
Grid<int, vector> myGrid;
myGrid.at(1, 2) = 3;
println("{}", myGrid.at(1, 2).value_or(0));
Grid<int, vector> myGrid2 { myGrid };
Grid<int, deque> myDequeGrid;
\end{cpp}

这一节展示了可以将模板作为类型参数传递给其他模板,但这个语法看起来有点复杂,确实如此。我建议避免使用双重模板参数。实际上,标准库本身从不使用双重模板参数。

\mySubsubsection{26.1.3.}{非类型模板参数}

可能希望允许用户指定一个,用于初始化Grid中每个单元格的默认元素。下面是一个实现这一目标的完美合理的方法,使用 T\{\} 的零初始化语法作为第二个模板参数的默认值。

\begin{cpp}
export template <typename T, T DEFAULT = T{}>
class Grid { /* Identical as before. */ };
\end{cpp}

这个定义是合法的。可以使用第一个参数中的类型 T 作为第二个参数的类型,可以使用这个 T 的初始值来初始化Grid中的每个单元格:

\begin{cpp}
template <typename T, T DEFAULT>
Grid<T, DEFAULT>::Grid(std::size_t width, std::size_t height)
    : m_width { width }, m_height { height }
{
    m_cells.resize(m_width * m_height, DEFAULT);
}
\end{cpp}

其他成员函数定义保持不变,除了必须将第二个模板参数添加到模板声明中,所有 Grid<T> 的实例变为 Grid<T, DEFAULT>。做出这些更改之后,可以用所有元素的初始值实例化Grid:

\begin{cpp}
Grid<int> myIntGrid; // Initial value is int{}, i.e., 0
Grid<int, 10> myIntGrid2; // Initial value is 10
\end{cpp}

初始值可以是任何整数,假设按以下方式为 SpreadsheetCell 创建一个 Grid:

\begin{cpp}
SpreadsheetCell defaultCell;
Grid<SpreadsheetCell, defaultCell> mySpreadsheet; // WILL NOT COMPILE
\end{cpp}

第二行将导致编译错误,因为模板参数 DEFAULT 的值必须在编译时已知;但defaultCell 的值直到运行时才能知道,所以它不是 DEFAULT 可接受的值。

\begin{myNotic}{NOTE}
C++20 之前,非类型模板参数不能是对象,甚至不能是双精度浮点数或浮点数。它们仅限于整型、枚举、指针和引用。从 C++20 开始,这些限制有所放宽,现在允许非类型模板参数为浮点类型,甚至某些类类型。但这样的类类型有很多限制,本书中不再讨论。简单地说,SpreadsheetCell 类不符合这些限制。
\end{myNotic}





























