
模板在C++中提供了超出简单类和函数模板的能力,其中之一就是模板递归。模板递归与函数递归类似,函数递归中,通过调用自身来解决一个稍微简单版本的问题来进行的。本节首先提供了模板递归的动机,然后展示了如何实现。

\mySubsubsection{26.4.1.}{N维网格:第一次尝试}

到目前为止,Grid类模板只支持二维,这限制了它的实用性。如果想编写一个3D井字游戏,或者编写一个具有四维矩阵的数学程序呢?当然可以为每个维度编写一个模板化或非模板化的类,那将重复很多代码。另一种方法只编写一个一维网格,可以通过用另一个Grid作为其元素类型来实例化Grid,来创建任意维度的Grid。这个Grid元素类型本身可以实例化为以其自身为元素类型的Grid,依此类推。以下是OneDGrid类模板的实现,只是前面示例中Grid类模板的一维版本,增加了resize()成员函数,并用operator[]替换了at()。就像标准库容器vector一样,operator[]实现不执行边界检查。对于此示例,m\_elements存储T的实例,而不是std::optional<T>的实例。

\begin{cpp}
export template <typename T>
class OneDGrid final
{
    public:
        explicit OneDGrid(std::size_t size = DefaultSize) { resize(size); }

        T& operator[](std::size_t x) { return m_elements[x]; }
        const T& operator[](std::size_t x) const { return m_elements[x]; }

        void resize(std::size_t newSize) { m_elements.resize(newSize); }
        std::size_t getSize() const { return m_elements.size(); }

        static constexpr std::size_t DefaultSize { 10 };
    private:
        std::vector<T> m_elements;
};
\end{cpp}

有了这个OneDGrid实现,就可以创建多维网格了:

\begin{cpp}
OneDGrid<int> singleDGrid;
OneDGrid<OneDGrid<int>> twoDGrid;
OneDGrid<OneDGrid<OneDGrid<int>>> threeDGrid;
singleDGrid[3] = 5;
twoDGrid[3][3] = 5;
threeDGrid[3][3][3] = 5;
\end{cpp}

这段代码可以正常工作,但是声明很混乱。

\mySubsubsection{26.4.2.}{真正的N维网格}

可以使用模板递归来编写一个“真正的”N维网格,因为网格的维数本质上递归。可以在以下声明中看到:

\begin{cpp}
OneDGrid<OneDGrid<OneDGrid<int>>> threeDGrid;
\end{cpp}

可以将每个嵌套的OneDGrid视为一个递归步骤,以OneDGrid<int>为基本情况。换句话说,一个三维网格是一个由,单维网格组成的单维网格的单维网格的int。可以让类模板为完成这个递归,可以像这样创建N维网格:

\begin{cpp}
NDGrid<int, 1> singleDGrid;
NDGrid<int, 2> twoDGrid;
NDGrid<int, 3> threeDGrid;
\end{cpp}

NDGrid类模板接受一个元素类型和一个指定其“维数”的整数。关键在于,NDGrid的元素类型不是模板参数列表中指定的元素类型,而实际上是一个维数比当前维数少一的NDGrid。换句话说,一个三维网格是一个二维网格的vector;每个二维网格是单维网格的vector。

有了递归,就可以为维数为1的NDGrid编写一个偏特化,其中元素类型不是另一个NDGrid,而是实际上模板参数中指定的元素类型。

下面显示了NDGrid类模板的定义和实现,与上一节中的OneDGrid不同。m\_elements数据成员现在是NDGrid<T, N-1>的vector;这就开始递归了。此外,operator[]返回对元素类型的引用,该元素类型是NDGrid<T, N-1>,而不是T。

除了模板递归本身之外,实现中最棘手的部分是适当地设置网格的每个维度的大小。这个实现在创建N维网格时,使每个维度的大小都相等。指定每个维度的单独大小很困难,用户应该能够创建具有指定大小(如20或50)的网格。因此,构造函数接受一个整数大小参数。resize()成员函数被修改为调整m\_elements的大小,并使用NDGrid<T, N-1> \{ newSize \}初始化每个元素,这会递归地将网格的所有维度调整为新的大小。

\begin{cpp}
export template <typename T, std::size_t N>
class NDGrid final
{
    public:
        explicit NDGrid(std::size_t size = DefaultSize) { resize(size); }

        NDGrid<T, N-1>& operator[](std::size_t x) { return m_elements[x]; }
        const NDGrid<T, N-1>& operator[](std::size_t x) const {
            return m_elements[x]; }

        void resize(std::size_t newSize)
        {
            m_elements.resize(newSize, NDGrid<T, N-1> { newSize });
        }

        std::size_t getSize() const { return m_elements.size(); }

        static constexpr std::size_t DefaultSize { 10 };
    private:
        std::vector<NDGrid<T, N-1>> m_elements;
};
\end{cpp}

模板递归是对维度为1的模板进行偏特化,下面展示了定义和实现。由于特化从不从主模板继承代码,因此必须重写大量的代码。

\begin{cpp}
export template <typename T>
class NDGrid<T, 1> final
{
    public:
        explicit NDGrid(std::size_t size = DefaultSize) { resize(size); }

        T& operator[](std::size_t x) { return m_elements[x]; }
        const T& operator[](std::size_t x) const { return m_elements[x]; }

        void resize(std::size_t newSize) { m_elements.resize(newSize); }
        std::size_t getSize() const { return m_elements.size(); }

        static constexpr std::size_t DefaultSize { 10 };
    private:
        std::vector<T> m_elements;
};
\end{cpp}

这里,递归结束:元素类型是T,而不是另一个模板实例化。

现在,可以编写如下代码:

\begin{cpp}
NDGrid<int, 3> my3DGrid { 4 };
my3DGrid[2][1][2] = 5;
my3DGrid[1][1][1] = 5;
println("{}", my3DGrid[2][1][2]);
\end{cpp}

为了避免主模板和特化之间的代码重复,可以将重复的代码提取到一个基类中,然后从该基类派生主模板和特化;但在这个小型示例中,增加的开销将超过节省的代码。













