第12章中展示的 const char* 类型的 Grid 类模板特化称为全特化模板,它特化了 Grid 模板的所有模板参数,特化中不再有模板参数。这并不是唯一可以特化类的方式;还可以编写部分类模板特化,在其中特化一些模板参数而不特化其他参数。例如,带有宽度(width)和高度(height)非类型参数的 Grid 模板:

\begin{cpp}
export template <typename T, std::size_t WIDTH, std::size_t HEIGHT>
class Grid
{
    public:
        Grid() = default;
        virtual ~Grid() = default;

        // Explicitly default a copy constructor and assignment operator.
        Grid(const Grid& src) = default;
        Grid& operator=(const Grid& rhs) = default;

        // Explicitly default a move constructor and assignment operator.
        Grid(Grid&& src) = default;
        Grid& operator=(Grid&& rhs) = default;

        std::optional<T>& at(std::size_t x, std::size_t y);
        const std::optional<T>& at(std::size_t x, std::size_t y) const;

        std::size_t getHeight() const { return HEIGHT; }
        std::size_t getWidth() const { return WIDTH; }
    private:
        void verifyCoordinate(std::size_t x, std::size_t y) const;
        std::optional<T> m_cells[WIDTH][HEIGHT];
};.
\end{cpp}

可以将这个类模板特化为const char*:

\begin{cpp}
export template <std::size_t WIDTH, std::size_t HEIGHT>
class Grid<const char*, WIDTH, HEIGHT>
{
    public:
        Grid() = default;
        virtual ~Grid() = default;

        // Explicitly default a copy constructor and assignment operator.
        Grid(const Grid& src) = default;
        Grid& operator=(const Grid& rhs) = default;

        // Explicitly default a move constructor and assignment operator.
        Grid(Grid&& src) = default;
        Grid& operator=(Grid&& rhs) = default;

        std::optional<std::string>& at(std::size_t x, std::size_t y);
        const std::optional<std::string>& at(std::size_t x, std::size_t y) const;

        std::size_t getHeight() const { return HEIGHT; }
        std::size_t getWidth() const { return WIDTH; }
    private:
        void verifyCoordinate(std::size_t x, std::size_t y) const;

        std::optional<std::string> m_cells[WIDTH][HEIGHT];
};
\end{cpp}

这种情况下,没有特化所有模板参数,模板声明看起来像这样:

\begin{cpp}
export template <std::size_t WIDTH, std::size_t HEIGHT>
class Grid<const char*, WIDTH, HEIGHT>
\end{cpp}

这个类模板只有两个参数:WIDTH 和 HEIGHT,但正在为一个有三个参数的 Grid 类进行实现:T、WIDTH 和 HEIGHT。模板参数列表包含两个参数,而显式的 Grid<const char*, WIDTH, HEIGHT> 包含三个参数。实例化模板时,必须指定三个参数,不能仅用高度和宽度来实例化模板。

\begin{cpp}
Grid<int, 2, 2> myIntGrid; // Uses the original Grid
Grid<const char*, 2, 2> myStringGrid; // Uses the partial specialization
Grid<2, 3> test; // DOES NOT COMPILE! No type specified.
\end{cpp}

嗯...这个语法可能令人困惑。此外,偏特化与全特化不同,必须在每个成员函数定义前,包含模板声明:

\begin{cpp}
template <std::size_t WIDTH, std::size_t HEIGHT>
const std::optional<std::string>&
    Grid<const char*, WIDTH, HEIGHT>::at(std::size_t x, std::size_t y) const
{
    verifyCoordinate(x, y);
    return m_cells[x][y];
}
\end{cpp}

需要带有两个参数的模板声明,来标识这个成员函数是针对这两个参数进行的参数化。无论在哪里引用完整的类名,都必须使用 Grid<const char*, WIDTH, HEIGHT>。

前面的例子并没有显示出偏特化化的强大之处,可以编写特化为可能类型的子集的实现,而不必特化单个类型。例如,一个特化为所有指针类型的 Grid 类模板。这个特化的复制构造函数和赋值运算符,对指针指向的对象执行深拷贝,而不是浅拷贝。

以下是类定义,假设正在特化仅有一个模板参数的Grid。这个实现中,Grid 成为提供数据的所有者,它会在需要时自动释放内存,复制/移动构造函数和复制/移动赋值运算符是必需的。复制赋值运算符使用复制并交换习语,移动赋值运算符使用移动并交换习语,如第9章中介绍的那样,这需要一个 noexcept swap() 成员函数。

\begin{cpp}
export template <typename U>
class Grid<U*>
{
    public:
        explicit Grid(std::size_t width = DefaultWidth,
            std::size_t height = DefaultHeight);
        virtual ~Grid() = default;

        // Copy constructor and copy assignment operator.
        Grid(const Grid& src);
        Grid& operator=(const Grid& rhs);

        // Move constructor and move assignment operator.
        Grid(Grid&& src) noexcept;
        Grid& operator=(Grid&& rhs) noexcept;

        void swap(Grid& other) noexcept;

        std::unique_ptr<U>& at(std::size_t x, std::size_t y);
        const std::unique_ptr<U>& at(std::size_t x, std::size_t y) const;

        std::size_t getHeight() const { return m_height; }
        std::size_t getWidth() const { return m_width; }

        static constexpr std::size_t DefaultWidth { 10 };
        static constexpr std::size_t DefaultHeight { 10 };
    private:
        void verifyCoordinate(std::size_t x, std::size_t y) const;

        std::vector<std::unique_ptr<U>> m_cells;
        std::size_t m_width { 0 }, m_height { 0 };
};
\end{cpp}

以下两行是问题的关键:

\begin{cpp}
export template <typename U>
class Grid<U*>
\end{cpp}

这个语法表明这个类模板是 Grid 类模板,是对所有指针类型的特化。当有一个如 Grid<int*> 的实例化时,U 是 int,而不是 int*。这可能有点违反直觉,但这就是它的工作方式。

以下是使用这个偏特化的例子:

\begin{cpp}
Grid<int> myIntGrid; // Uses the non-specialized grid.
Grid<int*> psGrid { 2, 2 }; // Uses the partial specialization for pointer types.

psGrid.at(0, 0) = make_unique<int>(1);
psGrid.at(0, 1) = make_unique<int>(2);
psGrid.at(1, 0) = make_unique<int>(3);

Grid<int*> psGrid2 { psGrid };
Grid<int*> psGrid3;
psGrid3 = psGrid2;

auto& element { psGrid2.at(1, 0) };
if (element != nullptr) {
    println("{}", *element);
    *element = 6;
}
println("{}", *psGrid.at(1, 0)); // psGrid is not modified.
println("{}", *psGrid2.at(1, 0)); // psGrid2 is modified.
\end{cpp}

这是输出结果:

\begin{shell}
3
3
6
\end{shell}

成员函数的实现相当简单,除了复制构造函数,使用单个元素的复制构造函数可进行深拷贝:

\begin{cpp}
template <typename U>
Grid<U*>::Grid(const Grid& src)
    : Grid { src.m_width, src.m_height }
{
    // The ctor-initializer of this constructor delegates first to the
    // non-copy constructor to allocate the proper amount of memory.

    // The next step is to copy the data.
    for (std::size_t i { 0 }; i < m_cells.size(); ++i) {
        // Make a deep copy of the element by using its copy constructor.
        if (src.m_cells[i] != nullptr) {
            m_cells[i] = std::make_unique<U>(*src.m_cells[i]);
        }
    }
}
\end{cpp}






































