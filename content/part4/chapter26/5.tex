
普通模板只能接受固定数量的模板参数,变长模板可以接受可变数量的模板参数。以下代码定义了一个模板,使用一个名为 Types 的参数包,可以接受任意数量的模板参数:

\begin{cpp}
template <typename... Types>
class MyVariadicTemplate { };
\end{cpp}

\begin{myNotic}{NOTE}
typename 后面的三个点不是错误。这是定义变长模板参数包的语法。参数包是可以接受可变数量参数的东西。
\end{myNotic}

可以用任意数量的模板实参实例化 MyVariadicTemplate:

\begin{cpp}
MyVariadicTemplate<int> instance1;
MyVariadicTemplate<string, double, vector<int>> instance2;
\end{cpp}

甚至可以用零个模板实参实例化:

\begin{cpp}
MyVariadicTemplate<> instance3;
\end{cpp}

要禁止用零个模板实参实例化变长模板:

\begin{cpp}
template <typename T1, typename... Types>
class MyVariadicTemplate { };
\end{cpp}

有了这个定义,尝试用零个模板实参实例化 MyVariadicTemplate 时,将出现编译错误。

不可能直接迭代变长模板给定的参数,唯一的方式是借助模板递归或折叠表达式。以下部分展示了这两种方法。

\mySubsubsection{26.5.1.}{类型安全的可变长度参数列表}

变长模板允许创建类型安全的可变长度参数列表。以下示例定义了一个名为 processValues() 的变长模板,以类型安全的方式接受任意数量的参数,这些参数具有不同的类型。processValues() 函数处理可变长度参数列表中的每个值,并为每个单一参数执行一个名为 handleValue() 的函数。所以,需要为每个想要处理的类型(这个例子中是 int、double 和 string)实现 handleValue() 的重载版本:

\begin{cpp}
void handleValue(int value) { println("Integer: {}", value); }
void handleValue(double value) { println("Double: {}", value); }
void handleValue(const string& value) { println("String: {}", value); }

void processValues() // Base case to stop recursion
{ /* Nothing to do in this base case. */ }

template <typename T1, typename... Tn>
void processValues(const T1& arg1, const Tn&... args)
{
    handleValue(arg1);
    processValues(args...);
}
\end{cpp}

这个例子展示了 triple dots (...) 操作符的双重使用。这个操作符在三个地方出现,有两个不同的含义。首先,模板参数列表中 typename 后面,和函数参数列表中的类型 Tn 后面,都表示参数包。

参数包可以接受可变数量的参数。

操作符的第二个使用是在函数体中的参数名 args 后面,意味着参数包展开;操作符将参数包展开为单独的参数,并用逗号分隔:

\begin{cpp}
processValues(args...);
\end{cpp}

这个语句将 args 参数包展开为它的单独参数,并用逗号分隔,然后用展开后的参数列表调用 processValues() 函数。模板始终需要至少一个参数,T1。递归调用 processValues() 带 args... 所以每次调用都少一个参数。

因为 processValues() 函数的实现是递归的,所以需要有一种方法来停止递归。可以通过实现一个不接受参数的 processValues() 函数来实现。

可以像这样测试 processValues() 变长模板:

\begin{cpp}
processValues(1, 2, 3.56, "test", 1.1f);
\end{cpp}

这个例子生成的递归调用如下所示:

\begin{cpp}
processValues(1, 2, 3.56, "test", 1.1f);
    handleValue(1);
        processValues(2, 3.56, "test", 1.1f);
        handleValue(2);
        processValues(3.56, "test", 1.1f);
            handleValue(3.56);
            processValues("test", 1.1f);
                handleValue("test");
                processValues(1.1f);
                    handleValue(1.1f);
                    processValues();
\end{cpp}

这种可变长度参数列表的实现完全类型安全。processValues() 函数会根据实际类型会自动调用正确的 handleValue() 重载。当以某种类型调用 processValues() 且未定义对应的 handleValue() 重载时,编译器会报错。

也可以在 processValues() 的实现中使用在第 12 章介绍的转发引用,以下实现使用了转发引用 T\&\& 和 std::forward() 进行完美转发所有参数。完美转发意味着如果传递给 processValues() 的参数是 右值,会转发 右值 引用。如果传递的是 左值,会转发 左值 引用。

\begin{cpp}
void processValues() // Base case to stop recursion
{ /* Nothing to do in this base case.*/ }

template <typename T1, typename... Tn>
void processValues(T1&& arg1, Tn&&... args)
{
    handleValue(forward<T1>(arg1));
    processValues(forward<Tn>(args)...);
}
\end{cpp}

需要进一步解释的是:

\begin{cpp}
processValues(forward<Tn>(args)...);
\end{cpp}

操作符 ... 用于展开参数包,对参数包中的每个单独参数使用 std::forward() 并将其与逗号分隔开。args 是一个包含三个参数 a1、a2 和 a3 的参数包,这些参数分别属于三种类型 A1、A2 和 A3。展开如下所示:

\begin{cpp}
processValues(forward<A1>(a1),
              forward<A2>(a2),
              forward<A3>(a3));
\end{cpp}

使用了参数包的函数体内,可以使用 sizeof...(pack) 来检索参数包中的参数数量。这并不是使用 ... 进行的包展开,而是使用了特殊的关键字样式的 sizeof... 语法。

\begin{cpp}
int numberOfArguments { sizeof...(args) };
\end{cpp}

使用变长模板的实际例子是,编写一个安全且类型安全的 printf() 样式的函数模板。这是一个很好的实践练习,读者们可以自行尝试一下。

\mySamllsection{constexpr if}

constexpr if 语句是在编译时执行的 if 语句,而不是在运行时。如果 constexpr if 语句的某个分支从未采用,将永远不会编译,这种编译时的决策在变长模板中非常有用。例如,processValues() 的早期实现需要一个基础情况来停止递归(void processValues() \{\})。使用 constexpr if,可以避免这种基础情况。这个特性正式称为 constexpr if,但在实际代码中写为 if constexpr。

\begin{cpp}
template <typename T1, typename... Tn>
void processValues(T1&& arg1, Tn&&... args)
{
    handleValue(forward<T1>(arg1));
    if constexpr (sizeof...(args) > 0) {
        processValues(forward<Tn>(args)...);
    }
}
\end{cpp}

这个实现中,递归会在 可变 参数包 args 变为空时立即停止。与之前的实现相比,唯一的区别是不能再调用 processValues(),否则做会导致编译错误。

\mySubsubsection{26.5.2.}{可变数量的混合类}

参数包几乎可以用于任何地方,以下代码使用参数包来为 MyClass 定义可变数量的混合类。第 5 章介绍了混合类的概念。

\begin{cpp}
class Mixin1
{
    public:
        explicit Mixin1(int i) : m_value { i } {}
        virtual void mixin1Func() { println("Mixin1: {}", m_value); }
    private:
        int m_value;
};

class Mixin2
{
    public:
        explicit Mixin2(int i) : m_value { i } {}
        virtual void mixin2Func() { println("Mixin2: {}", m_value); }
    private:
        int m_value;
};

template <typename... Mixins>
class MyClass : public Mixins...
{
    public:
        explicit MyClass(const Mixins&... mixins) : Mixins { mixins }... {}
        virtual ~MyClass() = default;
};
\end{cpp}

这段代码首先定义了两个混合类:Mixin1 和 Mixin2,这个例子中的定义相当简单。构造函数接受一个整数,将其存储,并且有一个函数来打印有关特定类实例的信息。MyClass 变长模板使用参数包 typename... Mixins 来接受可变数量的混合类,在该类从所有这些混合类中继承,并且构造函数接受相同数量的参数来初始化每个继承的混合类。... 展开操作符是将操作符左侧的内容,重复为参数包中的每个模板参数进行,并用逗号分隔。该类可以这样使用:

\begin{cpp}
MyClass<Mixin1, Mixin2> a { Mixin1 { 11 }, Mixin2 { 22 } };
a.mixin1Func();
a.mixin2Func();

MyClass<Mixin1> b { Mixin1 { 33 } };
b.mixin1Func();
//b.mixin2Func(); // Error: does not compile.

MyClass<> c;
//c.mixin1Func(); // Error: does not compile.
//c.mixin2Func(); // Error: does not compile.
\end{cpp}

尝试在 b 上调用 mixin2Func() 时,会得到一个编译错误,因为 b 没有从 Mixin2 类继承。该程序的输出如下所示:

\begin{shell}
Mixin1: 11
Mixin2: 22
Mixin1: 33
\end{shell}

\mySubsubsection{26.5.3.}{折叠表达式}

C++支持折叠表达式,这使得在变长模板中处理参数包变得更加容易。折叠表达式可以用于对参数包中的每个值应用特定操作,将参数包中的所有值减少为单一值,以及更多操作。

以下表格列出了支持的四种折叠类型。在这个表格中,Ѳ可以是以下任一操作符:+, -, *, /, \%, \^{}, \&, |, <{}<, >{}>, +=, -=, *=, /=, \%=, \^{}=, \&=, |=, <{}<=, >{}>=, =, ==, !=, <, >, <=, >=, \&\&, ||, .*,d ->*.

% Please add the following required packages to your document preamble:
% \usepackage{longtable}
% Note: It may be necessary to compile the document several times to get a multi-page table to line up properly
\begin{longtable}{|l|l|l|}
\hline
\textbf{名称}             & \textbf{表达式} & \textbf{展开为}             \\ \hline
\endfirsthead
%
\endhead
%
\textbf{一元右折叠} & (pack Ѳ . . .)      & $pack_0$ Ѳ (. . . Ѳ ($pack_{n-1}$ Ѳ $pack_n$)) \\ \hline
\textbf{一元左折叠}  & (. . . Ѳ pack)      & (($pack_0$ Ѳ $pack_1$) Ѳ . . .) Ѳ $pack_n$   \\ \hline
\textbf{二元右折叠} & (pack Ѳ . . . Ѳ Init) & $pack_0$ Ѳ (. . . Ѳ ($pack_{n-1}$ Ѳ ($pack_n$ Ѳ Init))) \\ \hline
\textbf{二元左折叠}  & (Init Ѳ . . . Ѳ pack) & (((Init Ѳ $pack_0$) Ѳ $pack_1$) Ѳ . . .) Ѳ $pack_n$   \\ \hline
\end{longtable}

来看一些例子。以前,processValue()函数模板可递归定义:

\begin{cpp}
void processValues() { /* Nothing to do in this base case.*/ }

template <typename T1, typename... Tn>
void processValues(T1&& arg1, Tn&&... args)
{
    handleValue(forward<T1>(arg1));
    processValues(forward<Tn>(args)...);
}
\end{cpp}

因为递归定义,所以需要一个基情况来停止递归。使用折叠表达式,这可以实现为一个函数模板,使用逗号操作符的一元右折叠:

\begin{cpp}
template <typename... Tn>
void processValues(Tn&&... args) { (handleValue(forward<Tn>(args)) , ...); }
\end{cpp}

函数体中的三个点触发了用Ѳ对逗号操作符的折叠。那一行展开为对参数包中的每个参数调用handleValue(),每个调用handleValue()之间由逗号分隔。args是一个包含三个参数的参数包,分别为a1, a2, 和 a3,类型分别为A1, A2, 和 A3。一元右折叠的展开如下所示:

\begin{cpp}
(handleValue(forward<A1>(a1)) ,
    (handleValue(forward<A2>(a2)) , handleValue(forward<A3>(a3))));
\end{cpp}

这里有一个其他的例子。printValues()函数模板将其所有参数写入控制台,并用换行符分隔。

\begin{cpp}
template <typename... Values>
void printValues(const Values&... values) { (println("{}", values) , ...); }
\end{cpp}

假设values是一个包含三个参数的参数包,分别为v1, v2和 v3。一元右折叠的展开:

\begin{cpp}
(println("{}", v1) , (println("{}", v2) , println("{}", v3)));
\end{cpp}

你可以用任意数量的参数调用printValues(),例如:

\begin{cpp}
printValues(1, "test", 2.34);
\end{cpp}

之前的示例中,折叠是通过逗号操作符完成,但它可以与任何类型的操作符一起使用。以下代码定义了一个使用二元左折叠的变长函数模板,用于计算传递给它的所有值的和,二元左折叠总是需要一个Init值(请参阅之前的概述表)。因此,sumValues()有两个模板类型参数:一个普通的参数来指定Init的类型,以及一个可以接受0个或多个参数的参数包。

\begin{cpp}
template <typename T, typename... Values>
auto sumValues(const T& init, const Values&... values)
{ return (init + ... + values);}
\end{cpp}

假设values是一个包含三个参数的参数包,分别为v1, v2, 和 v3。这种情况下,二元左折叠的展开为:

\begin{cpp}
return (((init + v1) + v2) + v3);
\end{cpp}

sumValues()函数模板可以按照以下方式进行测试:

\begin{cpp}
println("{}", sumValues(1, 2, 3.3));
println("{}", sumValues(1));
\end{cpp}

sumValues()函数模板也可以用一元左折叠来定义。

\begin{cpp}
template <typename... Values>
auto sumValues(const Values&... values) { return (... + values); }
\end{cpp}

第12章中讨论的概念也可以是变长的。可以限制sumValues()函数模板,以便只能用相同类型的参数集调用。

\begin{cpp}
template <typename T, typename... Us>
concept SameTypes = (std::same_as<T, Us> && ...);

template <typename T, typename... Values>
    requires SameTypes<T, Values...>
auto sumValues(const T& init, const Values&... values)
{ return (init + ... + values); }
\end{cpp}

按照以下方式调用受约束的版本:

\begin{cpp}
println("{}", sumValues(1.1, 2.2, 3.3)); // OK: 3 doubles, output is 6.6
println("{}", sumValues(1)); // OK: 1 integer, output is 1
println("{}", sumValues("a"s, "b"s)); // OK: 2 strings, output is ab
\end{cpp}

以下调用失败,因为参数列表包含一个整数和两个双精度浮点数:

\begin{cpp}
println("{}", sumValues(1, 2.2, 3.3)); // Error
\end{cpp}

参数包长度为零时允许进行一元折叠,但仅在与逻辑与(\&\&)、逻辑或(||)和逗号(,)操作符的组合中。对于空参数包,\&\&得到true,||得到false,,得到void(),即一个空操作。例如:

\begin{cpp}
template <typename... Values>
bool allEven(const Values&... values) { return ((values % 2 == 0) && ...); }

template <typename... Values>
bool anyEven(const Values&... values) { return ((values % 2 == 0) || ...); }

int main()
{
    println("{} {} {}", allEven(2,4,6), allEven(2,3), allEven());//true false true
    println("{} {} {}", anyEven(1,2,3), anyEven(1,3), anyEven());//true false false
}
\end{cpp}
















