
如果正在编写多线程应用程序,必须对操作的顺序敏感。线程读取和写入共享数据,这可能会成为一个问题。避免这个问题的方法有很多,例如:永远不要实际在多个线程之间共享数据。如果无法避免共享数据,就必须提供同步,以便每次只有一个线程可以更改数据。

像布尔和整数这样的标量,可以通过原子操作正确地进行同步;但当数据更复杂,多个线程要使用该数据时,就需要显式的提供同步。

标准库在互斥锁和锁类中提供了对互斥的支持。这些可以用于实现线程之间的同步,并在以下各节中进行讨论。

\mySubsubsection{27.4.1.}{互斥锁类}

互斥锁代表互斥访问,基本机制为:

\begin{itemize}
\item
一个线程尝试锁定一个与另一个线程共享的内存,以便读取或写入。如果另一个线程当前持有锁,则新的线程将阻塞,直到锁释放或超时。

\item
当线程获得了锁,就可以自由地使用共享内存。这里,假设所有使用共享数据的线程都参与到互斥机制中。

\item
线程完成对共享内存的读取/写入后,会释放锁,以便其他线程有机会获得互斥锁。如果有两个或更多的线程正在等待锁,不保证哪个线程将先获得锁。
\end{itemize}

C++标准库提供了非定时和定时互斥锁类,以及递归和非递归版本。讨论所有这些之前,先来看一下自旋锁。

\mySamllsection{自旋锁}

自旋锁是一种同步机制,其中线程使用忙循环(自旋)来尝试获取锁,执行其工作,然后释放锁。自旋过程中,线程保持活跃但不做任何工作。另一方面,如果互斥锁无法立即获取,互斥锁可能会阻塞线程。阻塞线程是一个昂贵的操作,自旋锁避免了这种操作。自旋锁在知道锁将只持有很短时间的情况下有用。自旋锁完全可以自己实现,如以下代码所示,自旋锁可以使用单个原子类型:atomic\_flag。

\begin{cpp}
static constexpr unsigned NumberOfThreads { 50 };
static constexpr unsigned LoopsPerThread { 100 };
void dowork(unsigned threadNumber, vector<unsigned>& data, atomic_flag& spinlock)
{
    for (unsigned i { 0 }; i < LoopsPerThread; ++i) {
        while (spinlock.test_and_set()) { } // Spins until lock is acquired.
        // Safe to handle shared data...
        data.push_back(threadNumber);
        spinlock.clear(); // Releases the acquired lock.
    }
}

int main()
{
    vector<unsigned> data;
    atomic_flag dataSpinlock;
    vector<jthread> threads;
    for (unsigned i { 0 }; i < NumberOfThreads; ++i) {
        threads.emplace_back(dowork, i, ref(data), ref(dataSpinlock));
    }
    for (auto& t : threads) { t.join(); }
    println("data contains {} elements, expected {}.", data.size(),
        NumberOfThreads * LoopsPerThread);
}
\end{cpp}

每个线程通过重复调用test\_and\_set()在一个atomic\_flag上,直到成功获取锁,这就是忙循环。

\begin{myWarning}{WARNING}
由于自旋锁使用忙等待循环,应该只在确定线程只会短暂地锁定自旋锁时才作为候选。
\end{myWarning}

现在来看看标准库提供了哪些互斥锁类。

\mySamllsection{非定时互斥锁类}

标准库有三个非定时互斥锁类:std::mutex、recursive\_mutex和shared\_mutex。前两个类在<mutex>中定义,最后一个在<shared\_mutex>中定义。每个互斥锁都支持以下成员函数:

\begin{itemize}
\item
lock(): 调用线程尝试获取锁,并在锁被获取之前一直阻塞,会无限期地阻塞。如果希望限制线程阻塞的时间,应该使用定时互斥锁,这将在下一节讨论。

\item
try\_lock(): 调用线程尝试获取锁。如果锁当前由另一个线程持有,调用立即返回。如果已获取锁,try\_lock()返回true;否则,返回false。

\item
unlock(): 调用线程释放当前持有的锁,使其对另一个线程可用。
\end{itemize}

std::mutex是一个具有独占所有权的标准互斥锁类,只能有一个线程拥有这个互斥锁。如果另一个线程想要获取这个互斥锁的所有权,要么在使用lock()时阻塞,要么在使用try\_lock()时失败。已经拥有互斥锁的线程,不允许再次对该互斥锁调用lock()或try\_lock();否则,会死锁!

std::recursive\_mutex的行为与mutex相同,只是已经拥有recursive\_mutex所有权的线程允许再次对同一个recursive\_mutex调用lock()或try\_lock()。调用线程应该调用unlock()成员函数,其次数与获取recursive\_mutex锁的次数相同。

shared\_mutex类支持共享锁所有权,也称为读写锁。线程可以获取锁的独占所有权或共享所有权。独占所有权,也称为写锁,只能在没有其他线程具有独占或共享所有权时获取。共享所有权,也称为读锁,可以在没有其他线程具有独占所有权时获取,即使其他线程已经获取了自己的共享所有权。shared\_mutex类支持lock()、try\_lock()和unlock(),这些成员函数获取和释放独占锁。此外,还有以下与共享所有权相关的成员函数:lock\_shared()、try\_lock\_shared()和unlock\_shared()。这些函数的工作方式与前面的函数类似,但尝试获取或释放共享所有权。

已经对shared\_mutex拥有锁的线程,不能再次尝试获取该互斥锁的第二个锁。这可能导致死锁!

讨论如何使用这些互斥锁类的示例之前,还有一些其他主题需要首先讨论。示例将在本章后面的“使用互斥锁的示例”部分介绍。

\begin{myWarning}{WARNING}
不要手动调用互斥锁类的lock和unlock成员函数。互斥锁是一种资源,就像所有资源一样,它们应该使用 Resource Acquisition Is Initialization (RAII) 范式。C++标准库提供了一系列RAII锁类,将在本章后面的“锁”部分介绍,这对于避免死锁至关重要。会在锁对象超出作用域时自动解锁互斥锁,无需在正确的时间手动调用unlock()。
\end{myWarning}

\mySamllsection{定时互斥锁类}

调用互斥锁类的lock()时会阻塞,直到锁可以获取。另一方面,调用那些互斥锁类的try\_lock()尝试获取锁,但如果不成功,则立即返回。还有定时互斥锁类,可以在一定时间内尝试获取锁,但在时间耗尽后放弃。

标准库提供了三个定时互斥锁类:std::timed\_mutex、recursive\_timed\_mutex和shared\_timed\_mutex。前两个类在<mutex>中定义,最后一个在<shared\_mutex>中定义。都支持lock()、try\_lock()和unlock()成员函数;shared\_timed\_mutex还支持lock\_shared()、try\_lock\_shared()和unlock\_shared(),行为与前一节中描述的相同。此外,还支持以下成员函数:

\begin{itemize}
\item
try\_lock\_for(rel\_time): 调用线程尝试在一定相对时间内获取锁。如果在给定的超时时间内无法获取锁,调用失败并返回false。如果在超时时间内可以获取锁,调用成功并返回true。超时时间是一个std::chrono::duration。

\item
try\_lock\_until(abs\_time): 调用线程尝试在系统时间等于或超过指定绝对时间之前获取锁。如果在给定的绝对时间之前可以获取锁,调用返回true。如果系统时间超过给定的绝对时间,函数停止尝试获取锁并返回false。绝对时间为std::chrono::time\_point。
\end{itemize}

shared\_timed\_mutex还支持try\_lock\_shared\_for()和try\_lock\_shared\_until()。

已经拥有定时互斥锁或共享定时互斥锁所有权的线程,不能再次在同一互斥锁上获取锁。

recursive\_timed\_mutex允许线程多次获取锁,就像recursive\_mutex一样。

\mySubsubsection{27.4.2.}{锁}

锁类是一种RAII类,使正确获取和释放互斥锁变得更加容易;锁类的析构函数会自动释放关联的互斥锁。C++标准定义了四种锁:std::lock\_guard、unique\_lock、shared\_lock和scoped\_lock。

\mySamllsection{lock\_guard}

lock\_guard,定义在<mutex>中,是一个简单的锁,具有两个构造函数:

\begin{cpp}
explicit lock_guard(mutex_type& m);
\end{cpp}

接受一个互斥锁类型的引用作为参数的构造函数。尝试获取互斥锁上的锁并阻塞,直到获取锁。

\begin{cpp}
lock_guard(mutex_type& m, adopt_lock_t);
\end{cpp}

接受一个互斥锁类型的引用和一个第二个参数,该参数等于std::adopt\_lock。这是std::adopt\_lock\_t类型的全局常量,由标准库提供,专用于此目的。锁假定调用线程已经对引用的互斥锁调用了lock()。lock\_guard“收纳”互斥锁,并在lock\_guard销毁时自动释放互斥锁。

\mySamllsection{unique\_lock}

std::unique\_lock,定义在<mutex>中,是一个更复杂的锁,允许推迟获取锁,直到执行的后期,即声明之后很长时间。可以使用owns\_lock()成员函数或unique\_lock的布尔转换操作符来查看锁是否已经获取。本书后面章节中会演示使用此转换操作符的一个示例,unique\_lock有多个构造函数:

\begin{cpp}
explicit unique_lock(mutex_type& m);
\end{cpp}

接受一个互斥锁类型的引用作为参数。尝试获取互斥锁上的锁,并阻塞,直到获取锁。

\begin{cpp}
unique_lock(mutex_type& m, defer_lock_t) noexcept;
\end{cpp}

接受一个互斥锁类型的引用和一个std::defer\_lock\_t的实例,例如std::defer\_lock。unique\_lock存储互斥锁的引用,但不会立即尝试获取锁,可以在稍后获取锁。

\begin{cpp}
unique_lock(mutex_type& m, try_to_lock_t);
\end{cpp}

接受一个互斥锁类型的引用和一个std::try\_to\_lock\_t的实例,例如std::try\_to\_lock。尝试获取引用互斥锁上的锁,但如果失败,不会阻塞,可以在稍后获取锁。

\begin{cpp}
unique_lock(mutex_type& m, adopt_lock_t);
\end{cpp}

接受一个互斥锁类型的引用和一个std::adopt\_lock\_t的实例,例如std::adopt\_lock。假定调用线程已经对引用的互斥锁调用了lock()。unique\_lock“收纳”互斥锁,并在unique\_lock销毁时自动释放互斥锁。

\begin{cpp}
unique_lock(mutex_type& m, const chrono::time_point<Clock, Duration>& abs_time);
\end{cpp}

接受一个互斥锁类型的引用和一个绝对时间。尝试获取锁,直到系统时间超过给定的时间。

\begin{cpp}
unique_lock(mutex_type& m, const chrono::duration<Rep, Period>& rel_time);
\end{cpp}

接受一个互斥锁类型的引用和一个相对时间,尝试在给定的相对超时时间内获取锁。

unique\_lock类还具有lock()、try\_lock()、try\_lock\_for()、try\_lock\_until()和unlock()等成员函数,其行为如本章前面“定时互斥锁类”部分所述。

\mySamllsection{shared\_lock}

std::shared\_lock类,定义在<shared\_mutex>中,具有与unique\_lock相同的构造函数和成员函数。区别在于shared\_lock调用在底层共享互斥锁上的共享所有权相关成员函数。shared\_lock的成员函数是lock()、try\_lock()等,但在底层共享互斥锁上,会调用lock\_shared()、try\_lock\_shared()等。这样做是为了使shared\_lock具有与unique\_lock相同的接口,可以作为unique\_lock的替换使用。但获取的是共享锁,而不是独占锁。

\mySamllsection{同时获取多个锁}

C++有两个通用的锁函数,可以同时获取多个互斥锁对象,而不会产生死锁的风险。这两个函数都定义在std命名空间中,并且都是变参函数模板,如第26章所讨论。

第一个函数lock()在未指定的顺序中锁定所有给定的互斥锁对象,而不会产生死锁。如果其中一个互斥锁调用抛出异常,则对所有已经获取的锁调用unlock()。其原型如下所示:

\begin{cpp}
template <class L1, class L2, class... Ln> void lock(L1&, L2&, Ln&...);
\end{cpp}

try\_lock()具有类似的原型,通过按顺序对每个给定的互斥锁调用try\_lock()来尝试获取所有给定互斥锁上的锁。如果所有对try\_lock()的调用都成功,返回-1。如果try\_lock()失败,则对所有已经获取的锁调用unlock(),返回值是对try\_lock()失败的互斥锁参数的零基索引。

以下示例演示了如何使用通用lock()函数。process()函数首先为每个互斥锁创建两个锁,并将std::defer\_lock\_t的实例作为第二个参数传递给unique\_lock,以指示在构造期间不要获取锁,再调用std::lock()无死锁风险地获取两个锁。

\begin{cpp}
mutex mut1;
mutex mut2;

void process()
{
    unique_lock lock1 { mut1, defer_lock };
    unique_lock lock2 { mut2, defer_lock };
    lock(lock1, lock2);
    // Locks acquired.
} // Locks automatically released.
\end{cpp}

\mySamllsection{scoped\_lock}

std::scoped\_lock,定义在<mutex>中,类似于lock\_guard,但接受可变数量的互斥锁,这简化了获取多个锁的过程。例如,前面章节中的process()函数示例可以使用scoped\_lock重写:

\begin{cpp}
mutex m1;
mutex m2;

void process()
{
    scoped_lock locks { m1, m2 }; // Uses class template argument deduction, CTAD.
    // Locks acquired.
} // Locks automatically released.
\end{cpp}

\begin{myNotic}{NOTE}
scoped\_lock简化了获取多个锁的过程,不需要担心以正确的顺序获取它们。
\end{myNotic}

scoped\_lock是一个变参类模板,能够锁定任意数量的互斥锁。假设有一个包含互斥锁的std::array,并且需要一次性获取所有这些互斥锁的锁。为了使这个过程变得简单,可以写一个辅助的变参函数模板,结合使用std::index\_sequence和make\_index\_sequence。这里是示例:

\begin{cpp}
// Helper function to create the actual scoped_lock instance.
template <size_t N, size_t... Is>
auto make_scoped_lock(array<mutex, N>& mutexes, index_sequence<Is...>)
{
    return scoped_lock { mutexes[Is]... };
}

// Helper function to make it easy to use.
template <size_t N>
auto make_scoped_lock(array<mutex, N>& mutexes)
{
    return make_scoped_lock(mutexes, make_index_sequence<N>{});
}

int main()
{
    array<std::mutex, 4> mutexes;
    auto lockAll { make_scoped_lock(mutexes) };
}
\end{cpp}

\mySubsubsection{27.4.3.}{std::call\_once}

可以使用std::call\_once()与std::once\_flag结合,确保某个函数或成员函数仅调用一次,无论多少线程尝试使用相同的once\_flag调用call\_once()。实际上只有一次call\_once()会调用给定的函数。如果给定的函数没有抛出异常,这次调用就是有效的call\_once()调用。如果给定的函数抛出异常,异常会传播回调用者,然后选择另一个调用者来执行函数。特定once\_flag实例的有效调用,所有其他同一once\_flag上的call\_once()调用完成之前完成。其他尝试对同一once\_flag调用call\_once()的线程会阻塞,直到有效调用完成。图27.4说明了这一点,涉及三个线程。线程1执行有效的call\_once()调用,因为线程1的已有已经完成调用,线程2阻塞直到调用完成,而线程3不会阻塞。

\myGraphic{0.9}{content/part4/chapter27/images/4.png}{图 27.4}

以下示例演示了call\_once()的使用。示例启动三个线程,运行processingFunction(),并使用一些共享资源。这些共享资源通过调用initializeSharedResources()一次进行初始化。为了实现,每个线程都使用一个全局once\_flag调用call\_once()。结果是只有一个线程实际执行initializeSharedResources(),且只执行一次。当这个call\_once()调用正在进行时,其他线程会阻塞,直到initializeSharedResources()完成。

\begin{cpp}
once_flag g_onceFlag;
void initializeSharedResources()
{
    // ... Initialize shared resources to be used by multiple threads.
    println("Shared resources initialized.");
}

void processingFunction()
{
    // Make sure the shared resources are initialized.
    call_once(g_onceFlag, initializeSharedResources);
    // ... Do some work, including using the shared resources
    println("Processing");
}

int main()
{
    // Launch 3 threads.
    vector<jthread> threads { 3 };
    for (auto& t : threads) {
        t = jthread { processingFunction };
    }
    // No need to manually call join(), as we are using jthread.
}
\end{cpp}

此代码的输出为:

\begin{shell}
Shared resources initialized.
Processing
Processing
Processing
\end{shell}

可以在启动线程之前在main()函数的开始处调用一次initializeSharedResources();然而,这样就不会对call\_once()进行使用了。

\mySubsubsection{27.4.4.}{使用互斥锁的示例}

接下来的几节将给出几个示例,说明如何使用互斥锁来同步多个线程。

\mySamllsection{线程安全的写入流}

本章“线程”部分,有一个名为Counter的类示例。该示例提到C++流,如cout,默认情况下线程安全,但来自多个线程的输出可能会交错。这里有两种解决方案来解决这个问题:

\begin{itemize}
\item
使用同步流。

\item
使用互斥锁确保一次只有一个线程读写流对象。
\end{itemize}

\mySamllsection{同步流}

<syncstream>定义了std::basic\_osyncstream,它为char和wchar\_t流提供了预定义的类型别名osyncstream和wosyncstream,这些类名中的o代表输出。这些类保证通过它们进行的所有输出在同步流销毁的那一刻,出现在最终输出流中。保证输出不会因其他线程的输出交错,只要那些线程也使用自己的osyncstream对象。就线程安全而言,osyncstream与ostream之间的关系,与atomic\_ref<int>与int之间的关系完全相同。

先前Counter类的函数调用操作符可以按照以下方式实现,使用osyncstream来防止输出交错:

\begin{cpp}
class Counter
{
    public:
        explicit Counter(int id, int numIterations)
            : m_id { id }, m_numIterations { numIterations } { }

        void operator()() const
        {
            for (int i { 0 }; i < m_numIterations; ++i) {
                osyncstream syncedCout { cout };
                syncedCout << format("Counter {} has value {}", m_id, i);
                syncedCout << endl;
                // Upon destruction, syncedCout atomically flushes
                // its contents into cout.
            }
        }
    private:
        int m_id { 0 };
        int m_numIterations { 0 };
};
\end{cpp}

\mySamllsection{使用互斥锁}

如果不能使用同步流,可以按照以下代码使用互斥锁来同步Counter类中对cout的所有访问,需要添加了一个静态的互斥锁数据成员。它应该是静态的,类的所有实例都应该使用相同的互斥锁实例。向cout写入之前,使用lock\_guard获取互斥锁的锁。

\begin{cpp}
class Counter
{
    public:
        explicit Counter(int id, int numIterations)
            : m_id { id }, m_numIterations { numIterations } { }

        void operator()() const
        {
            for (int i { 0 }; i < m_numIterations; ++i) {
                lock_guard lock { ms_mutex };
                cout << format("Counter {} has value {}", m_id, i) << endl;
            }
        }
    private:
        int m_id { 0 };
        int m_numIterations { 0 };
        inline static mutex ms_mutex;
};
\end{cpp}

代码在for循环的每次迭代中创建一个lock\_guard实例。建议尽可能限制持有锁的时间;否则,会阻塞其他线程太长时间。

如果lock\_guard实例在for循环开始前一次性创建,基本上会失去这个代码的多线程效果,因为一个线程会持有锁直到for循环结束,而所有其他线程都会等待这个锁释放。

\begin{myNotic}{NOTE}
尽量减少持有锁的时间,避免在持有锁时执行缓慢的操作,例如打印消息到控制台、从文件中读取数据、访问数据库、进行长时间计算、执行显式休眠等。
\end{myNotic}

\mySamllsection{使用定时互斥锁}

以下示例演示了如何使用定时互斥锁。这是之前相同的Counter类,但这次它使用定时互斥锁与unique\_lock的组合。unique\_lock构造函数给出了200毫秒的相对时间,在200毫秒内尝试获取锁。如果在超时区间内无法获取锁,构造函数将返回,可以检查锁是否已获取。可以使用if语句在锁变量上进行此操作,因为unique\_lock定义了一个bool转换操作符。

\begin{cpp}
class Counter
{
    public:
        explicit Counter(int id, int numIterations)
            : m_id { id }, m_numIterations { numIterations } { }

        void operator()() const
        {
            for (int i { 0 }; i < m_numIterations; ++i) {
                unique_lock lock { ms_timedMutex, 200ms };
                if (lock) {
                    cout << format("Counter {} has value {}", m_id, i) << endl;
                } else {
                    // Lock not acquired in 200ms, skip output.
                }
            }
        }
    private:
        int m_id { 0 };
        int m_numIterations { 0 };
        inline static timed_mutex ms_timedMutex;
};
\end{cpp}


\mySamllsection{双重检查锁定}

双重检查锁定模式实际上是一个反模式,一定要避免使用!可能会在现有的代码库中遇到它,所以这里进行了展示。双重检查锁定模式是尝试避免使用互斥锁。这是一种夹生的尝试,试图编写比使用互斥锁更高效的代码。当试图使它比即将展示的示例更快时(例如,使用自由松散的原子类型(不进一步讨论)、使用普通的布尔类型而不是atomic<bool>类型等),可能会出错。这种模式容易受到数据竞争的影响,而且很难正确实现。讽刺的是,通常使用call\_once()会更快,使用魔法静态变量(如果适用)会更快。

\begin{myNotic}{NOTE}
函数局部静态变量称为魔法静态变量或线程安全的静态变量。C++保证这些局部静态变量以线程安全的方式初始化,因此不需要任何手动线程同步。有关使用魔法静态变量的示例,请参见第33章。
\end{myNotic}

\begin{myWarning}{WARNING}
避免使用双重检查锁定模式!使用其他机制,如简单锁、原子变量、call\_once()、魔法静态变量等。
\end{myWarning}

双重检查锁定可以用于确保资源只初始化一次。以下示例展示了如何实现,称为双重检查锁定模式,因为会两次检查g\_initialized变量的值,一次是在获取锁之前,一次是在获取锁之后。第一次g\_initialized检查用于防止在不需要时获取锁,第二次检查是为了确保在第一次g\_initialized检查和获取锁之间没有其他线程执行初始化。

\begin{cpp}
void initializeSharedResources()
{
    // ... Initialize shared resources to be used by multiple threads.
    println("Shared resources initialized.");
}

atomic<bool> g_initialized { false };
mutex g_mutex;

void processingFunction()
{
    if (!g_initialized) {
        unique_lock lock { g_mutex };
        if (!g_initialized) {
            initializeSharedResources();
            g_initialized = true;
        }
    }
    print("1");
}

int main()
{
    vector<jthread> threads;
    for (int i { 0 }; i < 5; ++i) {
        threads.emplace_back(processingFunction);
    }
}
\end{cpp}

输出清楚地显示只有其中一个线程初始化了共享资源:

\begin{shell}
Shared resources initialized.
11111
\end{shell}


\begin{myNotic}{NOTE}
对于这个示例,建议使用call\_once(),而非双重检查锁定!
\end{myNotic}





