
条件变量允许线程在满足某个条件、系统时间达到指定时间或由另一个线程设置时阻塞,直到满足该条件,这些变量允许进行显式的线程间通信。如果熟悉使用Win32 API进行的多线程编程,可以将条件变量与Windows中的事件对象进行比较。

<condition\_variable>提供了两个条件变量:

\begin{itemize}
\item
std::condition\_variable: 只能等待unique\_lock<mutex>的条件变量,根据C++标准,这在某些平台上可以达到最高效率。

\item
std::condition\_variable\_any: 可以等待任何类型的对象的条件下变量,包括自定义锁类型。
\end{itemize}

condition\_variable支持以下成员函数:

\begin{cpp}
notify_one();
\end{cpp}

唤醒在条件变量上等待的一个线程,这与Windows中的自动重置事件类似。

\begin{cpp}
notify_all();
\end{cpp}

唤醒在条件变量上等待的所有线程。

\begin{cpp}
wait(unique_lock<mutex>& lk);
\end{cpp}

调用wait()的线程应该已经获取了lk上的锁。调用wait()的效果是原子地调用lk.unlock()并阻塞线程,等待通知。当另一个线程通过调用notify\_one()或notify\_all()唤醒该线程时,函数会再次调用lk.lock(),需要阻塞直到获取锁,然后返回。

\begin{cpp}
wait_for(unique_lock<mutex>& lk, const chrono::duration<Rep, Period>& rel_time);
\end{cpp}

类似于wait(),只是线程会在调用notify\_one()、notify\_all()或指定的超时时间到期时唤醒。

\begin{cpp}
wait_until(unique_lock<mutex>& lk, const chrono::time_point<Clock, Duration>& abs_time);
\end{cpp}

类似于wait(),只是线程会在调用notify\_one()、notify\_all()或系统时间超过给定的绝对时间时被唤醒。

wait()、wait\_for()和wait\_until()也有接受谓词参数的重载。例如,接受谓词的wait()重载等效于:

\begin{cpp}
while (!predicate())
    wait(lk);
\end{cpp}

condition\_variable\_any类支持与condition\_variable相同的成员函数,只是接受的锁不是仅unique\_lock<mutex>,使用的锁类必须具有lock()和unlock()成员函数。

\mySubsubsection{27.5.1.}{伪唤醒}

条件变量上等待的线程可以在另一个线程调用notify\_one()或notify\_all()时唤醒,或在达到相对超时时,或在系统时间达到某个时间时。也可能伪唤醒,线程可能会在没有其他线程调用notify成员函数,并且没有达到超时的情况下唤醒。因此,当线程在条件变量上等待并唤醒时,需要检查唤醒的原因。检查方法之一是使用接受谓词的wait()版本,如下所示。

\mySubsubsection{27.5.2.}{使用条件变量}

例如,条件变量可以用于后台线程处理队列中的项目。可以定义一个队列,在其中插入要处理的项目。后台线程等待队列中是否有项目。当项目插入队列时,会唤醒线程,处理项目,然后回到睡眠状态,等待下一个项目。假设有以下队列:

\begin{cpp}
queue<string> m_queue;
\end{cpp}

为了确保在给定时间只修改这个队列的线程,添加了一个互斥锁:

\begin{cpp}
mutex m_mutex;
\end{cpp}

为了能够在添加项目时通知后台线程,还添加了一个条件变量:

\begin{cpp}
condition_variable m_condVar;
\end{cpp}

想要将项目添加到队列的线程首先获取互斥锁上的锁,然后将项目添加到队列,并通知后台线程。调用notify\_one()或notify\_all()可以在当前是否有锁的情况下进行;两者都有效。

\begin{cpp}
// Lock mutex and add entry to the queue.
lock_guard lock { m_mutex };
m_queue.push(entry);
// Notify condition variable to wake up thread.
m_condVar.notify_all();
\end{cpp}

后台线程在一个无限循环中等待通知,如下所示。注意使用wait()接受一个谓词来正确处理伪唤醒,谓词检查队列中是否真的有东西。当调用wait()返回时,可以确定队列中确实有东西。

\begin{cpp}
unique_lock lock { m_mutex };
while (true) {
    // Wait for a notification.
    m_condVar.wait(lock, [this]{ return !m_queue.empty(); });
    // Whenever we reach this line, the mutex is locked and the queue is non-empty.
    // Process queue item...
    m_queue.pop();
}
\end{cpp}

本章末尾的“示例:多线程日志类”部分提供了一个完整的示例,说明如何使用条件变量向其他线程发送通知。

C++标准还定义了一个名为std::notify\_all\_at\_thread\_exit(cond, lk)的辅助函数,其中cond是一个条件变量,lk是一个unique\_lock<mutex>实例。调用此函数的线程应该已经获得了锁lk。当线程退出时,会自动执行以下操作:

\begin{cpp}
lk.unlock();
cond.notify_all();
\end{cpp}

\begin{myNotic}{NOTE}
锁lk保持锁定状态直到线程退出,所以需要确保这不会导致死锁。例如,错误的锁定顺序。
\end{myNotic}




