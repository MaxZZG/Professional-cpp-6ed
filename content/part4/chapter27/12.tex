协程是一种可以在执行过程中暂停,并在稍后时间点恢复执行的函数。函数体内包含以下关键字的函数都是协程:

\begin{itemize}
\item
co\_await: 等待计算完成时暂停协程的执行。当计算完成时,执行恢复。

\item
co\_return: 从协程返回(协程中不允许使用普通的return)。执行co\_return后的协程不能再次恢复执行。

\item
co\_yield: 从协程返回一个值给调用者,并暂停协程。随后再次调用协程,将会在暂停的地方继续执行。
\end{itemize}

协程有两种类型:有栈和无栈。有栈协程可以从嵌套调用的任意深处暂停。另一方面,无栈协程只能从函数体的最顶层栈帧暂停。当无栈协程暂停时,只有函数体内的自动存储持续期的变量和临时变量会保存;调用栈不会保存。无栈协程的内存使用非常小,可以同时运行数百万甚至数十亿个协程,C++只支持无栈协程。

公平地说,协程并不一定与多线程有直接关系,只是提供了一种函数可以在稍后时间点暂停和恢复执行的方式。如果需要,协程也可以在多线程环境中使用。

协程可以用同步编程风格实现异步操作。使用场景包括:

\begin{itemize}
\item
生成器

\item
异步I/O

\item
延迟计算

\item
事件驱动应用程序

\item
状态机
\end{itemize}

\CXXTwentythreeLogo{-40}{-50}

不幸的是,尽管所有的底层语言构建块都可用以编写自己的协程,但在高级协程工具方面却没有什么内容。C++23标准库引入了一个标准的高级协程设施,即生成器std::generator。生成器提供了一种机制,使得一个线程可以在生成结果和处理这些结果之间切换,而不涉及多个线程。

以下代码演示了<generator>中定义的std::generator类模板的使用:

\begin{cpp}
generator<int> getSequenceGenerator(int startValue, int numberOfValues)
{
    for (int i { startValue }; i < startValue + numberOfValues; ++i) {
        // Print the local time to standard out, see Chapter 22.
        auto currentTime { system_clock::now() };
        auto localTime { current_zone()->to_local(currentTime) };
        print("{:%H:%M:%OS}: ", localTime);
        // Yield a value to the caller, and suspend the coroutine.
        co_yield i;
    }
}

int main()
{
    auto gen { getSequenceGenerator(10, 5) };
    for (const auto& value : gen) {
        print("{} (Press enter for next value)", value);
        cin.ignore();
    }
}
\end{cpp}

当运行应用程序时,会得到以下输出:

\begin{shell}
16:35:42: 10 (Press enter for next value)
\end{shell}

按下回车键添加另一行:

\begin{shell}
16:35:42: 10 (Press enter for next value)
16:36:03: 11 (Press enter for next value)
\end{shell}

再次按下回车键添加另一行:

\begin{shell}
16:35:42: 10 (Press enter for next value)
16:36:03: 11 (Press enter for next value)
16:36:21: 12 (Press enter for next value)
\end{shell}

每次按下回车键后,都会从生成器请求一个新值。这导致协程恢复执行,执行getSequenceGenerator()中的for循环的下一轮,打印本地时间,并返回下一个值。使用co\_yield返回值,这会返回值并随后暂停协程。值本身在main()函数中打印,后面跟着按回车键以获取下一个值的提示,输出清楚地显示协程的多次暂停和恢复。

不幸的是,这本书的范围内关于协程的介绍就到此为止了。编写自己的协程,如std::generator非常复杂,超出了本书的讨论范围。我建议使用专家编写的现有协程。如果需要标准库提供的生成器之外的其他协程设施,有许多第三方库可用,如\href{https://github.com/lewissbaker/cppcoro}{cppcoro1}和\href{https://github.com/David-Haim/concurrencpp}{concurrencpp},它们提供了一系列高级协程。本节的目标是介绍这个概念,让你知道它存在,也许未来的C++标准会引入更多高级标准化的协程工具。




