通过解决下面的练习,可以练习本章讨论的内容。所有练习的解决方案都可以在本书的网站\url{www.wiley.com/go/proc++6e}下载到源码。若在练习中卡住了,可以考虑先重读本章的部分内容,试着自己找到答案,再在从网站上寻找解决方案。

\begin{itemize}
\item
\textbf{练习 27-1}: 编写一个应用程序,该程序无限期地每隔三秒钟哔哔声。三秒的延迟必须作为参数传递给你的线程函数。提示:可以通过向标准输出打印\verb|\|a来使计算机发出哔哔声。

\item
\textbf{练习 27-2}: 修改解决方案以使应用程序在用户按下回车键时停止。

\item
\textbf{练习 27-3}: 修改解决方案以使哔哔声在用户按下回车键后继续,直到用户再次按下回车键。当用户按下回车键,哔哔声应该暂停,直到用户再次按下回车键。用户可以暂停和恢复哔哔声,次数不限。

\item
\textbf{练习 27-4}: 编写一个应用程序,它可以并发计算多个斐波那契数。例如,代码应该能够并行计算斐波那契数列中的第4、9、14和17个数字。斐波那契数列从0和1开始,后续值都是前两个值的和,所以:0, 1, 1, 2, 3, 5, 8, 13, 21, 34, 55, 89等。当所有结果都可用,将它们输出到标准输出。最后,使用受限的标准库算法计算它们的和。

\item
\textbf{练习 27-5}: 改进本章前面部分的机器人示例。可以在源代码库中找到该代码,文件名为Ch27\verb|\|05\_barrier\verb|\|barrier.cpp。改进它,使主线程启动所有机器人线程,等待所有机器人启动,准备第一次迭代,然后指示所有等待的机器人开始工作。

\item
\textbf{练习 27-6}: 使用compare\_exchange\_strong()实现一个函数atomicMin(a, b),以原子方式将a设置为min(a, b),其中a是一个原子<int>,b是一个int。
\end{itemize}







