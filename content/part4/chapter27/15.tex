通过解决下面的练习,可以练习本章讨论的内容。所有练习的解决方案都可以在本书的网站\url{www.wiley.com/go/proc++6e}下载到源码。然而,若在练习中卡住了,在从网站上寻找解决方案之前,可以考虑先重读本章的部分内容,试着自己找到答案。

\begin{itemize}
\item
Exercise 27-1: Write an application that beeps every three seconds indefinitely. The threesecond delay must be passed as an argument to your thread function. Tip: You can make your computer beep by printing \a to the standard output.

\item
Exercise 27-2: Modify your solution to Exercise 27-1 so that the application can be stopped when the user presses the enter key.

\item
Exercise 27-3: Modify your solution to Exercise 27-1 so that the beeping continues until the user presses the Enter key. Once the Enter key is pressed, beeping should be paused, until the user presses the Enter key again. The user can pause and resume the beeping as many times as she wants.

\item
Exercise 27-4: Write an application that can calculate a number of Fibonacci numbers concurrently. For example, your code should be able to calculate the $4^{th}$, $9^{th}$, $14^{th}$, and $17^{th}$ number in the Fibonacci series in parallel. The Fibonacci series starts with 0 and 1, and any subsequent value is the sum of the two previous values, so: 0, 1, 1, 2, 3, 5, 8, 13, 21, 34, 55, 89, and so on. Once all results are available, output them to the standard output. Finally, calculate their sum using a constrained Standard Library algorithm.

\item
Exercise 27-5: Improve the robot example from earlier in this chapter. You can find that code in the downloadable source code archive in Ch27\verb|\|05\_barrier\verb|\|barrier.cpp. Improve it so that the main thread starts all the robot threads, waits for all the robots to have started, prepares the first iteration, and then instructs all waiting robots to start working.

\item
Exercise 27-6: Use compare\_exchange\_strong() to implement a function atomicMin(a, b) that sets a to min(a, b) atomically, and where a is an atomic<int> and b an int.
\end{itemize}







