这个部分展示了如何使用线程、互斥锁、锁和条件变量来编写一个多线程的Logger类,这个类允许从不同的线程向队列中添加日志消息。Logger类本身在一个后台线程中处理这个队列,并依次将日志消息写入文件。这个类在两个迭代中设计,以展示编写多线程代码时会遇到的一些问题。

C++标准没有线程安全的队列,所以必须使用某种同步机制来保护对队列的访问,以避免多个线程同时读写队列。这个示例使用互斥锁和条件变量来提供同步,Logger类的定义如下:

\begin{myWarning}{WARNING}
这个Logger类使用std::thread,而非jthread来演示,1)不当使用线程时的灾难性结果,2)使用线程很容易出错。
\end{myWarning}

\begin{cpp}
export class Logger
{
    public:
        // Starts a background thread writing log entries to a file.
        Logger();
        // Prevent copy construction and assignment.
        Logger(const Logger&) = delete;
        Logger& operator=(const Logger&) = delete;
        // Add log entry to the queue.
        void log(std::string entry);
    private:
        // The function running in the background thread.
        void processEntries();
        // Helper member function to process a queue of entries.
        void processEntriesHelper(std::queue<std::string>& queue,
            std::ofstream& ofs) const;
        // Mutex and condition variable to protect access to the queue.
        std::mutex m_mutex;
        std::condition_variable m_condVar;
        std::queue<std::string> m_queue;
        // The background thread.
        std::thread m_thread;
};
\end{cpp}

实现如下。这个初始设计存在几个问题。当运行它时,可能行为不正确并崩溃。这将在Logger类的下一个设计迭代中讨论和解决。processEntries()成员函数中的while循环值得注意,它处理队列中当前的所有消息。在持有锁的情况下,将当前条目队列的内容与栈上的空本地队列进行交换。之后,释放锁,以便其他线程不再阻塞,以添加新的条目到现在的空队列中。释放锁后,将处理本地队列中的所有条目。这时不再需要锁,因为其他线程不会触及这个本地队列。

\begin{cpp}
Logger::Logger()
{
    // Start background thread.
    m_thread = thread { &Logger::processEntries, this };
}

void Logger::log(string entry)
{
    // Lock mutex and add entry to the queue.
    lock_guard lock { m_mutex };
    m_queue.push(move(entry));
    // Notify condition variable to wake up thread.
    m_condVar.notify_all();
}

void Logger::processEntries()
{
    // Open log file.
    ofstream logFile { "log.txt" };
    if (logFile.fail()) {
        println(cerr, "Failed to open logfile.");
        return;
    }

    unique_lock lock { m_mutex }; // Acquire a lock on m_mutex.
    while (true) { // Start processing loop.
        // Wait for a notification.
        m_condVar.wait(lock);

        // Condition variable is notified, so something might be in the queue.

        // While we still have the lock, swap the contents of the current queue
        // with an empty local queue on the stack.
        queue<string> localQueue;
        localQueue.swap(m_queue);

        // Now that all entries have been moved from the current queue to the
        // local queue, we can release the lock so other threads are not blocked
        // while we process the entries.
        lock.unlock();

        // Process the entries in the local queue on the stack. This happens after
        // having released the lock, so other threads are not blocked anymore.
        processEntriesHelper(localQueue, logFile);
        lock.lock();
    }
}

void Logger::processEntriesHelper(queue<string>& queue, ofstream& ofs) const
{
    while (!queue.empty()) {
        ofs << queue.front() << endl;
        queue.pop();
    }
}
\end{cpp}

\begin{myWarning}{WARNING}
从这个相当简单的任务中看到,编写正确的多线程代码是困难的!不幸的是,C++标准库还没有提供并发数据结构,至少现在还没有。

Logger类只是一个示例,展示了这些基本构建块。对于生产质量的代码,我建议使用合适的第三方并发数据结构,而不是自己编写。例如,开源的boost C++库(boost.org)有一个无锁队列的实现,允许并发使用而不需要显式同步。
\end{myWarning}

Logger类可以通过以下测试代码进行测试。它启动了一定数量的线程,所有线程都向同一个Logger实例添加了一些日志消息。

\begin{cpp}
void logSomeMessages(int id, Logger& logger)
{
    for (int i { 0 }; i < 10; ++i) {
        logger.log(format("Log entry {} from thread {}", i, id));
        this_thread::sleep_for(50ms);
    }
}

int main()
{
    Logger logger;
    vector<jthread> threads;
    // Create a few threads all working with the same Logger instance.
    for (int i { 0 }; i < 10; ++i) {
        threads.emplace_back(logSomeMessages, i, ref(logger));
    }
}
\end{cpp}

如果构建并运行这个简单的初始版本,会注意到应用程序突然终止。这是因为应用程序从未调用join()或detach()在Logger后台线程上。一个可汇入的线程对象的析构函数,即尚未调用join()或detach(),会调用std::terminate()来终止所有运行的线程和应用程序本身。所以队列中仍然存在的消息没有写入磁盘上的文件,一些运行时库甚至在这种应用程序终止时,会发出错误或生成崩溃转储文件。需要添加一种机制,以优雅地关闭后台线程,并在后台线程完全关闭后才终止应用程序。可以通过向类中添加析构函数和布尔数据成员来实现,新定义如下所示:

\begin{cpp}
export class Logger
{
    public:
        // Gracefully shut down background thread.
        virtual ~Logger();
        // Other public members omitted for brevity.
    private:
        // Boolean telling the background thread to terminate.
        bool m_exit { false };
        // Other members omitted for brevity.
};
\end{cpp}

析构函数将m\_exit设置为true,唤醒后台线程,然后等待线程关闭。析构函数在设置m\_exit为true之前获取m\_mutex上的锁,这是为了避免与processEntries()的竞争条件和死锁,processEntries()可能在其while循环的起点检查m\_exit,然后等待通知。如果主线程在那个时调用Logger的析构函数(假设析构函数尚未编写以获取m\_mutex上的锁),那么析构函数将m\_exit设置为true,并在processEntries()检查m\_exit之后、在processEntries()等待条件变量之前调用notify\_all()。因此,processEntries()将不会看到m\_exit的新值,并且会错过通知。这时,应用程序死锁,因为析构函数正在等待join()调用,而后台线程正在等待条件变量。析构函数可以在持有锁的情况下调用notify\_all(),或者在释放锁后调用,但是锁必须在调用join()之前释放,这就解释了为什么使用花括号。

\begin{cpp}
Logger::~Logger()
{
    {
        lock_guard lock { m_mutex };
        // Gracefully shut down the thread by setting m_exit to true.
        m_exit = true;
    }
    // Notify condition variable to wake up thread.
    m_condVar.notify_all();
    // Wait until thread is shut down. This should be outside the above code
    // block because the lock must be released before calling join()!
    m_thread.join();
}
\end{cpp}

processEntries()成员函数需要检查这个布尔变量,并在其为true时终止处理循环。还应该只在队列为空时调用wait()。

\begin{cpp}
void Logger::processEntries()
{
    // Omitted for brevity.

    unique_lock lock { m_mutex }; // Acquire a lock on m_mutex.
    while (true) { // Start processing loop.
        if (!m_exit) { // Only wait for notifications if we don't have to exit.
            if (m_queue.empty()) { // Only wait if the queue is empty.
                m_condVar.wait(lock);
            }
        } else {
            // We have to exit, process the remaining entries in the queue.
            processEntriesHelper(m_queue, logFile);
            break;
        }

        // Condition variable is notified, so something might be in the queue
        // and/or we need to shut down this thread.

        queue<string> localQueue;
        localQueue.swap(m_queue);
        lock.unlock();
        processEntriesHelper(localQueue, logFile);

        lock.lock();
    }
}
\end{cpp}

不能仅在while循环的外部条件中检查m\_exit,因为即使m\_exit为true,队列中可能仍然有需要写入日志文件的日志条目。

可以在多线程代码的特定位置添加人工延迟,以触发某些行为。这样的延迟只应该添加到测试中,必须从最终代码中删除!例如,为了测试与析构函数的条件竞争需要解决,可以从主程序中删除对log()的调用,使其可调用Logger类的析构函数,并添加以下延迟:

\begin{cpp}
void Logger::processEntries()
{
    // Omitted for brevity.
    while (true) {
        if (!m_exit) { // Only wait for notifications if we don't have to exit.
            this_thread::sleep_for(1000ms);
            if (m_queue.empty()) { // Only wait if the queue is empty.
                m_condVar.wait(lock);
            }
    // Remaining code omitted, same as before.
}
\end{cpp}

\begin{myNotic}{NOTE}
我建议使用jthread,而不使用thread,因为它在其析构函数中会自动加入。使用jthread,没必要在Logger析构函数中的显式调用join()。
\end{myNotic}



