本章简要概述了使用标准C++线程支持库进行多线程编程。了解了如何使用std::thread启动线程,以及如何使用jthread使取消线程更加安全、更少出错、更容易书写。还了解了如何使用原子类型和原子操作来操作共享数据,而无需使用显式同步机制。如果无法使用这些原子类型和操作,可以使用互斥锁和条件变量来确保不同线程之间,对共享数据的读/写访问的适当同步。还有同步原语:信号量、锁存器和栅栏。还有,promise和future代表了一个简单的线程间通信通道;可以使用future更容易地从线程中获取结果。本章最后,对协程进行了简要的介绍,并对多线程应用程序设计的提供了一些最佳实践指南。

如引言中所述,本章试图涵盖标准库提供的所有基本多线程构建块,但由于篇幅限制,无法深入讨论多线程编程的所有细节。有关多线程的书籍有很多,请参阅附录B以获取一些参考资料。
