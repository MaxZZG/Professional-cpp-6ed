信号量是轻量级的同步原语,可以作为其他同步机制(如互斥锁、门闩和栅栏)的构建块,信号量由一个表示槽位的计数器组成。计数器在构造函数中初始化。如果获取一个槽位,计数器将递减,而释放一个槽位则递增计数器。<semaphore>中定义了两个信号量类:std::counting\_semaphore和binary\_semaphore。

前者模拟一个非负的资源计数,后者只有一个槽位;因此,槽位要么是空闲的,要么不是,非常适合作为互斥锁的构建块。两者都提供以下成员函数:

% Please add the following required packages to your document preamble:
% \usepackage{longtable}
% Note: It may be necessary to compile the document several times to get a multi-page table to line up properly
\begin{longtable}{|l|l|}
\hline
\textbf{成员函数} &
\textbf{描述} \\ \hline
\endfirsthead
%
\endhead
%
acquire() &
\begin{tabular}[c]{@{}l@{}}递减计数器。计数器为零时阻塞,直到能够递减计数器,\\然后执行递减操作。\end{tabular} \\ \hline
try\_acquire() &
\begin{tabular}[c]{@{}l@{}}尝试递减计数器,但当计数器已经是零时,不会阻塞。\\如果计数器可以递减,则返回true;否则返回false。\end{tabular} \\ \hline
try\_acquire\_for() &
与try\_acquire()相同,但尝试会间隔一定的时间。\\ \hline
try\_acquire\_until() &
\begin{tabular}[c]{@{}l@{}}与try\_acquire()相同,但尝试直到系统时间达到\\给定的时间。\end{tabular} \\ \hline
release() &
\begin{tabular}[c]{@{}l@{}}通过给定的数量递增计数器,并释放在acquire()\\调用中阻塞的线程。\end{tabular} \\ \hline
\end{longtable}

计数信号量允许精确控制同时运行的线程数,以下代码允许最多四个线程以并行方式运行。从输出中,可以清楚地看到,只有四个线程能够同时获取信号量。

\begin{cpp}
counting_semaphore semaphore { 4 };
vector<jthread> threads;
for (int i { 0 }; i < 10; ++i) {
    threads.emplace_back([&semaphore] {
        semaphore.acquire();
        // ... Slot acquired ... (at most 4 threads concurrently)
        print("{}", i);
        this_thread::sleep_for(5s);
        semaphore.release();
    });
}
\end{cpp}

信号量的另一个用途是实现线程的通知机制,而不是条件变量。可以在信号量的构造函数中将计数器初始化为0。调用acquire()的线程都将阻塞,直到其他线程在信号量上调用release()。
