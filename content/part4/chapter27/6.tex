锁存器是一种单次使用的线程协调点,多个线程在锁存器点处阻塞。达到预定的线程数到达锁存器点,所有线程都会释放并允许继续执行。基本上,它是一个计数器,随着每个线程到达锁存器点而递减。当计数器达到零,锁存器将保持“开放”,所有阻塞的线程将释放,并且随后到达锁存器点的线程都可以立即继续。

锁存器由std::latch实现,定义在<latch>中。构造函数接受需要到达锁存器点的线程数。到达锁存器点的线程可以调用arrive\_and\_wait(),这将递减锁存器计数器并阻塞,直到锁存器获得释放信号。线程也可以通过调用wait()在锁存器点处阻塞,而不递减计数器。try\_wait()成员函数可以用来检查计数器是否达到零。如果需要,可以通过调用count\_down()在不阻塞的情况下递减计数器。

第一个示例演示了锁存器点的一个使用场景,其中数据以并行方式处理。以下代码片段启动了一定数量的工人线程,每个线程都执行一些工作。当工人线程完成其工作,会调用count\_down()在锁存器上,以信号化其工作已完成。主线程调用wait()在锁存器上等待,直到锁存器计数器达到零,信号表示所有工人线程都已完成自己的任务。

\begin{cpp}
// Launch a number of threads to do some work.
constexpr unsigned numberOfWorkerThreads { 10 };
latch latch { numberOfWorkerThreads };
vector<jthread> threads;
for (unsigned i { 0 }; i < numberOfWorkerThreads; ++i) {
    threads.emplace_back([&latch, i] {
        // Do some work...
        print("{} ", i);
        this_thread::sleep_for(1s);
        print("{} ", i);
        // When work is done, decrease the latch counter.
        latch.count_down();
    });
}
// Wait for all worker threads to finish.
latch.wait();
println("\nAll worker threads are finished.");
\end{cpp}

第二个示例演示了另一个锁存器点的使用场景,其中一些数据需要加载到内存中(I/O绑定),随后在多个线程中以并行方式处理。进一步,线程在启动时需要执行一些CPU绑定的初始化,并且在开始处理数据之前才能开始。通过首先启动线程并执行CPU绑定的初始化,并行加载数据(I/O绑定),从而提高性能。代码初始化了一个计数为1的锁存器对象,并启动了10个线程,它们都执行一些初始化,然后阻塞在锁存器上,直到锁存器计数器达到零。启动10个线程后,代码加载了一些数据(从磁盘),这是一个I/O绑定的步骤。当所有数据都已加载,锁存器计数器回递减到0,就回释放所有(10个)等待的线程。

\begin{cpp}
latch startLatch { 1 };
vector<jthread> threads;
for (int i { 0 }; i < 10; ++i) {
  threads.emplace_back([&startLatch] {
     // Do some initialization... (CPU bound)
     // Wait until the latch counter reaches zero.
     startLatch.wait();
     // Process data...
  });
}
// Load data... (I/O bound)

// Once all data has been loaded, decrement the latch counter
// which then reaches zero and unblocks all waiting threads.
startLatch.count_down();
\end{cpp}























