通过解决下面的练习,可以练习本章讨论的内容。所有练习的解决方案都可以在本书的网站\url{www.wiley.com/go/proc++6e}下载到源码。然而,若在练习中卡住了,在从网站上寻找解决方案之前,可以考虑先重读本章的部分内容,试着自己找到答案。

\begin{itemize}
\item
\textbf{练习 25-1}: 编写一个名为transform\_if()的算法,类似于第20章中讨论的标准库中的transform()算法。不同之处在于,transform\_if()应该接受一个谓词,并且只对谓词返回true的元素进行转换,其他元素保持不变。为了测试算法实现,创建一个整数数组,然后使用transform\_if()将整数复制到vector中,同时将所有奇数值乘以2。

\item
\textbf{练习 25-2}: 编写一个名为generate\_fibonacci()的算法,填充给定范围为一个斐波那契数列的数字。[连续两个斐波那契数的比值收敛于黄金比例,1.618034 ... 斐波那契数和黄金比例经常出现在自然界中,例如:树木的分枝、朝鲜蓟的开花、花瓣、贝壳等。黄金比例对人类有艺术吸引力,因此经常被建筑师用于设计房间、花园中植物的布局等。]斐波那契数列以0和1开始,任何后续值都是前两个值的和,因此:0, 1, 1, 2, 3, 5, 8, 13, 21, 34, 55, 89等。实现不允许包含手动编写的循环或递归算法,并且应该使用标准库的generate()算法来完成大部分工作。

\item
\textbf{练习 25-3}: 为directed\_graph类模板实现一个find(const T\&)成员函数。

\item
\textbf{练习 25-4}: 所有关联容器都有一个contains()成员函数,如果给定的元素在容器中则返回true,否则返回false。这对有向图也很有用,可以在有向图上添加contains()的实现。
\end{itemize}











