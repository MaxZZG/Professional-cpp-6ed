标准库中的每个容器都接受一个Allocator类型作为模板类型参数,默认情况下通常就足够了。例如,vector模板定义如下:

\begin{cpp}
template <class T, class Allocator = allocator<T>> class vector;
\end{cpp}

容器构造函数允许传递一个Allocator对象的实例,可以自定义容器分配内存的方式。容器执行的每次内存分配都是通过Allocator对象的allocate()成员函数进行的,而每次释放内存都是通过调用Allocator对象的deallocate()成员函数进行。当标准库容器接受Allocator参数时,如果没有提供,该参数总是默认为std::allocator<T>。std::allocator<T>的allocate()和deallocate()成员函数是对new和delete的简单包装。

注意,allocate()只是分配一个足够大的未初始化内存块,并不调用对象的构造函数。同样,deallocate()只是释放内存块,并不调用析构函数。当分配了内存块,就可以使用new操作符(参见第15章)在构造对象。以下代码片段展示了一个构造的示例。第29章展示了如何在对象池中使用Allocator。

\begin{cpp}
class MyClass {};
int main()
{
    // Create an allocator to use.
    std::allocator<MyClass> alloc;
    // Allocate an uninitialized memory block for 1 instance of MyClass.
    auto* memory { alloc.allocate(1) };
    // Use placement new operator to construct a MyClass in place.
    ::new (memory) MyClass{};
    // Destroy MyClass instance.
    std::destroy_at(memory);
    // Deallocate memory block.
    alloc.deallocate(memory, 1);
    memory = nullptr;
}
\end{cpp}

如果希望程序中的容器使用自定义的内存分配和释放方案,可以编写自己的Allocator类。使用自定义分配器有多种原因,例如:底层分配器的性能不可接受,可以有替代方案。当必须分配特定于操作系统的功能,如共享内存段时,使用自定义分配器可以使标准库容器在这些共享内存段中使用。使用自定义分配器很复杂,可能会出现许多潜在的问题,需要谨慎。

提供allocate()、deallocate(),以及其他所需成员函数和类型别名的类,都可以替换默认的分配器。

此外,标准库还有一个多态内存分配器的概念。容器分配器指定为模板类型参数的问题在于,两个相似但具有不同分配器类型的容器,是完全不同的类型。例如,vector<int, A1>和vector<int, A2>是不同的类型,不能相互赋值。

多态内存分配器,定义在std::pmr命名空间中的<memory\_resource>,有助于解决这个问题。类std::pmr::polymorphic\_allocator是一个真正的Allocator类,满足所有分配器要求,如具有allocate()和deallocate()成员函数。polymorphic\_allocator的分配行为取决于其在构造时给出的memory\_resource,而不是模板类型参数。不同的polymorphic\_allocators在分配和释放内存时,尽管都有相同的类型,也可以有不同的方式行为,即polymorphic\_allocator。标准库提供了一些内置的memory\_resource,可以用来初始化polymorphic\_allocator:synchronized\_pool\_resource、unsynchronized\_pool\_resource和monotonic\_buffer\_resource。标准库还提供了模板类型别名,如std::pmr::vector<T>,用于std::vector<T, std::pmr::polymorphic\_allocator<T>{}>。一个std::pmr::vector<T>与一个memory\_resource关联仍然是不同的类型,并且不能从std::vector<T>赋值。但与不同memory\_resource关联的std::pmr::vector<T>是相同的类型,并且可以从另一个std::pmr::vector<T>对象赋值。

根据我的经验,自定义分配器和多态内存分配器在日常工作编码中是非常高级,且很少使用的特性。我自己从未使用过,所以详细的介绍会超出了本书的范围。更多信息,请参考附录B中列出有关C++标准库的相关书籍。



















