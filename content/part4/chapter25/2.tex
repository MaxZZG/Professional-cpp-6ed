
标准库包含许多有用的容器、算法和迭代器,可以在应用程序中使用。但任何库都不可能包含所有工具,满足所有客户的需求,所以最好的库是可扩展的:允许客户适应和添加基本功能,以获得所需的功能。由于标准库将数据与操作算法分离的基本结构,因此具有可扩展性。可以通过提供符合标准库指南的迭代器来编写自己的容器,使其与标准库算法一起工作。也可以编写自己的算法,与标准容器的迭代器一起工作。但请记住,不允许将自己的容器和算法放在std命名空间中。

\begin{myNotic}{NOTE}
本书通常使用这样的约定来命名函数和成员函数:不带下划线,名称中的每个单词都大写,除了第一个单词,例如:getIndex()。这一章讨论的是扩展标准库,因此使用了标准库的命名约定。函数和成员函数名称都是小写的,用下划线分隔单词,例如get\_index()。类名也使用标准库的命名约定。
\end{myNotic}

\mySubsubsection{25.2.1.}{为什么要扩展标准库?}

如果自定义C++中的算法或容器,可以选择让其遵循标准库约定,也可以不遵循。对于简单的容器和算法,遵循标准库的要求可能不值得付出努力。然而,对于计划重用的代码,这种努力是值得的。因为遵循了成熟的接口指南,其他C++开发者能更容易理解这段代码。其次,能够使用容器或算法与标准库的其他部分(算法或容器)一起使用,而无需提供特殊的中介或适配器。最后,这将使开发者采取严谨的态度,来开发可靠的代码。

\mySubsubsection{25.2.2.}{实现一个标准库算法}

第16章和第20章描述了标准库中一组有用的算法,但不可避免地会遇到需要新算法的情况。这种情况发生时,实现自己的算法并不困难,该算法与标准库迭代器一起工作,就像标准算法一样。

\mySamllsection{find\_all}

假设要找到给定范围内的所有元素,包括它们的位置,匹配一个谓词。find()和find\_if()算法是最有可能的候选者,每个都只返回一个元素的迭代器。可以使用copy\_if()找到所有匹配给定谓词的元素,但会用找到的元素的副本填充输出,丢失其位置。如果想避免复制,可以使用copy\_if()与back\_insert\_iterator(参见第17章)到一个vector<reference\_wrapper<T>{}>中,但这也不会找到的元素的位置。实际上,没有一个标准算法可以通过迭代器,获取匹配谓词的所有元素,所以可以自己实现的这个功能,暂时称为find\_all()。

第一部分中,我们将了解find\_all()的遵循传统无约束算法模型的实现。开始实现工作,会了解到如何扩展和调整,以遵循现代约束算法模型,并添加对投影和更多特性的支持。

第一项任务是定义函数原型。可以遵循copy\_if()的模型,即一个带有三个模板类型参数的函数模板:输入迭代器类型、输出迭代器类型和谓词类型。函数的参数是输入序列的开始和结束迭代器,输出序列的开始迭代器,以及一个谓词对象。与copy\_if()一样,算法返回一个迭代器,指向输出序列中最后一个元素之后的元素。现代C++代码中,建议为模板类型参数添加适当的约束,我们遵循这个建议:

\begin{cpp}
template <forward_iterator ForwardIterator,
          output_iterator<ForwardIterator> OutputIterator,
          indirect_unary_predicate<ForwardIterator> Predicate>
OutputIterator find_all(ForwardIterator first, ForwardIterator last,
                        OutputIterator dest, Predicate pred);
\end{cpp}

forward\_iterator概念指定一个迭代器必须可以解引用和递增等,output\_iterator<ForwardIterator>概念要求迭代器是一个接受ForwardIterator类型的值的输出迭代器。indirect\_unary\_predicate是一个预定义的需求集,算法可以用它来指定一元谓词参数的要求。这里需要间接使用,因为需求应用于其模板类型参数所引用的类型,这个案例中是ForwardIterator,不是ForwardIterator本身。

另一种设计选择是省略输出迭代器,并返回一个迭代器,该迭代器在输入序列中迭代所有匹配的元素。要求实现自己的迭代器类,这将在本章后面讨论。

下一项任务是编写实现。find\_all()算法遍历输入序列中的所有元素,使用invoke()在每个元素上调用谓词,并将匹配元素的迭代器存储在输出序列中。下面是实现:

\begin{cpp}
template <forward_iterator ForwardIterator,
          output_iterator<ForwardIterator> OutputIterator,
          indirect_unary_predicate<ForwardIterator> Predicate>
OutputIterator find_all(ForwardIterator first, ForwardIterator last,
                        OutputIterator dest, Predicate pred)
{
    while (first != last) {
        if (invoke(pred, *first)) {
            *dest = first;
            ++dest;
        }
        ++first;
    }
    return dest;
}
\end{cpp}

类似于copy\_if(),该算法只覆盖输出序列中已存在的元素,请确保输出序列足够大以容纳结果,或者使用迭代器适配器,如back\_insert\_iterator,如下面的代码所示。在找到所有匹配的元素后,代码计算找到的元素数量,即matches中的迭代器数量。遍历结果,打印每个元素。

\begin{cpp}
vector<int> vec { 5, 4, 5, 4, 10, 6, 5, 8, 10 };
vector<vector<int>::iterator> matches;

find_all(begin(vec), end(vec), back_inserter(matches),
    [](int i){ return i == 10; });

println("Found {} matching elements: ", matches.size());
for (const auto& it : matches) {
    println("{} at position {}", *it, distance(begin(vec), it));
}
\end{cpp}

输出为:

\begin{shell}
Found 2 matching elements:
10 at position 4
10 at position 8
\end{shell}

\mySamllsection{现代化的 find\_all}

如第17章所述,大多数约束算法接受一个投影参数。find\_all()算法可以现代化以支持这样的投影参数。此外,约束算法通常没有相同类型的开始和结束迭代器。相反,开始迭代器通常是正常的,但结束标记可以是不同的类型,称为“哨兵”。以下是新实现的算法:

\begin{cpp}
template <forward_iterator ForwardIterator,
    sentinel_for<ForwardIterator> Sentinel,
    output_iterator<ForwardIterator> OutputIterator,
    typename Projection = std::identity,
    indirect_unary_predicate<projected<ForwardIterator, Projection>> Predicate>
OutputIterator find_all(ForwardIterator first, Sentinel last,
                        OutputIterator dest, Predicate pred, Projection proj = {})
{
    while (first != last) {
        if (invoke(pred, invoke(proj, *first))) {
            *dest = first;
            ++dest;
        }
        ++first;
    }
    return dest;
}
\end{cpp}

sentinel\_for约束确保表达式first != last是有效的。Projection模板类型参数是新的,并且有一个默认值,即恒等操作。indirect\_unary\_predicate的模板类型参数稍微改为projected<ForwardIterator, Projection>,这代表将Projection函数应用于ForwardIterator类型的解引用。

现代化的find\_all()可以按以下方式进行测试:

\begin{cpp}
find_all(begin(vec), end(vec), back_inserter(matches),
    [](int i) { return i == 10; },
    [](int i) { return i * 2; });
\end{cpp}

这个find\_all()的调用与上一节的调用类似,只不过现在包括了投影。对于每个元素,算法首先使用这个投影函数转换,然后将其传递给给定的谓词。这个案例中,首先加倍每个元素,然后检查转换后的元素是否等于10。现在的输出如下(可以与上一节的输出进行比较)。

\begin{shell}
Found 3 matching elements:
5 at position 0
5 at position 2
5 at position 6
\end{shell}

\mySubsubsection{25.2.3.}{实现标准库容器}

C++标准库包含了一组任何容器必须满足的要求,才能资格成为标准库容器。如果希望自定义容器是顺序的(如vector)、有序关联的(如map)或无序关联的(如unordered\_map),必须符合要求。

编写自定义容器时,我的建议是首先编写基本容器,遵循标准库的一般规则,比如使其成为一个类模板,但不必过于担心标准库符合性的细节。开发了基本实现之后,添加迭代器支持,使其能够与标准库框架一起工作。接下来,添加成员函数和类型别名以满足所有基本容器要求。最后,满足容器的要求。本章采用这种方法来开发有向图数据结构,也称为digraph。

\mySamllsection{简介——有向图}

某些C++标准库容器可能在它们的实现中使用图,但标准库并没有向用户提供类似图的数据结构。实现自己的图,似乎是编写符合标准库的自定义容器的完美示例。

开始编写代码之前,首先了解一下有向图是什么样的数据结构,以及如何在内存中表示其数据。图25.1显示了有向图示例的视觉表示。基本上,有向图由一组节点(也称为顶点)组成,这些节点通过边连接,每条边都有方向。

\myGraphic{0.4}{content/part4/chapter25/images/1.png}{图 25.1}

内存中存储此类数据结构有多种方法,例如边列表、邻接矩阵和邻接列表。此实现使用邻接列表。节点存储在vector中,每个节点都有一个邻接列表,列出其相邻节点。假设图25.2中的有向图,使用邻接列表表示此图会产生以下数据结构:

\myGraphic{0.25}{content/part4/chapter25/images/2.png}{图 25.2}

% Please add the following required packages to your document preamble:
% \usepackage{longtable}
% Note: It may be necessary to compile the document several times to get a multi-page table to line up properly
\begin{longtable}{|l|l|}
\hline
\textbf{节点} & \textbf{邻接列表} \\ \hline
\endfirsthead
%
\endhead
%
A             & B, C                    \\ \hline
B             & C                       \\ \hline
C             &                         \\ \hline
\end{longtable}

这可以存储为一个vector,其中vector的每个元素代表表的行,每个vector元素代表一个节点及其对应的邻接列表。第一部分实现了一个简单,但功能齐全的directed\_graph<T>,其中T是存储在单个节点中的值的类型,存储在directed\_graph中的所有值必须唯一。这可能不是最好或最高效的有向图实现,但这不是本章的重点。本章的目的是演示按照标准库的思想,制作数据结构的过程。

\mySamllsection{让directed\_graph成为标准库的容器}

directed\_graph实现使用了节点的概念,要实现的第一段代码是表示图的单个节点的数据结构。节点具有一个值和一个相邻节点的列表,该列表存储为这些相邻节点的索引set,使用set可以确保列表中不存储重复的相邻索引。该类有一个构造函数,用于为给定值构造新的graph\_node,并提供一个value()成员函数来检索节点的值。只提供了value()的const重载,因为值不可更改。定义位于名为graph\_node.cppm的directed\_graph:node分区文件中,该文件位于名为details的命名空间中,并且没有从模块中导出,因为directed\_graphs的用户不应直接使用graph\_nodes。这是graph\_node的接口。注意,这里使用了第1章中介绍的[[nodiscard]]属性:

\begin{cpp}
export module directed_graph:node;
...
namespace details
{
    template <typename T>
    class graph_node
    {
        public:
            // Constructs a graph_node for the given value.
            explicit graph_node(directed_graph<T>* graph, T t)
            : m_graph { graph }, m_data(std::move(t)) { }

            // Returns a reference to the stored value.
            [[nodiscard]] const T& value() const noexcept { return m_data; }

            // Type alias for the container type used to store the adjacency list.
            using adjacency_list_type = std::set<std::size_t>;

            // Returns a reference to the adjacency list.
            [[nodiscard]] auto& get_adjacent_nodes_indices() {
                return m_adjacentNodeIndices; }
            [[nodiscard]] const auto& get_adjacent_nodes_indices() const {
                return m_adjacentNodeIndices; }
        private:
            // A pointer to the graph this node is in.
            directed_graph<T>* m_graph;

            T m_data;
            adjacency_list_type m_adjacentNodeIndices;
    };
}
\end{cpp}

这个定义中,没有对模板类型参数T施加任何约束。就像vector一样,元素的要求取决于实际对容器执行的操作。

构造函数初始化中,m\_data的初始化使用了m\_data(std::move(t))。使用统一的初始化语法m\_data\{std::move(t)\}可能不会编译,因为T是用户定义的类型。

现在已经有了自己的graph\_node实现,让来看看directed\_graph类本身。

\mySamllsection{directed\_graph的接口}

directed\_graph支持三种基本操作:插入、删除和查找,还可以交换,定义在directed\_graph模块中。以下是directed\_graph类模板的public部分:

\begin{cpp}
export module directed_graph;
...
export template <typename T>
class directed_graph
{
    public:
        // For insert to be successful, the value shall not be in the graph yet.
        // Returns true if a new node with given value has been added to
        // the graph, and false if there was already a node with the given value.
        bool insert(T node_value);

        // Returns true if the given node value was erased, false otherwise.
        bool erase(const T& node_value);

        // Returns true if the edge was successfully created, false otherwise.
        bool insert_edge(const T& from_node_value, const T& to_node_value);

        // Returns true if the given edge was erased, false otherwise.
        bool erase_edge(const T& from_node_value, const T& to_node_value);

        // Removes all nodes from the graph.
        void clear() noexcept;

        // Returns a reference to the value in the node with given index
        // without bounds checking.
        const T& operator[](std::size_t index) const;

        // Two directed graphs are equal if their sets of nodes are equal (where
        // nodes with the same T value are considered equal) and the same number
        // of edges between each corresponding pair of nodes.
        // The order in which the nodes and edges have been added does not
        // affect equality.
        bool operator==(const directed_graph& rhs) const;

        // Swaps all nodes between this graph and the given graph.
        void swap(directed_graph& other_graph) noexcept;

        // Returns the number of nodes in the graph.
        [[nodiscard]] std::size_t size() const noexcept;

        // Returns a set with the values of the adjacent nodes of a given node.
        // If the given node does not exist, an empty set is returned.
        [[nodiscard]] std::set<T> get_adjacent_nodes_values(
            const T& node_value) const;
    private:
        // Implementation details not shown yet.
};
\end{cpp}

定义中,没有对模板类型参数T施加任何约束,就像标准库中的vector容器一样,元素的要求取决于实际对容器执行的操作。

这个接口看起来很简单,这个接口没有定义任何用户定义的复制构造函数、复制赋值操作符、移动构造函数或移动赋值操作符,即该类遵循第9章中讨论的零规则。现在来看看public成员函数的具体实现。

\mySamllsection{实现}

确定有向图接口后,需要选择实现模型。正如前面讨论的,这个实现将有向图存储为节点列表,其中每个节点包含其值和相邻节点的索引set。由于相邻节点列表包含到其他节点的索引,节点应基于其索引可访问,vector是最适合存储节点的容器。每个节点都用graph\_node实例表示,最终结构是一个graph\_node vector。以下是directed\_graph类的第一个private成员:

\begin{cpp}
private:
    using node_container_type = std::vector<details::graph_node<T>>;
    node_container_type m_nodes;
\end{cpp}

\mySamllsection{搜索节点}

图的插入和删除操作需要找到具有给定节点值的元素,有一个private辅助成员函数执行此任务。提供了常量和非常量的重载:

\begin{cpp}
// Helper member function to return an iterator to the given node, or the
// end iterator if the given node is not in the graph.
typename node_container_type::iterator find_node(const T& node_value);
typename node_container_type::const_iterator find_node(const T& node_value) const;
\end{cpp}

find\_node()非常量重载的实现如下,常量重载具有相同的实现。

\begin{cpp}
template <typename T>
typename directed_graph<T>::node_container_type::iterator
    directed_graph<T>::find_node(const T& node_value)
{
    return std::find_if(std::begin(m_nodes), std::end(m_nodes),
        [&](const auto& node) { return node.value() == node_value; });
}
\end{cpp}

该成员函数的主体并不复杂。使用标准库中的find\_if()算法,如第20章所述,在图中的所有节点中搜索具有等于node\_value参数的值的节点。如果在图中找到了这样的节点,返回该节点的迭代器;否则,返回结束迭代器。

这个成员函数头中的语法有些令人困惑,特别是对typename关键字的用法。当使用依赖于模板参数的类型时,必须始终使用typename关键字。类型node\_container\_type::iterator,即vector<details::graph\_node<T>{}>::iterator,依赖于T模板类型参数。

\mySamllsection{插入节点}

insert()必须检查具有给定值的节点是否已经存在于图中。如果不存在,可以为给定值创建一个新的节点。public接口提供了一个接受T值的insert()成员函数。这是最优的方式,并遵循第9章部分中的建议。调用emplace\_back()构建一个新的graph\_node,通过传递指向directed\_graph的指针和节点的值到graph\_node构造函数:

\begin{cpp}
template <typename T>
bool directed_graph<T>::insert(T node_value)
{
    auto iter { find_node(node_value) };
    if (iter != std::end(m_nodes)) {
        // Value is already in the graph, return false.
        return false;
    }
    m_nodes.emplace_back(this, std::move(node_value));
    // Value successfully added to the graph, return true.
    return true;
}
\end{cpp}

\mySamllsection{插入边}

将节点添加到图中时,这些节点之间的边就可以构建出来,以构建出一个有向图。为此,提供了insert\_edge()成员函数,该函数需要两个参数:边的起点节点的值,以及边指向的终点节点的值。成员函数首先搜索图中的起点和终点节点。如果其中一个节点在图中找不到,则返回false。如果两个都找到了,通过调用private辅助函数get\_index\_of\_node()计算包含to\_node\_value的节点的索引,并最终将这个索引添加到包含from\_node\_value的节点的邻接列表中。insert\_edge()成员函数只有在图改变时才返回true。第18章中,set的insert()返回一个pair<iterator,bool>,其中布尔值表示插入是否成功,这就是为什么要在返回语句中使用.second。

\begin{cpp}
template <typename T>
bool directed_graph<T>::insert_edge(const T& from_node_value,
    const T& to_node_value)
{
    const auto from { find_node(from_node_value) };
    const auto to { find_node(to_node_value) };
    if (from == std::end(m_nodes) || to == std::end(m_nodes)) {
        return false;
    }
    const std::size_t to_index { get_index_of_node(to) };
    return from->get_adjacent_nodes_indices().insert(to_index).second;
}
\end{cpp}

get\_index\_of\_node()辅助成员函数的实现如下所示:

\begin{cpp}
template <typename T>
std::size_t directed_graph<T>::get_index_of_node(
    typename node_container_type::const_iterator node) const noexcept
{
    return node - std::cbegin(m_nodes);
}
\end{cpp}

\mySamllsection{删除节点}

erase()遵循与insert()相同的模式:首先尝试通过调用find\_node()找到给定的节点。如果节点存在,从图中删除它;否则,不做任何事情。从图中删除现有节点是一个两步过程:

\begin{enumerate}
\item
从所有其他节点的邻接列表中删除节点的索引。

\item
从节点列表中删除实际的节点。
\end{enumerate}

为了第一步,向graph\_node添加了一个辅助成员函数remove\_node\_index(),该函数从节点的邻接列表中删除给定的节点索引,并更新剩余的索引以考虑索引的移动,实现如下。棘手的部分是,邻接列表是一个set,而set不允许对其值进行修改。因此,实现中的第二步将set转换为vector,使用for\_each()算法更新所有需要更新的索引,最后清空set并插入更新的索引。这可能不是最有效的实现,但正如前面提到的,这不是当前讨论的重点。

\begin{cpp}
template <typename T>
void graph_node<T>::remove_node_index(std::size_t node_index)
{
    // First, remove references to the to-be-deleted node.
    m_adjacentNodeIndices.erase(node_index);

    // Second, modify all adjacency indices to account for the removal of a node.
    // std::set doesn't let us modify its elements in place,
    // so we rebuild the set from scratch.
    std::vector<std::size_t> indices(std::begin(m_adjacentNodeIndices),
    std::end(m_adjacentNodeIndices));
    std::for_each(std::begin(indices), std::end(indices),
        [node_index](std::size_t& index) {
            if (index > node_index) { --index; }
        });
    m_adjacentNodeIndices.clear();
    m_adjacentNodeIndices.insert(std::begin(indices), std::end(indices));
}
\end{cpp}

接下来,向directed\_graph添加了一个remove\_all\_links\_to()辅助成员函数。这个成员函数更新所有节点的剩余邻接节点索引,以反映从图中移除一个节点的情况。首先,计算node\_index,即给定节点在节点vector中的索引。然后,遍历所有节点,并从每个节点的邻接列表中移除node\_index。

\begin{cpp}
template <typename T>
void directed_graph<T>::remove_all_links_to(
    typename node_container_type::const_iterator node_iter)
{
    const std::size_t node_index { get_index_of_node(node_iter) };
    for (auto&& node : m_nodes) { node.remove_node_index(node_index); }
}
\end{cpp}

有了这个辅助成员函数,实际的erase()成员函数的实现就变简单了:

\begin{cpp}
template <typename T>
bool directed_graph<T>::erase(const T& node_value)
{
    auto iter { find_node(node_value) };
    if (iter == std::end(m_nodes)) { return false; }
    remove_all_links_to(iter);
    m_nodes.erase(iter);
    return true;
}
\end{cpp}

\mySamllsection{删除边}

删除边的程序与添加边的过程非常相似。如果起点或终点节点找不到,则不执行任何操作;否则,从起点节点的邻接列表中,移除具有to\_node\_value值节点的索引:

\begin{cpp}
template <typename T>
bool directed_graph<T>::erase_edge(const T& from_node_value,
const T& to_node_value)
{
    const auto from { find_node(from_node_value) };
    const auto to { find_node(to_node_value) };
    if (from == std::end(m_nodes) || to == std::end(m_nodes)) {
        return false; // nothing to erase
    }
    const std::size_t to_index { get_index_of_node(to) };
    from->get_adjacent_nodes_indices().erase(to_index);
    return true;
}
\end{cpp}

\mySamllsection{移除所有元素}

clear()简单地清空整个图:

\begin{cpp}
template <typename T>
void directed_graph<T>::clear() noexcept
{
    m_nodes.clear();
}
\end{cpp}

\mySamllsection{交换图}

directed\_graph只有一个数据成员,即一个vector容器,交换两个directed\_graph就是交换其数据成员:

\begin{cpp}
template <typename T>
void directed_graph<T>::swap(directed_graph& other_graph) noexcept
{
    m_nodes.swap(other_graph.m_nodes);
}
\end{cpp}

以下独立导出的swap()函数也提供了,简单地将请求转发到公共swap()成员函数:

\begin{cpp}
export template <typename T>
void swap(directed_graph<T>& first, directed_graph<T>& second) noexcept
{
    first.swap(second);
}
\end{cpp}

\mySamllsection{访问节点}

directed\_graph的公共接口支持基于其索引访问节点,使用operator[]。实现很简单,与vector一样,操作符不对请求的索引进行边界检查:

\begin{cpp}
template <typename T>
const T& directed_graph<T>::operator[](std::size_t index) const
{
    return m_nodes[index].value();
}
\end{cpp}

\mySamllsection{比较图}

两个directed\_graphs相等,当包含相同的一组节点,以及这些节点之间相同的一组边。由于两个directed\_graph可能以不同的顺序添加节点,实现不能仅比较m\_node数据成员,还需要做更多的工作。

代码首先检查两个directed\_graph的大小。如果大小不同,两个图不相同。如果它们的大小相同,代码遍历其中一个图的所有节点。对于每个节点,尝试在另一个图中找到相同的节点。如果找不到这样的节点,则两个图不相同。如果找到这样的节点,使用一个名为get\_adjacent\_nodes\_values()的辅助成员函数将相邻节点的索引转换为相邻节点的值,然后比较这些值是否相等。

\begin{cpp}
template <typename T>
bool directed_graph<T>::operator==(const directed_graph& rhs) const
{
    if (m_nodes.size() != rhs.m_nodes.size()) { return false; }

    for (auto&& node : m_nodes) {
        const auto rhsNodeIter { rhs.find_node(node.value()) };
        if (rhsNodeIter == std::end(rhs.m_nodes)) { return false; }

        const auto adjacent_values_lhs { get_adjacent_nodes_values(
            node.get_adjacent_nodes_indices()) };
        const auto adjacent_values_rhs { rhs.get_adjacent_nodes_values(
            rhsNodeIter->get_adjacent_nodes_indices()) };
        if (adjacent_values_lhs != adjacent_values_rhs) { return false; }
    }
    return true;
}

template <typename T>
std::set<T> directed_graph<T>::get_adjacent_nodes_values(
    const typename details::graph_node<T>::adjacency_list_type& indices) const
{
    std::set<T> values;
    for (auto&& index : indices) { values.insert(m_nodes[index].value()); }
    return values;
}
\end{cpp}

\mySamllsection{获取相邻节点}

public接口提供了一个名为get\_adjacent\_nodes\_values()的成员函数,接受一个const T类型的引用作为参数,该函数返回一个包含给定节点相邻节点的值的集合。如果给定的节点不存在,则返回一个空集合。实现使用了前面章节中接受索引列表get\_adjacent\_nodes\_values()的重载:

\begin{cpp}
template <typename T>
std::set<T> directed_graph<T>::get_adjacent_nodes_values(const T& node_value) const
{
    auto iter { find_node(node_value) };
    if (iter == std::end(m_nodes)) { return {}; }
    return get_adjacent_nodes_values(iter->get_adjacent_nodes_indices());
}
\end{cpp}

\mySamllsection{查询图的大小}

最后,size()成员函数返回图中的节点数量:

\begin{cpp}
template <typename T>
std::size_t directed_graph<T>::size() const noexcept
{
    return m_nodes.size();
}
\end{cpp}

\mySamllsection{打印图}

图可以以一种称为DOT的标准格式打印出来,这是一种图形描述语言。有工具可以理解DOT格式的图,并将它们转换为图形表示。为了方便测试directed\_graph代码,可以使用以下to\_dot()转换函数,其使用示例在下一节给出。

\begin{cpp}
// Returns a given graph in DOT format.
export template <typename T>
std::string to_dot(const directed_graph<T>& graph, std::string_view graph_name)
{
    std::ostringstream output;
    std::println(output, "digraph {} {{", graph_name);
            for (std::size_t index { 0 }; index < graph.size(); ++index) {
                const auto& node_value { graph[index] };
                const auto adjacent_values { graph.get_adjacent_nodes_values(node_value) };
                if (adjacent_values.empty()) {
                    std::println(output, "{}", node_value);
                } else {
                    for (auto&& neighbor : adjacent_values) {
                        std::println(output, "{} -> {}", node_value, neighbor);
                    }
                }
            }
            std::println(output, "}}");
    return std::move(output).str();
}
\end{cpp}

\mySamllsection{有向图的使用}

已经完成了基本有向图类的完整实现,是时候对这个类进行一次测试了。下面是一个展示基本有向图类模板的小程序:

\begin{cpp}
directed_graph<int> graph;
// Insert some nodes and edges.
graph.insert(11);
graph.insert(22);
graph.insert(33);
graph.insert(44);
graph.insert(55);
graph.insert_edge(11, 33);
graph.insert_edge(22, 33);
graph.insert_edge(22, 44);
graph.insert_edge(22, 55);
graph.insert_edge(33, 44);
graph.insert_edge(44, 55);
println("{}", to_dot(graph, "Graph1"));

// Remove an edge and a node.
graph.erase_edge(22, 44);
graph.erase(44);
println("{}", to_dot(graph, "Graph1"));

// Print the size of the graph.
println("Size: {}", graph.size());
\end{cpp}

输出为:

\begin{shell}
digraph Graph1 {
11 -> 33
22 -> 33
22 -> 44
22 -> 55
33 -> 44
44 -> 55
55
}
digraph Graph1 {
11 -> 33
22 -> 33
22 -> 55
33
55
}
Size: 4
\end{shell}

\mySamllsection{标准库的其他功能}

前面几节中实现的基本有向图遵循了标准库的思想,但并未完全遵循其具体规定。对于大多数用途,前面的实现已经足够好。如果想使用标准库算法处理有向图,必须做更多的努力。C++标准规定了类模板必须提供的成员函数和类型别名,以符合标准库容器的资格。

\mySamllsection{必需的类型别名}

C++标准要求每个标准库容器都必须提供以下公共类型别名:

% Please add the following required packages to your document preamble:
% \usepackage{longtable}
% Note: It may be necessary to compile the document several times to get a multi-page table to line up properly
\begin{longtable}{|l|l|}
\hline
\textbf{类型名称} & \textbf{描述}                                                     \\ \hline
\endfirsthead
%
\endhead
%
value\_type        & 存储在容器中的元素类型                                               \\ \hline
reference          & 对容器中存储的元素类型的引用                                         \\ \hline
const\_reference   & 对容器中存储的元素类型的常量引用                                     \\ \hline
iterator           & 遍历容器中元素类型的类型                                             \\ \hline
const\_iterator    & 用于遍历容器中常量元素的迭代器类型的版本                               \\ \hline
size\_type &
\begin{tabular}[c]{@{}l@{}}可以表示容器中元素数量的类型;\\这通常是size\_t(<cstddef>)\end{tabular} \\ \hline
difference\_type &
\begin{tabular}[c]{@{}l@{}}可以表示容器中两个迭代器差异的类型;\\这通常是ptrdiff\_t (<cstddef>)\end{tabular} \\ \hline
\end{longtable}

以下是为directed\_graph类模板定义的所有这些类型别名,除了迭代器和const\_iterator。关于迭代器的编写将在本章的后面详细介绍。

\begin{cpp}
export template <typename T>
class directed_graph
{
    public:
        using value_type = T;
        using reference = value_type&;
        using const_reference = const value_type&;
        using size_type = std::size_t;
        using difference_type = std::ptrdiff_t;
        // Remainder of class definition omitted for brevity.
};
\end{cpp}

有了这些类型别名,一些成员函数可以稍微修改。这是对operator[]的早期定义:

\begin{cpp}
const T& operator[](std::size_t index) const;
\end{cpp}

使用新的类型别名:

\begin{cpp}
const_reference operator[](size_type index) const;
\end{cpp}

\mySamllsection{必需的成员函数}

除了必需的类型别名之外,容器还必须提供以下成员函数:

% Please add the following required packages to your document preamble:
% \usepackage{longtable}
% Note: It may be necessary to compile the document several times to get a multi-page table to line up properly
\begin{longtable}{|l|l|l|}
\hline
\textbf{成员函数} &
\textbf{描述} &
\textbf{\begin{tabular}[c]{@{}l@{}}最坏情况下的\\时间复杂度\end{tabular}} \\ \hline
\endfirsthead
%
\endhead
%
默认构造函数 &
构造一个空的容器 &
常数 \\ \hline
复制构造函数 &
执行容器的深度复制 &
线性 \\ \hline
移动构造函数 &
执行移动构造操作 &
常数 \\ \hline
复制赋值操作符 &
执行容器的深度复制 &
线性 \\ \hline
移动赋值操作符 &
执行移动赋值操作 &
常数 \\ \hline
析构函数 &
\begin{tabular}[c]{@{}l@{}}销毁容器中剩余的元素,\\并释放它们的内存\end{tabular} &
线性 \\ \hline
\begin{tabular}[c]{@{}l@{}}iterator begin();\\ const\_iterator\\ begin() const;\end{tabular} &
\begin{tabular}[c]{@{}l@{}}返回一个迭代器或常量迭代器,\\指向容器中的第一个元素\end{tabular} &
常数 \\ \hline
\begin{tabular}[c]{@{}l@{}}iterator end();\\ const\_iterator\\ end() const;\end{tabular} &
\begin{tabular}[c]{@{}l@{}}返回一个迭代器或常量迭代器,\\指向容器中最后一个元素的下\\一个位置\end{tabular} &
常数 \\ \hline
\begin{tabular}[c]{@{}l@{}}const\_iterator\\ cbegin() const;\end{tabular} &
与begin() const相同 &
常数 \\ \hline
\begin{tabular}[c]{@{}l@{}}const\_iterator\\ cend() const;\end{tabular} &
与end() const相同 &
常数 \\ \hline
operator== &
\begin{tabular}[c]{@{}l@{}}比较操作符,用于比较两个容器\end{tabular} &
线性 \\ \hline
\begin{tabular}[c]{@{}l@{}}void swap(Container\&)\\ noexcept;\end{tabular} &
\begin{tabular}[c]{@{}l@{}}交换传递给成员函数的容器与调用\\成员函数对象的内容\end{tabular} &
常数 \\ \hline
size\_type size() const; &
\begin{tabular}[c]{@{}l@{}}返回容器中的元素数量\end{tabular} &
常数 \\ \hline
size\_type max\_size() const; &
\begin{tabular}[c]{@{}l@{}}返回容器可以容纳的最大元素数量\end{tabular} &
常数 \\ \hline
bool empty() const; &
\begin{tabular}[c]{@{}l@{}}返回容器是否包含元素\end{tabular} &
常数 \\ \hline
\end{longtable}

directed\_graph实现遵循零规则(参见第9章),不需要显式的复制/移动构造函数、复制/移动赋值操作符或析构函数。

以下代码展示了size()、max\_size()和empty()成员函数的声明。与迭代器相关的成员函数,begin()、end()、cbegin()和cend(),将在下一节关于编写迭代器中介绍。

\begin{cpp}
export template <typename T>
class directed_graph
{
    public:
        [[nodiscard]] size_type size() const noexcept;
        [[nodiscard]] size_type max_size() const noexcept;
        [[nodiscard]] bool empty() const noexcept;
        // Other member functions omitted for brevity.
};
\end{cpp}

这些三个成员函数的实现很简单,可以直接转发到m\_node容器具有相同名称的成员函数,size\_type是类模板中定义的类型别名。它是一个类模板的成员,返回类型在实现中必须使用typename directed\_graph<T>进行限定。

\begin{cpp}
template <typename T>
typename directed_graph<T>::size_type directed_graph<T>::size() const noexcept
{
    return m_nodes.size();
}

template <typename T>
typename directed_graph<T>::size_type directed_graph<T>::max_size() const noexcept
{
    return m_nodes.max_size();
}

template <typename T>
bool directed_graph<T>::empty() const noexcept
{
    return m_nodes.empty();
}
\end{cpp}

当前的directed\_graph实现有operator[],用于根据其索引获取对节点的访问。这个操作符,就像vector中的operator[]一样,不执行边界检查,传递一个越界索引可能会使应用程序崩溃。与vector一样,directed\_graph可以通过添加一个at()成员函数来进行增强,该函数执行边界检查,并在传递的索引越界时抛出std::out\_of\_range异常。这是定义:

\begin{cpp}
const_reference at(size_type index) const;
\end{cpp}

实现只是转发到m\_node:

\begin{cpp}
template <typename T>
typename directed_graph<T>::const_reference
    directed_graph<T>::at(size_type index) const
{
    return m_nodes.at(index).value();
}
\end{cpp}

\mySamllsection{实现迭代器}

最重要的容器要求是迭代器支持。为了与泛型算法一起工作,每个容器都必须提供一种访问容器中元素的迭代器类型。迭代器通常应提供重载的operator++、*、->、==以及一些其他操作,具体取决于其特定行为,只要迭代器提供基本的迭代操作就好。

关于自定义迭代器,首先需要做出的决定是,将是什么类型的:前向、双向、随机访问或连续。双向迭代器支持似乎是针对有向图迭代器的良好选择,所以必须提供operator-{}-。另一种选择是为有向图实现随机访问迭代器,这涉及添加+、-、+=、-=、<、>、<=、>=和[]操作符。这可以是一个很好的练习,以练习编写迭代器。有关随机访问迭代器要求的更多详细信息,请参阅第17章。

第二个决定是如何对容器中的元素进行排序。有向图是无序的,因此按排序顺序遍历将不是有效率的做法。对于符合标准库规范的容器来说,最重要的是从begin()到end()遍历其元素时,每个元素都会访问一次;仅仅因为它是序列,并不意味着它是特定的序列。有向图迭代器可以简单地按添加到图中的顺序遍历节点,这与std::unordered\_set的迭代方式相同。

第三个决定是如何在内部表示迭代器,表示通常非常依赖于容器的内部实现。迭代器的主要目的是引用容器中的单个元素。有向图的情况下,所有节点都存储在m\_nodes中,因此有向图迭代器可能是对指向元素vector迭代器的包装。

当选择了实现方式,必须决定一致地表示结束迭代器。回想一下,结束迭代器应该真的是“超出范围”的标记:通过将++应用于指向容器中最后一个元素的迭代器所达到的迭代器。有向图迭代器可以使用m\_nodes的结束迭代器作为其结束迭代器。

最后,容器需要提供迭代器和const\_iterator类型别名。此实现定义了这两个类型别名,都是基于const\_directed\_graph\_iterator\_impl类模板。原因在于在图中不能更改值,这与std::set遵循的原则相同。

如果需要为自定义数据结构,提供不同的迭代器和const\_iterator类型,迭代器必须可转换为const\_iterator。

\mySamllsection{const\_directed\_graph\_iterator\_impl类模板}

鉴于上一节做出的决定,现在是定义const\_directed\_graph\_iterator\_impl类模板的时候了。首先,每个const\_directed\_graph\_iterator\_impl对象都是特定于directed\_graph实例的迭代器。为了提供一对一的映射,const\_directed\_graph\_iterator\_impl也必须是一个类模板,其中directed\_graph类型作为模板类型参数,称为DirectedGraph。

主要问题是如何符合双向迭代器的要求,任何行为像迭代器的东西都是迭代器(鸭子定律)。迭代器不需要从另一个类派生就可以作为双向迭代器,但如果希望自定义迭代器可以在泛型算法中使用,必须指定其特性。第17章解释说,iterator\_traits是一个类模板,为每个迭代器类型定义了五个类型别名:value\_type、difference\_type、iterator\_category、pointer和reference。iterator\_traits类模板的默认实现,只是从迭代器本身获取这五个类型别名,可以直接为自定义迭代器定义这些类型别名。const\_directed\_graph\_iterator\_impl是一个双向迭代器,所以指定bidirectional\_iterator\_tag作为迭代器类别。其他合法的迭代器类别包括input\_iterator\_tag、output\_iterator\_tag、forward\_iterator\_tag、random\_access\_iterator\_tag和contiguous\_iterator\_tag。对于const\_directed\_graph\_iterator\_impl,元素类型(value\_type)是typename DirectedGraph::value\_type。

\begin{myNotic}{NOTE}
过去,建议从<iterator>中定义的std::iterator类模板派生自定义迭代器。目前这个类模板已经弃用,不应该再使用。
\end{myNotic}

以下是const\_directed\_graph\_iterator\_impl类模板的定义:

\begin{cpp}
template <typename DirectedGraph>
class const_directed_graph_iterator_impl
{
    public:
        using value_type = typename DirectedGraph::value_type;
        using difference_type = std::ptrdiff_t;
        using iterator_category = std::bidirectional_iterator_tag;
        using pointer = const value_type*;
        using reference = const value_type&;
        using node_container_iterator =
            typename DirectedGraph::node_container_type::const_iterator;

        // Bidirectional iterators must supply a default constructor.
        const_directed_graph_iterator_impl() = default;

        explicit const_directed_graph_iterator_impl(node_container_iterator it);

        reference operator*() const;

        // Return type must be something to which -> can be applied.
        // So, return a pointer.
        pointer operator->() const;

        const_directed_graph_iterator_impl& operator++();
        const_directed_graph_iterator_impl operator++(int);

        const_directed_graph_iterator_impl& operator--();
        const_directed_graph_iterator_impl operator--(int);

        // Defaulted operator==.
        bool operator==(const const_directed_graph_iterator_impl&) const = default;

    private:
        friend class directed_graph<value_type>;

        node_container_iterator m_nodeIterator;
};
\end{cpp}

如果对重载操作符的定义和实现(在下一节中展示)感到困惑,请参考第15章以获取详细信息。const\_directed\_graph\_iterator\_impl的实现不需要复制/移动构造函数和复制/移动赋值操作符,默认行为就是我们想要的。该类也不需要显式的析构函数,因为没有需要清理的东西,所以这个类也遵循零规则。

\mySamllsection{const\_directed\_graph\_iterator\_impl成员函数的实现}

const\_directed\_graph\_iterator\_impl构造函数初始化数据成员:

\begin{cpp}
template <typename DirectedGraph>
const_directed_graph_iterator_impl<DirectedGraph>::
    const_directed_graph_iterator_impl(node_container_iterator it)
        : m_nodeIterator { it } { }
\end{cpp}

默认构造函数被默认,以便客户可以在不初始化的情况下声明const\_directed\_graph\_iterator\_impl变量。使用默认构造函数构建的迭代器不需要引用值,并且可以对其进行任何操作,其结果可能未定义。

解引用操作符的实现简洁,但看起来会有些棘手。第15章解释说operator*和->是不对称的:

\begin{itemize}
\item
operator* 返回对实际底层值的引用,是对迭代器所引用的元素的引用。

\item
operator-> 必须返回可以再次应用箭头操作符的东西,因此返回指向元素的指针。编译器然后将->应用于指针,从而访问元素的特定字段或成员函数。
\end{itemize}

\begin{cpp}
// Return a reference to the actual element.
template <typename DirectedGraph>
typename const_directed_graph_iterator_impl<DirectedGraph>::reference
    const_directed_graph_iterator_impl<DirectedGraph>::operator*() const
{
    return m_nodeIterator->value();
}

// Return a pointer to the actual element, so the compiler can
// apply -> to it to access the actual desired field.
template <typename DirectedGraph>
typename const_directed_graph_iterator_impl<DirectedGraph>::pointer
    const_directed_graph_iterator_impl<DirectedGraph>::operator->() const
{
    return &m_nodeIterator->value();
}
\end{cpp}

增加操作符的实现如下。递减操作符没有显示,可以以类似的方式实现。

\begin{cpp}
template <typename DirectedGraph>
const_directed_graph_iterator_impl<DirectedGraph>&
    const_directed_graph_iterator_impl<DirectedGraph>::operator++()
{
    ++m_nodeIterator;
    return *this;
}

template <typename DirectedGraph>
const_directed_graph_iterator_impl<DirectedGraph>
    const_directed_graph_iterator_impl<DirectedGraph>::operator++(int)
{
    auto oldIt { *this };
    ++*this;
    return oldIt;
}
\end{cpp}

迭代器不需要比原始指针更安全,不需要对增加一个已经到达末尾的迭代器等操作进行错误进行检查。

const\_directed\_graph\_iterator\_impl的node\_container\_iterator类型别名使用了directed\_graph的private node\_container\_type类型别名。因此,directed\_graph类模板必须声明const\_directed\_graph\_iterator\_impl为友元:

\begin{cpp}
export template <typename T>
class directed_graph
{
    // Other member functions omitted for brevity.
    private:
        friend class const_directed_graph_iterator_impl<directed_graph>;
};
\end{cpp}

\mySamllsection{迭代器类型别名和访问成员函数}

为directed\_graph提供迭代器支持的最后一部分是,提供必要的类型别名在directed\_graph类模板中,并编写begin()、end()、cbegin()和cend()成员函数。类型别名和成员函数原型如下所示:

\begin{cpp}
export template <typename T>
class directed_graph
{
    public:
        // Other type aliases omitted for brevity.
        using iterator = const_directed_graph_iterator_impl<directed_graph>;
        using const_iterator = const_directed_graph_iterator_impl<directed_graph>;

        // Iterator member functions.
        iterator begin() noexcept;
        iterator end() noexcept;
        const_iterator begin() const noexcept;
        const_iterator end() const noexcept;
        const_iterator cbegin() const noexcept;
        const_iterator cend() const noexcept;
        // Remainder of class definition omitted for brevity.
};
\end{cpp}

iterator和const\_iterator都是const\_directed\_graph\_iterator\_impl的类型别名,所以用户无法修改directed\_graph迭代器所引用的值。directed\_graph中的节点值必须唯一,允许用户通过迭代器修改节点的值,将允许引入重复值的可能性。这与std::set的原则相同,在std::set中,也不能修改元素。

由于iterator和const\_iterator都是const\_directed\_graph\_iterator\_impl的类型别名,返回迭代器的非const begin()和end()成员函数并不严格必要;const重载就足够了。然而,标准库要求容器必须提供非const begin()和end()的重载。

directed\_graph类模板将其所有节点存储在一个简单的vector中。begin()和end()可以简单地将工作转发给vector的同名成员函数,并将结果包装在const\_directed\_graph\_iterator\_impl中:

\begin{cpp}
template <typename T>
typename directed_graph<T>::iterator
    directed_graph<T>::begin() noexcept { return iterator{ std::begin(m_nodes) }; }

template <typename T>
typename directed_graph<T>::iterator
    directed_graph<T>::end() noexcept { return iterator { std::end(m_nodes) }; }

template <typename T>
typename directed_graph<T>::const_iterator
    directed_graph<T>::begin() const noexcept
{ return const_iterator { std::begin(m_nodes) }; }

template <typename T>
typename directed_graph<T>::const_iterator
    directed_graph<T>::end() const noexcept
{ return const_iterator { std::end(m_nodes) }; }
\end{cpp}

cbegin()和cend()成员函数将请求转发到begin()和end()的const重载:

\begin{cpp}
template <typename T>
typename directed_graph<T>::const_iterator
    directed_graph<T>::cbegin() const noexcept { return begin(); }

template <typename T>
typename directed_graph<T>::const_iterator
    directed_graph<T>::cend() const noexcept { return end(); }
\end{cpp}

\mySamllsection{修改其他成员函数以使用迭代器}

现在directed\_graph支持了迭代器,其他成员函数可以稍微修改以使用迭代器,从而遵循标准库的指南。首先看看insert()成员函数。早期的实现中,是这样定义的:

\begin{cpp}
// For an insert to be successful, the value shall not be in the graph yet.
// Returns true if a new node with given value has been added to // the graph, and false if there was already a node with the given value.
bool insert(T node_value);
\end{cpp}

为了更贴近标准库的思想,可以将其修改为返回一个std::pair<iterator, bool>,其中布尔值表示元素是否已添加到图中,迭代器对应该新添加的元素或已存在的元素。

\begin{cpp}
std::pair<iterator, bool> insert(T node_value);
\end{cpp}

实现如下:

\begin{cpp}
template <typename T>
std::pair<typename directed_graph<T>::iterator, bool>
    directed_graph<T>::insert(T node_value)
{
    auto iter { find_node(node_value) };
    if (iter != std::end(m_nodes)) {
        // Value is already in the graph.
        return { iterator { iter }, false };
    }
    m_nodes.emplace_back(this, std::move(node_value));
    // Value successfully added to the graph.
    return { iterator { std::prev(std::end(m_nodes)) }, true };
}
\end{cpp}

此外,提供了一个接受迭代器提示的insert()的重载。这个成员对于有向图来说没什么用,但提供了对称性。

\begin{cpp}
template <typename T>
typename directed_graph<T>::iterator
    directed_graph<T>::insert(const_iterator hint, T node_value)
{
    // Ignore the hint, just forward to another insert().
    return insert(std::move(node_value)).first;
}
\end{cpp}

最后,提供了一个接受迭代器范围的insert()的重载。这个重载是一个成员函数模板,以便接受容器,而不仅仅是其他有向图的迭代器范围。实际实现使用了第17章中的insert\_iterator。

\begin{cpp}
template <typename T>
template <std::input_iterator Iter>
void directed_graph<T>::insert(Iter first, Iter last)
{
    // Copy each element in the range by using an insert_iterator adapter.
    // Give end() as a dummy position -- insert ignores it anyway.
    std::copy(first, last, std::insert_iterator { *this, end() });
}
\end{cpp}

erase()成员函数应修改以使用迭代器。早期的定义有一个节点值作为参数并返回一个布尔值:

\begin{cpp}
// Returns true if the given node value was erased, false otherwise.
bool erase(const T& node_value);
\end{cpp}

为了遵循标准库的原则,directed\_graph修改为提供两个erase()成员函数:一个删除迭代器所指向的节点,另一个删除给定迭代器范围的节点。两者都返回最后一个删除节点的迭代器:

\begin{cpp}
// Returns an iterator to the element after the last deleted element.
iterator erase(const_iterator pos);
iterator erase(const_iterator first, const_iterator last);
\end{cpp}

以下是实现:

\begin{cpp}
template <typename T>
typename directed_graph<T>::iterator
    directed_graph<T>::erase(const_iterator pos)
{
    if (pos.m_nodeIterator == std::end(m_nodes)) {
        return end();
    }
    remove_all_links_to(pos.m_nodeIterator);
    return iterator { m_nodes.erase(pos.m_nodeIterator) };
}

template <typename T>
typename directed_graph<T>::iterator
    directed_graph<T>::erase(const_iterator first, const_iterator last)
{
    for (auto iter { first }; iter != last; ++iter) {
        remove_all_links_to(iter.m_nodeIterator);
    }
    return iterator { m_nodes.erase(first.m_nodeIterator, last.m_nodeIterator) };
}
\end{cpp}

最后,可以实现find()成员函数,返回一个迭代器。这个实现在本章的最后留作练习。

\begin{cpp}
const_iterator find(const T& node_value) const;
\end{cpp}

\mySamllsection{使用directed\_graph迭代器}

现在directed\_graph支持迭代器,可以像使用其他标准库容器一样遍历其元素,并将迭代器传递给成员函数和函数。以下是一些示例:

\begin{cpp}
directed_graph<int> graph;
// Populate the graph, omitted (see downloadable source code archive) ...

// Try to insert a duplicate, and use structured bindings for the result.
auto [iter22, inserted] { graph.insert(22) };
if (!inserted) { println("Duplicate element."); }

// Print nodes using a for loop and iterators.
for (auto iter { graph.cbegin() }; iter != graph.cend(); ++iter) {
    print("{} ", *iter);
}
println("");

// Print nodes using a for loop and iterators retrieved with the non-member
// functions cbegin() and cend().
for (auto iter { cbegin(graph) }; iter != cend(graph); ++iter) {
    print("{} ", *iter);
}
println("");

// Print nodes using a range-based for loop.
for (auto& node : graph) { print("{} ", node); }
println("");

// Search a node using the find() Standard Library algorithm.
auto result { find(begin(graph), end(graph), 22) };
if (result != end(graph)) { println("Node 22 found.");}
else { println("Node 22 NOT found."); }

// Count all nodes with values > 22.
auto count { count_if(begin(graph), end(graph),
    [](const auto& node) { return node > 22; }) };
println("{} nodes > 22", count);

// Use the iterator-based erase() member function in combination with find().
graph.erase(find(begin(graph), end(graph), 44));
\end{cpp}

这个代码还显示,由于支持迭代器,标准库算法可以与directed\_graph一起使用。由于directed\_graph只支持const迭代器,因此只支持非修改的标准库算法,就像std::set一样。以下使用remove-erase模式的代码无法编译:

\begin{cpp}
graph.erase(remove_if(begin(graph), end(graph),
    [](const auto& node) { return node > 22; }), end(graph));
\end{cpp}

\mySamllsection{添加对逆向迭代器的支持}

如果容器提供双向迭代器,则认为可逆。可逆容器应该有两个类型别名:

% Please add the following required packages to your document preamble:
% \usepackage{longtable}
% Note: It may be necessary to compile the document several times to get a multi-page table to line up properly
\begin{longtable}{|l|l|}
\hline
\textbf{类型名称}       & \textbf{描述}                                                                                                                           \\ \hline
\endfirsthead
%
\endhead
%
reverse\_iterator        & \begin{tabular}[c]{@{}l@{}}用于按反序遍历容器中元素的类型\end{tabular}                             \\ \hline
const\_reverse\_iterator & \begin{tabular}[c]{@{}l@{}}用于按反序遍历容器中常量元素的reverse\_iterator\end{tabular} \\ \hline
\end{longtable}

此外,容器应该提供rbegin()和rend(),与begin()和end()对称;以及crbegin()和crend(),与cbegin()和cend()对称。

directed\_graph迭代器是双向的,应该支持逆向遍历。以下代码段突出显示了必要的更改,这两个新的类型别名使用标准库提供的std::reverse\_iterator适配器(第17章中描述),将directed\_graph迭代器转换为逆向迭代器。

\begin{cpp}
export template <typename T>
class directed_graph
{
    public:
    // Other type aliases omitted for brevity.
    using reverse_iterator = std::reverse_iterator<iterator>;
    using const_reverse_iterator = std::reverse_iterator<const_iterator>;

    // Reverse iterator member functions.
    reverse_iterator rbegin() noexcept;
    reverse_iterator rend() noexcept;
    const_reverse_iterator rbegin() const noexcept;
    const_reverse_iterator rend() const noexcept;
    const_reverse_iterator crbegin() const noexcept;
    const_reverse_iterator crend() const noexcept;
    // Remainder of class definition omitted for brevity.
};
\end{cpp}

反向迭代器成员函数的实现:

\begin{cpp}
template <typename T>
typename directed_graph<T>::reverse_iterator
    directed_graph<T>::rbegin() noexcept { return reverse_iterator { end() }; }

template <typename T>
typename directed_graph<T>::reverse_iterator
    directed_graph<T>::rend() noexcept { return reverse_iterator { begin() }; }

template <typename T>
typename directed_graph<T>::const_reverse_iterator
    directed_graph<T>::rbegin() const noexcept
{ return const_reverse_iterator { end() }; }

template <typename T>
typename directed_graph<T>::const_reverse_iterator
    directed_graph<T>::rend() const noexcept
{ return const_reverse_iterator { begin() }; }

template <typename T>
typename directed_graph<T>::const_reverse_iterator
    directed_graph<T>::crbegin() const noexcept { return rbegin(); }

template <typename T>
typename directed_graph<T>::const_reverse_iterator
    directed_graph<T>::crend() const noexcept { return rend(); }
\end{cpp}

下面的代码段展示了如何以相反的顺序打印一个图的所有节点:

\begin{cpp}
for (auto iter { graph.rbegin() }; iter != graph.rend(); ++iter) {
    print("{} ", *iter);
}
\end{cpp}

\mySamllsection{遍历相邻节点}

有向图使用一个vector来存储节点,其中每个节点包含节点的值和一组相邻节点。我们改进有向图接口,以支持在不将相邻节点复制到另一个容器的情况下,遍历给定节点的相邻节点。首先需要添加的是一个名为const\_adjacent\_nodes\_iterator\_impl的类模板,遵循与const\_directed\_graph\_iterator\_impl相同的原则。请从本书的代码库下载源码,以查看完整代码。

下一步是在有向图接口中增加一个新的类型别名、一个辅助结构来表示相邻节点的范围,以及一个成员函数来获取给定节点值的相邻节点。nodes\_adjacent\_to()成员函数在给定的节点值无法找到时,返回一个空的optional。

\begin{cpp}
export template <typename T>
class directed_graph
{
    public:
        // Other type aliases omitted for brevity.
        using const_adjacent_nodes_iterator =
            const_adjacent_nodes_iterator_impl<directed_graph>;

        // Helper structure to represent a range of adjacent nodes.
        struct nodes_adjacent_to_result
        {
            const_adjacent_nodes_iterator m_begin;
            const_adjacent_nodes_iterator m_end;
            const_adjacent_nodes_iterator begin() const noexcept{ return m_begin; }
            const_adjacent_nodes_iterator end() const noexcept { return m_end; }
        };

        // Returns a range with the adjacent nodes for the given node value.
        std::optional<nodes_adjacent_to_result> nodes_adjacent_to(
            const T& node_value) const noexcept;
        // Remainder of class definition omitted for brevity.
};
\end{cpp}

以下是nodes\_adjacent\_to()成员函数的实现:

\begin{cpp}
template <typename T>
std::optional<typename directed_graph<T>::nodes_adjacent_to_result>
    directed_graph<T>::nodes_adjacent_to(const T& node_value) const noexcept
{
    auto iter { find_node(node_value) };
    if (iter == std::end(m_nodes)) { return {}; }
    return nodes_adjacent_to_result {
        const_adjacent_nodes_iterator {
            std::cbegin(iter->get_adjacent_nodes_indices()), this },
        const_adjacent_nodes_iterator {
            std::cend(iter->get_adjacent_nodes_indices()), this }
    };
}
\end{cpp}

有了nodes\_adjacent\_to(),访问给定节点的所有相邻节点变得轻而易举。以下是一个示例,用于打印值为22节点的所有相邻节点。

\begin{cpp}
print("Adjacency list for node 22: ");
auto nodesAdjacentTo22 { graph.nodes_adjacent_to(22) };
if (!nodesAdjacentTo22.has_value()) {
    println("Value 22 not found.");
} else {
    for (const auto& node : *nodesAdjacentTo22) { print("{} ", node); }
}
\end{cpp}

\mySamllsection{打印图}

现在directed\_graph支持nodes\_adjacent\_to(),用于打印图的to\_dot()辅助函数模板可以简化为:

\begin{cpp}
export template <typename T>
std::string to_dot(const directed_graph<T>& graph, std::string_view graph_name)
{
    std::ostringstream output;
    std::println(output, "digraph {} {{", graph_name);
    for (auto&& node : graph) {
        auto adjacent_nodes { graph.nodes_adjacent_to(node) };
        if (adjacent_nodes->begin() == adjacent_nodes->end()) {
            std::println(output, "{}", node);
        } else {
            for (const auto& adjacent_node : *adjacent_nodes) {
                std::println(output, "{} -> {}", node, adjacent_node);
            }
        }
    }
    std::println(output, "}}");
    return std::move(output).str();
}
\end{cpp}

\mySamllsection{其他类似标准库的功能}

directed\_graph类模板可以添加一些其他标准库功能。先添加类似于vector的assign()成员函数,接受迭代器范围的assign()成员函数,再是一个成员函数模板:

\begin{cpp}
template <std::input_iterator Iter>
void assign(Iter first, Iter last);

void assign(std::initializer_list<T> il);
\end{cpp}

可以将给定迭代器范围或初始化列表中的所有元素,分配给一个有向图。赋值意味着清除当前图,并插入新节点。尽管语法不同,但实现很简单:

\begin{cpp}
template <typename T>
template <std::input_iterator Iter>
void directed_graph<T>::assign(Iter first, Iter last)
{
    clear();
    for (auto iter { first }; iter != last; ++iter) { insert(*iter); }
}

template <typename T>
void directed_graph<T>::assign(std::initializer_list<T> il)
{
    assign(std::begin(il), std::end(il));
}
\end{cpp}

还提供了insert()的initializer\_list重载:

\begin{cpp}
template <typename T>
void directed_graph<T>::insert(std::initializer_list<T> il)
{
    insert(std::begin(il), std::end(il));
}
\end{cpp}

有了这个insert()的重载,节点可以这样添加:

\begin{cpp}
graph.insert({ 66, 77, 88 });
\end{cpp}

接下来,可以添加初始化列表构造函数和赋值操作符。由于这是第一个显式的directed\_graph构造函数和赋值操作符,默认、复制和移动构造函数,以及复制和移动赋值操作符也需要显式默认。

\begin{cpp}
// Default, copy, and move constructors.
directed_graph() = default;
directed_graph(const directed_graph&) = default;
directed_graph(directed_graph&&) noexcept = default;

// initializer_list constructor.
directed_graph(std::initializer_list<T> il);

// Copy and move assignment operators.
directed_graph& operator=(const directed_graph&) = default;
directed_graph& operator=(directed_graph&&) noexcept = default;

// initializer_list assignment operator.
directed_graph& operator=(std::initializer_list<T> il);
\end{cpp}

下面是initializer\_list构造函数和赋值操作符的实现:

\begin{cpp}
template <typename T>
directed_graph<T>::directed_graph(std::initializer_list<T> il)
{
    assign(std::begin(il), std::end(il));
}

template <typename T>
directed_graph<T>& directed_graph<T>::operator=(
    std::initializer_list<T> il)
{
    // Use a copy-and-swap-like algorithm to guarantee strong exception safety.
    // Do all the work in a temporary instance.
    directed_graph new_graph { il };
    swap(new_graph); // Commit the work with only non-throwing operations.
    return *this;
}
\end{cpp}

有了这些,directed\_graph就可以使用统一的初始化构造:

\begin{cpp}
directed_graph<int> graph { 11, 22, 33 };
\end{cpp}

而非:

\begin{cpp}
directed_graph<int> graph;
graph.insert(11);
graph.insert(22);
graph.insert(33);
\end{cpp}

可以给图分配如下内容:

\begin{cpp}
graph = { 66, 77, 88 };
\end{cpp}

由于initializer\_list构造器和类模板参数推导(CTAD),可以在构造directed\_graph时删除元素类型,就像vector一样:

\begin{cpp}
directed_graph graph { 11, 22, 33 };
\end{cpp}

也可以添加一个接受迭代器范围元素的构造函数。这又是一个成员函数模板,类似于接受迭代器范围的assign(),简单地将工作转发给assign():

\begin{cpp}
template <typename T>
template <std::input_iterator Iter>
directed_graph<T>::directed_graph(Iter first, Iter last)
{
    assign(first, last);
}
\end{cpp}

C++23为大多数标准库容器添加了一个insert\_range()成员函数,也可以添加到directed\_graph中。这个实现使用了std::ranges::input\_range概念来约束Range模板类型参数。

\begin{cpp}
template <typename T>
template <std::ranges::input_range Range>
void directed_graph<T>::insert_range(Range&& range)
{
    insert(std::ranges::begin(range), std::ranges::end(range));
}
\end{cpp}

使用insert\_range(),可以向范围插入元素:

\begin{cpp}
vector moreNodes { 66, 77 };
graph.insert_range(moreNodes);
\end{cpp}

最后,可以添加一个erase()重载,接受一个节点值。类似于std::set,返回删除的节点数量,对于directed\_graph来说,这个数量总是0或1。

\begin{cpp}
template <typename T>
typename directed_graph<T>::size_type directed_graph<T>::erase(
    const T& node_value)
{
    const auto iter { find_node(node_value) };
    if (iter != std::end(m_nodes)) {
        remove_all_links_to(iter);
        m_nodes.erase(iter);
        return 1;
    }
    return 0;
}
\end{cpp}

\mySamllsection{进一步的改进}

可以对directed\_graph类模板进行一些改进。以下是几个示例:

\begin{itemize}
\item
当前实现没有检查图中的循环。可以添加这样的检查,以使图成为有向无环图。

\item
可以实现对随机访问迭代器的支持,而不是支持双向迭代器。

\item
标准库关联容器支持与节点相关的功能,参见第18章。directed\_graph类模板可以改进,包括一个node\_type类型别名和extract()成员函数。

\item
一个更复杂的改进可能是添加对自定义分配器的支持,就像所有标准库容器都有的一样。这需要使用std::allocator\_traits<A>提供的标准库功能,例如construct()、destroy()、propagate\_on\_container\_move\_assignment、propagate\_on\_container\_copy\_assignment、propagate\_on\_container\_swap等。
\end{itemize}

\mySamllsection{其他容器类型}

directed\_graph类模板基本上是一个顺序容器,但由于图的性质,确实实现了某些关联容器的功能,例如:insert()成员函数的返回类型。

还可以编写一个纯顺序容器、无序关联容器或有序关联容器,也需要遵循标准库规定的特定要求。列出这些要求在这里比较麻烦,但需要指出的是,deque容器几乎完全遵循规定的顺序容器要求,唯一的区别是它提供resize()成员函数(标准库不需要)。一个有序关联容器的示例是map,可以基于它来建模自己的有序关联容器,基于unordered\_map的话,就是一个无序关联容器了。


