
通过解决下面的练习,可以练习本章讨论的内容。所有练习的解决方案都可以在本书的网站\url{www.wiley.com/go/proc++6e}下载到源码。若在练习中卡住了,可以考虑先重读本章的部分内容,试着自己找到答案,再在从网站上寻找解决方案。

\begin{itemize}
\item
\textbf{练习2-1}:编写程序,要求用户输入两个字符串,然后使用三向比较操作符按字母顺序输出出来。要向用户请求字符串,可以使用std::cin流,在第1章中简要介绍过。第13章详细介绍了输入和输出,但这里是如何从控制台中读取字符串。要终止该输入,只需按Enter键。

\begin{cpp}
std::string s;
getline(cin, s1);
\end{cpp}

\item
\textbf{练习2-2}:编写程序,要求用户输入一个源字符串(= haystack)、一个要在源字符串中查找的字符串(= needle)和一个替换字符串。编写一个带有三个参数(haystack、needle和replacement string)的函数,返回一个haystack的副本,其中所有的needle都替换为目标字符串。只使用std::string,不使用string\_view。可以使用哪种参数类型,为什么?从main()调用这个函数,并输出出所有的字符串进行验证。

\item
\textbf{练习2-3}:修改练习2-2中的程序,并在尽可能多的地方使用std::string\_view。

\item
\textbf{练习2-4}:编写程序,要求用户输入未知数量的浮点数,并将所有数字存储在vector中,每个数字都应在输入后加一行。当用户输入数字0时,停止请求更多的数字。要从控制台中读取浮点数,使用cin的方法与第1章中使用cin输入整数值的方法相同。用一对格式化表中的数字,其中每列以不同的格式输出数字。表中的每一行对应一个输入的数字。

\item
\textbf{练习2-5}:编写程序,要求用户输入一个未知数量的单词。当用户输入*时停止输入。将所有单独的单词存储在一个vector中,可以用下列方法输入单个单词:

\begin{cpp}
std::string word;
cin >> word;
\end{cpp}

当输入完成后,计算最长单词的长度。最后,输出列中的所有单词,每行5个。列的宽度基于最长的单词。输出列中居中的单词,并用|字符分隔列。
\end{itemize}


