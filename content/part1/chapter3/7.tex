Many programmers begin a new project by pledging that this time they will do everything right. Any time a variable or parameter shouldn’t be changed, it’ll be marked const. All variables will have clear, concise, readable names. Every developer will put the left curly brace on the subsequent line and will adopt the standard text editor and its conventions for tabs and spaces.

For a number of reasons, it is difficult to sustain this level of stylistic consistency. In the case of const, sometimes programmers just aren’t educated about how to use it. You will eventually come across old code or a library function that isn’t const-savvy. For example, suppose you are writing a function accepting a const parameter, and you need to call a legacy function accepting a non-const parameter. If you cannot modify the legacy code to make it const aware, maybe because it’s a third-party library, and you are absolutely certain that the legacy function will not modify its non-const argument, then a good programmer will use const\_cast() (see Chapter 1) to temporarily suspend the const property of the parameter, but an inexperienced programmer will start to unwind the const property back from the calling function, once again ending up with a program that never uses const.

Other times, standardization of style comes up against programmers’ individual tastes and biases. Perhaps the culture of your team makes it impractical to enforce strict style guidelines. In such situations, you may have to decide which elements you really need to standardize (such as variable names and tabs) and which ones are safe to leave up to individuals (perhaps spacing and commenting style). You can even obtain or write scripts that will automatically correct style “bugs” or flag stylistic problems along with code errors. Some development environments, such as Microsoft Visual C++, support automatic formatting of code according to rules that you specify. This makes it trivial to write code that always follows the guidelines that have been configured.