许多开发者开始一个新项目时,都会承诺会保证一切正确。每当一个变量或参数不应该改变时,都会标记为 const。所有的变量都将有清晰、简洁、可读的名字。每个开发者都将左大括号放在下一行,并采用标准的文本编辑器和其对制表符和空格的约定。

出于多种原因,保持这种风格的一致性非常困难。对于 const,有时开发者并不了解如何使用,最终会遇到旧代码或一个不支持 const 的库函数。例如,假设正在编写一个接受 const 参数的函数,并且需要调用一个接受非 const 参数的遗留函数。若不能修改历史代码以使其支持 const,可能是因为它是第三方库,并且您确信遗留函数不会修改其非 const 参数,那么可以使用 const\_cast()(参见第 1 章)来暂时取消参数的 const 属性,但一个经验不足的开发者将从调用函数开始添加 const 属性,再次陷入一个从不使用 const 的程序。

其他情况下,风格的标准化与开发者的个人喜好和偏见发生了冲突。也许您的团队的氛围,使得强制执行严格的风格指南不切实际。这种情况下,可能需要决定哪些元素真正需要标准化(例如:变量名和制表符),哪些可以留给个人决定(可能是空格和注释风格)。甚至可以获得或编写脚本来自动纠正风格“错误”或标记风格问题,并与代码错误一起显示。一些开发环境,如 Microsoft Visual C++,支持根据指定的规则自动格式化代码。这使得编写始终遵循已配置的指南的代码变得轻而易举。