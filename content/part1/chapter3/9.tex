By solving the following exercises, you can practice the material discussed in this chapter. Solutions to all exercises are available with the code download on the book’s website at www.wiley.com/go/ proc++6e. However, if you are stuck on an exercise, first reread parts of this chapter to try to find an answer yourself before looking at the solution from the website.

Code comments and coding style are subjective. The following exercises do not have a single perfect answer. The solutions from the website provide one of many possible correct answers to the exercises.


\begin{itemize}
\item
Exercise 3-1: Chapter 1 discusses an example of an employee records system. That system has a database, and one of the member functions of the database is displayCurrent(). Here is the implementation of that member function with some comments:

\begin{cpp}
void Database::displayCurrent() const // The displayCurrent() member function
{
    for (const auto& employee : m_employees) { // For each employee...
        if (employee.isHired()) { // If the employee is hired
            employee.display(); // Then display that employee
        }
    }
}
\end{cpp}

Do you see anything wrong with these comments? Why? Can you come up with better comments?


\item
Exercise 3-2: The employee records system from Chapter 1 contains a Database class. The following is a snippet of that class with only three member functions. Add proper JavaDocstyle comments to this code snippet. Consult Chapter 1 to brush up on what exactly these member functions do.

\begin{cpp}
class Database
{
    public:
    Employee& addEmployee(const std::string& firstName,
    const std::string& lastName);
    Employee& getEmployee(int employeeNumber);
    Employee& getEmployee(const std::string& firstName,
    const std::string& lastName);
    // Remainder omitted...
};
\end{cpp}

\item
Exercise 3-3: The following class has a number of naming issues. Can you spot them all and propose better names?

\begin{cpp}
class xrayController
{
    public:
        // Gets the active X-ray current in μA.
        double getCurrent() const;
        // Sets the current of the X-rays to the given current in μA.
        void setIt(double Val);
        // Sets the current to 0 μA.
        void 0Current();
        // Gets the X-ray source type.
        const std::string& getSourceType() const;
        // Sets the X-ray source type.
        void setSourceType(std::string_view _Type);
    private:
        double d; // The X-ray current in μA.
        std::string m_src__type; // The type of the X-ray source.
};
\end{cpp}

\item
Exercise 3-4: Given the following code snippet, reformat the snippet three times: first put curly braces on their own lines, then indent the curly braces themselves, and finally remove the curly braces for single-statement code blocks. This exercise allows you to get a feeling of different formatting styles and what the impact is on code readability.

\begin{cpp}
Employee& Database::getEmployee(int employeeNumber)
{
    for (auto& employee : m_employees) {
        if (employee.getEmployeeNumber() == employeeNumber) {
            return employee;
        }
    }
    throw logic_error { "No employee found." };
}
\end{cpp}
\end{itemize}