
编写风格“好”的代码需要时间,可能不需要花费太多时间来编写一个快速而又粗糙的程序来解析XML文件。使用功能分解、足够的注释和干净的结构编写相同的程序会花费更多的时间。这真的值得吗?

\mySubsubsection{3.1.1.}{提前思考}

若一个菜鸟开发者要在一年后使用你的代码,你对自己的代码有多大信心?我的一个朋友,面对越来越混乱的web应用程序代码,鼓励其团队考虑一个假设的实习生,他将在一年后开始工作。若没有文档,只有诸多可怕的函数,这个可怜的实习生怎么能跟上代码库的开发进度呢?当编写代码时,想象一下将来有新的人,甚至自己需要对其进行维护。你还会记得它是怎么运作的吗?若没空帮忙怎么办?编写良好的代码避免了这些问题,还因为其易于阅读和理解。

\mySubsubsection{3.1.2.}{良好风格的要素}

很难列举出使代码“具有良好的风格”的特征。随着时间的推移,会发现自己喜欢的风格,并注意到其他人编写的代码中有用的技术。也许更重要的是,在遇到可怕的代码时,应该避免什么。然而,好的代码有几个共同的原则,这些原则将在本章中探讨:

\begin{itemize}
\item
文档

\item
分解

\item
命名

\item
语法的使用

\item
格式
\end{itemize}





