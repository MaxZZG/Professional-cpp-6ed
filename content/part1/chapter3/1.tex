
Writing code that is stylistically “good” takes time. You probably don’t need much time to whip together a quick-and-dirty program to parse an XML file. Writing the same program with functional decomposition, adequate comments, and a clean structure would take you more time. Is it really worth it?

\mySubsubsection{3.1.1.}{Thinking Ahead}

How confident would you be in your code if a new programmer had to work with it a year from now? A friend of mine, faced with a growing mess of web application code, encouraged his team to think about a hypothetical intern who would be starting in a year. How would this poor intern ever get up to speed on the code base if there were no documentation and scary multiple-page functions? When you’re writing code, imagine that somebody new or even you yourself will have to maintain it in the future. Will you even still remember how it works? What if you’re not available to help? Wellwritten code avoids these problems because it is easy to read and understand.

\mySubsubsection{3.1.2.}{Elements of Good Style}

It is difficult to enumerate the characteristics of code that make it “stylistically good.” Over time, you’ll find styles that you like and notice useful techniques in code that others wrote. Perhaps more important, you’ll encounter horrible code that teaches you what to avoid. However, good code shares several universal tenets that are explored in this chapter:

\begin{itemize}
\item
Documentation

\item
Decomposition

\item
Naming

\item
Use of the language

\item
Formatting
\end{itemize}





