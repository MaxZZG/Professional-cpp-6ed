
C++ 允许各种极其难以阅读的事情,来看看这段古怪的代码:

\begin{cpp}
i++ + ++i;
\end{cpp}

这段代码可读性很差,但更重要的是,其行为在 C++ 标准中是未定义的。问题在于 i++ 使用了 i 的值,但同时有一个副作用,即增加 i 的值。标准并没有说这个增加应该在什么时候进行,只是说副作用(增加)应该在序列点 ; 之后可见,而编译器可以在那个语句执行期间进行增加操作。

不可能知道哪个 i 的值将用于 ++i,使用不同的编译器和平台运行这段代码可能会有不同的结果。

像下面这样的表达式

\begin{cpp}
a[i] = ++i;
\end{cpp}

从 C++17 开始,这样的表达式定义良好, C++17 保证在计算赋值运算符右侧的所有操作之前完成计算左侧,所以首先 i 递增,然后用作 a[i] 的索引。即便如此,为了清晰起见,仍然不建议使用此类表达式。

考虑到 C++ 语言提供的所有功能,如何利用这些语言特性来实现良好的风格非常重要。

\mySubsubsection{3.5.1.}{使用常数}

糟糕的代码通常充满了“魔法数字”。某些函数中,代码可能会使用 2.71828、24、3600 等数值。为什么?这些值代表什么?具有数学背景的人可能清楚 2.71828 代表了 e 的近似值,但大多数人并不知道这一点。C++ 提供了常量,用以给那些不变的值赋予符号名称,例如 2.71828、24、3600 等。这里有一些例子:

\begin{cpp}
const double ApproximationForE { 2.71828182845904523536 };
const int HoursPerDay { 24 };
const int SecondsPerHour { 3'600 };
\end{cpp}

\begin{myNotic}{NOTE}
标准库包含了一组预定义的数学常量,这些常量都在<numbers>头文件中定义,位于 std::numbers 命名空间内。例如,std::numbers::e(自然对数的底数 e)、pi(圆周率)、sqrt2(2 的平方根)、phi(黄金分割比)等等。
\end{myNotic}

\mySubsubsection{3.5.2.}{使用引用,而非指针}

过去,C++ 开发者通常会先学习 C 语言。C 语言中,指针是唯一的按引用传递机制,但在C++中可以切换到使用引用。若先学习了 C 语言,可能会认为引用并没有真正为语言添加新功能,可能会认为只是为指针提供的功能引入了一种新的语法。

使用引用有几个优点。首先,引用比指针更安全,不直接处理内存地址,并且不能为 nullptr。其次,从风格上讲,引用比指针更清晰,使用与栈变量相同的语法;即不需要使用 \& 明确地取地址,也不需要使用 * 明确地解引用。也易于使用,应该不会和开发者代码风格相冲突。不过,一些开发者认为,若在函数调用中看到 \&,他们就知道调用的函数将修改对象,若没有 \&,则一定是按值传递。对于引用,他们不知道函数是否会更改对象,除非查看函数实现。这是一种错误的思维方式。传递指针并不意味着对象将进行修改,因为参数可能是 const T*。传递指针和引用都可能修改对象,也可能不修改,这取决于函数参数是 const T*、T*、const T\& 还是 T\&。无论如何都需要查看函数实现,以确定函数是否可能修改传入的对象。

引用的另一个好处是,明确了内存的所有权。若正在编写一个函数,另一个开发者传递给过来一个对象的引用,可以明确地读取和可能修改该对象,但没有办法来释放其内存。若传递的是指针,这可能会不太清楚。需要删除对象以清理内存吗?还是调用者会这样做?在现代 C++ 中,含义是明确的:任何原始指针都不具有所有权,处理所有权和所有权转移是使用智能指针完成的,这将在第 7 章介绍。

C++ 中,引用是一种更加直观和安全的传递复杂对象的方式,提供了一种直接的语法来表示对对象的引用,而不是对象的副本。引用的一个重要特性是在初始化后,不能重新绑定到另一个对象,这有助于减少错误并提高代码的清晰度。此外,引用提供了一种清晰的方式来指示函数是否会修改传入的参数,因为函数参数列表中的 const 引用表明不会修改传入参数,而非 const 引用则表示对象可能会进行修改。

\mySubsubsection{3.5.3.}{使用自定义异常}

C++可以忽略异常。语言的语法并没有强制处理异常,理论上可以使用传统机制编写容错的程序,比如返回特殊值(例如 -1、nullptr 等)或设置错误标志。当返回特殊值来表示错误时,可以使用第 1 章中的 [[nodiscard]] 属性来强制函数调用者对返回的值进行处理。

异常为错误处理提供了一个更丰富的机制,自定义异常允许根据需求定制这个机制。例如,网络浏览器的自定义异常类型可以包括指定包含错误的网页、错误发生时的网络状态,以及带有上下文信息的字段。

\begin{myNotic}{NOTE}
语言特性存在的目的是为了帮助开发者,请理解并利用那些有助于良好编程风格的语言特性。
\end{myNotic}












