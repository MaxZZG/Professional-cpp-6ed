By solving the following exercises, you can practice the material discussed in this chapter. Solutions to all exercises are available with the code download on the book’s website at www.wiley.com/go/ proc++6e. However, if you are stuck on an exercise, first reread parts of this chapter to try to find an answer yourself before looking at the solution from the website.

\begin{itemize}
\item
Exercise 1-1: Modify the Employee structure from the beginning of this chapter by putting it in a namespace called HR. What modifications do you have to make to the code in main() to work with this new implementation? Additionally, modify the code to use designated initializers.

\item
Exercise 1-2: Build further on the result of Exercise 1-1 and add an enumeration data member title to Employee to specify whether a certain employee is a Manager, Senior Engineer, or Engineer. Which kind of enumeration will you use and why? Whatever you need to add, add it to the HR namespace. Test your new Employee data member in the main() function. Use a switch statement to print out a human-readable string for the title.

\item
Exercise 1-3: Use an std::array to store three Employee instances from Exercise 1-2 with different data. Subsequently, use a range-based for loop to print out the employees in the array.

\item
Exercise 1-4: Do the same as Exercise 1-3, but use an std::vector instead of an array, and use push\_back() to insert elements into the vector.

\item
Exercise 1-5: Take your solution for Exercise 1-4 and replace the data members for the first and last initials with strings to represent the full first and last name.

\item
Exercise 1-6: Now that you know about const and references, and what they are used for, modify the AirlineTicket class from earlier in this chapter to use references wherever possible and to be const correct.

\item
Exercise 1-7: Modify the AirlineTicket class from Exercise 1-6 to include an optional frequent-flyer number. What is the best way to represent this optional data member? Add a setter and a getter to set and retrieve the frequent-flyer number. Modify the main() function to test your implementation.
\end{itemize}