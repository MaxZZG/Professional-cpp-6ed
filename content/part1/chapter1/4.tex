通过解决下面的练习,可以练习本章讨论的内容。所有练习的解决方案都可以在本书的网站\url{www.wiley.com/go/proc++6e}下载到源码。若在练习中卡住了,可以考虑先重读本章的部分内容,试着自己找到答案,再在从网站上寻找解决方案。

\begin{itemize}
\item
\textbf{练习1-1}:修改本章开头的Employee结构,将其放入名为HR的命名空间中。必须对main()中的代码进行哪些修改才能使用这个新实现?另外,修改代码,并使用指定的初始化。

\item
\textbf{练习1-2}:在练习1-1的结果基础上进一步构建,并向Employee添加枚举数据成员标题,以指定某个员工是Manager、Senior Engineer还是Engineer。使用哪一种枚举?为什么?无论需要添加什么,请将其添加到HR命名空间中。main()函数中测试新的Employee数据成员,使用switch语句为标题输出人类可读的字符串。

\item
\textbf{练习1-3}:使用一个std::array来存储练习1-2中具有不同数据的三个Employee实例,使用基于范围的for循环打印出数组中的雇员。

\item
\textbf{练习1-4}:执行与练习1-3相同的操作,但要使用std::vector,并使用push\_back()将元素插入vector。

\item
\textbf{练习1-5}:采用练习1-4的解决方案,将首字母和尾字母的数据成员替换为字符串,以表示完整的姓和名。

\item
\textbf{练习1-6}:现在已经了解了const和引用,以及其用途,修改本章前面的AirlineTicket类,使其尽可能使用引用,并且保持const的正确性。

\item
\textbf{练习1-7}:修改练习1-6中的AirlineTicket类,包括一个可选的常旅客号码。表示这个可选数据成员的最佳方法是什么?并添加setter和getter来设置和检索常旅客号,修改main()函数进行测试。
\end{itemize}


