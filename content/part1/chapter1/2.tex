
以下程序基于本章结构讨论中早些时候使用的员工数据库示例。这次,将得到一个完全功能的C++程序,该程序使用了本章讨论的许多特性。这个实际示例包括使用类、异常、流、vector、命名空间、引用,以及其他语言特性。

\mySubsubsection{1.2.1.}{员工档案系统}

接下来实现一个程序来管理公司的员工记录与以下功能:

\begin{itemize}
\item
增加和解雇员工

\item
晋升和降职员工

\item
查看所有员工,过去和现在

\item
查看所有当前员工

\item
查看所有前雇员
\end{itemize}

该程序的设计将代码分为三个部分。Employee类封装了描述单个员工的个人信息,Database类管理公司的所有员工。最后,一个单独的UserInterface文件提供了程序的交互性。

\mySubsubsection{1.2.2.}{Employee类}

Employee类维护雇员的所有信息,成员函数提供了查询和更改该信息的方法。Employee也知道如何在控制台上进行输出,还存在成员函数来调整员工的工资和雇佣状态。

\mySamllsection{Employee.cppm}

Employee.cppm模块接口文件定义了Employee类。接下来,文本将逐个描述此文件的部分。文件的前几行如下所示:

\begin{cpp}
export module employee;
import std;
namespace Records {
\end{cpp}

第一行是模块声明,声明该文件导出一个名为employee的模块,后面跟着对标准库功能的导入。此代码还声明了随花括号内的后续代码位于Records命名空间中,Records是本程序中用于应用程序特定代码的命名空间。

接下来,在Records命名空间内部定义了以下两个常量。本书不对常量添加任何特殊前缀,以大写字母起始,以便更好地与变量区分开。

\begin{cpp}
const int DefaultStartingSalary { 30'000 };
export const int DefaultRaiseAndDemeritAmount { 1'000 };
\end{cpp}

第一个常量代表新员工的默认起始工资。因为模块外的代码不需要访问它,所以这个常量没有导出。员工模块内的代码可以作为Records::DefaultStartingSalary来访问这个常量。

第二个常量是晋升或降职员工的默认金额。因为导出了这个常量,所以模块外的代码可以访问,例如,可以通过两倍的默认金额来晋升一个员工。

接下来,定义并导出了Employee类,及其公共成员函数:

\begin{cpp}
export class Employee
{
    public:
        Employee(const std::string& firstName,
                 const std::string& lastName);
        void promote(int raiseAmount = DefaultRaiseAndDemeritAmount);
        void demote(int demeritAmount = DefaultRaiseAndDemeritAmount);
        void hire(); // Hires or rehires the employee
        void fire(); // Dismisses the employee
        void display() const; // Prints employee info to console

        // Getters and setters
        void setFirstName(const std::string& firstName);
        const std::string& getFirstName() const;

        void setLastName(const std::string& lastName);
        const std::string& getLastName() const;

        void setEmployeeNumber(int employeeNumber);
        int getEmployeeNumber() const;

        void setSalary(int newSalary);
        int getSalary() const;

        bool isHired() const;
\end{cpp}

提供了一个构造函数,接受一个首字母和姓。promote()和demote()成员函数都有整数参数,其默认值等于DefaultRaiseAndDemeritAmount。这样,其他代码可以省略参数,默认值会自动使用。提供了一些成员函数来雇佣和解雇员工,以及一个显示员工信息的成员函数。一些setter和getter提供了修改员工信息或查询员工当前信息的功能。

数据成员声明为私有,其他代码不能直接进行修改:

\begin{cpp}
    private:
        std::string m_firstName;
        std::string m_lastName;
        int m_employeeNumber { -1 };
        int m_salary { DefaultStartingSalary };
        bool m_hired { false };
    };
}
\end{cpp}

setter和getter提供了修改或查询这些值的唯一方法。数据成员在类定义中直接初始化,而不是在构造函数中。默认情况下,新员工没有姓名,员工编号为-1,默认起薪,状态为未雇用。

\mySamllsection{Employee.cpp}

模块实现文件的前几行如下所示:

\begin{cpp}
module employee;
import std;
using namespace std;
\end{cpp}

第一行指定这个源文件属于哪个模块,后面是一个std的导入和一个using指令。

接受姓和名的构造函数只设置相应的数据成员:

\begin{cpp}
namespace Records {
    Employee::Employee(const string& firstName, const string& lastName)
    : m_firstName { firstName }, m_lastName { lastName }
    {
    }
\end{cpp}

promote()和demote()成员函数只是用一个新值调用setSalary()。这里不重复整型参数的默认值,只允许在函数声明中使用,而不允许在定义中使用。

\begin{cpp}
void Employee::promote(int raiseAmount)
{
    setSalary(getSalary() + raiseAmount);
}

void Employee::demote(int demeritAmount)
{
    setSalary(getSalary() - demeritAmount);
}
\end{cpp}

hire()和fire()成员函数只是设置了m\_hired的状态:

\begin{cpp}
void Employee::hire() { m_hired = true; }
void Employee::fire() { m_hired = false; }
\end{cpp}

display()成员函数使用println()来显示当前员工的信息。这段代码是Employee类的一部分,所以可以直接访问数据成员,比如m\_salary,而不是使用getter,比如getSalary()。然而,当getter和setter存在时,即使是在类中,使用它们是一种良好的编程风格。

\begin{cpp}
void Employee::display() const
{
    println("Employee: {}, {}", getLastName(), getFirstName());
    println("-------------------------");
    println("{}", (isHired() ? "Current Employee" : "Former Employee"));
    println("Employee Number: {}", getEmployeeNumber());
    println("Salary: ${}", getSalary());
    println("");
}
\end{cpp}

最后,一些getter和setter执行获取和设置值的任务:

\begin{cpp}
    // Getters and setters
    void Employee::setFirstName(const string& firstName) {m_firstName = firstName;}
    const string& Employee::getFirstName() const { return m_firstName; }

    void Employee::setLastName(const string& lastName) { m_lastName = lastName; }
    const string& Employee::getLastName() const { return m_lastName; }

    void Employee::setEmployeeNumber(int employeeNumber) {
        m_employeeNumber = employeeNumber; }
    int Employee::getEmployeeNumber() const { return m_employeeNumber; }

    void Employee::setSalary(int salary) { m_salary = salary; }
    int Employee::getSalary() const { return m_salary; }

    bool Employee::isHired() const { return m_hired; }
}
\end{cpp}

尽管这些成员函数看似微不足道,但拥有getter和setter总比将数据成员公开的好。例如,将来可以在setSalary()成员函数中执行边界检查。getter和setter可以将断点插入,会让调试更容易。另一个原因是,当决定更改在类中存储数据的方式时,只需要修改这些getter和setter即可。

\mySamllsection{EmployeeTest.cpp}

编写单个类时,需要对它们进行隔离测试。下面的代码包含一个main()函数,该函数使用Employee类执行一些简单的操作。当确定Employee类可以正常工作,就应该删除或注释掉这个文件,这样就不会试图用多个main()函数编译代码。

\begin{cpp}
import std;
import employee;

using namespace std;
using namespace Records;

int main()
{
    println("Testing the Employee class.");
    Employee emp { "Jane", "Doe" };
    emp.setFirstName("John");
    emp.setLastName("Doe");
    emp.setEmployeeNumber(71);
    emp.setSalary(50'000);
    emp.promote();
    emp.promote(50);
    emp.hire();
    emp.display();
}
\end{cpp}

另一种,并且更好的测试单个类的方法是单元测试。单元测试是用于测试特定功能的小代码块,其会保留在代码库中。所有的单元测试都会频繁执行;例如,由构建系统自动执行。这样做的好处是,若对现有功能进行了一些更改,单元测试会立即提示,是否破坏了某项功能。

\mySubsubsection{1.2.3.}{Database类}

接下来实现Database类,使用标准库中的std::vector类来存储Employee对象。

\mySamllsection{Database.cppm}

以下是数据库.cppm模块接口文件的前几行代码:

\begin{cpp}
export module database;
import std;
import employee;

namespace Records {
    const int FirstEmployeeNumber { 1'000 };
\end{cpp}

因为数据库将自动为新员工分配员工编号,所以定义了一个常量来指定编号的起始值。

接下来,定义并导出Database类:

\begin{cpp}
    export class Database
    {
        public:
            Employee& addEmployee(const std::string& firstName,
                                  const std::string& lastName);
            Employee& getEmployee(int employeeNumber);
            Employee& getEmployee(const std::string& firstName,
                                  const std::string& lastName);
\end{cpp}

数据库提供了一种简单的方法来添加新员工,通过提供首字母和姓,这个成员函数返回新员工的引用。外部代码还可以通过调用getEmployee()成员函数来获取员工的引用。这个成员函数声明了两个重载版本。一个允许按员工编号检索。另一个需要首字母和姓。

因为数据库是所有员工记录的中心存储库,所以有以下成员函数来显示所有员工、当前雇佣的员工和解雇的员工:

\begin{cpp}
            void displayAll() const;
            void displayCurrent() const;
            void displayFormer() const;
\end{cpp}

最后,私有数据成员定义如下:

\begin{cpp}
        private:
            std::vector<Employee> m_employees;
            int m_nextEmployeeNumber { FirstEmployeeNumber };
    };
}
\end{cpp}

m\_employees数据成员包含Employee对象,而m\_nextEmployeeNumber跟踪分配给新员工的员工编号,并使用FirstEmployeeNumber常量进行初始化。

\mySamllsection{Database.cpp}

下面是addEmployee()成员函数的实现:

\begin{cpp}
module database;
import std;

using namespace std;

namespace Records {
    Employee& Database::addEmployee(const string& firstName,
                                    const string& lastName)
    {
        Employee theEmployee { firstName, lastName };
        theEmployee.setEmployeeNumber(m_nextEmployeeNumber++);
        theEmployee.hire();
        m_employees.push_back(theEmployee);
        return m_employees.back();
    }
\end{cpp}

addEmployee()成员函数创建一个新的Employee对象,填充其信息,并将其添加到vector中。数据成员m\_nextEmployeeNumber在使用后递增,以便下一个员工将获得一个新号。vector的back()成员函数返回vector中最后一个元素的引用,即新添加的雇员。

getEmployee()成员函数之一实现如下。第二个重载的实现与此类似,因此没有展示。它们都使用基于范围的for循环遍历m\_employees中的所有员工,并检查Employee是否与传递给成员函数的信息相匹配。若没有找到匹配项,则抛出异常。注意,在基于范围的for循环中使用了auto\&,循环不想处理Employees的副本,而是要处理对m\_employees中的Employees的引用。

\begin{cpp}
    Employee& Database::getEmployee(int employeeNumber)
    {
        for (auto& employee : m_employees) {
            if (employee.getEmployeeNumber() == employeeNumber) {
                return employee;
            }
        }
        throw logic_error { "No employee found." };
    }
\end{cpp}

下面的display成员函数都使用类似的算法:循环遍历所有员工,并要求每个员工在显示条件匹配时将自己显示到终端。

\begin{cpp}
    avoid Database::displayAll() const
    {
        for (const auto& employee : m_employees) { employee.display(); }
    }

    void Database::displayCurrent() const
    {
        for (const auto& employee : m_employees) {
            if (employee.isHired()) { employee.display(); }
        }
    }

    void Database::displayFormer() const
    {
        for (const auto& employee : m_employees) {
            if (!employee.isHired()) { employee.display(); }
        }
    }
}
\end{cpp}

\mySamllsection{DatabaseTest.cpp}

下面是对数据库基本功能的一个简单测试:

\begin{cpp}
import std;
import database;

using namespace std;
using namespace Records;

int main()
{
    Database myDB;
    Employee& emp1 { myDB.addEmployee("Greg", "Wallis") };
    emp1.fire();

    Employee& emp2 { myDB.addEmployee("Marc", "White") };
    emp2.setSalary(100'000);

    Employee& emp3 { myDB.addEmployee("John", "Doe") };
    emp3.setSalary(10'000);
    emp3.promote();

    println("All employees:\n==============");
    myDB.displayAll();

    println("\nCurrent employees:\n==================");
    myDB.displayCurrent();

    println("\nFormer employees:\n=================");
    myDB.displayFormer();
}
\end{cpp}

\mySubsubsection{1.2.4.}{用户接口}

该程序的最后一部分是一个基于菜单的用户界面,使用户可以轻松地使用员工数据库。

下面的main()函数包含一个循环,该循环显示菜单,执行选定的操作,然后再次执行所有操作。对于大多数操作,都定义了单独的函数。对于更简单的操作,比如显示雇员,实际的代码会放在适当的位置。

\begin{cpp}
import std;
import database;
import employee;

using namespace std;
using namespace Records;

int displayMenu();
void doHire(Database& db);
void doFire(Database& db);
void doPromote(Database& db);

int main()
{
    Database employeeDB;
    bool done { false };
    while (!done) {
        int selection { displayMenu() };
        switch (selection) {
        case 0:
            done = true;
            break;
        case 1:
            doHire(employeeDB);
            break;
        case 2:
            doFire(employeeDB);
            break;
        case 3:
            doPromote(employeeDB);
            break;
        case 4:
            employeeDB.displayAll();
            break;
        case 5:
            employeeDB.displayCurrent();
            break;
        case 6:
            employeeDB.displayFormer();
            break;
        default:
            println(cerr, "Unknown command.");
            break;
        }
    }
}
\end{cpp}

displayMenu()函数打印菜单并从用户处获取输入。一个重要的注意事项是,这段代码假设用户会“表现得很好”,并在请求数字时输入一个数字。第13章将展示如何处理不良输入。

\begin{cpp}
int displayMenu()
{
    int selection;
    println("");
    println("Employee Database");
    println("-----------------");
    println("1) Hire a new employee");
    println("2) Fire an employee");
    println("3) Promote an employee");
    println("4) List all employees");
    println("5) List all current employees");
    println("6) List all former employees");
    println("0) Quit");
    println("");
    print("---> ");
    cin >> selection;
    return selection;
}
\end{cpp}

doHire()函数从用户那里获取新员工的姓名,并告诉数据库添加该员工:

\begin{cpp}
void doHire(Database& db)
{
    string firstName;
    string lastName;

    print("First name? ");
    cin >> firstName;

    print("Last name? ");
    cin >> lastName;

    auto& employee { db.addEmployee(firstName, lastName) };
    println("Hired employee {} {} with employee number {}.",
        firstName, lastName, employee.getEmployeeNumber());
}
\end{cpp}

doFire()和doPromote()都通过员工编号向数据库查询员工,然后使用employee对象的public成员函数进行更改:

\begin{cpp}
void doFire(Database& db)
{
    int employeeNumber;
    print("Employee number? ");
    cin >> employeeNumber;

    try {
        auto& emp { db.getEmployee(employeeNumber) };
        emp.fire();
        println("Employee {} terminated.", employeeNumber);
    } catch (const std::logic_error& exception) {
        println(cerr, "Unable to terminate employee: {}", exception.what());
    }
}

void doPromote(Database& db)
{
    int employeeNumber;
    print("Employee number? ");
    cin >> employeeNumber;
    int raiseAmount;
    print("How much of a raise? ");
    cin >> raiseAmount;

    try {
        auto& emp { db.getEmployee(employeeNumber) };
        emp.promote(raiseAmount);
    } catch (const std::logic_error& exception) {
        println(cerr, "Unable to promote employee: {}", exception.what());
    }
}
\end{cpp}

\mySubsubsection{1.2.5.}{评估项目}

程序涵盖了从简单到更复杂的一系列主题,有多种方式可以扩展这个程序。例如,用户界面并没有暴露出Database和Employee类的所有功能,可以修改UI以包括这些特性。还可以尝试实现这两个类的其他功能,这将是一个练习本章所学材料的好方法。

若对这个程序的某些部分感到困惑,请参阅本章的相关部分来复习这些主题。若仍然不清楚,学习最好的方式是编写代码并尝试各种操作。例如,不确定如何使用条件运算符,可以编写一个短小的main()函数来使用看看。




