\noindent
\textbf{WHAT’S IN THIS CHAPTER?}

\begin{itemize}
\item
An overview of C++ language features that are common, but for which you might have forgotten the syntax

\item
What RAII is and why it is a powerful concept

\item
What the double dispatch technique is and how to use it

\item
Various techniques to implement and use mixin classes

\item
What frameworks are

\item
The model-view-controller paradigm
\end{itemize}

\noindent
\textbf{WILEY.COM DOWNLOADS FOR THIS CHAPTER}

Please note that all the code examples for this chapter are available as part of this chapter’s code download on the book’s website at \url{www.wiley.com/go/proc++6e} on the Download Code tab.

One of the major themes of this book has been the adoption of reusable techniques and patterns. As a programmer, you tend to face similar problems repeatedly. With an arsenal of diverse approaches, you can save yourself time by applying the proper technique or pattern to a given problem.

This chapter is about design techniques, while the next chapter is about design patterns. Both represent standard approaches for solving particular problems; however, design techniques are specific to C++, whereas design patterns are less language-specific. Often, a design technique aims to overcome an annoying feature or language deficiency. Other times, a design technique is a piece of code that you use in many different programs to solve a common C++ problem.

Design techniques are C++ idioms that aren’t necessarily built-in parts of the language but are nonetheless frequently used. The first part of this chapter covers the language features in C++ that are common, but for which you might have forgotten the syntax. This material is a review, but it is a useful reference tool when the syntax escapes you. The topics covered include the following:

\begin{itemize}
\item
Starting a class from scratch

\item
Extending a class by deriving from it

\item
Writing a lambda expression

\item
Implementing the copy-and-swap idiom

\item
Throwing and catching exceptions

\item
Defining a class template

\item
Constraining class and function template parameters

\item
Writing to a file

\item
Reading from a file
\end{itemize}

The second part of this chapter focuses on higher-level techniques that build upon C++ language features. These techniques offer a better way to accomplish everyday programming tasks. Topics include the following:

\begin{itemize}
\item
Resource Acquisition Is Initialization (RAII)

\item
The double dispatch technique

\item
Mixin classes
\end{itemize}

The chapter concludes with an introduction to frameworks, a coding technique that greatly eases the development of large applications.
















