\noindent
\textbf{内筒概要}

\begin{itemize}
\item
C++语言常见特性的概述,但已经忘记了它们的语法

\item
什么是RAII,为什么它很强大

\item
什么是双重分派技术
\item
实现和使用混合(mixin)类

\item
什么是框架

\item
模型-视图-控制器范式
\end{itemize}

本章的所有代码示例都可以在\url{https://github.com/Professional-CPP/edition-6}获得。

这本书的一个主要主题是采用可重用技术和模式。作为一名开发者,往往会反复面临类似的问题。拥有一系列多样化的方法,可以通过将适当的技巧或模式应用于给定问题来节省时间。

本章讨论的是设计技术,而下一章讨论的是设计模式。两者都代表了解决特定问题的标准方法,但设计技术特定于C++,而设计模式则较少依赖于语言。通常,设计技术的目的是为了克服一个恼人的特性或语言缺陷。其他时候,设计技术是一段代码,可以在许多不同的程序中使用它,来解决常见的C++问题。

设计技术是C++的习惯用法,不一定是语言的内建部分,但经常使用。本章的第一部分涵盖了C++中常见的语言特性,但各位读者可能已经忘记了其语法。这些内容虽然是复习,但当忘记语法时,它们是一个有用的参考工具,包括:

\begin{itemize}
\item
从头开始编写一个类

\item
通过派生来扩展一个类

\item
编写Lambda表达式

\item
实现拷贝-交换习惯用法

\item
抛出和捕获异常

\item
定义一个类模板

\item
约束类和函数模板参数

\item
写入文件

\item
读取文件
\end{itemize}

本章的第二部分关注的是建立在C++语言特性之上的更高级技术。这些技术提供了一种更好的方法来完成日常编程任务,包括:

\begin{itemize}
\item
资源获取即初始化(RAII)

\item
双重分派技术

\item
混合类
\end{itemize}

本章最后介绍了框架,这是一种编码技术,可以大大简化大型应用程序的开发。
















