
当图形操作系统在20世纪80年代首次出现时,程序设计语言主要是面向过程的。当时,编写图形用户界面(GUI)应用程序通常涉及操作复杂的数据结构,并递给操作系统提供的函数。例如,要在窗口中绘制一个矩形,可能需要填充一个 Window 结构体,并将其传递给 drawRect() 函数。

随着面向对象编程(OOP)的流行,开发者开始寻找将OOP范式应用于GUI开发的方法。结果就是所谓的面向对象框架。框架是一组类,提供对某些底层功能的面向对象接口。框架通常指用于通用应用程序开发的大型类库,框架可以真正代表任何功能。如果编写一组类来为应用程序提供数据库功能,那些类可以认为是框架。

\mySubsubsection{32.3.1.}{使用框架}

框架定义的特征是提供技术和模式。框架通常需要一些学习才能开始使用,因为它们有自己的思维模型。开始使用大型应用程序框架之前,如微软基础类(MFC),需要了解它的世界观。

框架在抽象思想和实际实现上都有很大的不同。许多框架建立在遗留的过程式API之上,这可能影响它们的设计。其他框架则是从底层开始编写,以面向对象的设计为目标。一些框架可能会反对C++语言的某些方面。例如,一个框架可能会故意放弃多重继承的概念。

当开始使用新的框架时,首要任务是找出其核心。遵循哪些设计原则?开发者试图传达什么样的思维模型?广泛使用语言的哪些方面?这些都是至关重要的问题,尽管听起来像是在过程中会学到的东西。如果没有理解框架的设计、模型或语言特性,无法很快就进入框架之内。

了解框架的设计也能够扩展它。例如,如果框架缺少某个功能,比如打印支持,可以编写自己的打印类,使用与框架相同的模型。这样做可以在应用程序中保持一致的模型,并且有其他应用程序可重用的代码。

此外,框架可能会使用某些特定的数据类型。MFC框架使用 CString 数据类型来表示字符串,而不是使用标准库中的 std::string 类。这并不意味着必须将整个代码库中的数据类型更改为框架提供的数据类型,而可以在框架代码与代码库的其他部分之间的边界上转换数据类型。

\mySubsubsection{32.3.2.}{模型-视图-控制器范式}

如我之前所提到的,框架在面向对象设计方面有很大的不同。常见的范式是所谓的模型-视图-控制器(MVC),这个范式模拟了许多应用程序经常处理的数据集、对该数据的一个或多个视图,以及数据操作。

MVC中,一组数据称为模型。在赛车模拟器中,模型会跟踪各种统计数据,比如:汽车的当前速度和受损程度。实际上,模型通常以具有许多getter和setter类的形式出现。赛车模型的类定义可能如下所示:

\begin{cpp}
class RaceCar
{
    public:
        RaceCar();
        virtual ~RaceCar() = default;

        virtual double getSpeed() const;
        virtual void setSpeed(double speed);

        virtual double getDamageLevel() const;
        virtual void setDamageLevel(double damage);
    private:
        double m_speed { 0.0 };
        double m_damageLevel { 0.0 };
};
\end{cpp}

视图是数据的一种特定可视化方式。对于RaceCar,可能有两种视图。第一种是汽车的图形视图,第二种是显示随时间变化的损坏程度的图表。重要的是,这两种视图都在操作相同的数据——查看同一信息的不同方式。这是MVC范式的主要优点之一:通过将数据与其显示分离,可以使代码更组织化,并轻松创建其他视图。

MVC范式的最后一部分是控制器,控制器是响应某些事件而改变模型的代码部分。例如,当赛车模拟器中的驾驶员撞上混凝土障碍物时,控制器会指示模型增加汽车的损坏程度,并减少其速度。控制器还可以操纵视图,当用户在用户界面上滚动滚动条时,控制器会指示视图滚动其内容。

MVC的三个组成部分在一个反馈循环中相互作用。动作由控制器处理,调整模型和/或视图。如果模型发生变化,会通知视图进行更新。图32.4展示了这种交互。

\myGraphic{0.7}{content/part5/chapter32/images/4.png}{图 32.4}











