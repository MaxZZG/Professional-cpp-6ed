
当您阅读这段话时,世界各地的成千上万的C++程序员正在解决已经解决的问题。圣何塞的一个小隔间里有人,正在从头开始编写一个使用引用计数的智能指针实现。地中海岛屿上的一位年轻程序员,正在设计一个可以从混合类中获得好处的类层次结构。

作为一名专业的C++开发者,应该少花时间重新发明轮子,而多花时间以新的方式适应可重用的概念。本节给出了一些可以直接应用于自己的程序,或根据需要定制的通用方法示例。

\mySubsubsection{32.2.1.}{资源获取即初始化}

资源获取即初始化(RAII)是一个简单但强大的概念,用于获取某些资源的所有权,并在RAII实例离开作用域时,自动释放这些获取的资源,初始化和销毁都在确定的时间点发生。基本上,新RAII实例的构造函数获取某个资源的所有权,并使用该资源初始化实例,这就是为什么称为资源获取即初始化。当RAII实例销毁时,析构函数自动释放获取的资源。

下面是一个 File RAII 类的示例,安全地包装了C风格文件句柄(std::FILE),并在RAII实例离开作用域时自动关闭文件。RAII类还提供了 get()、release() 和 reset() 成员函数,这些函数的行为与标准库中某些类(如 std::unique\_ptr)上的同名成员函数相似。RAII类通常不允许复制构造和复制赋值,所以这个实现删除了这些成员。

\begin{cpp}
import std;

class File final
{
    public:
        explicit File(std::FILE* file) : m_file { file } { }
        ~File() { reset(); }

        // Prevent copy construction and copy assignment.
        File(const File& src) = delete;
        File& operator=(const File& rhs) = delete;

        // Allow move construction.
        File(File&& src) noexcept : m_file { std::exchange(src.m_file, nullptr) }
        {
        }

        // Allow move assignment.
        File& operator=(File&& rhs) noexcept
        {
            if (this != &rhs) {
                reset();
                m_file = std::exchange(rhs.m_file, nullptr);
            }
            return *this;
        }

        // get(), release(), and reset()
        std::FILE* get() const noexcept { return m_file; }
        [[nodiscard]] std::FILE* release() noexcept
        {
            return std::exchange(m_file, nullptr);
        }

        void reset(std::FILE* file = nullptr) noexcept
        {
            if (m_file) { std::fclose(m_file); }
            m_file = file;
        }

    private:
        std::FILE* m_file { nullptr };
};
\end{cpp}

可以按以下方式使用:

\begin{cpp}
File myFile { std::fopen("input.txt", "r") };
\end{cpp}

当myFile实例超出作用域,其析构函数就会调用,文件会自动关闭。

使用RAII类时,有一个重要的陷阱。可能会意外地编写一个语句,以为它在某个作用域内正确创建了RAII实例,但实际上它创建了一个临时对象,该语句执行完毕后立即销毁。例如,以下语句正确使用了File RAII类:

\begin{cpp}
File myFile { std::fopen("input.txt", "r") };
\end{cpp}

然而,可能会不小心忘记为RAII实例命名,如下所示:

\begin{cpp}
File { std::fopen("input.txt", "r") };
\end{cpp}

这个语句创建了一个临时的File实例,该实例在语句结束时立即销毁。编译器不会对此发出任何警告或错误。为了避免这种情况,应该使用[[nodiscard]]属性标记RAII类的构造函数。例如:

\begin{cpp}
[[nodiscard]] explicit File(std::FILE* file) : m_file{ file } { }
\end{cpp}

这样,创建一个没有命名的File实例会触发编译器警告,如下所示:

\begin{shell}
warning C4834: discarding return value of function with 'nodiscard' attribute
\end{shell}

当然,使用File RAII类时,可能永远不会忘记为它命名,很可能会想对打开的文件进行操作。但有时需要在一个特定的作用域内创建一个RAII类的实例,而不需要直接与该创建的实例进一步交互(例如,互斥锁)。来看一个使用标准库中的RAII类std::unique\_lock的示例(参见第27章),以下代码段展示了unique\_lock的正确使用。我最初没有使用统一初始化语法来初始化unique\_lock,但稍后的讨论中会提到这一点。

\begin{cpp}
class Foo
{
    public:
        void setData()
        {
            unique_lock<mutex> lock(m_mutex);
            // ...
        }
    private:
        mutex m_mutex;
};
\end{cpp}

setData()成员函数使用unique\_lock RAII类来构造一个局部锁对象,该对象锁定m\_mutex数据成员,并在函数结束时自动解锁该互斥量。

但因为定义锁变量后不直接使用,所以很容易犯以下错误:

\begin{cpp}
unique_lock<mutex>(m_mutex);
\end{cpp}

这段代码中,忘记了给unique\_lock一个名字。这可以编译,但它不会做想让它做的事!它实际上声明了一个名为m\_mutex的局部变量(隐藏了m\_mutex数据成员),并用调用unique\_lock的默认构造函数来初始化。结果是m\_mutex数据成员没有锁定!如果编译器的警告级别设置得足够高,会给出警告:

\begin{shell}
warning C4458: declaration of 'm_mutex' hides class member
\end{shell}

如果使用以下统一初始化语法,编译器不会生成“隐藏类成员”的警告,但也不会做你想做的事。以下创建了一个临时的m\_mutex锁,但由于它是临时的,这个锁在这个语句结束时立即释放。

\begin{cpp}
unique_lock<mutex> { m_mutex };
\end{cpp}

最近,一些编译器实际上标记了unique\_lock构造函数为[[nodiscard]],就像我推荐给File RAII示例一样。这样的编译器之一是Visual C++ 2022。如果使用这样的编译器,前面的语句会生成一个警告,指出构造函数的返回值丢弃了。

此外,可能会在作为参数传递的名称中犯一个打字错误,例如:

\begin{cpp}
unique_lock<mutex>(m);
\end{cpp}

在这里,忘记为锁提供一个名字,并且在参数的名称中犯了一个输入错误。这段代码只是声明了一个名为m的局部变量,并用unique\_lock的默认构造函数初始化它。编译器甚至不会生成警告,可能会生成一个m是未引用局部变量的警告。不过,如果使用以下统一初始化语法,编译器会发出错误,抱怨未声明的标识符m:

\begin{cpp}
unique_lock<mutex> { m };
\end{cpp}

\begin{myWarning}{WARNING}
确保为RAII实例命名!此外,我建议不要在RAII类中包含默认构造函数。这样可以避免这里讨论的问题。
\end{myWarning}

\mySubsubsection{32.2.2.}{双重分派}

双重分派是一种技术,为多态性的概念增加了一个维度。正如第5章中所述,多态性允许程序基于运行时类型确定行为。例如,可能有一个带有 move() 成员函数的 Animal 类。所有动物都会移动,但在移动方式上有所不同。move() 成员函数为 Animal 的每个派生类定义,以便在运行时为适当类型的动物调用适当的成员函数,而不需要在编译时知道动物的类型。第10章解释了,如何使用虚成员函数来实现这种运行时多态。

然而,有时需要一个成员函数根据两个对象的运行时类型来表现,而不仅仅是其中一个。例如,想为 Animal 类添加一个成员函数,如果动物吃掉另一个动物则返回 true,否则返回 false。这个决定基于两个因素:吃掉的动物的类型和被吃的动物的类型。不幸的是,C++没有提供语言机制来根据多于一个对象的运行时类型来选择行为。虚成员函数本身不足以模拟这种情况,因为它们仅根据接收对象的运行时类型,来确定成员函数或行为。

一些面向对象的语言提供了在运行时,基于两个或更多对象的运行时类型,来选择成员函数的能力,这个特性称为多方法。在C++中没有核心语言特性来支持多方法,但可以使用双重分派技术,其提供了一种使函数对多于一个对象虚拟化的方法。

\begin{myNotic}{NOTE}
双重分派实际上是多分派的一个特例,其中选择行为取决于两个或更多对象的运行时类型。实践中,双重分派通常足够使用,基于两个对象的运行时类型来选择行为。
\end{myNotic}

\mySamllsection{尝试 \#1: 简单暴力法}

实现一个其行为取决于两个不同对象运行时类型成员函数的最直接方法是,采取其中一个对象的视角,并使用一系列 if/else 构造来检查另一个对象的类型。例如,可以在每个从 Animal 派生的类中实现一个名为 eats() 的成员函数,该函数接受另一个动物作为参数。在基类中,该成员函数被声明为纯虚函数,如下所示:

\begin{cpp}
class Animal
{
    public:
        virtual bool eats(const Animal& prey) const = 0;
};
\end{cpp}

每个派生类实现 eats() 成员函数,并根据参数的类型返回适当的值。下面是几个派生类的 eats() 实现示例。请注意,TRex 避免使用任何 if 语句,因为——根据作者的说法——像食肉恐龙一样,TRex 吃任何东西。

\begin{cpp}
bool Bear::eats(const Animal& prey) const
{
    if (typeid(prey) == typeid(Fish)) { return true; }
    return false;
}

bool Fish::eats(const Animal& prey) const
{
    if (typeid(prey) == typeid(Fish)) { return true; }
    return false;
}

bool TRex::eats(const Animal& prey) const
{
    return true;
}
\end{cpp}

这种简单暴力法有效,对于少量的类来说可能是最直接的技术。然而,有几个原因可能要避免使用这种方法。

\begin{itemize}
\item
面向对象编程(OOP)的纯粹主义者通常不赞成显式查询对象的类型,因为这暗示了一个缺乏适当面向对象结构的设计。

\item
随着类型数量的增长,这样的代码可能会变得混乱和重复。

\item
这种方法并没有强制派生类考虑新的类型。例如,添加了一个 Donkey,Bear 类将继续编译,但当告知吃一个 Donkey 时会返回 false,尽管众所周知熊吃驴。一只熊会拒绝吃一头驴,因为没有显式检查 Donkey 的 else if 语句。
\end{itemize}

\mySamllsection{尝试 \#2: 使用重载的单多态性}

可能会尝试使用带有重载的多态性,来规避所有的级联if/else构造。为什么不将每个类的一个eats()成员函数改为每个Animal的派生类重载成员函数呢?基类定义将如下所示:

\begin{cpp}
class Animal
{
    public:
        virtual bool eats(const Bear&) const = 0;
        virtual bool eats(const Fish&) const = 0;
        virtual bool eats(const TRex&) const = 0;
};
\end{cpp}

由于基类中的成员函数是纯虚的,每个派生类都需要实现针对每种其他类型Animal的行为。例如,Bear类包含以下成员函数:

\begin{cpp}
class Bear : public Animal
{
    public:
        bool eats(const Bear&) const override { return false; }
        bool eats(const Fish&) const override { return true; }
        bool eats(const TRex&) const override { return false; }
};
\end{cpp}

这种方法最初看起来是可行的,但实际上只解决了半个问题。为了在Animal上调用适当的eats()成员函数,编译器需要知道被吃掉的动物的编译时类型。以下调用将是成功的,因为吃和被吃的动物的编译时类型都是已知的:

\begin{cpp}
Bear myBear;
Fish myFish;
println("Bear eats fish? {}", myBear.eats(myFish));
\end{cpp}

缺失的部分是解决方案只在一个方向上具有多态性。可以通过Animal引用访问myBear,并且将调用正确的成员函数:

\begin{cpp}
Animal& animalRef { myBear };
println("Bear eats fish? {}", animalRef.eats(myFish));
\end{cpp}

反之则不然。如果将Animal引用传递给eats()成员函数,将得到编译错误,没有接受Animal的eats()成员函数。编译器无法在编译时确定调用哪个版本。以下示例无法编译:

\begin{cpp}
Animal& animalRef { myFish };
println("Bear eats fish? {}",
    myBear.eats(animalRef)); // BUG! No member function Bear::eats(Animal&)
\end{cpp}

因为编译器需要在编译时知道,将要调用eats()成员函数的哪个重载版本,所以这个解决方案并不是真正的多态性。例如,正在遍历一个Animal引用数组并将每个引用传递给eats(),代码将无法工作。

\mySamllsection{尝试 \#3: 双重分派}

双重分派技术是解决多类型问题的真正多态解决方案。在C++中,多态性是通过派生类中覆盖成员函数来实现的。运行时,根据对象的实际类型调用成员函数。前面的单多态尝试没有成功,因为它试图使用多态性来确定调用哪个重载版本的成员函数,而不是用来确定调用哪个类的成员函数。

首先,关注一个派生类,也许是 Bear 类。该类需要一个具有以下声明的成员函数:

\begin{cpp}
bool eats(const Animal& prey) const override;
\end{cpp}

双重分派的关键是基于参数上的成员函数调用来确定结果。假设 Animal 类有一个名为 eatenBy() 的成员函数,接受一个 Animal 引用作为参数。如果当前的 Animal 传递进来的那个类吃掉,这个成员函数返回 true。有了这样的成员函数,eats() 的定义就变得简单了:

\begin{cpp}
bool Bear::eats(const Animal& prey) const
{
    return prey.eatenBy(*this);
}
\end{cpp}

乍一看,这个解决方案似乎给单多态成员函数增加了一个成员函数调用层。毕竟,每个派生类必须为 Animal 每个派生类实现一个eatenBy(),但有一个关键的区别。多态性发生了两次!当在 Animal 上调用 eats() 成员函数时,多态性确定是在调用 Bear::eats()、Fish::eats(),还是其他的。当调用 eatenBy() 时,多态性再次确定调用哪个类的版本的成员函数,调用的是 prey 对象的运行时类型的 eatenBy()。注意,*this 的运行时类型始终与编译时类型相同,这样编译器就可以为参数(本例中为 Bear)调用正确的重载版本的 eatenBy()。

下面是使用双重分派技术的 Animal 层次结构的类定义。前向类声明是必要的,因为基类使用了派生类的引用。每个从 Animal 派生的类都以相同的方式实现 eats() 成员函数,但它不能提升到基类中。原因是如果尝试这样做,编译器将不知道调用哪个重载版本的 eatenBy() 成员函数,因为 *this 将是一个 Animal,而不是特定的派生类。成员函数重载解析是根据对象的编译时类型确定的,而不是运行时类型。

\begin{cpp}
// Forward declarations.
class Fish;
class Bear;
class TRex;

class Animal
{
    public:
        virtual bool eats(const Animal& prey) const = 0;

        virtual bool eatenBy(const Bear&) const = 0;
        virtual bool eatenBy(const Fish&) const = 0;
        virtual bool eatenBy(const TRex&) const = 0;
};
class Bear : public Animal
{
    public:
        bool eats(const Animal& prey) const override{ return prey.eatenBy(*this); }

        bool eatenBy(const Bear&) const override { return false; }
        bool eatenBy(const Fish&) const override { return false; }
        bool eatenBy(const TRex&) const override { return true; }
};

class Fish : public Animal
{
    public:
        bool eats(const Animal& prey) const override{ return prey.eatenBy(*this); }

        bool eatenBy(const Bear&) const override { return true; }
        bool eatenBy(const Fish&) const override { return true; }
        bool eatenBy(const TRex&) const override { return true; }
};

class TRex : public Animal
{
    public:
        bool eats(const Animal& prey) const override{ return prey.eatenBy(*this); }

        bool eatenBy(const Bear&) const override { return false; }
        bool eatenBy(const Fish&) const override { return false; }
        bool eatenBy(const TRex&) const override { return true; }
};
\end{cpp}

双重分派是一个需要时间来适应的概念。我建议仔细研究这段代码,以便熟悉这个概念及其实现。

\mySubsubsection{32.2.3.}{混合类}

第5章和第6章介绍了混合类,回答了“这个类还能做什么?”的问题,答案往往以“-able”结尾。例如,Clickable、Drawable、Printable、Lovable等。混合类是一种在不承诺完全的is-a关系的情况下向类添加功能的方法。在C++中实现混合类有几种技术。本节将探讨以下内容:

\begin{itemize}
\item
使用多重继承

\item
使用类模板

\item
使用CRTP

\item
使用CRTP和“推断this”
\end{itemize}

\mySamllsection{使用多重继承}

本节探讨如何使用多重继承技术设计、实现和使用混合类。

\mySamllsection{设计混合类s}

混合类包含可以重用其他类的代码,单个混合类实现了一个定义良好的功能。例如,可能有一个名为Playable的混合类,用于某些类型的媒体对象。混合类可能包含与计算机声音驱动程序通信的大部分代码。通过混合该类,媒体对象将免费获得该功能。

设计混合类时,需要考虑添加的行为是否属于对象层次结构或单独类。以之前的例子为例,如果所有媒体类都可播放,基类应该从Playable派生,而不是将Playable类混合所有派生类。如果只有某些媒体类可播放,并且它们分布在层次结构中,那么使用混合类有意义。

混合类特别有用的情况之一是,当类在一个轴上组织成层次结构,而在另一个轴上它们也具有相似性时。例如,在一个网格上玩的一场战争模拟游戏。每个网格位置可以包含一个具有攻击和防御能力以及其他特征的Item。某些项目,如城堡是固定的。其他项目,如骑士或浮动城堡,可以在网格上移动。在最初设计对象层次结构时,可能会得到类似于图32.1的结构,根据攻击和防御能力来组织类。

\myGraphic{0.8}{content/part5/chapter32/images/1.png}{图 32.1}

图32.1中的层次结构忽略了某些类包含的运动功能。围绕运动构建层次结构会导致类似于图32.2的结构。

\myGraphic{0.8}{content/part5/chapter32/images/2.png}{图 32.2}

当然,图32.2的设计会丢弃图32.1的所有组织。一个好的面向对象的开发者应该怎么做呢?

假设选择第一个层次结构,围绕攻击者和防御者组织,需要一种方法将运动纳入方程。一种可能性是,尽管只有部分派生类支持运动,可以在Item基类中添加一个move()成员函数。默认实现将不做任何事情,但某些派生类将重写move()以实际改变在网格上的位置。

另一种方法是编写一个Movable混合类。图32.1中的优雅层次结构可以得到保留,但层次结构中的某些类,除了父类之外还从Movable派生。图32.3展示了这种设计。

\myGraphic{0.8}{content/part5/chapter32/images/3.png}{图 32.3}

还有一种方法是将层次结构扁平化,根本不使用运行时多态,可以在需要以多态方式处理类型子集的情况下,使用静态多态性和/或类型擦除。这些方法在本文中不再进一步讨论。

\mySamllsection{实现混合类}

编写一个混合类与编写一个普通类没有区别。实际上,会更简单。使用之前的战争模拟,Movable 混合类可能如下所示:

\begin{cpp}
class Movable
{
    public:
        virtual void move() { /* Implementation to move an item… */ }
};
\end{cpp}

这个 Movable 混合类实现了在网格上移动项目的实际代码,还为可以移动的项目提供了一个类型。可以创建一个包含所有可移动项目的数组,而无需知道或关心它们属于 Item 的哪个派生类。

\mySamllsection{使用混合类}

使用混合类的代码在语法上等同于多重继承。除了从主层次结构中派生出父类外,还从混合类派生:

\begin{cpp}
class FloatingCastle : public Castle, public Movable { /* … */ };
\end{cpp}

这将在 FloatingCastle 类中混合了 Movable 混合类提供的功能。现在有一个类,它存在于层次结构中合理的位置,但仍然与层次结构中其他位置的对象共享共性。

\mySamllsection{使用类模板}

C++ 中实现混合类的第二种方法是使混合类本身,成为一个接受模板类型参数的类模板,然后从该类型派生。

第 6 章解释了实现一个 SelfDrivable 混合类模板的机制,该混合类可以用于创建自驱动的汽车和卡车。现在已经精通类模板(第 12 章和第 26 章),第 6 章中的 SelfDrivable 混合示例应该很熟悉了。混合类定义如下:

\begin{cpp}
template <typename T>
class SelfDrivable : public T { /* … */ };
\end{cpp}

如果有 Car 和 Truck 类,可以轻松地定义一个自驱动的汽车和卡车如下:

\begin{cpp}
SelfDrivable<Car> selfDrivingCar;
SelfDrivable<Truck> selfDrivingTruck;
\end{cpp}

这样,功能可以添加到现有的类,Car 和 Truck,而无需修改这些类。

这是一个完整的示例:

\begin{cpp}
template <typename T>
class SelfDrivable : public T
{
    public:
        void drive() { this->setSpeed(1.2); }
};

class Car
{
    public:
        void setSpeed(double speed) { println("Car speed set to {}.", speed); }
};

class Truck
{
    public:
        void setSpeed(double speed) { println("Truck speed set to {}.", speed); }
};

int main()
{
    SelfDrivable<Car> car;
    SelfDrivable<Truck> truck;
    car.drive();
    truck.drive();
}
\end{cpp}


\mySamllsection{使用CRTP}

在C++中实现混合类的另一种技术是使用奇异重复模板模式(CRTP)。

混合类本身再次是一个类模板,但这次它接受一个模板类型参数,表示派生类的类型,并不从其他类继承。实现中需要使用static\_cast()将this转换为派生类的类型。

\begin{cpp}
template <typename Derived>
class SelfDrivable
{
    public:
        void drive()
        {
            auto& self { static_cast<Derived&>(*this) };
            self.setSpeed(1.2);
        }
};
\end{cpp}

具体的类,如Car和Truck,然后从SelfDrivable继承,并将它们的类型作为SelfDrivable的模板类型参数传递。

\begin{cpp}
class Car : public SelfDrivable<Car>
{
    public:
        void setSpeed(double speed) { println("Car speed set to {}.", speed); }
};
class Truck : public SelfDrivable<Truck>
{
    public:
        void setSpeed(double speed) { println("Truck speed set to {}.", speed); }
};
\end{cpp}

这些可以按以下方式使用:

\begin{shell}
Car car;
Truck truck;
car.drive();
truck.drive();
\end{shell}

\CXXTwentythreeLogo{-40}{-50}

\mySamllsection{使用CRTP和“推断this”}

前面的CRTP示例中,SelfDrivable::drive()的实现需要一个static\_cast()来获取对正确派生类型的访问权限。由于C++23的“推断this”特性,SelfDrivable类可以更优雅地实现。如下所示,使用一个显式的对象参数。这个实现中,SelfDrivable混合类不再是类模板,但SelfDrivable::drive()现在是成员函数模板。标记为this的参数称为显式对象参数(参见第8章)。

\begin{cpp}
class SelfDrivable
{
    public:
        void drive(this auto& self) { self.setSpeed(1.2); }
};
\end{cpp}

Car和Truck然后简单地从SelfDrivable继承:

\begin{cpp}
class Car : public SelfDrivable { /* Same as before */ };
class Truck : public SelfDrivable { /* Same as before */ };
\end{cpp}

模型-视图-控制器(MVC)模式在许多流行框架中获得了广泛的支持。即使是非传统的应用程序,如网络应用程序,也正在朝着MVC的方向发展,它强制实施了数据、数据操作和数据显示之间的清晰分离。

MVC模式已经演变成几种不同的版本,例如:模型-视图-呈现者(MVP)、模型-视图-适配器(MVA)、模型-视图-视图模型(MVVM)等。





