
第1章中说了,C++标准规范超过2000页,该标准定义了许多关键字和大量的语言特性。要记住这一切是不可能的,即使是C++专家有时也需要查阅资料。考虑到这一点,本节展示了所有C++程序中都会用到的编码技术示例。当记得概念但忘记了语法时,可以参考以下各节来复习。

\mySubsubsection{32.1.1.}{编写类}

不记得如何开始?没问题——这里是一个在模块接口文件中定义的Simple类:

\begin{cpp}
export module simple;

// A simple class that illustrates class definition syntax.
export class Simple
{
    public:
        Simple(); // Constructor
        virtual ~Simple() = default; // Defaulted virtual destructor

        // Disallow copy construction and copy assignment.
        Simple(const Simple& src) = delete;
        Simple& operator=(const Simple& rhs) = delete;

        // Explicitly default move constructor and move assignment operator.
        Simple(Simple&& src) = default;
        Simple& operator=(Simple&& rhs) = default;

        virtual void publicMemberFunction(); // Public member function
        int m_publicInteger; // Public data member

    protected:
        virtual void protectedMemberFunction();// Protected member function
        int m_protectedInteger { 41 }; // Protected data member

    private:
        virtual void privateMemberFunction(); // Private member function
        int m_privateInteger { 42 }; // Private data member
        static constexpr int Constant { 2 }; // Private constant
        static inline int ms_staticInt { 3 }; // Private static data member
};
\end{cpp}

\begin{myNotic}{NOTE}
这个类定义展示了一些可能,但并不推荐的做法。在自己的类定义中,应避免使用public或protected的数据成员。一个类应该封装其数据;因此,应该将数据成员设为private,并提供public或protected的getter和setter成员函数。
\end{myNotic}

正如第10章中解释的,如果类意图作为其他类的基类,必须将析构函数设为虚。非基类的析构函数为非虚,但这样一来,我建议将类标记为final,这样其他类就不能从它派生了。如果只想将析构函数设为虚,但不需要在析构函数内部有任何实现,那么可以像Simple类示例中那样显式地将其默认化。

这个例子还展示了可以显式地删除或默认特殊成员函数。删除复制构造函数和复制赋值操作符,以避免无意的复制,而显式地默认化移动构造函数和移动赋值操作符。

接下来,这是模块的实现文件:

\begin{cpp}
module simple;

Simple::Simple() : m_publicInteger { 40 }

{
    // Implementation of constructor
}

void Simple::publicMemberFunction() { /* Implementation */ }
void Simple::protectedMemberFunction() { /* Implementation */ }
void Simple::privateMemberFunction() { /* Implementation */ }
\end{cpp}

\begin{myNotic}{NOTE}
类成员函数的定义也可以直接出现在模块接口文件中,如下一部分所示。不必将类分割成模块接口文件和模块实现文件。
\end{myNotic}

第8章和第9章提供了编写自定义类的所有细节。

\mySubsubsection{32.1.2.}{从现有类派生}

要从现有类派生,需要声明一个扩展了另一个类的新类。这里是一个名为DerivedSimple的类的定义,从Simple派生而来,并在derived\_simple模块中定义:

\begin{cpp}
export module derived_simple;

export import simple;

// A class derived from the Simple class.
export class DerivedSimple : public Simple
{
    public:
        DerivedSimple() : Simple{} // Constructor
        { /* Implementation of constructor */ }

        void publicMemberFunction() override // Overridden member function
        {
            // Implementation of overridden member function
            Simple::publicMemberFunction(); // Access the base class implementation
        }

        virtual void anotherMemberFunction() // New member function
        { /* Implementation of new member function */ }
};
\end{cpp}

在第10章中查找有关继承技术的详细信息。

\mySubsubsection{32.1.3.}{编写Lambda表达式}

Lambda表达式允许您编写小型匿名内联函数,与C++标准库算法结合使用时尤其强大。以下代码片段展示了一个示例,使用count\_if()算法和一个Lambda表达式来计算vector中偶数值的数量。此外,Lambda表达式通过从其封闭作用域,捕获callCount变量来跟踪其调用的次数。

\begin{cpp}
vector values { 1, 2, 3, 4, 5, 6, 7, 8, 9 };
int callCount { 0 };
auto evenCount { ranges::count_if(values,
    [&callCount](int value) {
        ++callCount;
        return value % 2 == 0;
    })
};
println("There are {} even elements in the vector.", evenCount);
println("Lambda was called {} times.", callCount);
\end{cpp}

第19章详细讨论了Lambda表达式。

\mySubsubsection{32.1.4.}{使用复制-交换习语}

复制-交换习语在第9章中详细讨论,是一种在对象上实现可能抛出异常的操作的习语,具有强烈的异常安全保证,即要么全部成功,要么全部失败。只需创建对象的一个副本,修改该副本(可能是一个复杂的算法,可能会抛出异常)。最后,在没有抛出异常的情况下,使用非抛出swap()将副本与原始对象交换。赋值运算符就是可以使用复制-交换习语的操作的一个例子。赋值运算符首先制作源对象的一个本地副本,然后使用仅有的非抛出swap()实现将这个副本与当前对象交换。

以下是用于复制赋值运算符的复制-交换习语的一个简洁示例。该类定义了一个复制构造函数、一个复制赋值运算符和一个标记为noexcept的swap()成员函数。

\begin{cpp}
export module copy_and_swap;

export class CopyAndSwap final
{
    public:
        CopyAndSwap() = default;
        ~CopyAndSwap(); // Destructor

        CopyAndSwap(const CopyAndSwap& src); // Copy constructor
        CopyAndSwap& operator=(const CopyAndSwap& rhs); // Copy assignment operator

        void swap(CopyAndSwap& other) noexcept; // noexcept swap() member function
    private:
        // Private data members...
};
// Standalone noexcept swap() function
export void swap(CopyAndSwap& first, CopyAndSwap& second) noexcept;
\end{cpp}

以下是实现:

\begin{cpp}
CopyAndSwap::~CopyAndSwap() { /* Implementation of destructor. */ }

CopyAndSwap::CopyAndSwap(const CopyAndSwap& src)
{
    // This copy constructor can first delegate to a non-copy constructor
    // if any resource allocations have to be done. See the Spreadsheet
    // implementation in Chapter 9 for an example.

    // Make a copy of all data members...
}

void swap(CopyAndSwap& first, CopyAndSwap& second) noexcept
{
    first.swap(second);
}

void CopyAndSwap::swap(CopyAndSwap& other) noexcept
{
    using std::swap;
    // Swap each data member, for example:
    // swap(m_data, other.m_data);
}

CopyAndSwap& CopyAndSwap::operator=(const CopyAndSwap& rhs)
{
    // Copy-and-swap idiom.
    auto copy { rhs }; // Do all the work in a temporary instance.
    swap(copy); // Commit the work with only non-throwing operations.
    return *this;
}
\end{cpp}

更详细的讨论,请查阅第9章。

\mySubsubsection{32.1.5.}{抛出和捕获异常}

如果曾经在一个不使用异常的团队工作过,或者已经习惯了Java风格的异常,那么C++的语法可能会感到陌生。这里是一个使用内置异常类std::runtime\_error的复习。在大多数较大的程序中,需要编写自己的异常类。

\begin{cpp}
import std;
using namespace std;
void throwIf(bool should)
{
    if (should) {
        throw runtime_error { "Here's my exception" };
    }
}
int main()
{
    try {
        throwIf(false); // Doesn't throw.
        throwIf(true); // Throws.
    } catch (const runtime_error& e) {
        println(cerr, "Caught exception: {}", e.what());
        return 1;
    }
}
\end{cpp}

第14章更详细地讨论了异常。

\mySubsubsection{32.1.6.}{编写类模板}

模板语法可能会令人困惑。最容易遗忘的模板部分是,使用类模板的代码需要能够看到类模板定义,以及成员函数的实现,函数模板也是如此。实现技巧就是将类成员函数的实现,直接放在包含类模板定义的接口文件中。下面的例子演示了这一点,并实现了一个包装对对象引用的类模板,包括一个getter。这里是模块接口文件:

\begin{cpp}
export module simple_wrapper;
export template <typename T>
class SimpleWrapper
{
    public:
        explicit SimpleWrapper(T& object) : m_object { object } { }
        T& get() const { return m_object; }
    private:
        T& m_object;
};
\end{cpp}

可以这样进行测试:

\begin{cpp}
import simple_wrapper;
import std;
using namespace std;

int main()
{
    // Try wrapping an integer.
    int i { 7 };
    SimpleWrapper intWrapper { i }; // Using CTAD.
    // Or without class template argument deduction (CTAD).
    SimpleWrapper<int> intWrapper2 { i };
    i = 2;
    println("wrapped value is {}", intWrapper.get());
    println("wrapped value is {}", intWrapper2.get());

    // Try wrapping a string.
    string str { "test" };
    SimpleWrapper stringWrapper { str };
    str += "!";
    println("wrapped value is {}", stringWrapper.get());
}
\end{cpp}

关于模板的详细信息可以在第12章和第26章中找到。

\mySubsubsection{32.1.7.}{约束模板参数}

使用概念,可以对类和函数模板的模板参数施加约束。例如,下面的代码片段约束了上一节中SimpleWrapper类模板的模板类型参数T,使其只能是浮点类型或整型。指定一个不满足这些约束的T类型将导致编译错误。

\begin{cpp}
import std;

export template <typename T> requires (std::floating_point<T> || std::integral<T>)
class SimpleWrapper
{
    public:
        explicit SimpleWrapper(T& object) : m_object { object } { }
        T& get() const { return m_object; }
    private:
        T& m_object;
};
\end{cpp}

第12章详细解释了概念。

\mySubsubsection{32.1.8.}{写入文件}

以下程序将消息输出到一个文件,然后重新打开文件并追加另一个消息。更多细节可以在第13章中找到。

\begin{cpp}
import std;
using namespace std;

int main()
{
    ofstream outputFile { "FileWrite.out" };
    if (outputFile.fail()) {
        println(cerr, "Unable to open file for writing.");
        return 1;
    }
    outputFile << "Hello!" << endl;
    outputFile.close();

    ofstream appendFile { "FileWrite.out", ios_base::app };
    if (appendFile.fail()) {
        println(cerr, "Unable to open file for appending.");
        return 2;
    }
    appendFile << "World!" << endl;
}
\end{cpp}

\mySubsubsection{32.1.9.}{读取文件}

文件输入的细节在第13章中讨论。这里有一个快速示例程序,用于基本的文件读取。读取上一节程序写入的文件,并以每次一个由空格分隔的标记的形式输出。

\begin{cpp}
import std;
using namespace std;

int main()
{
    ifstream inputFile { "FileWrite.out" };
    if (inputFile.fail()) {
        println(cerr, "Unable to open file for reading.");
        return 1;
    }

    string nextToken;
    while (inputFile >> nextToken) {
        println("Token: {}", nextToken);
    }
}
\end{cpp}

以下代码使用一次getline()调用读取整个文本文件,这对于包含内容中的\verb|\|0字符的二进制文件不起作用。

\begin{cpp}
string fileContents;
getline(inputFile, fileContents, '\0');
println("{}", fileContents);
\end{cpp}

另一种方法是使用istreambuf\_iterator(参见第17章):

\begin{cpp}
string fileContents {
    istreambuf_iterator<char> { inputFile },
    istreambuf_iterator<char> { }
};
println("{}", fileContents);
\end{cpp}













