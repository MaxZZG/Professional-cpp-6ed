
设计编码时考虑效率是好的。如果可以通过一些常识或基于经验的直觉来避免编写明显低效的程序,就没有必要编写这些程序了。然而,我敦促在设计和编码阶段不要过于沉迷于性能。最好是首先制作一个清晰、结构良好的设计和实现,然后使用分析器,只优化分析器标记为性能瓶颈的部分。请记住在第4章中介绍的“90/10”规则,指出大多数程序的运行时间中有90\%是在程序的10\%的代码中花费的(Hennessy和Patterson的《计算机体系结构:定量方法》第四版[摩根·考夫曼出版社,2006年])。所以可以优化90\%的代码,但仍然只能将程序的运行时间提高10\%。显然,想要优化的是在特定工作负载下程序运行时,最频繁执行的部分。

通常很有帮助的是分析程序,以确定哪些代码部分需要优化。有许多分析工具可供选择,它们在程序运行时分析程序,以生成有关其性能的数据。大多数分析工具都提供按函数级别的分析,通过指定程序中每个函数花费的时间(或总执行时间的百分比)。在程序上运行分析器后,通常可以立即知道程序的哪些部分需要优化。在优化前后进行分析,是证明优化有效性的关键。

如果使用的是Microsoft Visual C++,已经有了一个很好的内置分析器,这将在本章的后面讨论。如果还没有使用Visual C++,Microsoft提供了一个社区版,它是免费的,适用于学生、开源开发者和个人开发者,用于创建免费和付费应用程序。对于小型组织,最多可免费使用五个用户。另一个非常好的分析工具是IBM的Rational PurifyPlus(\url{www.almtoolbox.com/purify.php})。还有许多小型免费分析工具可供选择:Very Sleepy(\url{www.codersnotes.com/sleepy})和Luke Stackwalker(\url{lukestackwalker.sourceforge.net})是Windows上流行的分析器,Valgrind(valgrind.org)和gprof(GNU分析器,\url{sourceware.org/binutils/docs/gprof})是Unix/Linux系统上广为人知的分析器,还有许多其他选择。本节将演示两个分析器:Linux上的gprof和Visual C++ 2022自带的分析器。

\mySubsubsection{29.4.1.}{使用gprof分析的示例}

分析的力量最好通过一个真实的编码示例来展示。作为免责声明,第一个实现的性能缺陷并不微妙!真正的效率问题可能更复杂,但一个足够长的程序来展示会太长,不适合这本书。

假设为美国社会保障管理局工作。每年,管理局都会在其网站上发布一个允许用户查找前一年最受欢迎的新婴儿名字的网站。我们的任务是编写一个后端程序,用于为用户查找名字,输入是一个包含每个新婴儿名字的文件。显然,这个文件会包含重复的名字。例如,在2003年男孩的文件中,名字Jacob是最受欢迎的,出现了29,195次。程序必须读取该文件以构建内存中的数据库,用户可以请求特定名字的婴儿数量或该名字在所有婴儿中的排名。

\mySamllsection{第一次尝试设计}

对于这个程序,合理的逻辑设计是包含以下公共成员函数的NameDB类:

\begin{cpp}
export class NameDB
{
    public:
        // Reads list of baby names in nameFile to populate the database.
        // Throws invalid_argument if nameFile cannot be opened or read.

        explicit NameDB(const std::string& nameFile);
        // Returns the rank of the name (1st, 2nd, etc).
        // Returns –1 if the name is not found.

        int getNameRank(const std::string& name) const;
        // Returns the number of babies with a given name.
        // Returns –1 if the name is not found.
        int getAbsoluteNumber(const std::string& name) const;

        // Private members not shown yet ...
};
\end{cpp}

困难的部分是为内存数据库选择一个好的数据结构,第一次尝试是一个名称/计数对的vector。pair是一个工具类,组合了两个可能不同类型的值。vector中的每个条目存储一个名称,以及该名称在原始数据文件中出现的次数计数。下面是这个设计的完整类定义:

\begin{cpp}
export class NameDB
{
    public:
        explicit NameDB(const std::string& nameFile);
        int getNameRank(const std::string& name) const;
        int getAbsoluteNumber(const std::string& name) const;
    private:
        std::vector<std::pair<std::string, int>> m_names;

        // Helper member functions
        bool nameExists(const std::string& name) const;
        void incrementNameCount(const std::string& name);
        void addNewName(const std::string& name);
};
\end{cpp}

以下是构造函数,以及辅助成员函数nameExists()、incrementNameCount()和addNewName()的实现。nameExists()和incrementNameCount()中的循环遍历vector的所有元素。

\begin{cpp}
// Reads the names from the file and populates the database.
// The database is a vector of name/count pairs, storing the
// number of times each name shows up in the raw data.
NameDB::NameDB(const string& nameFile)
{
    // Open the file and check for errors.
    ifstream inputFile { nameFile };
    if (!inputFile) {
        throw invalid_argument { "Unable to open file" };
    }

    // Read the names one at a time.
    string name;
    while (inputFile >> name) {
        // Look up the name in the database so far.
        if (nameExists(name)) {
            // If the name exists in the database, just increment the count.
            incrementNameCount(name);
        } else {
            // If the name doesn't yet exist, add it with a count of 1.
            addNewName(name);
        }
    }
}

// Returns true if the name exists in the database, false otherwise.
bool NameDB::nameExists(const string& name) const
{
    // Iterate through the vector of names looking for the name.
    for (auto& entry : m_names) {
        if (entry.first == name) {
            return true;
        }
    }
    return false;
}

// Precondition: name exists in the vector of names.
// Postcondition: the count associated with name is incremented.
void NameDB::incrementNameCount(const string& name)
{
    for (auto& entry : m_names) {
        if (entry.first == name) {
            entry.second += 1;
            return;
        }
    }
}
// Adds a new name to the database.
void NameDB::addNewName(const string& name)
{
    m_names.emplace_back(name, 1);
}
\end{cpp}

可以使用类似std::find\_if()的算法,来完成与nameExists()和incrementNameCount()中的循环相同的工作。这里明确显示了循环,以强调性能问题。

眼尖的读者可能已经注意到了一些性能问题。如果名字有成千上万怎么办?在填充数据库时,涉及的许多线性搜索将变得缓慢。

为了完成这个示例,以下是两个公共成员函数的实现:

\begin{cpp}
// Returns the rank of the name.
// First looks up the name to obtain the number of babies with that name.
// Then iterates through all the names, counting all the names with a higher
// count than the specified name. Returns that count as the rank.
int NameDB::getNameRank(const string& name) const
{
    // Make use of the getAbsoluteNumber() member function.
    int num { getAbsoluteNumber(name) };
    // Check if we found the name.
    if (num == -1) {
        return -1;
    }

    // Now count all the names in the vector that have a
    // count higher than this one. If no name has a higher count,
    // this name is rank number 1. Every name with a higher count
    // decreases the rank of this name by 1.
    int rank { 1 };
    for (auto& entry : m_names) {
        if (entry.second > num) {
            ++rank;
        }
    }
    return rank;
}

// Returns the count associated with the given name.
int NameDB::getAbsoluteNumber(const string& name) const
{
    for (auto& entry : m_names) {
        if (entry.first == name) {
            return entry.second;
        }
    }
    return -1;
}
\end{cpp}

\mySamllsection{对第一次设计尝试进行性能分析}

为了测试程序,需要一个main()函数:

\begin{cpp}
import name_db;
import std;
using namespace std;

int main()
{
    NameDB boys { "boys_long.txt" };
    println("{}", boys.getNameRank("Daniel"));
    println("{}", boys.getNameRank("Jacob"));
    println("{}", boys.getNameRank("William"));
}
\end{cpp}

这个main()函数创建了一个名为boys的NameDB数据库,告诉它用包含500,500个名字的文件boys\_long.txt填充自己。

使用gprof有三个步骤:

\begin{enumerate}
\item
编译name\_db模块后,您应该使用一个特殊标志来编译main程序,该标志会在运行时导致它记录原始执行信息。当使用GCC作为编译器时,标志是-pg:

\begin{shell}
> gcc -lstdc++ -std=c++2b -pg -fmodules-ts -o namedb NameDB.cpp NameDBTest.cpp
\end{shell}

\begin{myNotic}{NOTE}
撰写本文时,GCC尚不支持完整的C++模块。当GCC完全支持模块,请查看其文档以了解如何编译和使用模块。

此外,目前必须指定-std=c++2b以启用C++23特性。在未来,这将变为-std=c++23,详情请查看编译器的文档。
\end{myNotic}

\item
运行程序。这将在工作目录中生成一个名为gmon.out的文件。运行程序时请耐心等待,因为第一个版本很慢。

\item
运行gprof命令。这一最后步骤能够分析gmon.out的性能分析信息,并生成一个(某种程度上)可读的报告。gprof输出到标准输出,应该将输出重定向到一个文件中:

\begin{shell}
> gprof namedb gmon.out > gprof_analysis.out
\end{shell}
\end{enumerate}

现在可以分析数据了。但输出文件的内容很难理解,所以需要一段时间来学习如何解释它。gprof提供了两组信息,第一组总结了程序中每个函数执行所花费的时间。第二组更有用,总结了执行每个函数及其子函数所花费的时间,这组信息也称为调用图。这里是一些来自gprof\_analysis.out文件的输出,经过编辑使其更具可读性。

\begin{shell}
index   %time   self    children    called    name
[1]     100.0   0.00     14.06                main [1]
                0.00     14.00       1/1          NameDB::NameDB [2]
                0.00      0.04       3/3          NameDB::getNameRank [25]
                0.00      0.01       1/1          NameDB::~NameDB [28]
\end{shell}

以下列表解释了不同列的含义:

\begin{itemize}
\item
index: 索引,用于引用调用图中此条目的位置。

\item
\%time: 此函数及其子函数所需的程序总执行时间的百分比。

\item
self: 函数本身执行了多少秒。

\item
children: 此函数的子函数执行了多少秒。

\item
called: 该函数调用的次数。

\item
name: 函数的名称。如果函数名称后面跟着方括号中的数字,则该数字指的是调用图中的另一个索引。
\end{itemize}

上述摘录告诉您,main()及其子函数占据了程序总执行时间的100\%,总共14.06秒。第二行显示NameDB构造函数占据了总共14.06秒中的14.00秒,立即可以看出性能问题所在。要追踪哪个部分构造函数耗时较长,需要跳转到调用图中索引为2的条目,因为这是在最后一列名称后面的方括号中的索引。在我的测试中,索引为2的调用如下所示:

\begin{shell}
[2] 99.6   0.00   14.00         1            NameDB::NameDB [2]
           1.20    6.14    500500/500500         NameDB::nameExists [3]
           1.24    5.24    499500/499500         NameDB::incrementNameCount [4]
           0.00    0.18      1000/1000           NameDB::addNewName [19]
           0.00    0.00         1/1              vector::vector [69]
\end{shell}

NameDB::NameDB的嵌套条目显示了哪些函数的后代花费了最多的时间,可以看到nameExists()花费了6.14秒,而incrementNameCount()花费了5.24秒,这些时间是调用函数的总和。那些行中的第4列显示了调用函数的次数(500,500次调用nameExists()和499,500次调用incrementNameCount()),其他函数没有花费显著的时间。

进一步分析之前,有两件事情应该显而易见:

\begin{itemize}
\item
花费大约14秒来填充大约500,000个名字的数据库很慢,也许需要一个更好的数据结构。

\item
nameExists()和incrementNameCount()花费几乎相同的时间,并且调用次数几乎相同。如果考虑到应用程序领域,这是有道理的:输入文本文件中的大多数名字都是重复的,所以大多数调用nameExists()后都会调用incrementNameCount()。如果回头看一下代码,可以看到这两个函数几乎相同,也许可以合并。此外,它们所做的绝大部分是搜索vector,使用排序数据结构可能会减少搜索时间。
\end{itemize}

\mySamllsection{第二次设计尝试}

根据gprof输出的这两点观察,重新设计程序。新的设计使用了一个map而不是一个vector。标准库map会保持条目排序并提供O(log n)的查找,而不是vector中的O(n)搜索。一个很好的练习是尝试使用std::unordered\_map,具有预期的O(1)查找,并使用分析器查看对于这个应用程序来说,确定是否比std::map更快。

程序的新版本还将nameExists()和incrementNameCount()合并为一个incrementIfExists()。

以下是类的新定义:

\begin{cpp}
export class NameDB
{
    public:
        explicit NameDB(const std::string& nameFile);
        int getNameRank(const std::string& name) const;
        int getAbsoluteNumber(const std::string& name) const;
    private:
        std::map<std::string, int> m_names;
        bool incrementIfExists(const std::string& name);
        void addNewName(const std::string& name);
};
\end{cpp}

以下是新成员函数实现:

\begin{cpp}
// Reads the names from the file and populates the database.
// The database is a map associating names with their frequency.
NameDB::NameDB(const string& nameFile)
{
    // Open the file and check for errors.
    ifstream inputFile { nameFile };
    if (!inputFile) {
        throw invalid_argument { "Unable to open file" };
    }

    // Read the names one at a time.
    string name;
    while (inputFile >> name) {
        // Look up the name in the database so far.
        if (!incrementIfExists(name)) {
            // If the name exists in the database, the
            // member function incremented it, so we just continue.
            // We get here if it didn't exist, in which case
            // we add it with a count of 1.
            addNewName(name);
        }
    }
}

// Returns true if the name exists in the database, false
// otherwise. If it finds it, it increments it.
bool NameDB::incrementIfExists(const string& name)
{
    // Find the name in the map.
    auto res { m_names.find(name) };
    if (res != end(m_names)) {
        res->second += 1;
        return true;
    }
    return false;
}

// Adds a new name to the database.
void NameDB::addNewName(const string& name)
{
    m_names[name] = 1;
}

int NameDB::getNameRank(const string& name) const { /* Omitted, same as before */ }

// Returns the count associated with the given name.
int NameDB::getAbsoluteNumber(const string& name) const
{
    auto res { m_names.find(name) };
    if (res != end(m_names)) {
        return res->second;
    }
    return -1;
}
\end{cpp}

\mySamllsection{分析第二次设计尝试}

按照之前展示的相同步骤,可以获取新版本程序的gprof性能数据。这些数据相当鼓舞人心:


\begin{shell}
index   %time   self   children    called     name
[1]     100.0   0.00     0.21                 main [1]
                0.02     0.18      1/1        NameDB::NameDB [2]
                0.00     0.01      1/1        NameDB::~NameDB [13]
                0.00     0.00      3/3        NameDB::getNameRank [28]
[2]      95.2   0.02     0.18      1          NameDB::NameDB [2]
                0.02     0.16 500500/500500   NameDB::incrementIfExists
[3]             0.00     0.00   1000/1000     NameDB::addNewName [24]
                0.00     0.00      1/1        map::map [87]
\end{shell}

如果在机器上运行此命令,输出将会不同。甚至可能在输出中看不到任何NameDB成员函数的数据。由于第二次尝试的高效性,时间变得非常短,可能会在输出中看到更多的map成员函数,而不是NameDB成员函数。

在我的测试中,main()现在只花费了0.21秒——提高了67倍!当然,还可以对这个程序进行进一步的改进。例如,当前的构造函数执行了一次查找,以查看名称是否已在map中。如果没有,则将其添加到映射中。可以将这两个操作简单地合并为:

\begin{cpp}
m_names[name] += 1;
\end{cpp}

如果名称已在map中,这条语句只是增加了它的计数器。如果名称尚未在map中,这条语句首先在map中添加一个条目,将给定的名称作为键,初始值为零,然后增加值,结果是一个计数器为1。

通过这个改进,可以删除incrementIfExists()和addNewName()成员函数,并将构造函数更改为:

\begin{cpp}
NameDB::NameDB(const string& nameFile)
{
    // Open the file and check for errors.
    ifstream inputFile { nameFile };
    if (!inputFile) {
        throw invalid_argument { "Unable to open file" };
    }

    // Read the names one at a time.
    string name;
    while (inputFile >> name) {
        m_names[name] += 1;
    }
}
\end{cpp}

getNameRank()仍然使用一个循环,该循环遍历映射中的所有元素。一个练习是尝试找到另一种数据结构,以便避免getNameRank()中的线性迭代。

\mySubsubsection{29.4.2.}{使用Visual C++ 2022进行性能分析示例}

大多数版本的Microsoft Visual C++ 2022都有一个内置的强大的分析器,这个分析器在本节中简要讨论。VC++分析器具有完整的图形用户界面。

要在Visual C++ 2022中开始分析应用程序,首先需要在Visual Studio中打开项目。本例使用与前几个部分中第一个低效设计尝试相同的NameDB代码。项目在Visual Studio中打开,请确保配置设置为Release,而不是Debug,然后点击调试菜单,选择性能分析器。一个新的窗口出现,如图 29.1所示。

\myGraphic{0.4}{content/part5/chapter29/images/1.png}{图 29.1}

根据VC++版本,这个窗口将提供多个不同的分析工具。以下是非详尽的列表,这里解释了其中两个:

\begin{itemize}
\item
CPU使用情况:用于监控低开销的应用程序,分析应用程序的行为,不会对目标应用程序产生很大的性能影响。

\item
Instrumentation:向应用程序添加代码,以便准确计数函数调用次数和计时单个函数调用,这个工具对应用程序的性能影响更大。建议首先使用CPU使用工具来了解应用程序中的瓶颈。如果这个工具没有提供足够的信息,可以尝试Instrumentation工具。
\end{itemize}

对于这个分析示例,只需启用CPU使用工具,然后点击开始按钮。这将执行程序并分析其CPU使用情况。当程序执行完成时,Visual Studio会自动打开分析报告。图 29.2显示了当分析NameDB应用程序的第一次尝试时的性能报告。

\myGraphic{0.6}{content/part5/chapter29/images/2.png}{图 29.2}

报告中,可以立即看到热点路径。就像gprof一样,显示NameDB构造函数占据了程序运行的大部分时间。Visual Studio的分析报告是交互式的。可以通过点击图 29.2中的热点路径树中的NameDB::NameDB构造函数,来深入挖掘该函数。这将对该函数进行深入分析,如图 29.3所示。

\myGraphic{0.7}{content/part5/chapter29/images/3.png}{图 29.3}

这个深入视图在顶部显示热点路径,在底部显示实际成员函数的代码。代码视图显示了代码行所需要的时间百分比,占用大部分时间的行以红色显示。这个视图立即清楚地显示了,incrementNameCount()和nameExists()花费的时间基本相同。

这个报告的顶部,有一个下拉列表叫做“当前视图”,可以使用它来查看性能分析数据的不同视图。






