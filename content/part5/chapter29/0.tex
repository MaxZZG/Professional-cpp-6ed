\noindent
\textbf{WHAT’S IN THIS CHAPTER?}

\begin{itemize}
\item
What efficiency and performance mean

\item
What kind of language-level optimizations you can use

\item
Which design-level guidelines you can follow to design efficient programs

\item
What profiling tools are
\end{itemize}

\noindent
\textbf{WILEY.COM DOWNLOADS FOR THIS CHAPTER}

Please note that all the code examples for this chapter are available as part of this chapter’s code download on the book’s website at \url{www.wiley.com/go/proc++6e} on the Download Code tab.

The efficiency of your programs is important regardless of your application domain. If your product competes with others in the marketplace, speed can be a major differentiator: given the choice between a slower and a faster program, which one would you choose? No one would buy an operating system that takes two weeks to boot up. Even if you don’t intend to sell your products, they will have users. Those users will not be happy with you if they end up wasting time waiting for your programs to complete tasks.

Now that you understand the concepts of professional C++ design and coding and have tackled some of the more complex facilities that the language provides, you are ready to incorporate performance into your programs. Writing efficient programs involves thought at the design level, as well as details at the implementation level. Although this chapter falls late in this book, remember to consider performance from the beginning of your projects.














