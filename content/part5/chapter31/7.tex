通过解决下面的练习,可以练习本章讨论的内容。所有练习的解决方案都可以在本书的网站\url{www.wiley.com/go/proc++6e}下载到源码。若在练习中卡住了,可以考虑先重读本章的部分内容,试着自己找到答案,再在从网站上寻找解决方案。

\begin{itemize}
\item
\textbf{练习 31-1}: 调试的基本法则是什么?

\item
\textbf{练习 31-2}: 能发现以下代码中的什么问题吗?

\begin{cpp}
import std;
using namespace std;

int* getData(int value) { return new int { value * 2 }; }

int main()
{
    int* data { getData(21) };
    println("{}", *data);

    data = getData(42);
    println("{}", *data);
}
\end{cpp}

\item
\textbf{练习 31-3}: 给定以下代码段:

\begin{cpp}
import std;
using namespace std;

int sum(int* values, int count)
{
    int total { 0 };
    for (int i { 0 }; i <= count; ++i) { total += values[i]; }
    return total;
}

int main()
{
    int values[] { 1, 2, 3 };
    int total { sum(values, sizeof(values)) };
    println("{}", total);
}
\end{cpp}

它计算一组值的和。对于值1, 2和3,期望和为6;在我的机器上运行代码时,对于调试构建,结果是-2,对于每个发布构建的执行,结果都是不同的随机数。例如,调试构建的结果是-2,发布构建的结果是920865056, -321371431等。这是怎么回事?使用符号调试器和其逐步执行模式来确定错误的根本原因。查阅调试器的文档,了解如何逐行执行代码。

\item
\textbf{练习 31-4}: (高级) 修改本章前面提到的开始时间调试模式示例,使用第14章讨论的std::source\_location来替换旧的LOG()宏。这比听起来要复杂。问题在于Logger::log()是一个变参函数模板,所以不能只是在变参参数包之后添加一个命名源位置参数。一个技巧是使用一个辅助类,例如Log。构造函数接受一个变参参数包和一个源位置,并将工作转发给Logger::log()。技巧的最后一步是以下推导指南,参见第12章:

\begin{cpp}
template <typename... Ts>
Log(Ts&&...) -> Log<Ts...>;
\end{cpp}
\end{itemize}












