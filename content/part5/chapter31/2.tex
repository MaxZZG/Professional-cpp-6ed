A bug in a computer program is incorrect run-time behavior. This undesirable behavior includes both catastrophic and noncatastrophic bugs. Examples of catastrophic bugs are program crashes, data corruption, operating system failures, or some other horrific outcome. A catastrophic bug can also manifest itself external to the software or computer system running the software; for example, medical software might contain a catastrophic bug causing a massive radiation overdose to a patient. Noncatastrophic bugs are bugs that cause the program to behave incorrectly in more subtle ways; for example, a web browser might return the wrong web page, or a spreadsheet application might calculate the standard deviation of a column incorrectly. These are also called logical bugs.

There are also cosmetic bugs, where something is visually not correct, but otherwise works correctly. For example, a button in a user interface is kept enabled when it shouldn’t be—but clicking it does nothing. All computations are perfectly correct, the program does not crash, but it doesn’t look as “nice” as it should.

The underlying cause, or root cause, of a bug is the mistake in the program that causes this incorrect behavior. The process of debugging a program includes both determining the root cause of the bug and fixing the code so that the bug will not occur again.












