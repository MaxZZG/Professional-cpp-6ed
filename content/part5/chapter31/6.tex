The most important concept in this chapter is the fundamental law of debugging: avoid bugs when you’re coding, but plan for bugs in your code. The reality of programming is that bugs will appear. If you’ve prepared your program properly, with error logging, debug traces, and assertions, then the actual debugging will be significantly easier.

This chapter also presented specific approaches for debugging bugs. The most important rule when debugging is to reproduce the problem. Then, you can use a symbolic debugger, or log-based debugging, to track down the root cause. Memory errors present particular difficulties and account for the majority of bugs in legacy C++ code. This chapter described various categories of memory bugs and their symptoms and showed examples of debugging errors in a program.

Debugging is a hard skill to learn. To take your C++ skills to a professional level, you will have to practice debugging a lot.


