
找到缺陷的唯一方法是通过测试。从开发者的角度来看,最重要的测试类型之一就是单元测试。单元测试是代码的一部分,用于测试类或子系统的特定功能。这些是开发者可能编写的最细致的测试。理想情况下,对于代码能够执行的每一个低级任务,都应该有一个或多个单元测试存在。例如,正在编写一个可以进行加法和乘法的数学库。单元测试套件可能包含以下测试:

\begin{itemize}
\item
测试简单的加法,如 1+2

\item
测试大数字的加法

\item
测试负数的加法

\item
测试零加一个数字

\item
测试加法的交换律

\item
测试简单的乘法

\item
测试大数乘法

\item
测试负数的乘法

\item
测试乘以零

\item
测试乘法的交换律
\end{itemize}

编写良好的单元测试在许多方面都能提供保护:

\begin{itemize}
\item
单元测试证明了一项功能实际上是有效的。在实际使用类之前,其行为是不确定的。

\item
当最近引入的更改破坏了某些功能时,单元测试会在第一时间提供警报。这种特定的用途,称为回归测试,将在本章后面介绍。

\item
当作为开发过程的一部分使用时,测试迫使开发者从一开始就解决问题。如果不能在单元测试失败的情况下提交代码,就必须立即解决相应的问题。

\item
单元测试允许在其它代码就位之前尝试代码。当第一次开始编程时,可以编写整个程序,然后运行。专业的程序太大,不适合这种方法,所以需要能够独立测试的组件。

\item
最后但同样重要的是,单元测试提供了使用示例。如果有同事想要知道如何使用您的数学库进行矩阵乘法,可以向她展示适当的单元测试。
\end{itemize}

\mySubsubsection{30.2.1.}{单元测试的方法}

单元测试很难出错,除非不写或写得很差。一般来说,测试越多,覆盖范围就越广。覆盖范围越广,错误遗漏和告诉老板,或者更糟糕的是客户,“哦,我们从未测试过那个”的可能性就越小。

编写单元测试的有效方法有几种。第28章中解释的极限编程方法论,指导其追随者在编写代码之前就编写单元测试。

首先编写测试有助于明确组件的需求,并提供一个度量标准,用于确定编码何时完成。首先编写测试可能有些棘手,需要开发者的高度勤奋。对于一些程序员来说,这简单地与其编码风格不匹配。一种不那么严格的方法是在编码之前设计测试,但在过程的后续进行实施。开发者需要理解模块的需求,但不必为不存在类编写代码。

一些团队中,特定子系统的作者不编写该子系统的单元测试。如果为自己的代码编写测试,可能会下意识地绕过已知的问题,或者只覆盖知道自己的代码能处理好的特定情况。此外,有时候很难对刚编写的代码中的错误感到兴奋,所以可能只付出半心半力的努力。一个开发者编写另一个开发者代码的单元测试需要大量的协调开销。当这种协调完成时,有助于保证测试更加有效。

代码覆盖率是一个度量标准,用于测量单元测试覆盖了多少代码。这样的度量标准允许在编写单元测试时,最大化代码覆盖率。可以使用一个代码覆盖率工具,比如 gcov(\url{gcc.gnu.org/onlinedocs/gcc/Gcov.html}),会报告单元测试调用了代码的百分比。适当的测试代码有单元测试来测试,通过该代码段可能采取的所有代码路径,从而达到100\%的单元测试代码覆盖率。不同的代码覆盖率工具对代码覆盖率的定义不同。如果工具认为一行是一个单一的if语句,而主体从未执行过,则工具会认为这一行覆盖了吗?工具如何定义模板的覆盖率?工具是否支持分支覆盖率,以确保覆盖每个可能的方向?需要对将要使用的工具进行探究。

\mySubsubsection{30.2.2.}{单元测试过程}

代码提供单元测试的过程从开发初期就开始了,远远早于编写代码。在设计阶段考虑到单元测试的可能性,会影响为软件做出的设计决策。即使不认同在编写代码之前编写单元测试的方法,至少应该花时间考虑需要提供哪种类型的测试,即使仍然处于设计阶段。可以将任务分解为定义良好的块,每个块都有自己的经过测试验证的标准。例如,任务是编写一个数据库访问类,首先编写将数据插入数据库的功能。当用一套单元测试完全测试完毕,就可以继续编写支持更新、删除和选择代码,边写边测试每一部分。

以下步骤列表是设计和实施单元测试的建议方法。与编程方法论一样,最好的过程是产生最佳结果的过程。我建议尝试使用不同的单元测试方法,以找到最适合的方法。

\mySamllsection{定义测试的粒度}

编写单元测试需要时间,这是无法避免的。软件开发者通常时间紧迫。为了赶在截止日期前完成,开发者倾向于跳过编写单元测试,这样可以更快完成。不幸的是,这种想法并没有考虑到整个情况。长期来看,省略单元测试将会适得其反。在软件开发过程中越早发现错误,成本就越低。如果开发者在单元测试期间发现了一个错误,可以立即修复。如果错误是由质量保证(QA)发现的,这个错误就会变得更加昂贵。错误可能需要额外的开发周期,需要错误管理;必须返回开发团队进行修复,然后再次返回QA验证修复。如果错误在QA过程中未捕获并最终传递给客户,将变得更加昂贵。

测试的粒度指的是其范围。如下表所示,可以使用几个测试函数来初始单元测试数据库类,然后逐渐添加更多测试以确保一切正常工作:

% Please add the following required packages to your document preamble:
% \usepackage{longtable}
% Note: It may be necessary to compile the document several times to get a multi-page table to line up properly
\begin{longtable}{|l|l|l|}
\hline
\textbf{粗粒度测试} &
\textbf{中粒度测试} &
\textbf{细粒度测试} \\ \hline
\endfirsthead
%
\endhead
%
\begin{tabular}[c]{@{}l@{}}testConnection()\\ testInsert()\\ testUpdate()\\ testDelete()\\ testSelect()\end{tabular} &
\begin{tabular}[c]{@{}l@{}}testConnectionDropped()\\ testInsertBadData()\\ testInsertStrings()\\ testInsertIntegers()\\ testUpdateStrings()\\ testUpdateIntegers()\\ testDeleteNonexistentRow()\\ testSelectComplicated()\\ testSelectMalformed()\end{tabular} &
\begin{tabular}[c]{@{}l@{}}testConnectionThroughHTTP()\\ testConnectionLocal()\\ testConnectionErrorBadHost()\\ testConnectionErrorServerBusy()\\ testInsertWideCharacters()\\ testInsertLargeData()\\ testInsertMalformed()\\ testUpdateWideCharacters()\\ testUpdateLargeData()\\ testUpdateMalformed()\\ testDeleteWithoutPermissions()\\ testDeleteThenUpdate()\\ testSelectNested()\\ testSelectWideCharacters()\\ testSelectLargeData()\end{tabular} \\ \hline
\end{longtable}

每一列都引入了更具体的测试。当从粗粒度测试转移到更细粒度的测试时,需要考虑错误条件、不同的输入数据集和不同的操作模式。

当然,最初选择测试粒度时所做出的决定并非一成不变。也许数据库类只是作为一个概念验证而编写,甚至不会使用。现在几个简单的测试可能就足够了,可以随时添加更多。或者,使用情况可能在以后的日期发生变化,例如:数据库类最初可能并没有考虑到国际字符。添加了这些特性,就应该使用特定的有针对性的单元测试,对其进行测试。

将单元测试视为实际实现功能的一部分。当进行修改时,不要只是修改测试以使其继续工作(当然也应该这样做)。编写新的测试并重新评估现有的测试。当发现并修复错误时,添加新的单元测试,专门测试那些修复,称为回归测试。

\begin{myWarning}{WARNING}
单元测试是测试的子系统的一部分。当增强和完善子系统时,也要增强和完善测试。
\end{myWarning}

\mySamllsection{个人测试的头脑风暴}

随着时间的推移,将获得对哪些代码方面应该转化为单元测试的直觉。某些功能或输入会让人感觉应该测试。这种直觉是通过试错获得的,也是通过查看团队中其他人编写的单元测试获得的。挑选出哪些开发者是最优秀的单元测试员应该相当容易。他们的测试往往组织得很好,并且经常修改。

单元测试创建变得自然而然之前,通过头脑风暴的方式来解决编写哪些测试的问题。为了激发一些想法,请考虑以下问题:

\begin{itemize}
\item
这段代码是为了做什么?

\item
每个功能的典型调用方式是什么?

\item
调用者可能会违反函数的前置条件吗?

\item
如何误用每个功能?

\item
期望作为输入的数据类型是什么?

\item
不期望作为输入的数据类型是什么?

\item
边缘情况或异常条件是什么?
\end{itemize}

不需要对这些问题编写正式答案(除非您的经理是这本书或某些测试方法学的狂热信徒),但它们会让您生成一些单元测试的想法。数据库类的测试表包含测试函数,每个函数都源于这些问题之一。

当为使用测试生成了一些想法时,考虑如何将它们组织成类别;测试的细分将随之而来。在数据库类示例中,测试可以分为以下类别:

\begin{itemize}
\item
基本测试

\item
错误测试

\item
本地化测试

\item
非法输入测试

\item
复杂测试
\end{itemize}

将测试分为类别,可以使它们更容易识别和增强。这也可能使开发者更容易意识到哪些代码方面测试得很好,哪些可能需要更多的单元测试。

\begin{myWarning}{WARNING}
编写大量简单测试很容易,但不要忘记更复杂的情况!
\end{myWarning}

\mySamllsection{创建示例数据和结果}

编写单元测试时最常陷入的陷阱是,将测试与代码的行为相匹配,而不是使用测试来验证代码。如果编写了一个单元测试,用于在数据库中绝对存在的数据上进行数据库选择,并且测试失败,问题是出在代码上还是测试上?通常更容易假设代码是正确的,并修改测试以匹配。这种方法是错误的!

为了避免这个陷阱,在尝试之前了解测试的输入和预期输出。有时候说起来容易做起来难。例如,编写了一些代码,使用特定的密钥来加密文本块。一个合理的单元测试会取一个固定的文本字符串,并将其传递给加密模块,再检查结果是否正确加密。

当去编写这样的测试时,很可能会先尝试加密模块的行为并查看结果。如果看起来合理,可能会编写一个测试来查找该值,这样做实际上并没有证明什么!并没有真正测试代码,只是编写了一个保证会继续返回该相同值的测试。编写测试需要一些真正的工作;需要独立于加密模块对文本进行加密以获得准确的结果。如果不知道正在使用的加密算法,因为它来自第三方库,可以编写如下测试:decrypt(encrypt(x))==x 和 encrypt(a)!=encrypt(b)。

\begin{myWarning}{WARNING}
运行测试之前,确定测试的正确输出。
\end{myWarning}

\mySamllsection{编写测试}

测试背后的确切代码取决于使用的测试框架类型。本章后面将讨论一个框架,Microsoft Visual C++ 测试框架。无论实际实现如何,以下准则将确保测试有效:

\begin{itemize}
\item
确保每个测试只测试一件事情。如果测试失败,将指向一个特定的功能。

\item
测试内部要具体。测试失败是因为抛出了异常,还是因为返回了错误的值?

\item
测试代码中广泛使用日志记录。如果有一天测试失败了,将能够了解发生了什么。

\item
避免依赖早期测试或相互关联的测试,测试应该尽可能隔离和原子化。

\item
如果测试需要使用其他子系统,考虑编写测试桩或模拟来模拟这些子系统。测试桩或模拟实现与它们模拟的子系统相同的接口,就可以用来代替具体的子系统实现。例如,一个单元测试需要一个数据库,但该单元测试不是测试该数据库的子系统,则测试桩或模拟可以实现数据库接口,并模拟真实的数据库。这样,运行单元测试不需要连接到真实的数据库,真实的数据库实现中的错误,也不会对这个特定的单元测试产生影响。实际上,可以使用模拟真实数据库可能难以或无法可靠实现的错误条件!

\item
代码审查者不仅要审查代码,还要审查单元测试。进行代码审查时,告诉另一位工程师认为可以添加测试的地方。
\end{itemize}

本章后面,单元测试通常是简单的代码段。通常,编写一个单元测试只需要几分钟,这使得单元测试可成为最有生产力的工具。

\mySamllsection{运行测试}

编写完一个测试后,应该立即运行,以免对结果的期待变得难以承受。一个充满通过单元测试的屏幕的喜悦不应低估。对于大多数开发者来说,这是声明代码有用和(据您所知)正确定量数据的最简单方式。

即使采用编写测试后再编写代码的方法论,也应该在编写完测试后立即运行它们,可以证明给自己看测试最初是失败的。当代码就位,就有实际数据显示它是否完成了应该完成的事情。

编写的每个测试第一次运行时都不太可能得到预期的结果,如果在编写代码之前编写测试,则所有测试都应该失败。如果有一个通过了,要么代码神奇地出现了,要么测试有问题。如果代码已经完成而测试仍然失败(有人说如果测试失败,代码实际上并没有完成),有两种可能性:代码可能是错误的,或者测试可能错误。

运行单元测试必须自动化。这可以通过几种方式实现,一种选择是有一个专用的系统,每次持续集成构建后自动运行所有单元测试,或者至少每晚运行一次。这样的系统必须发送电子邮件通知开发者单元测试失败的情况。

另一种选择是设置本地开发环境,以便在每次编译代码时执行单元测试,单元测试应该保持小而高效。如果确实有运行时间较长的单元测试,可将它们分开,并让专门的测试系统来运行这些测试。

\mySubsubsection{30.2.3.}{单元测试实战}

已经理论上了解了单元测试,现在是实际编写一些测试的时候了。以下示例借鉴了第29章中的对象池实现。简要回顾一下,对象池是一个类,可以用来避免分配过多的对象。通过跟踪已经分配的对象,池充当了需要某种类型对象的代码和已经分配的此类对象之间的代理。

ObjectPool类模板的公共接口如下,详细内容请参见第29章:

\begin{cpp}
export
template <typename T, typename Allocator = std::allocator<T>>
class ObjectPool final
{
    public:
        ObjectPool() = default;
        explicit ObjectPool(const Allocator& allocator);
        ~ObjectPool();

        // Prevent move construction and move assignment.
        ObjectPool(ObjectPool&&) = delete;
        ObjectPool& operator=(ObjectPool&&) = delete;

        // Prevent copy construction and copy assignment.
        ObjectPool(const ObjectPool&) = delete;
        ObjectPool& operator=(const ObjectPool&) = delete;

        // Reserves and returns an object from the pool. Arguments can be
        // provided which are perfectly forwarded to a constructor of T.
        template <typename... Args>
        std::shared_ptr<T> acquireObject(Args&&... args);
};
\end{cpp}

\mySamllsection{引入Microsoft Visual C++测试框架}

Microsoft Visual C++自带了一个内置的测试框架。使用单元测试框架的优点是允许开发者专注于编写测试,而不是处理设置测试、构建测试逻辑和收集结果。以下讨论是基于Visual C++ 2022编写的测试。

\begin{myNotic}{NOTE}
如果不使用Visual C++,有许多开源单元测试框架可用。Google Test (\url{github.com/google/googletest}) 是C++的一个这样的框架,另一个是Boost Test Library (\url{www.boost.org/doc/libs/1_82_0/libs/test/}) 。它们都包含了许多对测试开发者有帮助的工具和控制结果自动输出的选项。
\end{myNotic}

开始使用Visual C++测试框架,必须创建一个测试项目。以下步骤说明了如何测试ObjectPool类模板:

\begin{enumerate}
\item
启动Visual C++,创建一个新项目,选择Native Unit Test Project,然后点击Next。

\item
给项目命名并点击Create。

\item
向导创建了一个新的测试项目,其中包括一个名为<ProjectName>.cpp的文件。在Solution Explorer中选择这个文件并删除它,添加自己的文件。如果Solution Explorer停靠窗口不可见了,请点击View -> Solution Explorer。

\item
在Solution Explorer中右键单击项目,然后单击Properties。转到Configuration Properties -> C/C++ -> Precompiled Headers,将Precompiled Header选项设置为Not Using Precompiled Headers,然后点击OK。此外,在Solution Explorer中选择pch.cpp和pch.h文件并删除它们。使用预编译头是Visual C++的一个特性,用于提高构建速度,但在此测试项目中不使用。

\item
向测试项目添加名为ObjectPoolTest.h和ObjectPoolTest.cpp的空文件。
\end{enumerate}

现在已经准备好开始向代码中添加单元测试。

最常见的技术是将单元测试划分为测试类的逻辑测试组,将创建一个名为ObjectPoolTest的测试类。ObjectPoolTest.h中用于开始的基本代码如下:

\begin{cpp}
#pragma once
#include <CppUnitTest.h>

TEST_CLASS(ObjectPoolTest)
{
    public:
};
\end{cpp}

这段代码定义了一个名为ObjectPoolTest的测试类,但与标准C++相比,语法有点不同。这样做是为了让框架能够自动发现所有测试。

如果需要在运行测试类中定义的测试之前执行任务,或者在执行测试后进行清理,可以实现一个initialize和cleanup成员函数。这里是一个例子:

\begin{cpp}
TEST_CLASS(ObjectPoolTest)
{
    public:
        TEST_CLASS_INITIALIZE(setUp);
        TEST_CLASS_CLEANUP(tearDown);
};
\end{cpp}

由于ObjectPool类模板的测试相对简单且独立,setUp()和tearDown()的空定义就足够了,或者可以删除它们。如果确实需要,ObjectPoolTest.cpp源文件的初始阶段如下所示:

\begin{cpp}
#include "ObjectPoolTest.h"
void ObjectPoolTest::setUp() { }
void ObjectPoolTest::tearDown() { }
\end{cpp}

这就是开始开发单元测试所需的初始代码。

\begin{myNotic}{NOTE}
实际场景中,通常将测试代码和要测试的代码分成不同的项目。为了保持这个示例的简洁,我在这里没有这样做。
\end{myNotic}

\mySamllsection{编写第一个测试}

这可能是第一次接触到 Visual C++ 测试框架,或者对单元测试整体而言,第一个测试将是一个简单的测试,测试 0 是否小于 1。

一个单独的单元测试就是一个测试类的成员函数。要创建一个简单的测试,将它的声明添加到 ObjectPoolTest.h 文件中:

\begin{cpp}
TEST_CLASS(ObjectPoolTest)
{
    public:
        TEST_CLASS_INITIALIZE(setUp);
        TEST_CLASS_CLEANUP(tearDown);
        TEST_METHOD(testSimple); // Your first test!
};
\end{cpp}

testSimple 测试的实现使用 Assert::IsTrue(),定义在 Microsoft::VisualStudio::CppUnitTestFramework 命名空间中,来进行实际的测试。Assert::IsTrue() 验证给定表达式是否返回 true。如果表达式返回 false,则测试失败。Assert 提供了许多更多的辅助函数,如 AreEqual(), IsNull(), Fail(), ExpectException()等。在 testSimple 的情况下,测试声称 0 小于 1。这里是更新的 ObjectPoolTest.cpp 文件:

\begin{cpp}
#include "ObjectPoolTest.h"

using namespace Microsoft::VisualStudio::CppUnitTestFramework;

void ObjectPoolTest::setUp() { }
void ObjectPoolTest::tearDown() { }

void ObjectPoolTest::testSimple()
{
    Assert::IsTrue(0 < 1);
}
\end{cpp}

单元测试大多会比这个简单的断言更有趣。常见的模式是执行某种计算,然后断言结果是期望的值。使用 Visual C++ 测试框架,甚至不需要担心异常;框架会捕获,并报告必要的异常。

\mySamllsection{构建和运行测试}

通过点击 Build -> Build Solution 构建解决方案,并打开 Test Explorer(Test -> Test Explorer),如图 30.4 所示。

\myGraphic{0.5}{content/part5/chapter30/images/4.png}{图 30.4}

构建解决方案后,Test Explorer 会自动显示所有发现的单元测试,显示了 testSimple 单元测试。可以通过点击窗口左上角的 Run All Tests 按钮来运行所有测试,Test Explorer 显示单元测试是成功还是失败。这时,单元测试成功,如图 30.5 所示。

\myGraphic{0.5}{content/part5/chapter30/images/5.png}{图 30.5}

如果修改代码,断言 1 小于 0,则测试失败,Test Explorer 报告失败,如图 30.6 所示。

\myGraphic{0.5}{content/part5/chapter30/images/6.png}{图 30.6}

Test Explorer 窗口的下半部分显示与所选单元测试相关的有用信息。单元测试失败的情况下,会说明哪里失败了,现在是断言失败了。还有一个在失败发生时捕获的堆栈跟踪,可以点击那个堆栈跟踪中的超链接,直接跳转到有问题的行——对于调试非常有用。

\mySamllsection{负测试}

可以编写负测试,即进行应该失败的操作的测试,可以编写一个负测试来测试某个函数抛出预期的异常。Visual C++测试框架提供了Assert::ExpectException()函数来处理预期的异常。例如,以下单元测试使用ExpectException()来执行一个抛出std::invalid\_argument异常的Lambda表达式,该异常定义在<stdexcept>中[在撰写本文时,Visual C++ 2022包含了一个bug,这阻止了将\#include <CppUnitTest.h>和import std;结合在一起。作为变通方案,将import std;语句替换为所需的头文件\#include语句,例如\#include <stdexcept>。有关不同源代码文件中所需的全套头文件,请参见源码库代码]。ExpectException()的模板类型参数指定了预期的异常类型。

\begin{cpp}
void ObjectPoolTest::testException()
{
    Assert::ExpectException<std::invalid_argument>(
        []{ throw std::invalid_argument { "Error" }; });
}
\end{cpp}

\mySamllsection{添加实际测试}

现在框架已经设置完毕,简单的测试也运行正常,是时候将注意力转向ObjectPool类模板,并编写一些实际测试代码了。以下所有测试都将添加到ObjectPoolTest.h和ObjectPoolTest.cpp中,就像之前初始的测试一样。

首先,将ObjectPool.cppm模块接口文件复制到创建的ObjectPoolTest.h文件旁边,然后将其添加到项目中。ObjectPool.cppm使用了C++23功能,在撰写本文时,Visual C++ 2022默认尚未启用。要启用它,请打开项目属性,转到Configuration Properties -> General,并将C++ Language Standard设置为“预览 - 最新C++工作草案的功能”。在Visual C++的下一个版本中,可以将此选项设置为“ISO C++23标准”。

编写测试之前,需要将辅助类与ObjectPool一起使用。ObjectPool创建一定类型的对象,并按请求将其分发给调用者。一些测试需要检查是否检索到的对象,与以前检索到的对象相同。一种方法是创建一个序列化对象池——具有单调递增序列号的对象。以下代码展示了在Serial.cppm模块接口文件如何定义此类:

\begin{cpp}
export module serial;

export class Serial
{
    public:
        // A new object gets a next serial number.
        Serial() : m_serialNumber { ms_nextSerial++ } { }
        unsigned getSerialNumber() const { return m_serialNumber; }
    private:
        static inline unsigned ms_nextSerial { 0 }; // The first serial number is 0
        unsigned m_serialNumber { 0 };
};
\end{cpp}

现在,开始编写测试!作为一个初步的健全性检查,可能想要一个测试来创建一个对象池。如果在创建过程中抛出任何异常,Visual C++测试框架将报告一个错误。代码按照AAA原则编写:Arrange, Act, Assert;测试首先为运行测试设置一切,然后进行一些工作,最后断言预期的结果,这通常也称为if-when-then原则。我建议在单元测试中添加以IF、WHEN和THEN开头的注释,以便测试的三个阶段清晰地显示。

\begin{cpp}
void ObjectPoolTest::testCreation()
{
    // IF nothing

    // WHEN creating an ObjectPool
    ObjectPool<Serial> myPool;

    // THEN no exception is thrown
}
\end{cpp}

别忘了在头文件中添加TEST\_METHOD(testCreation);语句,这适用于所有后续测试。还需要在ObjectPoolTest.cpp源文件中添加对object\_pool和serial模块的导入声明。

\begin{cpp}
import object_pool;
import serial;
\end{cpp}

第二个测试,testAcquire(),测试ObjectPool的特定公共功能:能够分发对象,没有什么可断言的。为了证明结果的Serial对象的合法性,测试断言其序列号大于或等于零。

\begin{cpp}
void ObjectPoolTest::testAcquire()
{
    // IF an ObjectPool has been created for Serial objects
    ObjectPool<Serial> myPool;
    // WHEN acquiring an object
    auto serial { myPool.acquireObject() };
    // THEN we get a valid Serial object
    Assert::IsTrue(serial->getSerialNumber() >= 0);
}
\end{cpp}

下一个测试更有趣,ObjectPool 应该不会两次给出相同的 Serial 对象。这个测试通过从池中获取多个对象来检查 ObjectPool 的独占属性,所有获取到的对象的序列号都被存储在一个集合中。如果池正确地分配了唯一的对象,它们的序列号中不会匹配。

\begin{cpp}
void ObjectPoolTest::testExclusivity()
{
    // IF an ObjectPool has been created for Serial objects
    ObjectPool<Serial> myPool;
    // WHEN acquiring several objects from the pool
    const size_t numberOfObjectsToRetrieve { 20 };
    set<unsigned> seenSerialNumbers;

    for (size_t i { 0 }; i < numberOfObjectsToRetrieve; ++i) {
        auto nextSerial { myPool.acquireObject() };
        seenSerialNumbers.insert(nextSerial->getSerialNumber());
    }

    // THEN all retrieved serial numbers are different.
    Assert::AreEqual(numberOfObjectsToRetrieve, seenSerialNumbers.size());
}
\end{cpp}

最后一个测试(目前为止)检查释放功能。一个对象释放时,ObjectPool 可以再次给创建。池在释放所有对象之前,不应该分配额外的对象。

测试首先从池中获取 10 个 Serial 对象,将其存储在vector中以保持活动状态,并记录每个获取到的 Serial 的原始指针。当所有 10 个对象都检索到,就会释放回池中。

测试的第二阶段再次从池中获取 10 个对象,并存储在vector中以保持它们的活动状态。所有这些检索到的对象必须有一个原始指针,该指针在测试的第一阶段已经出现过。这验证了对象池可以正确地重复使用。

\begin{cpp}
void ObjectPoolTest::testRelease()
{
    // IF an ObjectPool has been created for Serial objects
    ObjectPool<Serial> myPool;
    // AND we acquired and released 10 objects from the pool, while
    //     remembering their raw pointers
    const size_t numberOfObjectsToRetrieve { 10 };
    // A set to remember all raw pointers that have been handed out by the pool.
    set<Serial*> retrievedSerialPointers;
    vector<shared_ptr<Serial>> retrievedSerials;
    for (size_t i { 0 }; i < numberOfObjectsToRetrieve; ++i) {
        auto object { myPool.acquireObject() };
        retrievedSerialPointers.push_back(object.get());
        // Add the retrieved Serial to the vector to keep it 'alive'.
        retrievedSerials.push_back(object);
    }
    // Release all objects back to the pool.
    retrievedSerials.clear();

    // The above loop has created 10 Serial objects, with 10 different
    // addresses, and released all 10 Serial objects back to the pool.

    // WHEN again retrieving 10 objects from the pool, and
    //      remembering their raw pointers.
    set<Serial*> newlyRetrievedSerialPointers;
    for (size_t i { 0 }; i < numberOfObjectsToRetrieve; ++i) {
        auto object { myPool.acquireObject() };
        newlyRetrievedSerialPointers.push_back(object.get());
        // Add the retrieved Serial to the vector to keep it 'alive'.
        retrievedSerials.push_back(object);
    }
    // Release all objects back to the pool.
    retrievedSerials.clear();

    // THEN all addresses of the 10 newly acquired objects must have been
    //      seen already during the first loop of acquiring 10 objects.
    //      This makes sure objects are properly re-used by the pool.
    Assert::IsTrue(retrievedSerialPointers == newlyRetrievedSerialPointers);
}
\end{cpp}

如果添加了所有这些测试并运行,Test Explorer应该看起来像 图 30.7。当然,如果一个或多个测试失败,将面临单元测试中的经典问题:是测试,还是代码出现了问题?

\myGraphic{0.5}{content/part5/chapter30/images/7.png}{图 30.7}

\mySamllsection{调试测试}

Visual C++测试框架使调试失败的单元测试变得非常容易。Test Explorer显示了在单元测试失败时捕获的堆栈跟踪,包含直接指向错误行的超链接。

有时直接在调试器中运行单元测试有用,这样可以在运行时检查变量,逐行逐步通过代码等。单元测试中的某些代码行上设置断点,在Test Explorer中右键单击单元测试并点击调试。测试框架开始在调试器中运行选定的测试,并在断点处中断,就可以逐步调试代码了。

\mySamllsection{享受单元测试}

前面的测试应该提供了一个很好的开始,如何为真实代码编写专业级别的测试。这只是冰山一角,前面的示例应该能帮助各位读者,从而可以为ObjectPool类模板编写更多的测试。

例如,可以向ObjectPool添加一个capacity()成员函数,该函数返回分发出去的对象数量和可用而无需分配新内存块的对象数量之和。这类似于vector的capacity()成员函数,返回在没有重新分配的情况下,可以存储在vector中的元素总数。有了这样的成员函数,可以添加一个测试,验证池总是以比上次池增长时多一倍的数量增长。

对于给定的一段代码,可以编写无限多的单元测试,这就是单元测试最棒的地方。如果发现自己想知道的代码可能会如何响应某种情况,那这就是一个单元测试。如果发现子系统的某个特定方面似乎有问题,请增加对该特定区域的单元测试覆盖率。即使只是想从客户的角度出发,看看与自己的类一起工作是什么感觉,编写单元测试也是获得不同视角的好方法,甚至可能决定在实现代码之前编写单元测试。可以在实现之前开始使用计划的接口,这可能会展示之前没有考虑到的用例和错误条件。



















