
Large programming projects are rarely finished when a feature-complete goal is reached. There are always bugs to find and fix, both during and after the main development phase. It is essential to understand the shared responsibility of quality control and the life cycle of a bug to perform well in a group.

\mySubsubsection{30.1.1.}{Whose Responsibility Is Testing?}

Software development organizations have different approaches to testing. In a small startup, there may not be a group of people whose full-time job is testing the product. Testing may be the responsibility of the individual developers, or all the employees of the company may be asked to lend a hand and try to break the product before its release. In larger organizations, a full-time quality assurance staff probably qualifies a release by testing it according to a set of criteria. Nonetheless, some aspects of testing may still be the responsibility of the developers. Even in organizations where the developers have no role in formal testing, you still need to be aware of what your responsibilities are in the larger process of quality assurance.

\mySubsubsection{30.1.2.}{The Life Cycle of a Bug}

All good engineering groups recognize that bugs will occur in software both before and after its release. There are many different ways to deal with these problems. Figure 30.1 shows a formal bug process, expressed as a flow chart. In this particular process, a bug is always filed by a member of the QA team. The bug reporting software sends a notification to the development manager, who sets the priority of the bug and assigns the bug to the appropriate module owner. The module owner can accept the bug or explain why the bug actually belongs to a different module or is invalid, giving the development manager the opportunity to assign it to someone else.

Once the bug has found its rightful owner, a fix is made, and the developer marks the bug as “fixed.” At this point, the QA engineer verifies that the bug no longer exists and marks the bug as “closed” or reopens the bug if it is still present.

\myGraphic{0.7}{content/part5/chapter30/images/1.png}{FIGURE 30.1}

Figure 30.2 shows a less formal approach. In this workflow, anybody can file a bug and assign an initial priority and a module. The module owner receives the bug report and can either accept it or reassign it to another engineer or module. When a correction is made, the bug is marked as “fixed.” Toward the end of the testing phase, all the developers and QA engineers divide up the fixed bugs and verify that each bug is no longer present in the current build. The release is ready when all bugs are marked as “closed.”

\myGraphic{0.7}{content/part5/chapter30/images/2.png}{FIGURE 30.2}

\mySubsubsection{30.1.3.}{Bug-Tracking Tools}

There are many ways to keep track of software bugs, from informal spreadsheet- or e-mail-based schemes to expensive third-party bug-tracking software. The appropriate solution for your organization depends on the group’s size, the nature of the software, and the level of formality you want to build around bug fixing.

There are also a number of free open-source bug-tracking solutions available. One of the more popular free tools for bug tracking is Bugzilla (bugzilla.org), written by the authors of the Mozilla and Firefox web browser. As an open-source project, Bugzilla has gradually accumulated a number of useful features to the point where it now rivals expensive bug-tracking software packages. Here are just a few of its many features:

\begin{itemize}
\item
Customizable settings for a bug, including its priority, associated component, status, and so on

\item
E-mail notification of new bug reports or changes to an existing report

\item
Tracking of dependencies between bugs and resolution of duplicate bugs

\item
Reporting and searching tools

\item
A web-based interface for filing and updating bugs
\end{itemize}

Figure 30.3 shows a bug being entered into a Bugzilla project that was set up for the second edition of this book. For my purposes, each chapter was input as a Bugzilla component. The filer of the bug can specify the severity of the bug (that is, how big of a deal it is). A summary and description are included to make it possible to search for the bug or list it in a report format.

\myGraphic{0.7}{content/part5/chapter30/images/3.png}{FIGURE 30.3}

When writing a bug report, make sure to include as much information as possible in the report. For example, if your bug report is about an error you get, put the full error message in the bug report as text, not only as a screenshot. This allows other people to find your bug report if they come across the same error.

Bug-tracking tools like Bugzilla are essential components of a professional software development environment. In addition to supplying a central list of currently open bugs, bug-tracking tools provide an important archive of previous bugs and their fixes. A support engineer, for instance, might use the tool to search for a problem similar to one reported by a customer. If a fix was made, the support person will be able to tell the customer which version they need to update to or how to work around the problem.














