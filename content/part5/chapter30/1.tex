
大型编程项目很少在功能完整的目标达成时就结束。无论是在主要开发阶段期间还是之后,总是有错误需要发现和修复。为了在团队中表现良好,理解质量控制的共享责任和错误的生命周期也非常重要。

\mySubsubsection{30.1.1.}{谁负责测试?}

软件开发组织有不同的测试方法。一个小型初创公司中,可能没有一群人的全职工作是测试产品。测试可能是个人开发者的责任,或者公司所有员工都可能要求在产品发布前帮忙测试,并尝试破坏产品。较大的组织中,全职质量保证人员可能会根据一套标准测试产品以确定其是否可以发布。尽管如此,某些方面的测试可能仍然是由开发者的责任。即使在开发者不参与正式测试的组织中,仍然需要了解质量保证更大过程中的责任是什么。

\mySubsubsection{30.1.2.}{错误的生命周期}

所有优秀的工程团队都认识到,软件在发布前后都会出现错误。有许多不同的方法来处理这些问题。图 30.1 展示了一个正式的错误处理流程,用流程图表示。在这个特定的流程中,错误总是由 QA 团队的成员提交。错误报告软件会向开发经理发送通知,开发经理设置错误的优先级,并将错误分配给适当的模块负责人。模块负责人可以接受错误,或者解释为什么错误实际上属于另一个模块或无效,给开发经理机会将其分配给其他人。

\myGraphic{0.8}{content/part5/chapter30/images/1.png}{图 30.1}

当错误找到了其所有者,就会进行修复,开发者将错误标记为“已修复”。此时,QA 工程师验证错误,若已不再存可将错误标记为“已关闭”,如果再次错误出现,则重新打开错误。

图 30.2 展示了一个不太正式的方法。在这个工作流程中,任何人都可以提交错误并分配一个初始优先级和一个模块。模块负责人收到错误报告,可以接受它或者将其重新分配给另一个工程师或模块。进行更正后,错误标记为“已修复”。在测试阶段即将结束时,所有开发者和 QA 工程师分配修复的错误,并验证每个错误在当前构建中不再存在。当所有错误都标记为“已关闭”时,就可以准备发布了。

\myGraphic{0.5}{content/part5/chapter30/images/2.png}{图 30.2}

\mySubsubsection{30.1.3.}{错误跟踪工具}

有许多方法可以跟踪软件错误,从非正式的基于电子表格或电子邮件的计划,到昂贵的第三方错误跟踪软件。适合组织的解决方案取决于团队的大小、软件的性质,以及希望在错误修复周围构建的正式程度。

还有许多免费的开源错误跟踪解决方案可用。一个更受欢迎的免费错误跟踪工具是 Bugzilla(bugzilla.org),由 Mozilla 和 Firefox 网络浏览器的作者编写。作为一个开源项目,Bugzilla 逐渐积累了许多有用的功能,现在它甚至可以与昂贵的错误跟踪软件相媲美。以下只是其众多功能中的一小部分:

\begin{itemize}
\item
可自定义的错误设置,包括其优先级、关联组件、状态等

\item
关于新错误报告或现有报告变更的电子邮件通知

\item
跟踪错误之间的依赖关系并解决重复错误

\item
报告和搜索工具

\item
用于提交和更新错误的基于网页的界面
\end{itemize}

图 30.3 展示了一个错误被输入到为本书第二版设置的 Bugzilla 项目中。为了我的目的,每一章都可作为 Bugzilla 的一个组件,错误提交者可以指定错误的严重性(即它有多大影响)。包括摘要和描述,以便可以搜索错误或以报告格式列出它。

\myGraphic{0.8}{content/part5/chapter30/images/3.png}{图 30.3}

在编写错误报告时,请确保在报告中尽可能多地包含信息。例如,如果错误报告是关于收到的错误,请将完整的错误消息作为文本放入错误报告中,而不仅仅是截图。这样其他人在遇到相同的错误时,可以找到已有的错误报告。

像 Bugzilla 这样的错误跟踪工具,是专业软件开发环境的基本组成部分。除了提供当前打开错误的中央列表外,错误跟踪工具还提供了以前错误及其修复的重要档案。例如,支持工程师可能会使用该工具搜索与客户报告的问题类似的问题。如果进行了修复,支持人员将能够告诉客户他们需要更新到哪个版本,或者如何解决问题。














