通过解决下面的练习,可以练习本章讨论的内容。所有练习的解决方案都可以在本书的网站\url{www.wiley.com/go/proc++6e}下载到源码。若在练习中卡住了,可以考虑先重读本章的部分内容,试着自己找到答案,再在从网站上寻找解决方案。

\begin{itemize}
\item
\textbf{练习 30-1}: 测试的三个类型是什么?

\item
\textbf{练习 30-2}: 为以下代码列出可以想到的单元测试:

\begin{cpp}
export class Foo
{
    public:
        // Constructs a Foo. Throws invalid_argument if a >= b.
        explicit Foo(int a, int b) : m_a { a }, m_b { b }
        {
            if (a >= b) {
                throw std::invalid_argument { "a should be less than b." };
            }
        }
        int getA() const { return m_a; }
        int getB() const { return m_b; }
    private:
        int m_a { 0 };
        int m_b { 0 };
};
\end{cpp}

\item
\textbf{练习 30-3}: 如果使用 Visual C++,请使用 Visual C++ 测试框架实现在 练习 30-2 中列出的单元测试。

\item
\textbf{练习 30-4}: 假设编写了一个计算数字阶乘的函数。

数字 n 的阶乘,写作 n!,是所有 1 到 n 的数字的乘积。例如,3! 等于 1×2×3。你决定遵循本章中的建议,并为代码编写单元测试。运行代码来计算 10!;它产生了 36288000。编写一个单元测试,验证当要求计算 10 的阶乘时,代码会产生 36288000。这样设计的单元测试怎么样?
\end{itemize}
