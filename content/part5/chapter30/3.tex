Fuzz testing, also known as fuzzing, involves a fuzzer that automatically generates random input data for a program or component to try to find unhandled edge cases. Typically, a recipe is provided that specifies how input data needs to be structured so it can be used as input for the program. If clearly wrongly structured input is provided to a program, its input data parser will likely immediately reject it. A fuzzer’s job then is to try to generate input data that is not obviously wrongly structured, so it won’t be rejected immediately by the program, but that could trigger some faulty logic further along during the execution of the program. Since a fuzzer generates random input data, it requires a lot of resources to cover the entire input space. An option is to run such fuzz testing scenarios in a cloud. There are several libraries available for implementing fuzz testing, for example libFuzzer (\url{llvm.org/docs/LibFuzzer.html}) and honggfuzz (\url{github.com/google/honggfuzz}).