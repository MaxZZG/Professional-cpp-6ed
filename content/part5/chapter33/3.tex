
第二种与工厂相关的模式称为工厂方法模式。使用这种模式,具体工厂完全决定创建哪种对象。早期的抽象工厂示例中,ICarFactory 有一个成员函数,用于创建 SUV 或轿车。使用工厂方法模式,只需向工厂请求一辆汽车,具体工厂决定究竟要建造什么。让我们看看另一个汽车工厂的模拟。

\mySubsubsection{33.3.1.}{示例:模拟第二个汽车工厂}

现实世界中,当谈论驾驶汽车时,可以对特定类型的汽车不进行指明。可能是在讨论丰田或福特。这没关系,因为丰田和福特都是可驾驶的。现在,想要一辆新车,需要指定想要丰田还是福特,对吧?不一定。可以说,“我想要一辆车”,根据所在的地点,会得到一辆特定的车。如果在丰田工厂说“我想要一辆车”,那么可能会得到一辆丰田。(或者会被逮捕,取决于如何提出请求。)如果在福特工厂说“我想要一辆车”,会得到一辆福特。

同样的概念也适用于 C++ 编程。第一个概念,一个可驾驶的通用汽车,并不是什么新鲜事;这是标准的多态性,在第5章中描述。可以编写一个抽象的 ICar 接口,定义一个虚拟的 drive() 成员函数。丰田和福特都可以实现这样的接口。

程序可以在不知道它们是丰田还是福特的情况下驾驶汽车,但在使用标准面向对象编程时,需要指定丰田或福特的地方是在创建汽车时。这里,需要调用一个或另一个的构造函数。这时,不能只说“我想要一辆车”,但还有一个平行的汽车工厂类层次结构。CarFactory 基类可以定义一个公共的非虚拟 requestCar() 成员函数,将工作转发给一个private的虚拟 createCar() 成员函数。ToyotaFactory 和 FordFactory 派生类覆盖 createCar() 成员函数以构建丰田或福特。图 33.3 显示了 ICar 和 CarFactory 层次结构。

\myGraphic{0.8}{content/part5/chapter33/images/3.png}{图 33.3}

现在程序中有一个 CarFactory 对象。当程序中的代码,比如一个汽车经销商,想要一辆新车时,会调用 CarFactory 对象的 requestCar()。取决于这个汽车工厂实际上是 ToyotaFactory 还是 FordFactory,代码会得到一辆 Toyota 或 Ford。图 33.4 显示了使用 ToyotaFactory 的汽车经销商程序中的对象。

\myGraphic{0.8}{content/part5/chapter33/images/4.png}{图 33.4}

图 33.5 显示了相同的程序,但是用 FordFactory 代替了 ToyotaFactory。注意,CarDealer 对象及其与工厂的关系保持不变。

\myGraphic{0.8}{content/part5/chapter33/images/5.png}{图 33.5}

这个例子演示了使用工厂的多态性。当要求汽车工厂提供一辆汽车时,不知道是丰田工厂还是福特工厂,但无论如何它都会给你一辆可以驾驶的汽车。这种方法导致了易于扩展的程序,简单地改变工厂实例就可以使程序适用于完全不同的对象和类集。

\mySubsubsection{33.3.2.}{工厂方法的实现}

使用工厂的一个原因是,想要创建的对象类型可能取决于某些条件。例如,想要一辆汽车,可能想把订单放入到目前为止接收请求最少的工厂,而不考虑最终得到的汽车是 Toyota 还是 Ford。以下展示了如何在 C++ 中实现这样的工厂。

首先需要的是汽车的层次结构:

\begin{cpp}
export class ICar
{
    public:
        virtual ~ICar() = default; // Always a virtual destructor!
        virtual std::string info() const = 0;
};

export class Ford : public ICar
{
    public:
        std::string info() const override { return "Ford"; }
};

export class Toyota : public ICar
{
    public:
        std::string info() const override { return "Toyota"; }
};
\end{cpp}

CarFactory 基类更有趣一些,每个工厂都跟踪生产的汽车数量。当调用public非虚拟 requestCar() 成员函数时,工厂生产的汽车数量增加一,并调用private虚拟 createCar() 成员函数,后者创建并返回一个新的具体汽车。这种模式也称为非虚拟接口模式(NVI)。这个想法是,各个工厂覆盖 createCar() 以返回适当的类型的汽车。CarFactory 本身实现 requestCar(),负责更新生产的汽车数量。requestCar() 成员函数是模板方法设计模式的一个例子。

CarFactory 还提供了一个public成员函数,用于查询每个工厂生产的汽车数量。CarFactory 类和派生类的类定义如下:

\begin{cpp}
export class CarFactory
{
    public:
        virtual ~CarFactory() = default; // Always a virtual destructor!
        // Omitted defaulted default ctor, copy/move ctor, copy/move assignment op.

        std::unique_ptr<ICar> requestCar()
        {
            // Increment the number of cars produced and return the new car.
            ++m_numberOfCarsProduced;
            return createCar();
        }

        unsigned getNumberOfCarsProduced() const { return m_numberOfCarsProduced; }
    private:
        virtual std::unique_ptr<ICar> createCar() = 0;
        unsigned m_numberOfCarsProduced { 0 };
};

export class FordFactory final : public CarFactory
{
    private:
        std::unique_ptr<ICar> createCar() override {
            return std::make_unique<Ford>(); }
};

export class ToyotaFactory final : public CarFactory
{
    private:
        std::unique_ptr<ICar> createCar() override {
            return std::make_unique<Toyota>(); }
};
\end{cpp}

派生类简单地覆盖了 createCar(),以返回它们生产的特定类型的汽车。

\begin{myNotic}{NOTE}
工厂方法是实现虚拟构造函数的一种方式,虚拟构造函数是创建不同类型对象的成员函数。例如,requestCar() 成员函数根据调用它的具体工厂对象创建 Toyota 和 Ford。
\end{myNotic}

\mySubsubsection{33.3.3.}{使用工厂方法}

使用工厂的最简单方法是实例化,并调用适当的成员函数,如下面的代码段所示:

\begin{cpp}
ToyotaFactory myFactory;
auto myCar { myFactory.requestCar() };
println("{}", myCar->info()); // Outputs Toyota
\end{cpp}

一个更有趣的例子是利用虚拟构造函数,在生产的汽车最少的工厂中建造一辆汽车。为此,可以创建一个新的工厂,称为 LeastBusyFactory,它从 CarFactory 派生并在其构造函数中接受其他 CarFactory 对象的列表。与所有 CarFactory 类必须做的那样,LeastBusyFactory 覆盖了 createCar() 成员函数。它的实现查找传递给构造函数的工厂列表中最不繁忙的工厂,并要求该工厂创建一辆汽车。这是一个此类工厂的实现:

\begin{cpp}
class LeastBusyFactory final : public CarFactory
{
    public:
        // Constructs an instance, taking ownership of the given factories.
        explicit LeastBusyFactory(vector<unique_ptr<CarFactory>> factories);
    private:
        unique_ptr<ICar> createCar() override;
        vector<unique_ptr<CarFactory>> m_factories;
};

LeastBusyFactory::LeastBusyFactory(vector<unique_ptr<CarFactory>> factories)
    : m_factories { move(factories) }
{
    if (m_factories.empty()) {
        throw runtime_error { "No factories provided." };
    }
}

unique_ptr<ICar> LeastBusyFactory::createCar()
{
    auto leastBusyFactory { ranges::min_element(m_factories,
        [](const auto& factory1, const auto& factory2) {
            return factory1->getNumberOfCarsProduced() <
                factory2->getNumberOfCarsProduced(); }) };
    return (*leastBusyFactory)->requestCar();
}
\end{cpp}

以下代码使用这个工厂,让生产汽车数量最少的工厂建造10辆汽车,无论它们是什么品牌:

\begin{cpp}
vector<unique_ptr<CarFactory>> factories;

// Create 3 Ford factories and 1 Toyota factory.
factories.push_back(make_unique<FordFactory>());
factories.push_back(make_unique<FordFactory>());
factories.push_back(make_unique<FordFactory>());
factories.push_back(make_unique<ToyotaFactory>());

// To get more interesting results, preorder some cars from specific factories.
for (size_t i : {0, 0, 0, 1, 1, 2}) { factories[i]->requestCar(); }

// Create a factory that automatically selects the least busy
// factory from a list of given factories.
LeastBusyFactory leastBusyFactory { move(factories) };

// Build 10 cars from the least busy factory.
for (unsigned i { 0 }; i < 10; ++i) {
    auto theCar { leastBusyFactory.requestCar() };
    println("{}", theCar->info());
}
\end{cpp}

当程序执行时,会打印出每辆生产汽车的品牌。

\begin{shell}
Toyota
Ford
Toyota
Ford
Ford
Toyota
Ford
Ford
Ford
Toyota
\end{shell}

\mySubsubsection{33.3.4.}{其他用法}

可以使用工厂方法模式来模拟现实世界的工厂以外的更多场景。例如,一个文字处理器,希望支持不同语言的文档,其中每个文档使用一种语言。文字处理器有许多方面,选择文档语言需要不同的支持:文档中使用的字符集(是否需要带重音的字符)、拼写检查器、同义词词典以及文档的显示方式等。可以使用工厂来设计一个文字处理器,通过编写一个 LanguageFactory 基类和为每种感兴趣的语言派生的工厂,例如 EnglishLanguageFactory 和 FrenchLanguageFactory。当用户为文档指定一种语言时,程序使用适当的 LanguageFactory 来创建特定语言的功能实例。例如,在工厂上调用 createSpellchecker() 成员函数来创建一个特定语言的拼写检查器,再用新语言的新的拼写检查器替换,前一种语言的拼写检查器。







