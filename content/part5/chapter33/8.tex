
观察者模式用于让观察者被可观察对象(=主题)通知,具体的观察者注册到他们感兴趣的可观察对象上。当可观察对象的状态发生变化时,会通知所有注册的观察者这一变化。使用观察者模式的主要好处是减少了耦合。可观察类不需要知道正在观察它的具体观察者类型。

\mySubsubsection{33.8.1.}{示例:公开主题事件}

这个示例包括具有可变数量参数的通用事件。主题可以公开特定的事件,例如:当主题的数据修改时、当主题的数据删除时等。

\mySubsubsection{33.8.2.}{可观察对象的实现}

首先,定义了一个可变参数的类模板 Event,这个类存储了一个具有可变数量参数的函数map。提供一个 addObserver() 成员函数来注册一个新的观察者,形式为一个在事件触发时应该通知的函数。addObserver() 返回一个 EventHandle,随后可以传递给 removeObserver() 以取消注册观察者。这个 EventHandle 只是一个数字,每次注册一个观察者时都会增加。最后,raise() 成员函数通知所有注册的观察者事件已触发。

\begin{cpp}
using EventHandle = unsigned int;

template <typename... Args>
class Event final
{
    public:
        // Adds an observer. Returns an EventHandle to unregister the observer.
        [[nodiscard]] EventHandle addObserver(
        function<void(const Args&...)> observer)
        {
            auto number { ++m_counter };
            m_observers[number] = move(observer);
            return number;
        }

        // Unregisters the observer pointed to by the given handle.
        void removeObserver(EventHandle handle)
        {
            m_observers.erase(handle);
        }

        // Raise event: notifies all registered observers.
        void raise(const Args&... args)
        {
            for (const auto& [_, callback] : m_observers) { callback(args...); }
        }
    private:
        unsigned int m_counter { 0 };
        map<EventHandle, function<void(const Args&...)>> m_observers;
};
\end{cpp}

想要暴露的观察者可以注册自己的事件类,只需提供注册和取消注册成员函数即可。由于使用了可变参数的类模板,可以创建任意数量参数的 Event 实例。这使得可观察对象可以向观察者传递相关信息。这是一个示例:

\begin{cpp}
class ObservableSubject
{
    public:
        EventHandle registerDataModifiedObserver(const auto& observer) {
        return m_eventDataModified.addObserver(observer); }
        void unregisterDataModifiedObserver(EventHandle handle) {
            m_eventDataModified.removeObserver(handle); }

        EventHandle registerDataDeletedObserver(const auto& observer) {
            return m_eventDataDeleted.addObserver(observer); }
        void unregisterDataDeletedObserver(EventHandle handle) {
            m_eventDataDeleted.removeObserver(handle); }

        void modifyData()
        {
            // ...
            m_eventDataModified.raise(1, 2.3);
        }

        void deleteData()
        {
            // ...
            m_eventDataDeleted.raise();
        }
    private:
        Event<int, double> m_eventDataModified;
        Event<> m_eventDataDeleted;
};
\end{cpp}

\mySubsubsection{33.8.3.}{使用观察者}

以下是一些测试代码,演示了如何使用实现的观察者模式。假设有一个可以处理修改事件的独立全局函数 modified():

\begin{cpp}
void modified(int a, double b) { println("modified({}, {})", a, b); }
\end{cpp}

假设还有一个可以处理修改事件的类 Observer:

\begin{cpp}
class Observer final
{
    public:
        explicit Observer(ObservableSubject& subject) : m_subject { subject }
        {
            m_subjectModifiedHandle = m_subject.registerDataModifiedObserver(
                [this](int i, double d) { onSubjectModified(i, d); });
        }
        ~Observer()
        {
            m_subject.unregisterDataModifiedObserver(m_subjectModifiedHandle);
        }
    private:
        void onSubjectModified(int a, double b)
        {
            println("Observer::onSubjectModified({}, {})", a, b);
        }
        ObservableSubject& m_subject;
        EventHandle m_subjectModifiedHandle;
};
\end{cpp}

最后,可以构建一个 ObservableSubject 实例,并注册一些观察者:

\begin{cpp}
ObservableSubject subject;

auto handleModified { subject.registerDataModifiedObserver(modified) };
auto handleDeleted { subject.registerDataDeletedObserver(
    []{ println("deleted"); }) };
Observer observer { subject };

subject.modifyData();
subject.deleteData();

println("");

subject.unregisterDataModifiedObserver(handleModified);
subject.modifyData();
subject.deleteData();
\end{cpp}

输出如下:

\begin{shell}
modified(1, 2.3)
Observer::onSubjectModified(1, 2.3)
deleted

Observer::onSubjectModified(1, 2.3)
deleted
\end{shell}

\begin{myNotic}{NOTE}
使用观察者模式时,需要注意的问题是观察者与主题之间的生命周期耦合。之前的例子中,如果主题仍然存在,Observer 的析构函数才能正常工作。
\end{myNotic}





