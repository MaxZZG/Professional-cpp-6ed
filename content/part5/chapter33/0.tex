\noindent
\textbf{WHAT’S IN THIS CHAPTER?}

\begin{itemize}
\item
What a pattern is and what the difference is with a design technique

\item
How to use the following patterns:

\begin{itemize}
\item
Strategy

\item
Abstract factory

\item
Factory method

\item
Adapter

\item
Proxy

\item
Iterator

\item
Observer

\item
Decorator

\item
Chain of responsibility

\item
Singleton
\end{itemize}
\end{itemize}

\noindent
\textbf{WILEY.COM DOWNLOADS FOR THIS CHAPTER}

Please note that all the code examples for this chapter are available as part of this chapter’s code download on the book’s website at \url{www.wiley.com/go/proc++6e} on the Download Code tab.

A design pattern is a standard approach to program organization that solves a general problem. Design patterns are less language-specific than are techniques. The difference between a pattern and a technique is admittedly fuzzy, and different books employ different definitions. This book defines a technique as a strategy particular to the C++ language, while a pattern is a more general strategy for object-oriented design applicable to any object-oriented language, such as C++, C\#, Java, or Smalltalk. In fact, if you are familiar with C\# or Java programming, you will recognize many of these patterns.

Design patterns have names, and that’s a big advantage. The name carries meaning and therefore helps to more easily communicate about solutions. The names of patterns also help developers to more quickly understand a solution. However, certain patterns have several different names, and the distinctions between certain patterns is sometimes a bit vague with different sources describing and categorizing them slightly differently. In fact, depending on the books or other sources you use, you may find the same name applied to different patterns. There is even disagreement as to which design approaches qualify as patterns. With a few exceptions, this book follows the terminology used in the seminal book Design Patterns: Elements of Reusable Object-Oriented Software, by Erich Gamma et al. (Addison-Wesley Professional, 1994). Other pattern names and variations are noted when appropriate.

The design pattern concept is a simple but powerful idea. Once you are able to recognize the recurring object-oriented interactions that occur in a program, finding an elegant solution often becomes a matter of selecting the appropriate pattern to apply.

As there are books available discussing nothing but design patterns, this chapter briefly describes just a small selection of the more important design patterns in detail and presents sample implementations. This gives you a pretty good idea about what design patterns are all about.

Any aspect of design is likely to provoke debate among programmers, and I believe that is a good thing. Don’t simply accept these patterns as the only way to accomplish a task—draw on their approaches and ideas to refine them and form new patterns.









