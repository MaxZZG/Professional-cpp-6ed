\noindent
\textbf{内筒概要}

\begin{itemize}
\item
什么是模式,其与设计技术的区别

\item
如何使用以下模式:

\begin{itemize}
\item
策略

\item
抽象工厂

\item
工厂方法

\item
适配器

\item
代理

\item
迭代器

\item
观察者

\item
装饰器

\item
责任链

\item
单例
\end{itemize}
\end{itemize}

本章的所有代码示例都可以在\url{https://github.com/Professional-CPP/edition-6}获得。

设计模式是一种标准化的程序组织方法,用于解决通用问题,设计模式比技术更不依赖于特定的语言。模式与技术的区别确实有些模糊,不同的书籍采用不同的定义。本书将技术定义为特定于C++语言的策略,而模式是一种更通用的面向对象设计的策略,适用于面向对象的语言,如C++、C\#、Java或Smalltalk。实际上,如果熟悉C\#或Java编程,会更了解这些模式。

设计模式有名称,这是一个很大的优势。名称带有意义,因此有助于更轻松地交流解决方案。模式名称也有助于开发人员更快地理解解决方案,但某些模式有多个不同的名称,不同来源对某些模式的描述和分类有时略有不同。事实上,根据使用的书籍或其他来源,可能会发现相同的名称应用于不同的模式。甚至对于哪些设计方法可以称为模式,还存在一些争议。除了少数例外,本书遵循了《设计模式:可重用面向对象软件元素》(Design Patterns: Elements of Reusable Object-Oriented Software)一书中使用的术语,该书由Erich Gamma等人撰写(Addison-Wesley Professional,1994年),其他模式名称和变体也有所提及。

设计模式的概念虽然简单,但非常强大。当能够识别程序中反复出现的面向对象交互,找到优雅的解决方案,往往就变成了选择适当模式应用的问题。

由于有专门讨论设计模式的书籍,本章简要描述了更重要的设计模式中的一些详细情况,并提供了示例实现。可对对设计模式有一个很好的了解。

开发者之间很可能关于设计引发辩论,我相信这是一件好事。不要仅仅接受这些模式作为完成任务的唯一方式——利用方法和理念来完善它们,从而形成新的模式。









