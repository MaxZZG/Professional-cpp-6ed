通过解决下面的练习,可以练习本章讨论的内容。所有练习的解决方案都可以在本书的网站\url{www.wiley.com/go/proc++6e}下载到源码。若在练习中卡住了,可以考虑先重读本章的部分内容,试着自己找到答案,再在从网站上寻找解决方案。

本章练习的概念与其他章节不同。以下练习简要介绍了新的设计模式,并要求对这些模式进行研究,以深入理解。

\begin{itemize}
\item
\textbf{练习33-1}:虽然本章讨论了一些精选的设计模式,但显然还有更多的模式可供选择。其中一种模式是命令模式,将一个或多个操作封装在对象中。这个模式的一个主要应用场景是实现可撤销的操作。使用附录B中的一个与模式相关的参考资料来研究并了解命令模式,或者从维基百科文章\url{en.wikipedia.org/wiki/Software_design_pattern}开始研究。

\item
\textbf{练习33-2}:另一种模式是外观模式。通过这种模式,提供了一个易于使用的高级接口,以隐藏子系统的复杂性。这使得子系统更容易使用。研究外观模式以了解更多关于它的信息。

\item
\textbf{练习33-3}:原型模式允许通过构建特定类型对象的原型实例,来指定可创建的不同类型的对象。这些原型实例通常注册在某种类型的注册表中,客户端然后可以从注册表中请求特定类型对象的原型,并随后克隆该原型,以进一步使用。研究原型模式以了解更多关于它的信息。

\item
\textbf{练习33-4}:中介者模式用于控制一组对象之间的交互,提倡在各个子系统之间保持松耦合。研究中介者模式以了解更多关于它的信息。
\end{itemize}












