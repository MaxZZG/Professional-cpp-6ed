通过解决下面的练习,可以练习本章讨论的内容。所有练习的解决方案都可以在本书的网站\url{www.wiley.com/go/proc++6e}下载到源码。然而,若在练习中卡住了,在从网站上寻找解决方案之前,可以考虑先重读本章的部分内容,试着自己找到答案。

The concept of the exercises for this chapter is different compared to other chapters. The following exercises briefly introduce new patterns and ask you to do research into those patterns to learn more about them.

\begin{itemize}
\item
Exercise 33-1: Although this chapter discussed a nice selection of patterns, there are of course many more patterns available. One such pattern is the command pattern. It encapsulates an operation or operations in an object. One major use case of this pattern is to implement undoable operations. Use one of the patterns-related references in Appendix B, “Annotated Bibliography,” to research and learn about the command pattern, or, alternatively, start from the Wikipedia article, \url{en.wikipedia.org/wiki/Software_design_pattern}, to start your research.

\item
Exercise 33-2: Another pattern is the facade pattern. With this pattern, you provide an easyto-use higher-level interface to hide the complexity of a subsystem. This makes the subsystem easier to use. Research the facade pattern to learn more about it.

\item
Exercise 33-3: With the prototype pattern, you specify different kinds of objects that can be created by constructing prototypical instances of those objects. These prototypical instances are usually registered in some kind of registry. A client can then ask the registry for the prototype of a specific kind of object and subsequently clone the prototype for further use. Research the prototype pattern to learn more about it.

\item
Exercise 33-4: The mediator pattern is used to control the interactions between a set of objects. It advocates loose coupling between the different subsystems in play. Research the mediator pattern to learn more about it.
\end{itemize}












