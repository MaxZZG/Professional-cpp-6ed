
现实世界中,工厂生产的是有形的物体,比如桌子或汽车。同样地,面向对象编程中的工厂用于构造对象。当在程序中使用工厂时,想要创建特定对象的代码片段会向工厂请求对象的实例,而不是自己调用对象构造函数。例如,一个室内装饰程序可能有一个 FurnitureFactory 对象。当部分代码需要一件家具,比如桌子时,会调用 FurnitureFactory 对象的 createTable() 成员函数,返回一个桌子。这就是工厂的主要优点,抽象了对象创建过程。

乍一看,工厂似乎会导致设计复杂化。看起来只是在程序中增加了一个间接层,与其在 FurnitureFactory 上调用 createTable(),完全可以直接创建一个新的 Table 对象。然而,使用工厂的一个好处是,可以与类层次结构一起使用来构造对象,而无需知道其确切类型。工厂可以与类层次结构并行运行,这并不是说它们必须与类层次结构并行运行,工厂也可以只创建任意数量的具体类型。

使用工厂的另一个好处是,可以在代码中传递工厂,而不是直接创建各种对象,这样程序的不同部分就可以为特定的域创建相同类型的对象。

使用工厂的另一个原因是,当创建对象需要由工厂拥有,由于工厂的客户端不知道的信息、状态、资源等时。如果创建对象需要执行一系列复杂的步骤,并且需要按照正确的顺序执行,或者所有创建的对象都需要正确地与其他对象链接等,也可以使用工厂。

工厂可以交换,使用依赖注入,可以在程序中轻松替换不同的工厂。而且,正如可以对创建的对象使用多态性一样,也可以对工厂使用多态性。接下来的例子演示了这一点。

面向对象编程中有两种主要的与工厂相关的模式:抽象工厂模式和工厂方法模式。本节讨论抽象工厂模式,下一节讨论工厂方法模式。

\mySubsubsection{33.2.1.}{示例:模拟汽车工厂}

想象一个能够生产汽车的工厂,工厂创建请求它的类型的汽车。首先,需要一个层次结构来表示几种类型的汽车。图 33.1 引入了一个 ICar 接口,带有一个虚拟成员函数来检索有关特定汽车的信息。Toyota 和 Ford 汽车从 ICar 派生,最后,Ford 和 Toyota 都有一个轿车和一个 SUV 模型。

\myGraphic{0.8}{content/part5/chapter33/images/1.png}{图 33.1}

除了汽车层次结构,我们还需要一个工厂层次结构。抽象工厂只暴露了一个接口来创建一个独立于品牌的轿车或 SUV,具体工厂构建具体品牌的具体模型。图 33.2 显示了这个层次结构。

\myGraphic{0.5}{content/part5/chapter33/images/2.png}{图 33.2}

\mySubsubsection{33.2.2.}{抽象工厂的实现}

汽车层次的实现很简单:

\begin{cpp}
export class ICar
{
    public:
    virtual ~ICar() = default; // Always a virtual destructor!
    virtual std::string info() const = 0;
};

export class Ford : public ICar { };

export class FordSedan : public Ford
{
    public:
    std::string info() const override { return "Ford Sedan"; }
};

export class FordSuv : public Ford
{
    public:
    std::string info() const override { return "Ford SUV"; }
};

export class Toyota : public ICar { };

export class ToyotaSedan : public Toyota
{
    public:
    std::string info() const override { return "Toyota Sedan"; }
};

export class ToyotaSuv : public Toyota
{
    public:
    std::string info() const override { return "Toyota SUV"; }
};
\end{cpp}

接下来是 ICarFactory 接口。简单地暴露了成员函数,来创建一个不知道具体工厂或汽车的轿车或 SUV。

\begin{cpp}
export class ICarFactory
{
    public:
        virtual ~ICarFactory() = default; // Always a virtual destructor!
        virtual std::unique_ptr<ICar> makeSuv() = 0;
        virtual std::unique_ptr<ICar> makeSedan() = 0;
};
\end{cpp}

有了具体的工厂,创建具体的汽车模型。只显示了 FordFactory,与ToyotaFactory 类似。

\begin{cpp}
export class FordFactory : public ICarFactory
{
    public:
        std::unique_ptr<ICar> makeSuv() override {
            return std::make_unique<FordSuv>(); }
        std::unique_ptr<ICar> makeSedan() override {
            return std::make_unique<FordSedan>(); }
};
\end{cpp}

这个例子中使用的这种方法称为抽象工厂,创建的对象类型取决于使用的具体工厂。

\mySubsubsection{33.2.3.}{使用抽象工厂}

以下示例展示了如何使用已实现的工厂,有一个函数,接受一个抽象汽车工厂,并使用它来构建一辆轿车和一辆 SUV,并打印出每辆生产汽车的信息。这个函数对具体工厂或具体汽车一无所知;即,只使用接口。main() 函数创建了两个工厂,一个用于 Fords,一个用于 Toyotas,然后要求 createSomeCars() 函数使用这些工厂来创建一些汽车。

\begin{cpp}
void createSomeCars(ICarFactory& carFactory)
{
    auto sedan { carFactory.makeSedan() };
    auto suv { carFactory.makeSuv() };
    println("Sedan: {}", sedan->info());
    println("SUV: {}", suv->info());
}

int main()
{
    FordFactory fordFactory;
    ToyotaFactory toyotaFactory;
    createSomeCars(fordFactory);
    createSomeCars(toyotaFactory);
}
\end{cpp}

代码的输出如下:

\begin{shell}
Sedan: Ford Sedan
SUV: Ford SUV
Sedan: Toyota Sedan
SUV: Toyota SUV
\end{shell}















