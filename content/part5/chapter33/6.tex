
代理模式是一种将类的抽象与其底层表示分离的多个模式,代理对象作为真实对象的替身。这类对象通常在访问真实对象代价较高或不可能时使用。有一个文档编辑器,文档可能包含几个大对象,如图片。而不是在打开文档时加载所有这些图片,文档编辑器可以用代理对象替换所有图片。这些代理不会立即加载图片,只有在用户在文档中滚动并到达图片时,文档编辑器才会要求图片代理绘制自己。此时,代理将工作委派给真实的图片对象,后者加载图片。

代理还可以用于正确地屏蔽某些功能,同时确保客户端无法使用强制类型转换来绕过屏蔽。

\mySubsubsection{33.6.1.}{示例:隐藏网络连接问题}

考虑一个具有玩家类(Player)的网络游戏,该类代表互联网上加入游戏的玩家。玩家类包括需要网络连接的功能,如即时消息功能。如果玩家的连接变得无响应,代表该玩家的玩家对象将无法再接收即时消息。

因为不想向用户暴露网络问题,可能希望有一个单独的类来隐藏玩家的网络部分。这个玩家代理对象将代替实际的玩家对象。客户可能始终使用玩家代理类,作为通往真实玩家类的入口,或者系统在玩家不可用时替换玩家代理。在网络故障期间,玩家代理对象仍然可以显示玩家的名字和最后已知状态,并继续在原始玩家对象无法工作时执行。因此,代理类隐藏了底层玩家类的一些不受欢迎的语义。

\mySubsubsection{33.6.2.}{代理的实现}

第一步是定义一个 IPlayer 接口,包含玩家的公共接口:

\begin{cpp}
class IPlayer
{
    public:
        virtual ~IPlayer() = default; // Always virtual destructor.
        virtual string getName() const = 0;
        // Sends an instant message to the player over the network and
        // returns the reply as a string.
        virtual string sendInstantMessage(string_view message) const = 0;
};
\end{cpp}

玩家类实现了 IPlayer 接口。在这个例子中,sendInstantMessage() 需要网络连接才能正常工作,并且如果网络连接断开,则抛出异常。

\begin{cpp}
class Player : public IPlayer
{
    public:
        string getName() const override;
        // Network connectivity is required.
        // Throws an exception if network connection is down.
        string sendInstantMessage(string_view message) const override;
};
\end{cpp}

玩家代理类也实现了 IPlayer 接口,并包含另一个 IPlayer 实例(“真实”的玩家):

\begin{cpp}
class PlayerProxy : public IPlayer
{
    public:
        // Create a PlayerProxy, taking ownership of the given player.
        explicit PlayerProxy(unique_ptr<IPlayer> player);
        string getName() const override;
        // Network connectivity is optional.
        string sendInstantMessage(string_view message) const override;
    private:
        bool hasNetworkConnectivity() const;
        unique_ptr<IPlayer> m_player;
};
\end{cpp}

构造函数接受并拥有给定的 IPlayer 实例:

\begin{cpp}
PlayerProxy::PlayerProxy(unique_ptr<IPlayer> player)
    : m_player { move(player) } { }
\end{cpp}

getName() 成员函数简单地转发给底层的玩家:

\begin{cpp}
string PlayerProxy::getName() const { return m_player->getName(); }
\end{cpp}

玩家代理的 sendInstantMessage() 成员函数检查网络连接,并根据情况返回默认字符串或转发请求。这隐藏了底层玩家对象的 sendInstantMessage() 成员函数在网络连接断开时抛出异常的事实。

\begin{cpp}
string PlayerProxy::sendInstantMessage(string_view message) const
{
    if (hasNetworkConnectivity()) { return m_player->sendInstantMessage(message); }
    else { return "The player has gone offline."; }
}
\end{cpp}

\mySubsubsection{33.6.3.}{使用代理}

如果代理编写得很好,使用它与使用其他对象没有什么不同。对于 PlayerProxy 示例,使用代理的代码可能完全不知道它的存在。以下函数,旨在在玩家获胜时调用,可能与实际玩家或玩家代理对象打交道。代码能够以相同的方式处理这两种情况,因为代理能确保有一个结果。

\begin{cpp}
bool informWinner(const IPlayer& player)
{
    auto result { player.sendInstantMessage("You have won! Play again?") };
    if (result == "yes") {
        println("{} wants to play again.", player.getName());
        return true;
    } else {
        // The player said no, or is offline.
        println("{} does not want to play again.", player.getName());
        return false;
    }
}
\end{cpp}






