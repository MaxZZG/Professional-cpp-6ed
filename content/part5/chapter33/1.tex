
策略设计模式是支持依赖倒置原则(DIP)的一种方式;请参见第6章。通过这种模式,接口用来倒置依赖关系。为每个提供的服务创建接口。如果一个组件需要一组服务,这些服务的接口就会注入到组件中,这种机制称为依赖注入。使用策略模式可以使单元测试更容易,可以轻松地为服务模拟,本节介绍了使用策略模式实现的日志机制。

\mySubsubsection{33.1.1.}{示例:日志机制}

基于策略的日志记录器示例,使用了一个名为ILogger的接口或抽象基类。想要记录内容的代码都使用这个ILogger接口,这个接口的具体实现会注入到需要使用日志功能的代码中。单元测试也可以使用这种模式,为ILogger接口注入一个特殊的模拟实现,以验证是否记录了正确的信息。这种模式的一个巨大优势是,可以轻松地更换具体的日志记录器,而无需修改库代码;客户端代码只需传入想要使用的日志记录器即可。

\mySubsubsection{33.1.2.}{基于策略的日志记录器的实现}

实现提供了一个具有以下功能的Logger类:

\begin{itemize}
\item
记录单个字符串。

\item
每条日志消息都带有当前系统时间和相关的日志级别前缀。

\item
可以设置日志记录器只记录高于一定日志级别的消息。

\item
每条记录的消息都会刷新到磁盘,以便立即出现在文件中。
\end{itemize}

定义ILogger接口:

\begin{cpp}
export class ILogger
{
    public:
        virtual ~ILogger() = default; // Virtual destructor.

        // Enumeration for the different log levels.
        enum class LogLevel { Debug, Info, Error };

        // Sets the log level.
        virtual void setLogLevel(LogLevel level) = 0;

        // Logs a single message at the given log level.
        virtual void log(std::string_view message, LogLevel logLevel) = 0;
};
\end{cpp}

具体的Logger类实现:

\begin{cpp}
export class Logger : public ILogger
{
    public:
        explicit Logger(const std::string& logFilename);
        void setLogLevel(LogLevel level) override;
        void log(std::string_view message, LogLevel logLevel) override;
    private:
        // Converts a log level to a human readable string.
        std::string_view getLogLevelString(LogLevel level) const;

        std::ofstream m_outputStream;
        LogLevel m_logLevel { LogLevel::Error };
};
\end{cpp}

Logger类的实现。打开了日志文件,每条日志消息都会写入到文件,并带有日志级别前缀,然后刷新到磁盘。

\begin{cpp}
Logger::Logger(const string& logFilename)
{
    m_outputStream.open(logFilename, ios_base::app);
    if (!m_outputStream.good()) {
        throw runtime_error { "Unable to initialize the Logger!" };
    }
    println(m_outputStream, "{}: Logger started.", chrono::system_clock::now());
}

void Logger::setLogLevel(LogLevel level)
{
    m_logLevel = level;
}

string_view Logger::getLogLevelString(LogLevel level) const
{
    switch (level) {
        case LogLevel::Debug: return "DEBUG";
        case LogLevel::Info: return "INFO";
        case LogLevel::Error: return "ERROR";
    }
    throw runtime_error { "Invalid log level." };
}

void Logger::log(string_view message, LogLevel logLevel)
{
    if (m_logLevel > logLevel) { return; }
    println(m_outputStream, "{}: [{}] {}", chrono::system_clock::now(),
        getLogLevelString(logLevel), message);
}
\end{cpp}

\mySubsubsection{33.1.3.}{使用基于策略的日志记录器}

假设有一个名为Foo的类,想要使用日志功能。使用策略模式,具体的ILogger实例注入到类中,例如:通过构造函数。

\begin{cpp}
class Foo
{
    public:
        explicit Foo(ILogger* logger) : m_logger { logger }
        {
            if (m_logger == nullptr) {
                throw invalid_argument { "ILogger cannot be null." };
            }
        }
        void doSomething()
        {
            m_logger->log("Hello strategy!", ILogger::LogLevel::Info);
        }
    private:
        ILogger* m_logger;
};
\end{cpp}

创建Foo实例时,具体的ILogger可注入到其中:

\begin{cpp}
Logger concreteLogger { "log.out" };
concreteLogger.setLogLevel(ILogger::LogLevel::Debug);

Foo f { &concreteLogger };
f.doSomething();
\end{cpp}






























