\noindent
\textbf{WHAT’S IN THIS CHAPTER?}

\begin{itemize}
\item
How to write code that runs on multiple platforms

\item
How to mix different programming languages together
\end{itemize}

\noindent
\textbf{WILEY.COM DOWNLOADS FOR THIS CHAPTER}

Please note that all the code examples for this chapter are available as part of this chapter’s code download on the book’s website at \url{www.wiley.com/go/proc++6e} on the Download Code tab.

C++ programs can be compiled to run on a variety of computing platforms, and the language has been rigorously defined to ensure that programming in C++ for one platform is similar to programming in C++ for another. Yet, despite the standardization of the language, platform differences eventually come into play when writing professional-quality programs in C++. Even when development is limited to a particular platform, small differences in compilers can elicit major headaches. This chapter examines the necessary complication of programming in a world with multiple platforms and multiple programming languages.

The first part of this chapter surveys the platform-related issues that C++ programmers encounter. A platform is the collection of all the details that make up your development and/or run-time system. For example, your platform may be the Microsoft Visual C++ 2022 compiler running on Windows 11 on an Intel Core i7 processor. Alternatively, your platform might be the GCC 13.2 compiler running on Linux on a PowerPC processor. Both of these platforms are able to compile and run C++ programs, but there are significant differences between them.

The second part of this chapter looks at how C++ can interact with other programming languages. While C++ is a general-purpose language, it may not always be the right tool for the job. Through a variety of mechanisms, you can integrate C++ code with other languages that may better serve your needs.

