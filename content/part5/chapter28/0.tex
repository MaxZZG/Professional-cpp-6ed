\noindent
\textbf{内筒概要}

\begin{itemize}
\item
什么是软件生命周期模型,有瀑布模型、生鱼片模型、螺旋模型和敏捷的例子

\item
什么是软件工程方法,有统一过程、Rational统一过程、Scrum、极限编程和软件分类的例子

\item
什么是版本控制
\end{itemize}

本章开始介绍这本书的最后一部分,这部分是关于软件工程的。这一部分描述了软件工程方法、代码效率、测试、调试、设计技术和设计模式,以及如何针对多个平台进行开发。

最初学习编程时,可能有自己的时间表。如果愿意,可以最后一刻才开始,并且在实现过程中可以彻底改变起初的设计。在专业编程世界中,开发者很少有这样的灵活性。即使是最自由放任的工程经理也承认需要一定的流程。如今,了解软件工程过程与了解如何编码一样重要。

本章中,我将概述各种软件工程方法。不会深入探讨任何一种方法——关于软件工程流程的优秀书籍有很多。我的意图是粗略地介绍几种不同的流程,以便可以进行比较和对比。我不会提倡或反对特定的方法论。相反,我希望通过了解几种不同方法的权衡,能够构建一个对自己或团队都有效的流程。无论是独立承包商,还是团队由几个大陆上的数百名工程师组成,了解不同的软件开发方法将有助于日常工作的顺利进行。

本章的最后部分讨论了版本控制系统,它可以轻松地管理源代码并跟踪其历史。每家公司中,版本控制系统都是必不可少的,以避免源代码维护的噩梦。即使是个人项目,我也强烈建议使用这样的系统。








