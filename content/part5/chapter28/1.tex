软件开发史充满了失败项目的传说。从预算超标、市场推广不力的应用程序,到炒作的操作系统,似乎没有软件开发领域能够幸免于这种趋势。

即使软件成功到达用户手中,bug也变得很常见,以至于最终用户经常需要不断的更新和补丁。有时软件无法完成它应该完成的任务,或者以用户期望之外的方式工作。这些问题都指向一个事实:编写软件很难。

你可能会好奇为什么?软件工程似乎与其它形式的工程在失败频率上有所不同。尽管汽车也有bug,但很少看到它们因为缓冲区溢出,而突然停止并要求重新启动(尽管随着越来越多的汽车组件成为软件驱动,可能会看到)。你的电视可能不完美,但不需要升级到2.3版本来让6频道可以播放。

其他工程学科是否比软件开发更先进?一位土木工程师是否能够凭借桥梁建造的悠久历史建造一座工作桥梁?化学工程师是否能够成功构建一个化合物,因为大多数bug在早期世代中已经解决了?软件工程是否太新,或者它真的是一种具有内在特质导致bug、不可用的结果和注定失败的项目频发的学科?

软件工程似乎确实有些不同。首先,计算机技术变化迅速,给软件开发过程带来了不确定性。即使在你项目的整个过程中没有出现令人震惊的突破,IT行业的步伐也可能导致问题。软件往往需要快速开发,因为竞争非常激烈。

软件开发进度也可能不可预测。当单个棘手的bug可能需要几天甚至几周才能修复时,准确的项目进度规划几乎不可能。即使一切都在按计划进行,产品定义的广泛变化(称为功能膨胀或范围膨胀)也可能破坏这个过程。如果这种范围膨胀得不到控制,最终可能导致软件臃肿。

软件往往很复杂,没有简单准确的方法来证明程序没有bug。如果代码有bug或混乱,可能会对软件产生长期影响。软件系统往往非常复杂,以至于当人员流动发生时,没有人愿意接近一些混乱的遗留代码,这导致了无尽的修补和变通方法。

当然,标准商业风险也适用于软件,市场压力和沟通不畅也会造成干扰。许多开发者试图避开公司政治,但开发和产品营销团队之间出现敌意的情况很常见。

所有这些因素都对软件工程产品产生不利影响,表明需要某种流程。软件项目庞大、复杂且快速,为了避免失败,工程团队需要采用一种系统来控制这种难以驾驭的过程。

优雅设计的软件和干净、可维护的代码是可以开发的。我坚信这是可能的,但需要每个团队成员的努力,并且需要遵循正确的软件开发流程和实践。



















