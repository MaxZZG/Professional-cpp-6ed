
It’s unlikely that any book or engineering theory will perfectly match the needs of your project or organization. I recommend that you learn from as many approaches as you can and design your own process. Combining concepts from different approaches may be easier than you think. For example, RUP optionally supports an XP-like approach. Here are some tips for building the software engineering process of your dreams.

\mySubsubsection{28.4.1.}{Be Open to New Ideas}

Some engineering techniques seem crazy at first, or unlikely to work. Look at new innovations in software engineering methodologies as a way to refine your existing process. Try things out when you can. If XP sounds intriguing, but you’re not sure if it will work in your organization, see if you can work it in slowly, taking a few of the principles at a time or trying it out with a smaller pilot project.

\mySubsubsection{28.4.2.}{Bring New Ideas to the Table}

Most likely, your engineering team is made up of people from varying backgrounds. You may have startup veterans, long-time consultants, recent graduates, and PhDs all on your team. You all have different sets of experiences and different ideas of how a software project should be run. Sometimes the best processes turn out to be a combination of the way things are typically done in these very different environments.

\mySubsubsection{28.4.3.}{Recognize What Works and What Doesn’t Work}

At the end of a project (or better yet, during the project, as with the sprint retrospective of the Scrum methodology), get the team together to evaluate the process. Sometimes there’s a major problem that nobody notices until the entire team stops to think about it. Perhaps there’s a problem that everybody knows about but nobody has discussed.

Consider what isn’t working and see how those parts can be fixed. Some organizations require formal code reviews prior to any source code check-in. If code reviews are so long and boring that nobody does a good job, discuss code-reviewing techniques as a group.

Also consider what is going well and see how those parts can be extended. For example, if maintaining the feature tasks as a group-editable wiki is working, then maybe devote some time to making the website even better.

\mySubsubsection{28.4.4.}{Don’t Be a Renegade}

Whether a process is mandated by your manager or custom-built by the team, it’s there for a reason. If your process involves writing formal design documents, make sure you write them. If you think that the process is broken or too complex, talk to your manager about it. Don’t just avoid the process—it will come back to haunt you.


